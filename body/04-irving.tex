\chapter[Rev. Mr. Irving]{rev. mr. irving}


This gentleman has gained an almost unprecedented, and not an
altogether unmerited popularity as a preacher. As he is, perhaps,
though a burning and a shining light, not ``one of the fixed,'' we
shall take this opportunity of discussing his merits, while he is
at his meridian height; and in doing so, shall ``nothing extenuate,
nor set down aught in malice.''

Few circumstances shew the prevailing and preposterous rage for
novelty in a more striking point of view, than the success of
Mr. Irving's oratory. People go to hear him in crowds, and come
away with a mixture of delight and astonishment\textemdash they go
again to see if the effect will continue, and send others to try
to find out the mystery\textemdash and in the noisy conflict
between extravagant encomiums and splenetic objections, the true
secret escapes observation, which is, that the whole thing is,
nearly from beginning to end, a \emph{transposition of ideas}. If the
subject of these remarks had come out as a player, with all his
advantages of figure, voice, and action, we think he would have
failed: if, as a preacher, he had kept within the strict bounds of
pulpit-oratory, he would scarcely have been much distinguished
among his Calvinistic brethren: as a mere author, he would have
excited attention rather by his quaintness and affectation of an
obsolete style and mode of thinking, than by any thing else. But
he has contrived to jumble these several characters together in an
unheard-of and unwarranted manner, and the fascination is
altogether irresistible. Our Caledonian divine is equally an
anomaly in religion, in literature, in personal appearance, and in
public speaking. To hear a person spout Shakspeare on the stage is
nothing\textemdash the charm is nearly worn out\textemdash but to
hear any one spout Shakspeare (and that not in a sneaking
under-tone, but at the top of his voice, and with the full breadth
of his chest) from a Calvinistic pulpit, is new and wonderful. The
\emph{Fancy} have lately lost something of their gloss in public
estimation, and after the last fight, few would go far to see a
Neat or a Spring set-to;\textemdash but to see a man who is able
to enter the ring with either of them, or brandish a quarter-staff
with Friar Tuck, or a broad-sword with Shaw the Lifeguards' man,
stand up in a strait-laced old-fashioned pulpit, and bandy
dialectics with modern philosophers or give a \emph{cross-buttock} to a
cabinet minister, there is something in a sight like this also,
that is a cure for sore eyes. It is as if Crib or Molyneux had
turned Methodist parson, or as if a Patagonian savage were to come
forward as the patron-saint of Evangelical religion. Again, the
doctrine of eternal punishment was one of the staple arguments
with which, everlastingly drawled out, the old school of
Presbyterian divines used to keep their audiences awake, or lull
them to sleep; but to which people of taste and fashion paid
little attention, as inelegant and barbarous, till Mr. Irving,
with his cast-iron features and sledge-hammer blows, puffing like
a grim Vulcan, set to work to forge more classic thunderbolts, and
kindle the expiring flames anew with the very sweepings of
sceptical and infidel libraries, so as to excite a pleasing horror
in the female part of his congregation. In short, our popular
declaimer has, contrary to the Scripture-caution, put new wine
into old bottles, or new cloth on old garments. He has, with an
unlimited and daring licence, mixed the sacred and the profane
together, the carnal and the spiritual man, the petulance of the
bar with the dogmatism of the pulpit, the theatrical and
theological, the modern and the obsolete;\textemdash what wonder
that this splendid piece of patchwork, splendid by contradiction
and contrast, has delighted some and confounded others? The more
serious part of his congregation indeed complain, though not
bitterly, that their pastor has converted their meeting-house into
a play-house: but when a lady of quality, introducing herself and
her three daughters to the preacher, assures him that they have
been to all the most fashionable places of resort, the opera, the
theatre, assemblies, Miss Macauley's readings, and Exeter-Change,
and have been equally entertained no where else, we apprehend that
no remonstrances of a committee of ruling-elders will be able to
bring him to his senses again, or make him forego such sweet, but
ill-assorted praise. What we mean to insist upon is, that
Mr. Irving owes his triumphant success, not to any one quality for
which he has been extolled, but to a combination of qualities, the
more striking in their immediate effect, in proportion as they are
unlooked-for and heterogeneous, like the violent opposition of
light and shade in a picture. We shall endeavour to explain this
view of the subject more at large.

Mr. Irving, then, is no common or mean man. He has four or five
qualities, possessed in a moderate or in a paramount degree,
which, added or multiplied together, fill up the important space
he occupies in the public eye. Mr. Irving's intellect itself is of
a superior order; he has undoubtedly both talents and acquirements
beyond the ordinary run of every-day preachers. These alone,
however, we hold, would not account for a twentieth part of the
effect he has produced: they would have lifted him perhaps out of
the mire and slough of sordid obscurity, but would never have
launched him into the ocean-stream of popularity, in which he
``lies floating many a rood;''\textemdash but to these he adds
uncommon height, a graceful figure and action, a clear and
powerful voice, a striking, if not a fine face, a bold and fiery
spirit, and a most portentous obliquity of vision, which throw him
to an immeasurable distance beyond all competition, and
effectually relieve whatever there might be of common-place or
bombast in his style of composition. Put the case that Mr. Irving
had been five feet high\textemdash Would he ever have been heard
of, or, as he does now, have ``bestrode the world like a Colossus?''
No, the thing speaks for itself. He would in vain have lifted his
Lilliputian arm to Heaven, people would have laughed at his
monkey-tricks. Again, had he been as tall as he is, but had wanted
other recommendations, he would have been nothing.
\begin{quote}
  ``The player's province they but vainly try, \\
  Who want these powers, deportment, voice, and eye.''
\end{quote}
Conceive a rough, ugly, shock-headed Scotchman, standing up in the
Caledonian chapel, and dealing ``damnation round the land'' in a
broad northern dialect, and with a harsh, screaking voice, what
ear polite, what smile serene would have hailed the barbarous
prodigy, or not consigned him to utter neglect and derision? But
the Rev. Edward Irving, with all his native wildness, ``hath a
smooth aspect framed to make women'' saints; his very unusual size
and height are carried off and moulded into elegance by the most
admirable symmetry of form and ease of gesture; his sable locks,
his clear iron-grey complexion, and firm-set features, turn the
raw, uncouth Scotchman into the likeness of a noble Italian
picture; and even his distortion of sight only redeems the
otherwise ``faultless monster'' within the bounds of humanity, and,
when admiration is exhausted and curiosity ceases, excites a new
interest by leading to the idle question whether it is an
advantage to the preacher or not. Farther, give him all his actual
and remarkable advantages of body and mind, let him be as tall, as
strait, as dark and clear of skin, as much at his ease, as
silver-tongued, as eloquent and as argumentative as he is, yet
with all these, and without a little charlatanery to set them off,
he had been nothing. He might, keeping within the rigid line of
his duty and professed calling, have preached on for ever; he
might have divided the old-fashioned doctrines of election, grace,
reprobation, predestination, into his sixteenth, seventeenth, and
eighteenth heads, and his \emph{lastly} have been looked for as a
``consummation devoutly to be wished;'' he might have defied the
devil and all his works, and by the help of a loud voice and
strong-set person\pcdash{1}
\begin{quote}
  ``A lusty man to ben an Abbot able;''\textemdash
\end{quote}
have increased his own congregation, and been quoted among the
godly as a powerful preacher of the word; but in addition to this,
he went out of his way to attack Jeremy Bentham, and the town was
up in arms. The thing was new. He thus wiped the stain of musty
ignorance and formal bigotry out of his style. Mr. Irving must
have something superior in him, to look over the shining
close-packed heads of his congregation to have a hit at the \emph{Great
Jurisconsult} in his study. He next, ere the report of the former
blow had subsided, made a lunge at Mr. Brougham, and glanced an
eye at Mr. Canning; \emph{mystified} Mr. Coleridge, and \emph{stultified}
Lord Liverpool in his place\textemdash in the Gallery. It was rare
sport to see him, ``like an eagle in a dovecote, flutter the
Volscians in Corioli.'' He has found out the secret of attracting
by repelling. Those whom he is likely to attack are curious to
hear what he says of them: they go again, to show that they do not
mind it. It is no less interesting to the by-standers, who like to
witness this sort of \emph{onslaught}\textemdash like a charge of
cavalry, the shock, and the resistance. Mr. Irving has, in fact,
without leave asked or a licence granted, converted the Caledonian
Chapel into a Westminster Forum or Debating Society, with the
sanctity of religion added to it. Our spirited polemic is not
contented to defend the citadel of orthodoxy against all
impugners, and shut himself up in texts of Scripture and huge
volumes of the Commentators as an impregnable fortress;\textemdash
he merely makes use of the stronghold of religion as a
resting-place, from which he sallies forth, armed with modern
topics and with penal fire, like Achilles of old rushing from the
Grecian tents, against the adversaries of God and man. Peter
Aretine is said to have laid the Princes of Europe under
contribution by penning satires against them: so Mr. Irving keeps
the public in awe by insulting all their favourite idols. He does
not spare their politicians, their rulers, their moralists, their
poets, their players, their critics, their reviewers, their
magazine-writers; he levels their resorts of business, their
places of amusement, at a blow\textemdash their cities, churches,
palaces, ranks and professions, refinements, and
elegances\textemdash and leaves nothing standing but himself, a
mighty landmark in a degenerate age, overlooking the wide havoc he
has made! He makes war upon all arts and sciences, upon the
faculties and nature of man, on his vices and his virtues, on all
existing institutions, and all possible improvements, that nothing
may be left but the Kirk of Scotland, and that he may be the head
of it. He literally sends a challenge to all London in the name of
the \textsc{King} of \textsc{Heaven}, to evacuate its streets, to disperse its
population, to lay aside its employments, to burn its wealth, to
renounce its vanities and pomp; and for what?\textemdash that he
may enter in as the \emph{King of Glory}; or after enforcing his threat
with the battering-ram of logic, the grape-shot of rhetoric, and
the crossfire of his double vision, reduce the British metropolis
to a Scottish heath, with a few miserable hovels upon it, where
they may worship God according to \emph{the root of the matter}, and an
old man with a blue bonnet, a fair-haired girl, and a little child
would form the flower of his flock! Such is the pretension and the
boast of this new Peter the Hermit, who would get rid of all we
have done in the way of improvement on a state of barbarous
ignorance, or still more barbarous prejudice, in order to begin
again on a \emph{tabula rasa} of Calvinism, and have a world of his own
making. It is not very surprising that when nearly the whole mass
and texture of civil society is indicted as a nuisance, and
threatened to be pulled down as a rotten building ready to fall on
the heads of the inhabitants, that all classes of people run to
hear the crash, and to see the engines and levers at work which
are to effect this laudable purpose. What else can be the meaning
of our preacher's taking upon himself to denounce the sentiments
of the most serious professors in great cities, as vitiated and
stark-naught, of relegating religion to his native glens, and
pretending that the hymn of praise or the sigh of contrition
cannot ascend acceptably to the throne of grace from the crowded
street as well as from the barren rock or silent valley? Why put
this affront upon his hearers? Why belie his own aspirations?
\begin{quote}
  ``God made the country, and man made the town.''
\end{quote}
So says the poet; does Mr. Irving say so? If he does, and finds
the air of the city death to his piety, why does he not return
home again? But if he can breathe it with impunity, and still
retain the fervour of his early enthusiasm, and the simplicity and
purity of the faith that was once delivered to the saints, why not
extend the benefit of his own experience to others, instead of
taunting them with a vapid pastoral theory? Or, if our popular and
eloquent divine finds a change in himself, that flattery prevents
the growth of grace, that he is becoming the God of his own
idolatry by being that of others, that the glittering of
coronet-coaches rolling down Holborn-Hill to Hatton Garden, that
titled beauty, that the parliamentary complexion of his audience,
the compliments of poets, and the stare of peers discompose his
wandering thoughts a little; and yet that he cannot give up these
strong temptations tugging at his heart; why not extend more
charity to others, and shew more candour in speaking of himself?
There is either a good deal of bigoted intolerance with a
deplorable want of self-knowledge in all this; or at least an
equal degree of cant and quackery.

To whichever cause we are to attribute this hyperbolical tone, we
hold it certain he could not have adopted it, if he had been \emph{a
little man}.  But his imposing figure and dignified manner enable
him to hazard sentiments or assertions that would be fatal to
others. His controversial daring is \emph{backed} by his bodily
prowess; and by bringing his intellectual pretensions boldly into
a line with his physical accomplishments, he, indeed, presents a
very formidable front to the sceptic or the scoffer. Take a cubit
from his stature, and his whole manner resolves itself into an
impertinence. But with that addition, he \emph{overcrows} the town,
browbeats their prejudices, and bullies them out of their senses,
and is not afraid of being contradicted by any one \emph{less than
himself}. It may be said, that individuals with great personal
defects have made a considerable figure as public speakers; and
Mr. Wilberforce, among others, may be held out as an
instance. Nothing can be more insignificant as to mere outward
appearance, and yet he is listened to in the House of Commons. But
he does not wield it, he does not insult or bully it. He leads by
following opinion, he trims, he shifts, he glides on the silvery
sounds of his undulating, flexible, cautiously modulated voice,
winding his way betwixt heaven and earth, now courting popularity,
now calling servility to his aid, and with a large estate, the
``saints,'' and the population of Yorkshire to swell his influence,
never venturing on the forlorn hope, or doing any thing more than
``hitting the house between wind and water.'' Yet he is probably a
cleverer man than Mr. Irving.

There is a Mr. Fox, a Dissenting Minister, as fluent a speaker,
with a sweeter voice and a more animated and beneficent
countenance than Mr.  Irving, who expresses himself with manly
spirit at a public meeting, takes a hand at whist, and is the
darling of his congregation; but he is no more, because he is
diminutive in person. His head is not seen above the crowd the
length of a street off. He is the Duke of Sussex in miniature, but
the Duke of Sussex does not go to hear him preach, as he attends
Mr. Irving, who rises up against him like a martello tower, and is
nothing loth to confront the spirit of a man of genius with the
blood-royal. We allow there are, or may be, talents sufficient to
produce this equality without a single personal advantage; but we
deny that this would be the effect of any that our great preacher
possesses.  We conceive it not improbable that the consciousness
of muscular power, that the admiration of his person by strangers
might first have inspired Mr. Irving with an ambition to be
something, intellectually speaking, and have given him confidence
to attempt the greatest things. He has not failed for want of
courage. The public, as well as the fair, are won by a show of
gallantry. Mr. Irving has shrunk from no opinion, however
paradoxical. He has scrupled to avow no sentiment, however
obnoxious. He has revived exploded prejudices, he has scouted
prevailing fashions.  He has opposed the spirit of the age, and
not consulted the \emph{esprit de corps}. He has brought back the
doctrines of Calvinism in all their inveteracy, and relaxed the
inveteracy of his northern accents. He has turned religion and the
Caledonian Chapel topsy-turvy. He has held a play-book in one
hand, and a Bible in the other, and quoted Shakspeare and
Melancthon in the same breath. The tree of the knowledge of good
and evil is no longer, with his grafting, a dry withered stump; it
shoots its branches to the skies, and hangs out its blossoms to
the gale\pcdash{1}
\begin{quote}
  ``Miraturque novos fructus, et non sua poma.''
\end{quote}
He has taken the thorns and briars of scholastic divinity, and
garlanded them with the flowers of modern literature. He has done
all this, relying on the strength of a remarkably fine person and
manner, and through that he has succeeded\textemdash otherwise he
would have perished miserably.

Dr. Chalmers is not by any means so good a looking man, nor so
accomplished a speaker as Mr. Irving; yet he at one time almost
equalled his oratorical celebrity, and certainly paved the way for
him. He has therefore more merit than his admired pupil, as he has
done as much with fewer means. He has more scope of intellect and
more intensity of purpose. Both his matter and his manner, setting
aside his face and figure, are more impressive. Take the volume of
``Sermons on Astronomy,'' by Dr. Chalmers, and the ``Four Orations
for the Oracles of God'' which Mr. Irving lately published, and we
apprehend there can be no comparison as to their success. The
first ran like wild-fire through the country, were the darlings of
watering-places, were laid in the windows of inns,\footnote{We remember finding the volume in the orchard at
Burford-bridge near Boxhill, and passing a whole and very
delightful morning in reading it, without quitting the shade of an
apple-tree.  We have not been able to pay Mr. Irving's back the
same compliment of reading it at a sitting.} and were to
be met with in all places of public resort; while the ``Orations''
get on but slowly, on Milton's stilts, and are pompously announced
as in a Third Edition. We believe the fairest and fondest of his
admirers would rather see and hear Mr. Irving than read him. The
reason is, that the groundwork of his compositions is trashy and
hackneyed, though set off by extravagant metaphors and an affected
phraseology; that without the turn of his head and wave of his
hand, his periods have nothing in them; and that he himself is the
only \emph{idea} with which he has yet enriched the public mind! He
must play off his person, as Orator Henley used to dazzle his
hearers with his diamond-ring. The small frontispiece prefixed to
the ``Orations'' does not serve to convey an adequate idea of the
magnitude of the man, nor of the ease and freedom of his motions
in the pulpit. How different is Dr.  Chalmers! He is like ``a
monkey-preacher'' to the other. He cannot boast of personal
appearance to set him off. But then he is like the very genius or
demon of theological controversy personified. He has neither airs
nor graces at command; he thinks nothing of himself; he has
nothing theatrical about him (which cannot be said of his
successor and rival); but you see a man in mortal throes and agony
with doubts and difficulties, seizing stubborn knotty points with
his teeth, tearing them with his hands, and straining his eyeballs
till they almost start out of their sockets, in pursuit of a train
of visionary reasoning, like a Highland-seer with his second
sight. The description of Balfour of Burley in his cave, with his
Bible in one hand and his sword in the other, contending with the
imaginary enemy of mankind, gasping for breath, and with the cold
moisture running down his face, gives a lively idea of
Dr. Chalmers's prophetic fury in the pulpit. If we could have
looked in to have seen Burley hard-beset ``by the coinage of his
heat-oppressed brain,'' who would have asked whether he was a
handsome man or not? It would be enough to see a man haunted by a
spirit, under the strong and entire dominion of a wilful
hallucination. So the integrity and vehemence of Dr. Chalmers's
manner, the determined way in which he gives himself up to his
subject, or lays about him and buffets sceptics and gainsayers,
arrests attention in spite of every other circumstance, and fixes
it on that, and that alone, which excites such interest and such
eagerness in his own breast! Besides, he is a logician, has a
theory in support of whatever he chooses to advance, and weaves
the tissue of his sophistry so close and intricate, that it is
difficult not to be entangled in it, or to escape from
it. ``There's magic in the web.'' Whatever appeals to the pride of
the human understanding, has a subtle charm in it. The mind is
naturally pugnacious, cannot refuse a challenge of strength or
skill, sturdily enters the lists and resolves to conquer, or to
yield itself vanquished in the forms. This is the chief hold
Dr. Chalmers had upon his hearers, and upon the readers of his
``Astronomical Discourses.'' No one was satisfied with his
arguments, no one could answer them, but every one wanted to try
what he could make of them, as we try to find out a riddle. ``By
his so potent art,'' the art of laying down problematical premises,
and drawing from them still more doubtful, but not impossible,
conclusions, ``he could bedim the noonday sun, betwixt the green
sea and the azure vault set roaring war,'' and almost compel the
stars in their courses to testify to his opinions. The mode in
which he undertook to make the circuit of the universe, and demand
categorical information ``now of the planetary and now of the
fixed,'' might put one in mind of Hecate's mode of ascending in a
machine from the stage, ``midst troops of spirits,'' in which you
now admire the skill of the artist, and next tremble for the fate
of the performer, fearing that the audacity of the attempt will
turn his head or break his neck. The style of these ``Discourses''
also, though not elegant or poetical, was, like the subject,
intricate and endless. It was that of a man pushing his way
through a labyrinth of difficulties, and determined not to
flinch. The impression on the reader was proportionate; for,
whatever were the merits of the style or matter, both were new and
striking; and the train of thought that was unfolded at such
length and with such strenuousness, was bold, well-sustained, and
consistent with itself.

Mr. Irving wants the continuity of thought and manner which
distinguishes his rival\textemdash and shines by patches and in
bursts. He does not warm or acquire increasing force or rapidity
with his progress. He is never hurried away by a deep or lofty
enthusiasm, nor touches the highest point of genius or fanaticism,
but ``in the very storm and whirlwind of his passion, he acquires
and begets a temperance that may give it smoothness.'' He has the
self-possession and masterly execution of an experienced player or
fencer, and does not seem to express his natural convictions, or
to be engaged in a mortal struggle. This greater ease and
indifference is the result of vast superiority of personal
appearance, which ``to be admired needs but to be seen,'' and does
not require the possessor to work himself up into a passion, or to
use any violent contortions to gain attention or to keep it. These
two celebrated preachers are in almost all respects an antithesis
to each other. If Mr. Irving is an example of what can be done by
the help of external advantages, Dr. Chalmers is a proof of what
can be done without them. The one is most indebted to his mind,
the other to his body. If Mr. Irving inclines one to suspect
fashionable or popular religion of a little \emph{anthropomorphitism},
Dr. Chalmers effectually redeems it from that scandal.


