\chapter[The Late Mr. Horne Tooke]{the late mr. horne tooke}

Mr. Horne Tooke was one of those who may be considered as
connecting links between a former period and the existing
generation. His education and accomplishments, nay, his political
opinions, were of the last age; his mind, and the tone of his
feelings were \emph{modern}. There was a hard, dry materialism in the
very texture of his understanding, varnished over by the external
refinements of the old school. Mr. Tooke had great scope of
attainment, and great versatility of pursuit; but the same
shrewdness, quickness, cool self-possession, the same
\emph{literalness} of perception, and absence of passion and
enthusiasm, characterised nearly all he did, said, or wrote. He
was without a rival (almost) in private conversation, an expert
public speaker, a keen politician, a first-rate grammarian, and
the finest gentleman (to say the least) of his own party. He had
no imagination (or he would not have scorned it!)\textemdash no delicacy of
taste, no rooted prejudices or strong attachments: his intellect
was like a bow of polished steel, from which he shot sharp-pointed
poisoned arrows at his friends in private, at his enemies in
public. His mind (so to speak) had no \emph{religion} in it, and very
little even of the moral qualities of genius; but he was a man of
the world, a scholar bred, and a most acute and powerful
logician. He was also a wit, and a formidable one: yet it may be
questioned whether his wit was any thing more than an excess of
his logical faculty: it did not consist in the play of fancy, but
in close and cutting combinations of the understanding. ``The law
is open to every one: \emph{so},'' said Mr. Tooke, ``\emph{is the London
Tavern}!'' It is the previous deduction formed in the mind, and the
splenetic contempt felt for a practical sophism, that \emph{beats about
the bush for}, and at last finds the apt illustration; not the
casual, glancing coincidence of two objects, that points out an
absurdity to the understanding. So, on another occasion, when Sir
Allan Gardiner (who was a candidate for Westminster) had objected
to Mr. Fox, that ``he was always against the minister, \emph{whether
right or wrong},'' and Mr. Fox, in his reply, had overlooked this
slip of the tongue, Mr. Tooke immediately seized on it, and said,
``he thought it at least an equal objection to Sir Allan, that he
was always \emph{with} the minister, whether right or wrong.'' This
retort had all the effect, and produced the same surprise as the
most brilliant display of wit or fancy: yet it was only the
detecting a flaw in an argument, like a flaw in an indictment, by
a kind of legal pertinacity, or rather by a rigid and constant
habit of attending to the exact import of every word and clause in
a sentence.  Mr. Tooke had the mind of a lawyer; but it was
applied to a vast variety of topics and general trains of
speculation.

Mr. Horne Tooke was in private company, and among his friends, the
finished gentleman of the last age. His manners were as
fascinating as his conversation was spirited and delightful. He
put one in mind of the burden of the song of ``\emph{The King's Old
Courtier, and an Old Courtier of the King's}.'' He was, however, of
the opposite party. It was curious to hear our modern sciolist
advancing opinions of the most radical kind without any mixture of
radical heat or violence, in a tone of fashionable \emph{nonchalance},
with elegance of gesture and attitude, and with the most perfect
good-humour. In the spirit of opposition, or in the pride of
logical superiority, he too often shocked the prejudices or
wounded the self-love of those about him, while he himself
displayed the same unmoved indifference or equanimity. He said the
most provoking things with a laughing gaiety, and a polite
attention, that there was no withstanding. He threw others off
their guard by thwarting their favourite theories, and then
availed himself of the temperance of his own pulse to chafe them
into madness. He had not one particle of deference for the opinion
of others, nor of sympathy with their feelings; nor had he any
obstinate convictions of his own to defend\pcdash{1}
\begin{quote}
  ``Lord of himself, uncumbered with a \emph{creed}!''
\end{quote}
He took up any topic by chance, and played with it at will, like a
juggler with his cups and balls. He generally ranged himself on
the losing side; and had rather an ill-natured delight in
contradiction, and in perplexing the understandings of others,
without leaving them any clue to guide them out of the labyrinth
into which he had led them.  He understood, in its perfection, the
great art of throwing the \emph{onus probandi} on his adversary; and so
could maintain almost any opinion, however absurd or fantastical,
with fearless impunity. I have heard a sensible and well-informed
man say, that he never was in company with Mr. Tooke without being
delighted and surprised, or without feeling the conversation of
every other person to be flat in the comparison; but that he did
not recollect having ever heard him make a remark that struck him
as a sound and true one, or that he himself appeared to think
so. He used to plague Fuseli by asking him after the origin of the
Teutonic dialects, and Dr. Parr, by wishing to know the meaning of
the common copulative, \emph{Is}. Once at G\pcdash{2}'s, he defended Pitt
from a charge of verbiage, and endeavoured to prove him superior
to Fox. Some one imitated Pitt's manner, to show that it was
monotonous, and he imitated him also, to show that it was not. He
maintained (what would he not maintain?) that young Betty's acting
was finer than John Kemble's, and recited a passage from Douglas
in the manner of each, to justify the preference he gave to the
former. The mentioning this will please the living; it cannot hurt
the dead. He argued on the same occasion and in the same breath,
that Addison's style was without modulation, and that it was
physically impossible for any one to write well, who was
habitually silent in company. He sat like a king at his own table,
and gave law to his guests\textemdash and to the world! No man knew better
how to manage his immediate circle, to foil or bring them out. A
professed orator, beginning to address some observations to
Mr. Tooke with a voluminous apology for his youth and
inexperience, he said, ``Speak up, young man!''\textemdash and by taking him
at his word, cut short the flower of orations. Porson was the only
person of whom he stood in some degree of awe, on account of his
prodigious memory and knowledge of his favourite subject,
Languages. Sheridan, it has been remarked, said more good things,
but had not an equal flow of pleasantry. As an instance of
Mr. Horne Tooke's extreme coolness and command of nerve, it has
been mentioned that once at a public dinner when he had got on the
table to return thanks for his health being drank with a glass of
wine in his hand, and when there was a great clamour and
opposition for some time, after it had subsided, he pointed to the
glass to shew that it was still full. Mr. Holcroft (the author of
the \emph{Road to Ruin}) was one of the most violent and fiery-spirited
of all that motley crew of persons, who attended the Sunday
meetings at Wimbledon. One day he was so enraged by some paradox
or raillery of his host, that he indignantly rose from his chair,
and said, ``Mr. Tooke, you are a scoundrel!'' His opponent without
manifesting the least emotion, replied, ``Mr. Holcroft, when is it
that I am to dine with you? shall it be next Thursday?''\textemdash ``If you
please, Mr.  Tooke!'' answered the angry philosopher, and sat down
again.\textemdash It was delightful to see him sometimes turn from these
waspish or ludicrous altercations with over-weening antagonists to
some old friend and veteran politician seated at his elbow; to
hear him recal the time of Wilkes and Liberty, the conversation
mellowing like the wine with the smack of age; assenting to all
the old man said, bringing out his pleasant \emph{traits}, and
pampering him into childish self-importance, and sending him away
thirty years younger than he came!

As a public or at least as a parliamentary speaker, Mr. Tooke did
not answer the expectations that had been conceived of him, or
probably that he had conceived of himself. It is natural for men
who have felt a superiority over all those whom they happen to
have encountered, to fancy that this superiority will continue,
and that it will extend from individuals to public bodies. There
is no rule in the case; or rather, the probability lies the
contrary way. That which constitutes the excellence of
conversation is of little use in addressing large assemblies of
people; while other qualities are required that are hardly to be
looked for in one and the same capacity. The way to move great
masses of men is to shew that you yourself are moved. In a private
circle, a ready repartee, a shrewd cross-question, ridicule and
banter, a caustic remark or an amusing anecdote, whatever sets off
the individual to advantage, or gratifies the curiosity or piques
the self-love of the hearers, keeps attention alive, and secures
the triumph of the speaker\textemdash it is a personal contest, and depends
on personal and momentary advantages. But in appealing to the
public, no one triumphs but in the triumph of some public cause,
or by shewing a sympathy with the general and predominant feelings
of mankind. In a private room, a satirist, a sophist may provoke
admiration by expressing his contempt for each of his adversaries
in turn, and by setting their opinion at defiance\textemdash but when men
are congregated together on a great public question and for a
weighty object, they must be treated with more respect; they are
touched with what affects themselves or the general weal, not with
what flatters the vanity of the speaker; they must be moved
altogether, if they are moved at all; they are impressed with
gratitude for a luminous exposition of their claims or for zeal in
their cause; and the lightning of generous indignation at bad men
and bad measures is followed by thunders of applause\textemdash even in the
House of Commons. But a man may sneer and cavil and puzzle and
fly-blow every question that comes before him\textemdash be despised and
feared by others, and admired by no one but himself. He who thinks
first of himself, either in the world or in a popular assembly,
will be sure to turn attention away from his claims, instead of
fixing it there. He must make common cause with his hearers. To
lead, he must follow the general bias. Mr. Tooke did not therefore
succeed as a speaker in parliament. He stood aloof, he played
antics, he exhibited his peculiar talent\textemdash while he was on his
legs, the question before the House stood still; the only point at
issue respected Mr. Tooke himself, his personal address and
adroitness of intellect.

Were there to be no more places and pensions, because Mr. Tooke's
style was terse and epigrammatic? Were the Opposition benches to
be inflamed to an unusual pitch of ``sacred vehemence,'' because he
gave them plainly to understand there was not a pin to choose
between Ministers and Opposition? Would the House let him remain
among them, because, if they turned him out on account of his
\emph{black coat}, Lord Camelford had threatened to send his \emph{black
servant} in his place? This was a good joke, but not a practical
one. Would he gain the affections of the people out of doors, by
scouting the question of reform? Would the King ever relish the
old associate of Wilkes? What interest, then, what party did he
represent? He represented nobody but himself. He was an example of
an ingenious man, a clever talker, but he was out of his place in
the House of Commons; where people did not come (as in his own
house) to admire or break a lance with him, but to get through the
business of the day, and so adjourn! He wanted effect and
\emph{momentum}. Each of his sentences told very well in itself, but
they did not all together make a speech. He left off where he
began. His eloquence was a succession of drops, not a stream. His
arguments, though subtle and new, did not affect the main body of
the question. The coldness and pettiness of his manner did not
warm the hearts or expand the understandings of his
hearers. Instead of encouraging, he checked the ardour of his
friends; and teazed, instead of overpowering his antagonists. The
only palpable hit he ever made, while he remained there, was the
comparing his own situation in being rejected by the House, on
account of the supposed purity of his clerical character, to the
story of the girl at the Magdalen, who was told ``she must turn out
and qualify.''\footnote{``They receive him like a virgin at the
Magdalen\textemdash \emph{Go thou and do likewise}.''\textemdash \textsc{Junius}.} This met with laughter and loud applause. It was
a \emph{home} thrust, and the House (to do them justice) are obliged to
any one who, by a smart blow, relieves them of the load of grave
responsibility, which sits heavy on their shoulders.\textemdash At the
hustings, or as an election-candidate, Mr. Tooke did better. There
was no great question to move or carry\textemdash it was an affair of
political \emph{sparring} between himself and the other candidates. He
took it in a very cool and leisurely manner\textemdash watched his
competitors with a wary, sarcastic eye; picked up the mistakes or
absurdities that fell from them, and retorted them on their heads;
told a story to the mob; and smiled and took snuff with a
gentlemanly and becoming air, as if he was already seated in the
House. But a Court of Law was the place where Mr. Tooke made the
best figure in public. He might assuredly be said to be ``native
and endued unto that element.'' He had here to stand merely on the
defensive\textemdash not to advance himself, but to block up the way\textemdash not to
impress others, but to be himself impenetrable. All he wanted was
\emph{negative success}; and to this no one was better qualified to
aspire. Cross purposes, \emph{moot-points}, pleas, demurrers, flaws in
the indictment, double meanings, cases, inconsequentialities,
these were the play-things, the darlings of Mr. Tooke's mind; and
with these he baffled the Judge, dumb-founded the Counsel, and
outwitted the Jury. The report of his trial before Lord Kenyon is
a master-piece of acuteness, dexterity, modest assurance, and
legal effect. It is much like his examination before the
Commissioners of the Income-Tax\textemdash nothing could be got out of him
in either case! Mr. Tooke, as a political leader, belonged to the
class of \emph{trimmers}; or at most, it was his delight to make
mischief and spoil sport. He would rather be \emph{against} himself
than \emph{for} any body else. He was neither a bold nor a safe
leader. He enticed others into scrapes, and kept out of them
himself. Provided he could say a clever or a spiteful thing, he
did not care whether it served or injured the cause. Spleen or the
exercise of intellectual power was the motive of his patriotism,
rather than principle. He would talk treason with a saving clause;
and instil sedition into the public mind, through the medium of a
third (who was to be the responsible) party. He made Sir Francis
Burdett his spokesman in the House and to the country, often
venting his chagrin or singularity of sentiment at the expense of
his friend; but what in the first was trick or reckless vanity,
was in the last plain downright English honesty and singleness of
heart. In the case of the State Trials, in 1794, Mr. Tooke rather
compromised his friends to screen himself. He kept repeating that
``others might have gone on to Windsor, but he had stopped at
Hounslow,'' as if to go farther might have been dangerous and
unwarrantable. It was not the question how far he or others had
actually gone, but how far they had a right to go, according to
the law. His conduct was not the limit of the law, nor did
treasonable excess begin where prudence or principle taught him to
stop short, though this was the oblique inference liable to be
drawn from his line of defence. Mr. Tooke was uneasy and
apprehensive for the issue of the Government-prosecution while in
confinement, and said, in speaking of it to a friend, with a
morbid feeling and an emphasis quite unusual with him\textemdash ``They want
our blood\textemdash blood\textemdash blood!'' It was somewhat ridiculous to implicate
Mr. Tooke in a charge of High Treason (and indeed the whole charge
was built on the mistaken purport of an intercepted letter
relating to an engagement for a private dinnerparty)\textemdash his politics
were not at all revolutionary. In this respect he was a mere
pettifogger, full of chicane, and captious objections, and
unmeaning discontent; but he had none of the grand whirling
movements of the French Revolution, nor of the tumultuous glow of
rebellion in his head or in his heart. His politics were cast in a
different mould, or confined to the party distinctions and court-
intrigues and pittances of popular right, that made a noise in the
time of Junius and Wilkes\textemdash and even if his understanding had gone
along with more modern and unqualified principles, his cautious
temper would have prevented his risking them in practice. Horne
Tooke (though not of the same side in politics) had much of the
tone of mind and more of the spirit of moral feeling of the
celebrated philosopher of Malmesbury. The narrow scale and
fine-drawn distinctions of his political creed made his
conversation on such subjects infinitely amusing, particularly
when contrasted with that of persons who dealt in the sounding
\emph{common-places} and sweeping clauses of abstract politics. He knew
all the cabals and jealousies and heart-burnings in the beginning
of the late reign, the changes of administration and the springs
of secret influence, the characters of the leading men, Wilkes,
Barrè, Dunning, Chatham, Burke, the Marquis of Rockingham, North,
Shelburne, Fox, Pitt, and all the vacillating events of the
American war:\textemdash these formed a curious back-ground to the more
prominent figures that occupied the present time, and Mr. Tooke
worked out the minute details and touched in the evanescent
\emph{traits} with the pencil of a master. His conversation resembled a
political \emph{camera obscura}\textemdash as quaint as it was magical. To some
pompous pretenders he might seem to narrate \emph{fabellas aniles} (old
wives' fables)\textemdash but not to those who study human nature, and wish
to know the materials of which it is composed. Mr. Tooke's
faculties might appear to have ripened and acquired a finer
flavour with age. In a former period of his life he was hardly the
man he was latterly; or else he had greater abilities to contend
against. He no where makes so poor a figure as in his controversy
with Junius. He has evidently the best of the argument, yet he
makes nothing out of it. He tells a long story about himself,
without wit or point in it; and whines and whimpers like a
school-boy under the rod of his master. Junius, after bringing a
hasty charge against him, has not a single fact to adduce in
support of it; but keeps his ground and fairly beats his adversary
out of the field by the mere force of style. One would think that
``Parson Horne'' knew who Junius was, and was afraid of him. ``Under
him his genius is'' quite ``rebuked.'' With the best cause to defend,
he comes off more shabbily from the contest than any other person
in the \textsc{Letters}, except Sir William Draper, who is the very hero of
defeat.

The great thing which Mr. Horne Tooke has done, and which he has
left behind him to posterity, is his work on Grammar, oddly enough
entitled \textsc{The Diversions Of Purley}. Many people have taken it up as
a description of a game\textemdash others supposing it to be a novel. It is,
in truth, one of the few philosophical works on Grammar that were
ever written. The essence of it (and, indeed, almost all that is
really valuable in it) is contained in his \emph{Letter to Dunning},
published about the year 1775.  Mr. Tooke's work is truly
elementary. Dr. Lowth described Mr. Harris's \emph{Hermes} as ``the
finest specimen of analysis since the days of Aristotle''\textemdash a work
in which there is no analysis at all, for analysis consists in
reducing things to their principles, and not in endless details
and subdivisions. Mr. Harris multiplies distinctions, and
confounds his readers. Mr. Tooke clears away the rubbish of
school-boy technicalities, and strikes at the root of his
subject. In accomplishing his arduous task, he was, perhaps, aided
not more by the strength and resources of his mind than by its
limits and defects. There is a web of old associations wound round
language, that is a kind of veil over its natural features; and
custom puts on the mask of ignorance. But this veil, this mask the
author of \emph{The Diversions of Purley} threw aside and penetrated to
the naked truth of things, by the literal, matter-of-fact,
unimaginative nature of his understanding, and because he was not
subject to prejudices or illusions of any kind. Words may be said
to ``bear a charmed life, that must not yield to one of woman
born''\textemdash with womanish weaknesses and confused apprehensions. But
this charm was broken in the case of Mr. Tooke, whose mind was the
reverse of effeminate\textemdash hard, unbending, concrete, physical,
half-savage\textemdash and who saw language stripped of the clothing of
habit or sentiment, or the disguises of doting pedantry, naked in
its cradle, and in its primitive state. Our author tells us that
he found his discovery on Grammar among a number of papers on
other subjects, which he had thrown aside and forgotten. Is this
an idle boast? Or had he made other discoveries of equal
importance, which he did not think it worth his while to
communicate to the world, but chose to die the churl of knowledge?
The whole of his reasoning turns upon shewing that the Conjunction
\emph{That} is the pronoun \emph{That}, which is itself the participle of a
verb, and in like manner that all the other mystical and hitherto
unintelligible parts of speech are derived from the only two
intelligible ones, the Verb and Noun. ``I affirm \emph{that} gold is
yellow,'' that is, ``I affirm \emph{that} fact, or that proposition,
viz. gold is yellow.'' The secret of the Conjunction on which so
many fine heads had split, on which so many learned definitions
were thrown away, as if it was its peculiar province and inborn
virtue to announce oracles and formal propositions, and nothing
else, like a Doctor of Laws, is here at once accounted for,
inasmuch as it is clearly nothing but another part of speech, the
pronoun, \emph{that}, with a third part of speech, the noun, \emph{thing},
understood. This is getting at a solution of words into their
component parts, not glossing over one difficulty by bringing
another to parallel it, nor like saying with Mr. Harris, when it
is asked, ``what a Conjunction is?'' that there are conjunctions
copulative, conjunctions disjunctive, and as many other frivolous
varieties of the species as any one chooses to hunt out ``with
laborious foolery.'' Our author hit upon his parent-discovery in
the course of a law-suit, while he was examining, with jealous
watchfulness, the meaning of words to prevent being entrapped by
them; or rather, this circumstance might itself be traced to the
habit of satisfying his own mind as to the precise sense in which
he himself made use of words. Mr. Tooke, though he had no
objection to puzzle others, was mightily averse to being puzzled
or \emph{mystified} himself. All was, to his determined mind, either
complete light or complete darkness. There was no hazy, doubtful
\emph{chiaro-scuro} in his understanding. He wanted something ``palpable
to feeling as to sight.'' ``What,'' he would say to himself, ``do I
mean when I use the conjunction \emph{that?} Is it an anomaly, a class
by itself, a word sealed against all inquisitive attempts? Is it
enough to call it a \emph{copula}, a bridge, a link, a word connecting
sentences? That is undoubtedly its use, but what is its origin?''
Mr. Tooke thought he had answered this question satisfactorily,
and loosened the Gordian knot of grammarians, ``familiar as his
garter,'' when he said, ``It is the common pronoun, adjective, or
participle, \emph{that}, with the noun, \emph{thing or proposition},
implied, and the particular example following it.'' So he thought,
and so every reader has thought since, with the exception of
teachers and writers upon grammar. Mr. Windham, indeed, who was a
sophist, but not a logician, charged him with having found ``a
mare's-nest;'' but it is not to be doubted that Mr. Tooke's
etymologies will stand the test, and last longer than
Mr. Windham's ingenious derivation of the practice of bull-baiting
from the principles of humanity!

{\addfontfeature{LetterSpace=2.0}Having thus laid the cornerstone,
  he proceeded to apply the same method of reasoning to other
  undecyphered and impracticable terms.} Thus the word,
\emph{And}, he explained clearly enough to be the verb \emph{add},
or a corruption of the old Saxon, \emph{anandad}. ``Two \emph{and}
two make four,'' that is, ``two \emph{add} two make four.''
Mr. Tooke, in fact, treated words as the chemists do substances;
he separated those which are compounded of others from those which
are not decompoundable. He did not explain the obscure by the more
obscure, but the difficult by the plain, the complex by the
simple. This alone is proceeding upon the true principles of
science: the rest is pedantry and \emph{petit-maitreship.} Our
philosophical writer distinguished all words into \emph{names of
  things}, and directions added for joining them together, or
originally into \emph{nouns} and \emph{verbs}.  It is a pity that
he has left this matter short, by omitting to define the
Verb. After enumerating sixteen different definitions (all of
which he dismisses with scorn and contumely) at the end of two
quarto volumes, he refers the reader for the true solution to a
third volume, which he did not live to finish. This extraordinary
man was in the habit of tantalizing his guests on a Sunday
afternoon with sundry abstruse speculations, and putting them off
to the following week for a satisfaction of their doubts; but why
should he treat posterity in the same scurvy manner, or leave the
world without quitting scores with it?  I question whether
Mr. Tooke was himself in possession of his pretended
\emph{nostrum}, and whether, after trying hard at a definition of
the verb as a distinct part of speech, as a terrier-dog mumbles a
hedge-hog, he did not find it too much for him, and leave it to
its fate. It is also a pity that Mr. Tooke spun out his great work
with prolix and dogmatical dissertations on irrelevant matters;
and after denying the old metaphysical theories of language,
should attempt to found a metaphysical theory of his own on the
nature and mechanism of language.  The nature of words, he
contended (it was the basis of his whole system) had no connection
with the nature of things or the objects of thought; yet he
afterwards strove to limit the nature of things and of the human
mind by the technical structure of language. Thus he endeavours to
shew that there are no abstract ideas, by enumerating two thousand
instances of words, expressing abstract ideas, that are the past
participles of certain verbs. It is difficult to know what he
means by this. On the other hand, he maintains that ``a complex
idea is as great an absurdity as a complex star,'' and that words
only are complex. He also makes out a triumphant list of
metaphysical and moral non-entities, proved to be so on the pure
principle that the names of these non-entities are participles,
not nouns, or names of things. That is strange in so close a
reasoner and in one who maintained that all language was a
masquerade of words, and that the class to which they
grammatically belonged had nothing to do with the class of ideas
they represented.

It is now above twenty years since the two quarto volumes of the
\emph{Diversions of Purley} were published, and fifty since the same
theory was promulgated in the celebrated \emph{Letter to Dunning}. Yet
it is a curious example of the \emph{Spirit of the Age} that
Mr. Lindley Murray's Grammar (a work out of which Mr. C\pcdash{1}  helps
himself to English, and Mr.  M\pcdash{1}  to style\footnote{This work is not without merit in the details and
examples of English construction. But its fault even in that part
is that he confounds the genius of the English language, making it
periphrastic and literal, instead of elliptical and
idiomatic. According to Mr. Murray, hardly any of our best writers
ever wrote a word of English.}) has proceeded to
the thirtieth edition in complete defiance of all the facts and
arguments there laid down. He defines a noun to be the name of a
thing. Is quackery a thing, \emph{i.e.} a substance?  He defines a verb
to be a word signifying \emph{to be, to do, or to suffer}.  Are being,
action, suffering verbs? He defines an adjective to be the name of
a quality. Are not \emph{wooden, golden, substantial} adjectives? He
maintains that there are six cases in English nouns\footnote{At least, with only one change in the genitive case.}, that is,
six various terminations without any change of termination at all,
and that English verbs have all the moods, tenses, and persons
that the Latin ones have. This is an extraordinary stretch of
blindness and obstinacy.  He very formally translates the Latin
Grammar into English (as so many had done before him) and fancies
he has written an English Grammar; and divines applaud, and
schoolmasters usher him into the polite world, and English
scholars carry on the jest, while Horne Tooke's genuine anatomy of
our native tongue is laid on the shelf. Can it be that our
politicians smell a rat in the Member for Old Sarum? That our
clergy do not relish Parson Horne? That the world at large are
alarmed at acuteness and originality greater than their own? What
has all this to do with the formation of the English language or
with the first conditions and necessary foundation of speech
itself? Is there nothing beyond the reach of prejudice and
party-spirit? It seems in this, as in so many other instances, as
if there was a patent for absurdity in the natural bias of the
human mind, and that folly should be \emph{stereotyped}!
