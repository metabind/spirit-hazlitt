\chapter[Mr. Southey]{mr. southey}

Mr. Southey, as we formerly remember to have seen him, had a
hectic flush upon his cheek, a roving fire in his eye, a falcon
glance, a look at once aspiring and dejected\textemdash it was the
look that had been impressed upon his face by the events that
marked the outset of his life, it was the dawn of Liberty that
still tinged his cheek, a smile betwixt hope and sadness that
still played upon his quivering lip. Mr. Southey's mind is
essentially sanguine, even to over-weeningness. It is prophetic of
good; it cordially embraces it; it casts a longing, lingering look
after it, even when it is gone for ever. He cannot bear to give up
the thought of happiness, his confidence in his fellow-man, when
all else despair.  It is the very element, ``where he must live or
have no life at all.''  While he supposed it possible that a
better form of society could be introduced than any that had
hitherto existed, while the light of the French Revolution beamed
into his soul (and long after, it was seen reflected on his brow,
like the light of setting suns on the peak of some high mountain,
or lonely range of clouds, floating in purer ether!)  while he had
this hope, this faith in man left, he cherished it with child-like
simplicity, he clung to it with the fondness of a lover, he was an
enthusiast, a fanatic, a leveller; he stuck at nothing that he
thought would banish all pain and misery from the world\textemdash
in his impatience of the smallest error or injustice, he would
have sacrificed himself and the existing generation (a holocaust)
to his devotion to the right cause. But when he once believed
after many staggering doubts and painful struggles, that this was
no longer possible, when his chimeras and golden dreams of human
perfectibility vanished from him, he turned suddenly round, and
maintained that ``whatever \emph{is}, is right.'' Mr.  Southey has
not fortitude of mind, has not patience to think that evil is
inseparable from the nature of things. His irritable sense rejects
the alternative altogether, as a weak stomach rejects the food
that is distasteful to it. He hopes on against hope, he believes
in all unbelief. He must either repose on actual or on imaginary
good. He missed his way in \emph{Utopia}, he has found it at Old
Sarum\textemdash
\begin{quote}
  ``His generous \emph{ardour} no cold medium knows:''
\end{quote}
his eagerness admits of no doubt or delay. He is ever in extremes,
and ever in the wrong!

The reason is, that not truth, but self-opinion is the ruling
principle of Mr. Southey's mind. The charm of novelty, the
applause of the multitude, the sanction of power, the
venerableness of antiquity, pique, resentment, the spirit of
contradiction have a good deal to do with his preferences. His
inquiries are partial and hasty: his conclusions raw and
unconcocted, and with a considerable infusion of whim and humour
and a monkish spleen. His opinions are like certain wines, warm
and generous when new; but they will not keep, and soon turn flat
or sour, for want of a stronger spirit of the understanding to
give a body to them. He wooed Liberty as a youthful lover, but it
was perhaps more as a mistress than a bride; and he has since
wedded with an elderly and not very reputable lady, called
Legitimacy. \emph{A wilful man}, according to the Scotch proverb,
\emph{must have his way}. If it were the cause to which he was
sincerely attached, he would adhere to it through good report and
evil report; but it is himself to whom he does homage, and would
have others do so; and he therefore changes sides, rather than
submit to apparent defeat or temporary mortification. Abstract
principle has no rule but the understood distinction between right
and wrong; the indulgence of vanity, of caprice, or prejudice is
regulated by the convenience or bias of the moment. The
temperament of our politician's mind is poetical, not
philosophical. He is more the creature of impulse, than he is of
reflection. He invents the unreal, he embellishes the false with
the glosses of fancy, but pays little attention to ``the words of
truth and soberness.'' His impressions are accidental, immediate,
personal, instead of being permanent and universal. Of all mortals
he is surely the most impatient of contradiction, even when he has
completely turned the tables on himself. Is not this very
inconsistency the reason?  Is he not tenacious of his opinions, in
proportion as they are brittle and hastily formed? Is he not
jealous of the grounds of his belief, because he fears they will
not bear inspection, or is conscious he has shifted them? Does he
not confine others to the strict line of orthodoxy, because he has
himself taken every liberty? Is he not afraid to look to the right
or the left, lest he should see the ghosts of his former
extravagances staring him in the face? Does he not refuse to
tolerate the smallest shade of difference in others, because he
feels that he wants the utmost latitude of construction for
differing so widely from himself? Is he not captious, dogmatical,
petulant in delivering his sentiments, according as he has been
inconsistent, rash, and fanciful in adopting them? He maintains
that there can be no possible ground for differing from him,
because he looks only at his own side of the question! He sets up
his own favourite notions as the standard of reason and honesty,
because he has changed from one extreme to another! He treats his
opponents with contempt, because he is himself afraid of meeting
with disrespect! He says that ``a Reformer is a worse character
than a house-breaker,'' in order to stifle the recollection that
he himself once was one!

We must say that ``we relish Mr. Southey more in the Reformer''
than in his lately acquired, but by no means natural or becoming
character of poet-laureat and courtier. He may rest assured that a
garland of wild flowers suits him better than the laureat-wreath:
that his pastoral odes and popular inscriptions were far more
adapted to his genius than his presentation-poems. He is nothing
akin to birth-day suits and drawing-room fopperies. ``He is
nothing, if not fantastical.'' In his figure, in his movements, in
his sentiments, he is sharp and angular, quaint and
eccentric. Mr. Southey is not of the court, courtly. Every thing
of him and about him is from the people. He is not classical, he
is not legitimate. He is not a man cast in the mould of other
men's opinions: he is not shaped on any model: he bows to no
authority: he yields only to his own wayward peculiarities. He is
wild, irregular, singular, extreme. He is no formalist, not he!
All is crude and chaotic, self-opinionated, vain. He wants
proportion, keeping, system, standard rules. He is not \emph{teres
  et rotundus}. Mr. Southey walks with his chin erect through the
streets of London, and with an umbrella sticking out under his
arm, in the finest weather. He has not sacrificed to the Graces,
nor studied decorum. With him every thing is projecting, starting
from its place, an episode, a digression, a poetic license. He
does not move in any given orbit, but like a falling star, shoots
from his sphere. He is pragmatical, restless, unfixed, full of
experiments, beginning every thing a-new, wiser than his betters,
judging for himself, dictating to others. He is decidedly
\emph{revolutionary}. He may have given up the reform of the
State: but depend upon it, he has some other \emph{hobby} of the
same kind. Does he not dedicate to his present Majesty that
extraordinary poem on the death of his father, called \emph{The
  Vision of Judgment}, as a specimen of what might be done in
English hexameters? In a court-poem all should be trite and on an
approved model. He might as well have presented himself at the
levee in a fancy or masquerade dress. Mr. Southey was not \emph{to
  try conclusions} with Majesty\textemdash still less on such an
occasion. The extreme freedoms with departed greatness, the
party-petulance carried to the Throne of Grace, the unchecked
indulgence of private humour, the assumption of infallibility and
even of the voice of Heaven in this poem, are pointed instances of
what we have said. They shew the singular state of over-excitement
of Mr. Southey's mind, and the force of old habits of independent
and unbridled thinking, which cannot be kept down even in
addressing his Sovereign! Look at Mr. Southey's larger poems, his
\emph{Kehama}, his \emph{Thalaba}, his \emph{Madoc}, his
\emph{Roderic}. Who will deny the spirit, the scope, the splendid
imagery, the hurried and startling interest that pervades them?
Who will say that they are not sustained on fictions wilder than
his own Glendoveer, that they are not the daring creations of a
mind curbed by no law, tamed by no fear, that they are not rather
like the trances than the waking dreams of genius, that they are
not the very paradoxes of poetry? All this is very well, very
intelligible, and very harmless, if we regard the rank
excrescences of Mr. Southey's poetry, like the red and blue
flowers in corn, as the unweeded growth of a luxuriant and
wandering fancy; or if we allow the yeasty workings of an ardent
spirit to ferment and boil over\textemdash the variety, the
boldness, the lively stimulus given to the mind may then atone for
the violation of rules and the offences to bed-rid authority; but
not if our poetic libertine sets up for a law-giver and judge, or
an apprehender of vagrants in the regions either of taste or
opinion. Our motley gentleman deserves the strait-waistcoat, if he
is for setting others in the stocks of servility, or condemning
them to the pillory for a new mode of rhyme or reason. Or if a
composer of sacred Dramas on classic models, or a translator of an
old Latin author (that will hardly bear translation) or a
vamper-up of vapid cantos and Odes set to music, were to turn
pander to prescription and palliater of every dull, incorrigible
abuse, it would not be much to be wondered at or even
regretted. But in Mr. Southey it was a lamentable falling-off. It
is indeed to be deplored, it is a stain on genius, a blow to
humanity, that the author of \emph{Joan of Arc}\textemdash that
work in which the love of Liberty is exhaled like the breath of
spring, mild, balmy, heaven-born, that is full of tears and
virgin-sighs, and yearnings of affection after truth and good,
gushing warm and crimsoned from the heart\textemdash should ever
after turn to folly, or become the advocate of a rotten
cause. After giving up his heart to that subject, he ought not
(whatever others might do) ever to have set his foot within the
threshold of a court. He might be sure that he would not gain
forgiveness or favour by it, nor obtain a single cordial smile
from greatness. All that Mr. Southey is or that he does best, is
independent, spontaneous, free as the vital air he
draws\textemdash when he affects the courtier or the sophist, he
is obliged to put a constraint upon himself, to hold in his
breath, he loses his genius, and offers a violence to his
nature. His characteristic faults are the excess of a lively,
unguarded temperament:\textemdash oh! let them not degenerate into
cold-blooded, heartless vices! If we speak or have ever spoken of
Mr. Southey with severity, it is with ``the malice of old
friends,'' for we count ourselves among his sincerest and
heartiest well-wishers. But while he himself is anomalous,
incalculable, eccentric, from youth to age (the \emph{Wat Tyler}
and the \emph{Vision of Judgment} are the Alpha and Omega of his
disjointed career) full of sallies of humour, of ebullitions of
spleen, making \emph{jets-d'eaux,} cascades, fountains, and
water-works of his idle opinions, he would shut up the wits of
others in leaden cisterns, to stagnate and corrupt, or bury them
under ground\textemdash
\begin{quote}
  ``Far from the sun and summer gale!''
\end{quote}
He would suppress the freedom of wit and humour, of which he has
set the example, and claim a privilege for playing antics. He
would introduce an uniformity of intellectual weights and
measures, of irregular metres and settled opinions, and enforce it
with a high hand. This has been judged hard by some, and has
brought down a severity of recrimination, perhaps disproportioned
to the injury done. ``Because he is virtuous,'' (it has been
asked,) ``are there to be no more cakes and ale?'' Because he is
loyal, are we to take all our notions from the \emph{Quarterly
  Review}?  Because he is orthodox, are we to do nothing but read
the \emph{Book of the Church}? We declare we think his former
poetical scepticism was not only more amiable, but had more of the
spirit of religion in it, implied a more heartfelt trust in nature
and providence than his present bigotry.  We are at the same time
free to declare that we think his articles in the \emph{Quarterly
  Review,} notwithstanding their virulence and the talent they
display, have a tendency to qualify its most pernicious effects.
They have redeeming traits in them. ``A little leaven leaveneth
the whole lump:'' and the spirit of humanity (thanks to
Mr. Southey) is not quite expelled from the \emph{Quarterly
  Review}. At the corner of his pen, ``there hangs a vapourous
drop profound'' of independence and liberality, which falls upon
its pages, and oozes out through the pores of the public
mind. There is a fortunate difference between writers whose hearts
are naturally callous to truth, and whose understandings are
hermetically sealed against all impressions but those of
self-interest, and a man like Mr. Southey. \emph{Once a
  philanthropist and always a philanthropist}.  No man can
entirely baulk his nature: it breaks out in spite of him.  In all
those questions, where the spirit of contradiction does not
interfere, on which he is not sore from old bruises, or sick from
the extravagance of youthful intoxication, as from a last night's
debauch, our ``laureate'' is still bold, free, candid, open to
conviction, a reformist without knowing it. He does not advocate
the slave-trade, he does not arm Mr. Malthus's revolting ratios
with his authority, he does not strain hard to deluge Ireland with
blood. On such points, where humanity has not become obnoxious,
where liberty has not passed into a by-word, Mr. Southey is still
liberal and humane. The elasticity of his spirit is unbroken: the
bow recoils to its old position. He still stands convicted of his
early passion for inquiry and improvement. He was not regularly
articled as a Government-tool!\textemdash Perhaps the most
pleasing and striking of all Mr. Southey's poems are not his
triumphant taunts hurled against oppression, are not his glowing
effusions to Liberty, but those in which, with a mild melancholy,
he seems conscious of his own infirmities of temper, and to feel a
wish to correct by thought and time the precocity and sharpness of
his disposition. May the quaint but affecting aspiration expressed
in one of these be fulfilled, that as he mellows into maturer age,
all such asperities may wear off, and he himself become
\begin{quote}
  ``Like the high leaves upon the holly-tree!''
\end{quote}
Mr. Southey's prose-style can scarcely be too much praised. It is
plain, clear, pointed, familiar, perfectly modern in its texture,
but with a grave and sparkling admixture of \emph{archaisms} in
its ornaments and occasional phraseology. He is the best and most
natural prose-writer of any poet of the day; we mean that he is
far better than Lord Byron, Mr. Wordsworth, or Mr. Coleridge, for
instance. The manner is perhaps superior to the matter, that is,
in his Essays and Reviews. There is rather a want of originality
and even of \emph{impetus}: but there is no want of playful or
biting satire, of ingenuity, of casuistry, of

learning and of information. He is ``full of wise saws and modern'' (as
well as ancient) ``instances.'' Mr. Southey may not always convince his
opponents; but he seldom fails to stagger, never to gall them. In a
word, we may describe his style by saying that it has not the body or
thickness of port wine, but is like clear sherry with kernels of
old authors thrown into it!\textemdash He also excels as an historian and
prose-translator. His histories abound in information, and exhibit
proofs of the most indefatigable patience and industry. By no uncommon
process of the mind, Mr. Southey seems willing to steady the extreme
levity of his opinions and feelings by an appeal to facts. His
translations of the Spanish and French romances are also executed \emph{con
amore}, and with the literal fidelity and care of a mere linguist. That
of the \emph{Cid}, in particular, is a masterpiece. Not a word could be
altered for the better, in the old scriptural style which it adopts in
conformity to the original. It is no less interesting in itself, or as a
record of high and chivalrous feelings and manners, than it is worthy of
perusal as a literary curiosity.

Mr. Southey's conversation has a little resemblance to a
common-place book; his habitual deportment to a piece of
clock-work. He is not remarkable either as a reasoner or an
observer: but he is quick, unaffected, replete with anecdote,
various and retentive in his reading, and exceedingly happy in his
play upon words, as most scholars are who give their minds this
sportive turn. We have chiefly seen Mr. Southey in company where
few people appear to advantage, we mean in that of Mr.
Coleridge. He has not certainly the same range of speculation, nor
the same flow of sounding words, but he makes up by the details of
knowledge, and by a scrupulous correctness of statement for what
he wants in originality of thought, or impetuous declamation. The
tones of Mr. Coleridge's voice are eloquence: those of Mr. Southey
are meagre, shrill, and dry. Mr. Coleridge's \emph{forte} is
conversation, and he is conscious of this: Mr. Southey evidently
considers writing as his strong-hold, and if gravelled in an
argument, or at a loss for an explanation, refers to something he
has written on the subject, or brings out his port-folio, doubled
down in dog-ears, in confirmation of some fact. He is scholastic
and professional in his ideas. He sets more value on what he
writes than on what he says: he is perhaps prouder of his library
than of his own productions\textemdash themselves a library! He is
more simple in his manners than his friend Mr. Coleridge; but at
the same time less cordial or conciliating. He is less vain, or
has less hope of pleasing, and therefore lays himself less out to
please. There is an air of condescension in his civility. With a
tall, loose figure, a peaked austerity of countenance, and no
inclination to \emph{embonpoint}, you would say he has something
puritanical, something ascetic in his appearance. He answers to
Mandeville's description of Addison, ``a parson in a tye-wig.'' He
is not a boon companion, nor does he indulge in the pleasures of
the table, nor in any other vice; nor are we aware that Mr.
Southey is chargeable with any human frailty but\textemdash
\emph{want of charity}!  Having fewer errors to plead guilty to,
he is less lenient to those of others. He was born an age too
late. Had he lived a century or two ago, he would have been a
happy as well as blameless character. But the distraction of the
time has unsettled him, and the multiplicity of his pretensions
have jostled with each other. No man in our day (at least no man
of genius) has led so uniformly and entirely the life of a scholar
from boyhood to the present hour, devoting himself to learning
with the enthusiasm of an early love, with the severity and
constancy of a religious vow\textemdash and well would it have
been for him if he had confined himself to this, and not
undertaken to pull down or to patch up the State! However
irregular in his opinions, Mr. Southey is constant, unremitting,
mechanical in his studies, and the performance of his
duties. There is nothing Pindaric or Shandean here. In all the
relations and charities of private life, he is correct, exemplary,
generous, just.  We never heard a single impropriety laid to his
charge; and if he has many enemies, few men can boast more
numerous or stauncher friends.\textemdash The variety and piquancy
of his writings form a striking contrast to the mode in which they
are produced. He rises early, and writes or reads till
breakfast-time. He writes or reads after breakfast till dinner,
after dinner till tea, and from tea till bed-time\textemdash
\begin{verse}
  \vleftofline{``}And follows so the ever-running year\\
  With profitable labour to his grave\textemdash ''
\end{verse}
on Derwent's banks, beneath the foot of Skiddaw. Study serves him
for business, exercise, recreation. He passes from verse to prose,
from history to poetry, from reading to writing, by a
stop-watch. He writes a fair hand, without blots, sitting upright
in his chair, leaves off when he comes to the bottom of the page,
and changes the subject for another, as opposite as the
Antipodes. His mind is after all rather the recipient and
transmitter of knowledge, than the originator of it. He has hardly
grasp of thought enough to arrive at any great leading truth. His
passions do not amount to more than irritability. With some gall
in his pen, and coldness in his manner, he has a great deal of
kindness in his heart. Rash in his opinions, he is steady in his
attachments\textemdash and is a man, in many particulars
admirable, in all respectable\textemdash his political
inconsistency alone excepted!

