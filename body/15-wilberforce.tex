\chapter[Lord Eldon and Mr. Wilberforce]
{lord eldon {\normalsize and} mr. wilberforce}

Lord Eldon is an exceedingly good-natured man; but this does not
prevent him, like other good-natured people, from consulting his
own ease or interest. The character of \emph{good-nature}, as it
is called, has been a good deal mistaken; and the present
Chancellor is not a bad illustration of the grounds of the
prevailing error. When we happen to see an individual whose
countenance is ``all tranquillity and smiles;'' who is full of
good-humour and pleasantry; whose manners are gentle and
conciliating; who is uniformly temperate in his expressions, and
punctual and just in his every-day dealings; we are apt to
conclude from so fair an outside, that
\begin{quote}
  ``All is conscience and tender heart''

\end{quote}
within also, and that such a one would not hurt a fly. And neither
would he without a motive. But mere good-nature (or what passes in
the world for such) is often no better than indolent
selfishness. A person distinguished and praised for this quality
will not needlessly offend others, because they may retaliate; and
besides, it ruffles his own temper. He likes to enjoy a perfect
calm, and to live in an interchange of kind offices. He suffers
few things to irritate or annoy him. He has a fine oiliness in his
disposition, which smooths the waves of passion as they rise. He
does not enter into the quarrels or enmities of others; bears
their calamities with patience; he listens to the din and clang of
war, the earthquake and the hurricane of the political and moral
world with the temper and spirit of a philosopher; no act of
injustice puts him beside himself, the follies and absurdities of
mankind never give him a moment's uneasiness, he has none of the
ordinary causes of fretfulness or chagrin that torment others from
the undue interest they take in the conduct of their neighbours or
in the public good. None of these idle or frivolous sources of
discontent, that make such havoc with the peace of human life,
ever discompose his features or alter the serenity of his
pulse. If a nation is robbed of its rights,
\begin{quote}
  ``If wretches hang that Ministers may dine,''\textemdash

\end{quote}
the laughing jest still collects in his eye, the cordial squeeze
of the hand is still the same. But tread on the toe of one of
these amiable and imperturbable mortals, or let a lump of soot
fall down the chimney and spoil their dinners, and see how they
will bear it. All their patience is confined to the accidents that
befal others: all their good-humour is to be resolved into giving
themselves no concern about any thing but their own ease and
self-indulgence. Their charity begins and ends at home. Their
being free from the common infirmities of temper is owing to their
indifference to the common feelings of humanity; and if you touch
the sore place, they betray more resentment, and break out (like
spoiled children) into greater fractiousness than others, partly
from a greater degree of selfishness, and partly because they are
taken by surprise, and mad to think they have not guarded every
point against annoyance or attack, by a habit of callous
insensibility and pampered indolence.

An instance of what we mean occurred but the other day. An
allusion was made in the House of Commons to something in the
proceedings in the Court of Chancery, and the Lord Chancellor
comes to his place in the Court, with the statement in his hand,
fire in his eyes, and a direct charge of falsehood in his mouth,
without knowing any thing certain of the matter, without making
any inquiry into it, without using any precaution or putting the
least restraint upon himself, and all on no better authority than
a common newspaper report. The thing was (not that we are imputing
any strong blame in this case, we merely bring it as an
illustration) it touched himself, his office, the inviolability of
his jurisdiction, the unexceptionableness of his proceedings, and
the wet blanket of the Chancellor's temper instantly took fire
like tinder! All the fine balancing was at an end; all the doubts,
all the delicacy, all the candour real or affected, all the
chances that there might be a mistake in the report, all the
decencies to be observed towards a Member of the House, are
overlooked by the blindness of passion, and the wary Judge pounces
upon the paragraph without mercy, without a moment's delay, or the
smallest attention to forms! This was indeed serious business,
there was to be no trifling here; every instant was an age till
the Chancellor had discharged his sense of indignation on the head
of the indiscreet interloper on his authority. Had it been another
person's case, another person's dignity that had been compromised,
another person's conduct that had been called in question, who
doubts but that the matter might have stood over till the next
term, that the Noble Lord would have taken the Newspaper home in
his pocket, that he would have compared it carefully with other
newspapers, that he would have written in the most mild and
gentlemanly terms to the Honourable Member to inquire into the
truth of the statement, that he would have watched a convenient
opportunity good-humouredly to ask other Honourable Members what
all this was about, that the greatest caution and fairness would
have been observed, and that to this hour the lawyers' clerks and
the junior counsel would have been in the greatest admiration of
the Chancellor's nicety of discrimination, and the utter
inefficacy of the heats, importunities, haste, and passions of
others to influence his judgment? This would have been true; yet
his readiness to decide and to condemn where he himself is
concerned, shews that passion is not dead in him, nor subject to
the controul of reason; but that self-love is the main-spring that
moves it, though on all beyond that limit he looks with the most
perfect calmness and philosophic indifference.
\begin{verse}
  \vleftofline{``}Resistless passion sways us to the mood \\
  Of what it likes or loaths.''

\end{verse}
All people are passionate in what concerns themselves, or in what
they take an interest in. The range of this last is different in
different persons; but the want of passion is but another name for
the want of sympathy and imagination.

\fixspacing{2}{
The Lord Chancellor's impartiality and conscientious exactness is
proverbial}; and is, we believe, as inflexible as it is delicate in
all cases that occur in the stated routine of legal practice. The
impatience, the irritation, the hopes, the fears, the confident
tone of the applicants move him not a jot from his intended
course, he looks at their claims with the ``lack lustre eye'' of
prefessional indifference.  Power and influence apart, his next
strongest passion is to indulge in the exercise of professional
learning and skill, to amuse himself with the dry details and
intricate windings of the law of equity. He delights to balance a
straw, to see a feather turn the scale, or make it even again; and
divides and subdivides a scruple to the smallest fraction. He
unravels the web of argument and pieces it together again; folds
it up and lays it aside, that he may examine it more at his
leisure. He hugs indecision to his breast, and takes home a modest
doubt or a nice point to solace himself with it in protracted,
luxurious dalliance. Delay seems, in his mind, to be of the very
essence of justice. He no more hurries through a question than if
no one was waiting for the result, and he was merely a
\emph{dilettanti}, fanciful judge, who played at my Lord
Chancellor, and busied himself with quibbles and punctilios as an
idle hobby and harmless illusion. The phlegm of the Chancellor's
disposition gives one almost a surfeit of impartiality and
candour: we are sick of the eternal poise of childish
dilatoriness; and would wish law and justice to be decided at once
by a cast of the dice (as they were in Rabelais) rather than be
kept in frivolous and tormenting suspense. But there is a limit
even to this extreme refinement and scrupulousness of the
Chancellor. The understanding acts only in the absence of the
passions. At the approach of the loadstone, the needle trembles,
and points to it. The air of a political question has a wonderful
tendency to brace and quicken the learned Lord's faculties. The
breath of a court speedily oversets a thousand objections, and
scatters the cobwebs of his brain. The secret wish of power is a
thumping \emph{make-weight,} where all is so nicely-balanced
beforehand. In the case of a celebrated beauty and heiress, and
the brother of a Noble Lord, the Chancellor hesitated long, and
went through the forms, as usual: but who ever doubted, where all
this indecision would end? No man in his senses, for a single
instant!  We shall not press this point, which is rather a
ticklish one. Some persons thought that from entertaining a
fellow-feeling on the subject, the Chancellor would have been
ready to favour the Poet-Laureat's application to the Court of
Chancery for an injunction against Wat Tyler. His Lordship's
sentiments on such points are not so variable, he has too much at
stake. He recollected the year 1794, though Mr. Southey had
forgotten it!\textemdash

The personal always prevails over the intellectual, where the
latter is not backed by strong feeling and principle. Where remote
and speculative objects do not excite a predominant interest and
passion, gross and immediate ones are sure to carry the day, even
in ingenuous and well-disposed minds. The will yields necessarily
to some motive or other; and where the public good or distant
consequences excite no sympathy in the breast, either from
short-sightedness or an easiness of temperament that shrinks from
any violent effort or painful emotion, self-interest, indolence,
the opinion of others, a desire to please, the sense of personal
obligation, come in and fill up the void of public spirit,
patriotism, and humanity. The best men in the world in their own
natural dispositions or in private life (for this reason) often
become the most dangerous public characters, from their pliancy to
the unruly passions of others, and from their having no set-off in
strong moral \emph{stamina} to the temptations that are held out
to them, if, as is frequently the case, they are men of versatile
talent or patient industry.\textemdash Lord Eldon has one of the
best-natured faces in the world; it is pleasant to meet him in the
street, plodding along with an umbrella under his arm, without one
trace of pride, of spleen, or discontent in his whole demeanour,
void of offence, with almost rustic simplicity and honesty of
appearance\textemdash a man that makes friends at first sight, and
could hardly make enemies, if he would; and whose only fault is
that he cannot say \emph{Nay} to power, or subject himself to an
unkind word or look from a King or a Minister. He is a
thorough-bred Tory.  Others boggle or are at fault in their
career, or give back at a pinch, they split into different
factions, have various objects to distract them, their private
friendships or antipathies stand in their way; but he has never
flinched, never gone back, never missed his way, he is an
\emph{out-and-outer} in this respect, his allegiance has been
without flaw, like ``one entire and perfect chrysolite,'' his
implicit understanding is a kind of taffeta-lining to the Crown,
his servility has assumed an air of the most determined
independence, and he has
\begin{quote}
  ``Read his history in a Prince's eyes!''\textemdash

\end{quote}
There has been no stretch of power attempted in his time that he
has not seconded: no existing abuse, so odious or so absurd, that
he has not sanctioned it. He has gone the whole length of the most
unpopular designs of Ministers. When the heavy artillery of
interest, power, and prejudice is brought into the field, the
paper pellets of the brain go for nothing: his labyrinth of nice,
lady-like doubts explodes like a mine of gun-powder. The
Chancellor may weigh and palter\textemdash the courtier is
decided, the politician is firm, and rivetted to his place in the
Cabinet! On all the great questions that have divided party
opinion or agitated the public mind, the Chancellor has been found
uniformly and without a single exception on the side of
prerogative and power, and against every proposal for the
advancement of freedom. He was a strenuous supporter of the wars
and coalitions against the principles of liberty abroad; he has
been equally zealous in urging or defending every act and
infringement of the Constitution, for abridging it at home: he at
the same time opposes every amelioration of the penal laws, on the
alleged ground of his abhorrence of even the shadow of innovation:
he has studiously set his face against Catholic emancipation; he
laboured hard in his vocation to prevent the abolition of the
Slave Trade; he was Attorney General in the trials for High
Treason in 1794; and the other day in giving his opinion on the
Queen's Trial, shed tears and protested his innocence before God!
This was natural and to be expected; but on all occasions he is to
be found at his post, true to the call of prejudice, of power, to
the will of others and to his own interest.  In the whole of his
public career, and with all the goodness of his disposition, he
has not shewn ``so small a drop of pity as a wren's eye.''  He
seems to be on his guard against every thing liberal and humane as
his weak side. Others relax in their obsequiousness either from
satiety or disgust, or a hankering after popularity, or a wish to
be thought above narrow prejudices. The Chancellor alone is fixed
and immoveable.  Is it want of understanding or of principle?
No\textemdash it is want of imagination, a phlegmatic habit, an
excess of false complaisance and good-nature \ldots Common humanity
and justice are little better than vague terms to him: he acts
upon his immediate feelings and least irksome impulses. The King's
hand is velvet to the touch\textemdash the Woolsack is a seat of
honour and profit! That is all he knows about the matter. As to
abstract metaphysical calculations, the ox that stands staring at
the corner of the street troubles his head as much about them as
he does: yet this last is a very good sort of animal with no harm
or malice in him, unless he is goaded on to mischief, and then it
is necessary to keep out of his way, or warn others against him!

Mr. Wilberforce is a less perfect character in his way. He acts
from mixed motives. He would willingly serve two masters, God and
Mammon. He is a person of many excellent and admirable
qualifications, but he has made a mistake in wishing to reconcile
those that are incompatible.  He has a most winning eloquence,
specious, persuasive, familiar, silver-tongued, is amiable,
charitable, conscientious, pious, loyal, humane, tractable to
power, accessible to popularity, honouring the king, and no less
charmed with the homage of his fellow-citizens. ``What lacks he
then?'' Nothing but an economy of good parts. By aiming at too
much, he has spoiled all, and neutralised what might have been an
estimable character, distinguished by signal services to
mankind. A man must take his choice not only between virtue and
vice, but between different virtues. Otherwise, he will not gain
his own approbation, or secure the respect of others. The graces
and accomplishments of private life mar the man of business and
the statesman. There is a severity, a sternness, a self-denial,
and a painful sense of duty required in the one, which ill befits
the softness and sweetness which should characterise the
other. Loyalty, patriotism, friendship, humanity, are all virtues;
but may they not sometimes clash? By being unwilling to forego the
praise due to any, we may forfeit the reputation of all; and
instead of uniting the suffrages of the whole world in our favour,
we may end in becoming a sort of bye-word for affectation, cant,
hollow professions, trimming, fickleness, and effeminate
imbecility. It is best to choose and act up to some one leading
character, as it is best to have some settled profession or
regular pursuit in life.

We can readily believe that Mr. Wilberforce's first object and
principle of action is to do what he thinks right: his next (and
that we fear is of almost equal weight with the first) is to do
what will be thought so by other people. He is always at a game of
\emph{hawk and buzzard} between these two: his ``conscience will
not budge,'' unless the world goes with it. He does not seem
greatly to dread the denunciation in Scripture, but rather to
court it\textemdash ``Woe unto you, when all men shall speak well
of you!'' We suspect he is not quite easy in his mind, because
West-India planters and Guinea traders do not join in his
praise. His ears are not strongly enough tuned to drink in the
execrations of the spoiler and the oppressor as the sweetest
music. It is not enough that one half of the human species (the
images of God carved in ebony, as old Fuller calls them) shout his
name as a champion and a saviour through vast burning zones, and
moisten their parched lips with the gush of gratitude for
deliverance from chains\textemdash he must have a Prime-Minister
drink his health at a Cabinet-dinner for aiding to rivet on those
of his country and of Europe! He goes hand and heart along with
Government in all their notions of legitimacy and political
aggrandizement, in the hope that they will leave him a sort of
\emph{no-man's ground} of humanity in the Great Desert, where his
reputation for benevolence and public spirit may spring up and
flourish, till its head touches the clouds, and it stretches out
its branches to the farthest part of the earth. He has no mercy on
those who claim a property in negro-slaves as so much live-stock
on their estates; the country rings with the applause of his wit,
his eloquence, and his indignant appeals to common sense and
humanity on this subject\textemdash but not a word has he to say,
not a whisper does he breathe against the claim set up by the
Despots of the Earth over their Continental subjects, but does
every thing in his power to confirm and sanction it! He must give
no offence. Mr. Wilberforce's humanity will go all lengths that it
can with safety and discretion: but it is not to be supposed that
it should lose him his seat for Yorkshire, the smile of Majesty,
or the countenance of the loyal and pious. He is anxious to do all
the good he can without hurting himself or his fair fame. His
conscience and his character compound matters very amicably.  He
rather patronises honesty than is a martyr to it. His patriotism,
his philanthropy are not so ill-bred, as to quarrel with his
loyalty or to banish him from the first circles. He preaches vital
Christianity to untutored savages; and tolerates its worst abuses
in civilized states.  He thus shews his respect for religion
without offending the clergy, or circumscribing the sphere of his
usefulness. There is in all this an appearance of a good deal of
cant and tricking. His patriotism may be accused of being servile;
his humanity ostentatious; his loyalty conditional; his religion a
mixture of fashion and fanaticism. ``Out upon such half-faced
fellowship!'' Mr. Wilberforce has the pride of being familiar with
the great; the vanity of being popular; the conceit of an
approving conscience. He is coy in his approaches to power; his
public spirit is, in a manner, \emph{under the rose}. He thus
reaps the credit of independence, without the obloquy; and secures
the advantages of servility, without incurring any obligations. He
has two strings to his bow:\textemdash he by no means neglects his
worldly interests, while he expects a bright reversion in the
skies. Mr. Wilberforce is far from being a hypocrite; but he is,
we think, as fine a specimen of \emph{moral equivocation} as can
well be conceived. A hypocrite is one who is the very reverse of,
or who despises the character he pretends to be: Mr.  Wilberforce
would be all that he pretends to be, and he is it in fact, as far
as words, plausible theories, good inclinations, and easy services
go, but not in heart and soul, or so as to give up the appearance
of any one of his pretensions to preserve the reality of any
other. He carefully chooses his ground to fight the battles of
loyalty, religion, and humanity, and it is such as is always safe
and advantageous to himself! This is perhaps hardly fair, and it
is of dangerous or doubtful tendency. Lord Eldon, for instance, is
known to be a thorough-paced ministerialist: his opinion is only
that of his party.  But Mr. Wilberforce is not a party-man. He is
the more looked up to on this account, but not with sufficient
reason. By tampering with different temptations and personal
projects, he has all the air of the most perfect independence, and
gains a character for impartiality and candour, when he is only
striking a balance in his mind between the \emph{éclat} of
differing from a Minister on some 'vantage ground, and the risk or
odium that may attend it. He carries all the weight of his
artificial popularity over to the Government on vital points and
hard-run questions; while they, in return, lend him a little of
the gilding of court-favour to set off his disinterested
philanthropy and tramontane enthusiasm. As a leader or a follower,
he makes an odd jumble of interests. By virtue of religious
sympathy, he has brought the Saints over to the side of the
abolition of Negro slavery. This his adversaries think hard and
stealing a march upon them. What have the SAINTS to do with
freedom or reform of any kind?\textemdash Mr. Wilberforce's style
of speaking is not quite \emph{parliamentary}, it is halfway
between that and \emph{evangelical}. He is altogether a
\emph{double-entendre:} the very tone of his voice is a
\emph{double-entendre.} It winds, and undulates, and glides up and
down on texts of Scripture, and scraps from Paley, and trite
sophistry, and pathetic appeals to his hearers in a faltering,
inprogressive, sidelong way, like those birds of weak wing, that
are borne from their strait-forward course
\begin{quote}
  ``By every little breath that under heaven is blown.''

\end{quote}
Something of this fluctuating, time-serving principle was visible
even in the great question of the Abolition of the Slave Trade. He
was, at one time, half inclined to surrender it into Mr. Pitt's
dilatory hands, and seemed to think the gloss of novelty was gone
from it, and the gaudy colouring of popularity sunk into the
\emph{sable} ground from which it rose!  It was, however,
persisted in and carried to a triumphant conclusion.
Mr. Wilberforce said too little on this occasion of one, compared
with whom he was but the frontispiece to that great chapter in the
history of the world\textemdash the mask, the varnishing, and
painting\textemdash the man that effected it by Herculean labours
of body, and equally gigantic labours of mind was Clarkson, the
true Apostle of human Redemption on that occasion, and who, it is
remarkable, resembles in his person and lineaments more than one
of the Apostles in the \emph{Cartoons} of Raphael. He deserves to
be added to the Twelve!\footnote{After all, the best as well as
  most amusing comment on the character just described was that
  made by Sheridan, who being picked up in no very creditable
  plight by the watch, and asked rather roughly who he was, made
  answer\textemdash ``I am Mr. Wilberforce!'' The guardians of the
  night conducted him home with all the honours due to Grace and
  Nature.}
