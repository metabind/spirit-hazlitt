\chapter[Elia and Geoffrey Crayon]
{elia {\normalsize and} geoffrey crayon}

So Mr. Charles Lamb and Mr. Washington Irvine choose to designate
themselves; and as their lucubrations under one or other of these
\emph{noms de guerre} have gained considerable notice from the
public, we shall here attempt to discriminate their several styles
and manner, and to point out the beauties and defects of each in
treating of somewhat similar subjects.

Mr. Irvine is, we take it, the more popular writer of the two, or
a more general favourite: Mr. Lamb has more devoted, and perhaps
more judicious partisans. Mr. Irvine is by birth an American, and
has, as it were, \emph{skimmed the cream}, and taken off patterns
with great skill and cleverness, from our best known and happiest
writers, so that their thoughts and almost their reputation are
indirectly transferred to his page, and smile upon us from another
hemisphere, like ``the pale reflex of Cynthia's brow:'' he
succeeds to our admiration and our sympathy by a sort of
prescriptive title and traditional privilege. Mr. Lamb, on the
contrary, being ``native to the manner here,'' though he too has
borrowed from previous sources, instead of availing himself of the
most popular and admired, has groped out his way, and made his
most successful researches among the more obscure and intricate,
though certainly not the least pithy or pleasant of our
writers. Mr. Washington Irvine has culled and transplanted the
flowers of modern literature, for the amusement of the general
reader: Mr. Lamb has raked among the dust and cobwebs of a more
remote period, has exhibited specimens of curious relics, and
pored over moth-eaten, decayed manuscripts, for the benefit of the
more inquisitive and discerning part of the public. Antiquity
after a time has the grace of novelty, as old fashions revived are
mistaken for new ones; and a certain quaintness and singularity of
style is an agreeable relief to the smooth and insipid monotony of
modern composition. Mr. Lamb has succeeded not by conforming to
the \emph{Spirit of the Age}, but in opposition to it. He does not
march boldly along with the crowd, but steals off the pavement to
pick his way in the contrary direction. He prefers \emph{bye-ways}
to \emph{highways}. When the full tide of human life pours along
to some festive shew, to some pageant of a day, Elia would stand
on one side to look over an old book-stall, or stroll down some
deserted pathway in search of a pensive inscription over a
tottering door-way, or some quaint device in architecture,
illustrative of embryo art and ancient manners. Mr. Lamb has the
very soul of an antiquarian, as this implies a reflecting
humanity; the film of the past hovers for ever before him. He is
shy, sensitive, the reverse of every thing coarse, vulgar,
obtrusive, and \emph{common-place}. He would fain ``shuffle off
this mortal coil'', and his spirit clothes itself in the garb of
elder time, homelier, but more durable. He is borne along with no
pompous paradoxes, shines in no glittering tinsel of a fashionable
phraseology; is neither fop nor sophist. He has none of the
turbulence or froth of new-fangled opinions. His style runs pure
and clear, though it may often take an underground course, or be
conveyed through old-fashioned conduit-pipes. Mr. Lamb does not
court popularity, nor strut in gaudy plumes, but shrinks from
every kind of ostentatious and obvious pretension into the
retirement of his own mind.
\begin{verse}
  \vleftofline{``}The self-applauding bird, the peacock see:\textemdash \\
  Mark what a sumptuous pharisee is he!\\
  Meridian sun-beams tempt him to unfold\\
  His radiant glories, azure, green, and gold:\\
  He treads as if, some solemn music near,\\
  His measured step were governed by his ear:\\
  And seems to say\textemdash Ye meaner fowl, give place,\\
  I am all splendour, dignity, and grace!\\
  Not so the pheasant on his charms presumes,\\
  Though he too has a glory in his plumes.\\
  He, christian-like, retreats with modest mien\\
  To the close copse or far sequestered green,\\
  And shines without desiring to be seen.''
\end{verse}
These lines well describe the modest and delicate beauties of
Mr. Lamb's writings, contrasted with the lofty and vain-glorious
pretensions of some of his contemporaries. This gentleman is not
one of those who pay all their homage to the prevailing idol: he
thinks that
\begin{quote}
  ``New-born gauds are made and moulded of things past.''
\end{quote}
nor does he
\begin{verse}
  \vleftofline{``}Give to dust that is a little gilt \\
  More laud than gilt o'er-dusted.''
\end{verse}
His convictions ``do not in broad rumour lie,'' nor are they ``set
off to the world in the glistering foil'' of fashion; but ``live
and breathe aloft in those pure eyes, and perfect judgment of
all-seeing \emph{time}.''  Mr. Lamb rather affects and is
tenacious of the obscure and remote: of that which rests on its
own intrinsic and silent merit; which scorns all alliance, or even
the suspicion of owing any thing to noisy clamour, to the glare of
circumstances. There is a fine tone of \emph{chiaro-scuro}, a
moral perspective in his writings. He delights to dwell on that
which is fresh to the eye of memory; he yearns after and covets
what soothes the frailty of human nature. That touches him most
nearly which is withdrawn to a certain distance, which verges on
the borders of oblivion:\textemdash that piques and provokes his
fancy most, which is hid from a superficial glance. That which,
though gone by, is still remembered, is in his view more genuine,
and has given more ``vital signs that it will live,'' than a thing
of yesterday, that may be forgotten to-morrow. Death has in this
sense the spirit of life in it; and the shadowy has to our author
something substantial in it. Ideas savour most of reality in his
mind; or rather his imagination loiters on the edge of each, and a
page of his writings recals to our fancy the \emph{stranger} on
the grate, fluttering in its dusky tensity, with its idle
superstition and hospitable welcome!

Mr. Lamb has a distaste to new faces, to new books, to new
buildings, to new customs. He is shy of all imposing appearances,
of all assumptions of self-importance, of all adventitious
ornaments, of all mechanical advantages, even to a nervous
excess. It is not merely that he does not rely upon, or ordinarily
avail himself of them; he holds them in abhorrence, he utterly
abjures and discards them, and places a great gulph between him
and them. He disdains all the vulgar artifices of authorship, all
the cant of criticism, and helps to notoriety. He has no grand
swelling theories to attract the visionary and the enthusiast, no
passing topics to allure the thoughtless and the vain. He evades
the present, he mocks the future. His affections revert to, and
settle on the past, but then, even this must have something
personal and local in it to interest him deeply and thoroughly; he
pitches his tent in the suburbs of existing manners; brings down
the account of character to the few straggling remains of the last
generation; seldom ventures beyond the bills of mortality, and
occupies that nice point between egotism and disinterested
humanity. No one makes the tour of our southern metropolis, or
describes the manners of the last age, so well as Mr.
Lamb\textemdash with so fine, and yet so formal an air\textemdash
with such vivid obscurity, with such arch piquancy, such
picturesque quaintness, such smiling pathos. How admirably he has
sketched the former inmates of the South- Sea House; what ``fine
fretwork he makes of their double and single entries!'' With what
a firm, yet subtle pencil he has embodied \emph{Mrs.  Battle's
  Opinions on Whist}! How notably he embalms a battered
\emph{beau}; how delightfully an amour, that was cold forty years
ago, revives in his pages! With what well-disguised humour he
introduces us to his relations, and how freely he serves up his
friends! Certainly, some of his portraits are \emph{fixtures}, and
will do to hang up as lasting and lively emblems of human
infirmity. Then there is no one who has so sure an ear for ``the
chimes at midnight'', not even excepting Mr. Justice Shallow; nor
could Master Silence himself take his ``cheese and pippins'' with
a more significant and satisfactory air. With what a gusto
Mr. Lamb describes the inns and courts of law, the Temple and
Gray's-Inn, as if he had been a student there for the last two
hundred years, and had been as well acquainted with the person of
Sir Francis Bacon as he is with his portrait or writings! It is
hard to say whether St. John's Gate is connected with more intense
and authentic associations in his mind, as a part of old London
Wall, or as the frontispiece (time out of mind) of the Gentleman's
Magazine. He haunts Watling-street like a gentle spirit; the
avenues to the play-houses are thick with panting recollections,
and Christ's-Hospital still breathes the balmy breath of infancy
in his description of it! Whittington and his Cat are a fine
hallucination for Mr. Lamb's historic Muse, and we believe he
never heartily forgave a certain writer who took the subject of
Guy Faux out of his hands. The streets of London are his
fairy-land, teeming with wonder, with life and interest to his
retrospective glance, as it did to the eager eye of childhood; he
has contrived to weave its tritest traditions into a bright and
endless romance!

Mr. Lamb's taste in books is also fine, and it is peculiar. It is
not the worse for a little \emph{idiosyncrasy}. He does not go
deep into the Scotch novels, but he is at home in Smollett and
Fielding. He is little read in Junius or Gibbon, but no man can
give a better account of Burton's Anatomy of Melancholy, or Sir
Thomas Brown's Urn-Burial, or Fuller's Worthies, or John Bunyan's
Holy War. No one is more unimpressible to a specious declamation;
no one relishes a recondite beauty more. His admiration of
Shakespear and Milton does not make him despise Pope; and he can
read Parnell with patience, and Gay with delight. His taste in
French and German literature is somewhat defective: nor has he
made much progress in the science of Political Economy or other
abstruse studies, though he has read vast folios of controversial
divinity, merely for the sake of the intricacy of style, and to
save himself the pain of thinking. Mr. Lamb is a good judge of
prints and pictures. His admiration of Hogarth does credit to
both, particularly when it is considered that Leonardo da Vinci is
his next greatest favourite, and that his love of the
\emph{actual} does not proceed from a want of taste for the
\emph{ideal}. His worst fault is an over-eagerness of enthusiasm,
which occasionally makes him take a surfeit of his highest
favourites.\textemdash Mr. Lamb excels in familiar conversation
almost as much as in writing, when his modesty does not overpower
his self-possession. He is as little of a proser as possible; but
he \emph{blurts} out the finest wit and sense in the world. He
keeps a good deal in the back-ground at first, till some excellent
conceit pushes him forward, and then he abounds in whim and
pleasantry. There is a primitive simplicity and self-denial about
his manners; and a Quakerism in his personal appearance, which is,
however, relieved by a fine Titian head, full of dumb eloquence!
Mr. Lamb is a general favourite with those who know him. His
character is equally singular and amiable. He is endeared to his
friends not less by his foibles than his virtues; he insures their
esteem by the one, and does not wound their self-love by the
other. He gains ground in the opinion of others, by making no
advances in his own. We easily admire genius where the diffidence
of the possessor makes our acknowledgment of merit seem like a
sort of patronage, or act of condescension, as we willingly extend
our good offices where they are not exacted as obligations, or
repaid with sullen indifference.\textemdash The style of the
Essays of Elia is liable to the charge of a certain
\emph{mannerism}. His sentences are cast in the mould of old
authors; his expressions are borrowed from them; but his feelings
and observations are genuine and original, taken from actual life,
or from his own breast; and he may be said (if any one can) ``to
have coined his heart for \emph{jests},'' and to have split his
brain for fine distinctions! Mr. Lamb, from the peculiarity of his
exterior and address as an author, would probably never have made
his way by detached and independent efforts; but, fortunately for
himself and others, he has taken advantage of the Periodical
Press, where he has been stuck into notice, and the texture of his
compositions is assuredly fine enough to bear the broadest glare
of popularity that has hitherto shone upon them.  Mr. Lamb's
literary efforts have procured him civic honours (a thing unheard
of in our times), and he has been invited, in his character of
ELIA, to dine at a select party with the Lord Mayor. We should
prefer this distinction to that of being poet-laureat. We would
recommend to Mr. Waithman's perusal (if Mr. Lamb has not
anticipated us) the \emph{Rosamond Gray} and the \emph{John
  Woodvil} of the same author, as an agreeable relief to the noise
of a city feast, and the heat of city elections. A friend, a short
time ago, quoted some lines\footnote{The description of sports in
  the forest:
\begin{verse}
  \vleftofline{``}To see the sun to bed and to arise,\\
  Like some hot amourist with glowing eyes,'' \&c.
\end{verse}} from the last-mentioned of these works, which meeting
Mr. Godwin's eye, he was so struck with the beauty of the passage,
and with a consciousness of having seen it before, that he was
uneasy till he could recollect where, and after hunting in vain
for it in Ben Jonson, Beaumont and Fletcher, and other not
unlikely places, sent to Mr. Lamb to know if he could help him to
the author!

Mr. Washington Irvine's acquaintance with English literature
begins almost where Mr. Lamb's ends,\textemdash with the
Spectator, Tom Brown's works, and the wits of Queen Anne. He is
not bottomed in our elder writers, nor do we think he has tasked
his own faculties much, at least on English ground. Of the merit
of his \emph{Knicker-bocker,} and New York stories, we cannot
pretend to judge. But in his \emph{Sketch-book} and
\emph{Bracebridge-Hall} he gives us very good American copies of
our British Essayists and Novelists, which may be very well on the
other side of the water, and as proofs of the capabilities of the
national genius, but which might be dispensed with here, where we
have to boast of the originals. Not only Mr. Irvine's language is
with great taste and felicity modelled on that of Addison, Sterne,
Goldsmith, or Mackenzie; but the thoughts and sentiments are taken
at the rebound, and as they are brought forward at the present
period, want both freshness and probability. Mr. Irvine's writings
are literary \emph{anachronisms}. He comes to England for the
first time; and being on the spot, fancies himself in the midst of
those characters and manners which he had read of in the Spectator
and other approved authors, and which were the only idea he had
hitherto formed of the parent country. Instead of looking round to
see what \emph{we are}, he sets to work to describe us as \emph{we
  were}\textemdash at second hand. He has Parson Adams, or Sir
Roger de Coverley in his ``\emph{mind's eye}''; and he makes a
village curate, or a country 'squire in Yorkshire or Hampshire sit
to these admired models for their portraits in the beginning of
the nineteenth century. Whatever the ingenious author has been
most delighted with in the representations of books, he transfers
to his port-folio, and swears that he has found it actually
existing in the course of his observation and travels through
Great Britain. Instead of tracing the changes that have taken
place in society since Addison or Fielding wrote, he transcribes
their account in a different hand-writing, and thus keeps us
stationary, at least in our most attractive and praise-worthy
qualities of simplicity, honesty, hospitality, modesty, and
good-nature. This is a very flattering mode of turning fiction
into history, or history into fiction; and we should scarcely know
ourselves again in the softened and altered likeness, but that it
bears the date of 1820, and issues from the press in
Albemarle-street. This is one way of complimenting our national
and Tory prejudices; and coupled with literal or exaggerated
portraits of \emph{Yankee} peculiarities, could hardly fail to
please. The first Essay in the \emph{Sketch-book}, that on
National Antipathies, is the best; but after that, the sterling
ore of wit or feeling is gradually spun thinner and thinner, till
it fades to the shadow of a shade. Mr. Irvine is himself, we
believe, a most agreeable and deserving man, and has been led into
the natural and pardonable error we speak of, by the tempting bait
of European popularity, in which he thought there was no more
likely method of succeeding than by imitating the style of our
standard authors, and giving us credit for the virtues of our
forefathers.

\bulletrule

We should not feel that we had discharged our obligations to truth
or friendship, if we were to let this volume go without
introducing into it the name of the author of
\emph{Virginius}. This is the more proper, inasmuch as he is a
character by himself, and the only poet now living that is a mere
poet. If we were asked what sort of a man Mr. Knowles is, we could
only say, ``he is the writer of Virginius.'' His most intimate
friends see nothing in him, by which they could trace the work to
the author. The seeds of dramatic genius are contained and
fostered in the warmth of the blood that flows in his veins; his
heart dictates to his head. The most unconscious, the most
unpretending, the most artless of mortals, he instinctively obeys
the impulses of natural feeling, and produces a perfect work of
art. He has hardly read a poem or a play or seen any thing of the
world, but he hears the anxious beatings of his own heart, and
makes others feel them by the force of sympathy. Ignorant alike of
rules, regardless of models, he follows the steps of truth and
simplicity; and strength, proportion, and delicacy are the
infallible results. By thinking of nothing but his subject, he
rivets the attention of the audience to it. All his dialogue tends
to action, all his situations form classic groups. There is no
doubt that Virginius is the best acting tragedy that has been
produced on the modern stage. Mr.  Knowles himself was a player at
one time, and this circumstance has probably enabled him to judge
of the picturesque and dramatic effect of his lines, as we think
it might have assisted Shakespear. There is no impertinent
display, no flaunting poetry; the writer immediately conceives how
a thought would tell if he had to speak it himself. Mr.  Knowles
is the first tragic writer of the age; in other respects he is a
common man; and divides his time and his affections between his
plots and his fishing-tackle, between the Muses' spring, and those
mountain-streams which sparkle like his own eye, that gush out
like his own voice at the sight of an old friend. We have known
him almost from a child, and we must say he appears to us the same
boy-poet that he ever was. He has been cradled in song, and rocked
in it as in a dream, forgetful of himself and of the world!

