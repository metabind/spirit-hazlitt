\chapter[Sir James Mackintosh]{sir james mackintosh} %09

The subject of the present article is one of the ablest and most
accomplished men of the age, both as a writer, a speaker, and a
converser. He is, in fact, master of almost every known topic,
whether of a passing or of a more recondite nature. He has lived
much in society, and is deeply conversant with books. He is a man
of the world and a scholar; but the scholar gives the tone to all
his other acquirements and pursuits. Sir James is by education and
habit, and we were going to add, by the original turn of his mind,
a college-man; and perhaps he would have passed his time most
happily and respectably, had he devoted himself entirely to that
kind of life. The strength of his faculties would have been best
developed, his ambition would have met its proudest reward, in the
accumulation and elaborate display of grave and useful
knowledge. As it is, it may be said, that in company he talks
well, but too much; that in writing he overlays the original
subject and spirit of the composition, by an appeal to authorities
and by too formal a method; that in public speaking the logician
takes place of the orator, and that he fails to give effect to a
particular point or to urge an immediate advantage home upon his
adversary from the enlarged scope of his mind, and the wide career
he takes in the field of argument.

To consider him in the last point of view, first. As a political
partisan, he is rather the lecturer than the advocate. He is able
to instruct and delight an impartial and disinterested audience by
the extent of his information, by his acquaintance with general
principles, by the clearness and aptitude of his illustrations, by
vigour and copiousness of style; but where he has a prejudiced or
unfair antagonist to contend with, he is just as likely to put
weapons into his enemy's hands as to wrest them from him, and his
object seems to be rather to deserve than to obtain success. The
characteristics of his mind are retentiveness and comprehension,
with facility of production: but he is not equally remarkable for
originality of view, or warmth of feeling, or liveliness of
fancy. His eloquence is a little rhetorical; his reasoning chiefly
logical: he can bring down the account of knowledge on a vast
variety of subjects to the present moment, he can embellish any
cause he undertakes by the most approved and graceful ornaments,
he can support it by a host of facts and examples, but he cannot
advance it a step forward by placing it on a new and triumphant
'vantage-ground, nor can he overwhelm and break down the
artificial fences and bulwarks of sophistry by the irresistible
tide of manly enthusiasm. Sir James Mackintosh is an accomplished
debater, rather than a powerful orator: he is distinguished more
as a man of wonderful and variable talent than as a man of
commanding intellect. His mode of treating a question is critical,
and not parliamentary. It has been formed in the closet and the
schools, and is hardly fitted for scenes of active life, or the
collisions of party-spirit. Sir James reasons on the square; while
the arguments of his opponents are loaded with iron or gold. He
makes, indeed, a respectable ally, but not a very formidable
opponent. He is as likely, however, to prevail on a neutral, as he
is almost certain to be baffled on a hotly contested ground. On
any question of general policy or legislative improvement, the
Member for Nairn is heard with advantage, and his speeches are
attended with effect: and he would have equal weight and influence
at other times, if it were the object of the House to hear reason,
as it is his aim to speak it. But on subjects of peace or war, of
political rights or foreign interference, where the waves of party
run high, and the liberty of nations or the fate of mankind hangs
trembling in the scales, though he probably displays equal talent,
and does full and heaped justice to the question (abstractedly
speaking, or if it were to be tried before an impartial assembly),
yet we confess we have seldom heard him, on such occasions,
without pain for the event. He did not slur his own character and
pretensions, but he compromised the argument. He spoke \emph{the
truth, the whole truth, and nothing but the truth}; but the House
of Commons (we dare aver it) is not the place where the truth, the
whole truth, and nothing but the truth can be spoken with safety
or with advantage. The judgment of the House is not a balance to
weigh scruples and reasons to the turn of a fraction: another
element, besides the love of truth, enters into the composition of
their decisions, the reaction of which must be calculated upon and
guarded against. If our philosophical statesman had to open the
case before a class of tyros, or a circle of grey-beards, who
wished to form or to strengthen their judgments upon fair and
rational grounds, nothing could be more satisfactory, more
luminous, more able or more decisive than the view taken of it by
Sir James Mackintosh. But the House of Commons, as a collective
body, have not the docility of youth, the calm wisdom of age; and
often only want an excuse to do wrong, or to adhere to what they
have already determined upon; and Sir James, in detailing the
inexhaustible stores of his memory and reading, in unfolding the
wide range of his theory and practice, in laying down the rules
and the exceptions, in insisting upon the advantages and the
objections with equal explicitness, would be sure to let something
drop that a dextrous and watchful adversary would easily pick up
and turn against him, if this were found necessary; or if with so
many \emph{pros} and \emph{cons}, doubts and difficulties,
dilemmas and alternatives thrown into it, the scale, with its
natural bias to interest and power, did not already fly up and
kick the beam. There wanted unity of purpose, impetuosity of
feeling to break through the phalanx of hostile and inveterate
prejudice arrayed against him. He gave a handle to his enemies;
threw stumbling-blocks in the way of his friends. He raised so
many objections for the sake of answering them, proposed so many
doubts for the sake of solving them, and made so many concessions
where none were demanded, that his reasoning had the effect of
neutralizing itself; it became a mere exercise of the
understanding without zest or spirit left in it; and the provident
engineer who was to shatter in pieces the strong-holds of
corruption and oppression, by a well-directed and unsparing
discharge of artillery, seemed to have brought not only his own
cannon-balls, but his own wool-packs along with him to ward off
the threatened mischief. This was a good deal the effect of his
maiden speech on the transfer of Genoa, to which Lord Castlereagh
did not deign an answer, and which another Honourable Member
called ``a \emph{finical} speech.'' It was a most able, candid,
closely argued, and philosophical exposure of that unprincipled
transaction; but for this very reason it was a solecism in the
place where it was delivered. Sir James has, since this period,
and with the help of practice, lowered himself to the tone of the
House; and has also applied himself to questions more congenial to
his habits of mind, and where the success would be more likely to
be proportioned to his zeal and his exertions.

There was a greater degree of power, or of dashing and splendid
effect (we wish we could add, an equally humane and liberal
spirit) in the \emph{Lectures on the Law of Nature and Nations},
formerly delivered by Sir James (then Mr.) Mackintosh, in
Lincoln's-Inn Hall. He shewed greater confidence; was more at home
there. The effect was more electrical and instantaneous, and this
elicited a prouder display of intellectual riches, and a more
animated and imposing mode of delivery. He grew wanton with
success. Dazzling others by the brilliancy of his acquirements,
dazzled himself by the admiration they excited, he lost fear as
well as prudence; dared every thing, carried every thing before
him. The Modern Philosophy, counterscarp, outworks, citadel, and
all, fell without a blow, by ``the whiff and wind of his fell
\emph{doctrine},'' as if it had been a pack of cards. The volcano
of the French Revolution was seen expiring in its own flames, like
a bon-fire made of straw: the principles of Reform were scattered
in all directions, like chaff before the keen northern blast. He
laid about him like one inspired; nothing could withstand his
envenomed tooth. Like some savage beast got into the garden of the
fabled Hesperides, he made clear work of it, root and branch, with
white, foaming tusks\textemdash
\begin{quote} ``Laid waste the borders, and o'erthrew the
bowers.''
\end{quote} 
\fixspacing{1.1}{The havoc was amazing, the desolation was complete. As to our
visionary sceptics and Utopian philosophers}, they stood no chance
with our lecturer\textemdash he did not ``carve them as a dish fit
for the Gods, but hewed them as a carcase fit for hounds.'' Poor
Godwin, who had come, in the \emph{bonhommie} and candour of his
nature, to hear what new light had broken in upon his old friend,
was obliged to quit the field, and slunk away after an exulting
taunt thrown out at ``such fanciful chimeras as a golden mountain
or a perfect man.'' Mr. Mackintosh had something of the air, much
of the dexterity and self-possession, of a political and
philosophical juggler; and an eager and admiring audience gaped
and greedily swallowed the gilded bait of sophistry, prepared for
their credulity and wonder. Those of us who attended day after
day, and were accustomed to have all our previous notions
confounded and struck out of our hands by some metaphysical
legerdemain, were at last at some loss to know \emph{whether two
  and two made four}, till we had heard the lecturer's opinion on
that head. He might have some mental reservation on the subject,
some pointed ridicule to pour upon the common supposition, some
learned authority to quote against it. To anticipate the line of
argument he might pursue, was evidently presumptuous and
premature. One thing only appeared certain, that whatever opinion
he chose to take up, he was able to make good either by the foils
or the cudgels, by gross banter or nice distinctions, by a
well-timed mixture of paradox and common-place, by an appeal to
vulgar prejudices or startling scepticism.  It seemed to be
equally his object, or the tendency of his Discourses, to unsettle
every principle of reason or of common sense, and to leave his
audience at the mercy of the \emph{dictum} of a lawyer, the nod of
a minister, or the shout of a mob. To effect this purpose, he drew
largely on the learning of antiquity, on modern literature, on
history, poetry, and the belles-lettres, on the Schoolmen and on
writers of novels, French, English, and Italian. In mixing up the
sparkling julep, that by its potent operation was to scour away
the dregs and feculence and peccant humours of the body politic,
he seemed to stand with his back to the drawers in a metaphysical
dispensary, and to take out of them whatever ingredients suited
his purpose. In this way he had an antidote for every error, an
answer to every folly. The writings of Burke, Hume, Berkeley,
Paley, Lord Bacon, Jeremy Taylor, Grotius, Puffendorf, Cicero,
Aristotle, Tacitus, Livy, Sully, Machiavel, Guicciardini, Thuanus,
lay open beside him, and he could instantly lay his hand upon the
passage, and quote them chapter and verse to the clearing up of
all difficulties, and the silencing of all
oppugners. Mr. Mackintosh's Lectures were after all but a kind of
philosophical centos. They were profound, brilliant, new to his
hearers; but the profundity, the brilliancy, the novelty were not
his own. He was like Dr. Pangloss (not Voltaire's, but Coleman's)
who speaks only in quotations; and the pith, the marrow of Sir
James's reasoning and rhetoric at that memorable period might be
put within inverted commas. It, however, served its purpose and
the loud echo died away. We remember an excellent man and a sound
critic\footnote{The late Rev. Joseph Fawcett, of Walthamstow.}
going to hear one of these elaborate effusions; and on his want of
enthusiasm being accounted for from its not being one of the
orator's brilliant days, he replied, ``he did not think a man of
genius could speak for two hours without saying something by which
he would have been electrified.''  We are only sorry, at this
distance of time, for one thing in these Lectures\textemdash the
tone and spirit in which they seemed to have been composed and to
be delivered. If all that body of opinions and principles of which
the orator read his recantation was unfounded, and there was an
end of all those views and hopes that pointed to future
improvement, it was not a matter of triumph or exultation to the
lecturer or any body else, to the young or the old, the wise or
the foolish; on the contrary, it was a subject of regret, of slow,
reluctant, painful admission\textemdash
\begin{quote} ``Of lamentation loud heard through the rueful
air.''
\end{quote} The immediate occasion of this sudden and violent
change in Sir James's views and opinions was attributed to a
personal interview which he had had a little before his death with
Mr. Burke, at his house at Beaconsfield. In the latter end of the
year 1796, appeared the \emph{Regicide Peace}, from the pen of the
great apostate from liberty and betrayer of his species into the
hands of those who claimed it as their property by divine
right\textemdash a work imposing, solid in many respects,
abounding in facts and admirable reasoning, and in which all
flashy ornaments were laid aside for a testamentary gravity, (the
eloquence of despair resembling the throes and heaving and
muttered threats of an earthquake, rather than the loud
thunder-bolt)\textemdash and soon after came out a criticism on it
in \emph{The Monthly Review}, doing justice to the author and the
style, and combating the inferences with force and at much length;
but with candour and with respect, amounting to deference. It was
new to Mr.  Burke not to be called names by persons of the
opposite party; it was an additional triumph to him to be spoken
well of, to be loaded with well-earned praise by the author of the
\emph{Vindiciæ Gallicæ}. It was a testimony from an old, a
powerful, and an admired antagonist.\footnote{At the time when the
\emph{Vindiciae Gallicae} first made its appearance, as a reply to
the \emph{Reflections on the French Revolution}, it was cried up
by the partisans of the new school, as a work superior in the
charms of composition to its redoubted rival: in acuteness, depth,
and soundness of reasoning, of course there was supposed to be no
comparison.} He sent an invitation to the writer to come and see
him; and in the course of three days' animated discussion of such
subjects, Mr. Mackintosh became a convert not merely to the graces
and gravity of Mr. Burke's style, but to the liberality of his
views, and the solidity of his opinions.\textemdash The
Lincoln's-Inn Lectures were the fruit of this interview: such is
the influence exercised by men of genius and imaginative power
over those who have nothing to oppose to their unforeseen flashes
of thought and invention, but the dry, cold, formal deductions of
the understanding.  Our politician had time, during a few years of
absence from his native country, and while the din of war and the
cries of party-spirit ``were lost over a wide and unhearing
ocean,'' to recover from his surprise and from a temporary
alienation of mind; and to return in spirit, and in the mild and
mellowed maturity of age, to the principles and attachments of his
early life.

\fixspacing{1.1}{
The appointment of Sir James Mackintosh to a Judgeship in India
was one, which, however flattering to his vanity or favourable to
his interests, was entirely foreign to his feelings and habits. It
was an honourable exile. He was out of his element among black
slaves and sepoys, and Nabobs and cadets, and writers to India. He
had no one to exchange ideas with. The ``unbought grace of life,''
the charm of literary conversation was gone. It was the habit of
his mind, his ruling passion to enter into the shock and conflict
of opinions on philosophical, political, and critical
questions\textemdash not to dictate to raw tyros or domineer over
persons in subordinate situations\textemdash but to obtain the
guerdon and the laurels of superior sense and information by
meeting with men of equal standing, to have a fair field pitched,
to argue, to distinguish, to reply, to hunt down the game of
intellect with eagerness and skill, to push an advantage, to cover
a retreat, to give and take a fall\textemdash}
\begin{quote}
  ``And gladly would he learn, and gladly teach.''

\end{quote}
It is no wonder that this sort of friendly intellectual
gladiatorship is Sir James's greatest pleasure, for it is his
peculiar \emph{forte}. He has not many equals, and scarcely any
superior in it. He is too indolent for an author; too
unimpassioned for an orator: but in society he is just vain enough
to be pleased with immediate attention, good-humoured enough to
listen with patience to others, with great coolness and
self-possession, fluent, communicative, and with a manner equally
free from violence and insipidity. Few subjects can be started, on
which he is not qualified to appear to advantage as the gentleman
and scholar. If there is some tinge of pedantry, it is carried off
by great affability of address and variety of amusing and
interesting topics. There is scarce an author that he has not
read; a period of history that he is not conversant with; a
celebrated name of which he has not a number of anecdotes to
relate; an intricate question that he is not prepared to enter
upon in a popular or scientific manner. If an opinion in an
abstruse metaphysical author is referred to, he is probably able
to repeat the passage by heart, can tell the side of the page on
which it is to be met with, can trace it back through various
descents to Locke, Hobbes, Lord Herbert of Cherbury, to a place in
some obscure folio of the School-men or a note in one of the
commentators on Aristotle or Plato, and thus give you in a few
moments' space, and without any effort or previous notice, a
chronological table of the progress of the human mind in that
particular branch of inquiry. There is something, we think,
perfectly admirable and delightful in an exhibition of this kind,
and which is equally creditable to the speaker and gratifying to
the hearer.  But this kind of talent was of no use in India: the
intellectual wares, of which the Chief Judge delighted to make a
display, were in no request there. He languished after the friends
and the society he had left behind; and wrote over incessantly for
books from England. One that was sent him at this time was an
\emph{Essay on the Principles of Human Action}; and the way in
which he spoke of that dry, tough, metaphysical \emph{choke-pear},
shewed the dearth of intellectual intercourse in which he lived,
and the craving in his mind after those studies which had once
been his pride, and to which he still turned for consolation in
his remote solitude.\textemdash Perhaps to another, the novelty of
the scene, the differences of mind and manners might have atoned
for a want of social and literary \emph{agrèmens}: but Sir James
is one of those who see nature through the spectacles of books. He
might like to read an account of India; but India itself with its
burning, shining face would be a mere blank, an endless waste to
him. To persons of this class of mind things must be translated
into words, visible images into abstract propositions to meet
their refined apprehensions, and they have no more to say to a
matter-of-fact staring them in the face without a label in its
mouth, than they would to a hippopotamus!\textemdash We may add,
before we quit this point, that we cannot conceive of any two
persons more different in colloquial talents, in which they both
excel, than Sir James Mackintosh and Mr. Coleridge. They have
nearly an equal range of reading and of topics of conversation:
but in the mind of the one we see nothing but \emph{fixtures}, in
the other every thing is fluid. The ideas of the one are as formal
and tangible, as those of the other are shadowy and
evanescent. Sir James Mackintosh walks over the ground,
Mr. Coleridge is always flying off from it. The first knows all
that has been said upon a subject; the last has something to say
that was never said before. If the one deals too much in learned
\emph{common-places}, the other teems with idle fancies. The one
has a good deal of the \emph{caput mortuum} of genius, the other
is all volatile salt. The conversation of Sir James Mackintosh has
the effect of reading a well-written book, that of his friend is
like hearing a bewildered dream. The one is an Encyclopedia of
knowledge, the other is a succession of \emph{Sybilline Leaves}!

\fixspacing{1}{As an author, Sir James Mackintosh may claim the foremost rank
among those who pride themselves on artificial ornaments and
acquired learning, or who write what may be termed a
\emph{composite} style. His \emph{Vindciae Gallicae} is a work of
great labour, great ingenuity, great brilliancy, and great
vigour. It is a little too antithetical in the structure of its
periods, too dogmatical in the announcement of its opinions. Sir
James has, we believe, rejected something of the \emph{false
brilliant} of the one, as he has retracted some of the abrupt
extravagance of the other. We apprehend, however, that our author
is not one of those who draw from their own resources and
accumulated feelings, or who improve with age. He belongs to a
class (common in Scotland and elsewhere) who get up
school-exercises on any given subject in a masterly manner at
twenty, and who at forty are either where they were\textemdash or
retrograde, if they are men of sense and modesty. The reason is,
their vanity is weaned, after the first hey-day and animal spirits
of youth are flown, from making an affected display of knowledge,
which, however useful, is not their own, and may be much more
simply stated; they are tired of repeating the same arguments over
and over again, after having exhausted and rung the changes on
their whole stock for a number of times. Sir James Mackintosh is
understood to be a writer in the Edinburgh Review; and the
articles attributed to him there are full of matter of great pith
and moment. But they want the trim, pointed expression, the
ambitious ornaments, the ostentatious display and rapid volubility
of his early productions. We have heard it objected to his later
compositions, that his style is good as far as single words and
phrases are concerned, but that his sentences are clumsy and
disjointed, and that these make up still more awkward and
sprawling paragraphs. This is a nice criticism, and we cannot
speak to its truth: but if the fact be so, we think we can account
for it from the texture and obvious process of the author's
mind. All his ideas may be said to be given preconceptions. They
do not arise, as it were, out of the subject, or out of one
another at the moment, and therefore do not flow naturally and
gracefully from one another. They have been laid down beforehand
in a sort of formal division or frame-work of the understanding;
and the connexion between the premises and the conclusion, between
one branch of a subject and another, is made out in a bungling and
unsatisfactory manner. There is no principle of fusion in the
work: he strikes after the iron is cold, and there is a want of
malleability in the style. Sir James is at present said to be
engaged in writing a \emph{History of England} after the downfall
of the house of Stuart. May it be worthy of the talents of the
author, and of the principles of the period it is intended to
illustrate!}