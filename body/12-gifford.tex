\chapter[Mr. Gifford]{mr. gifford}

Mr. Gifford was originally bred to some handicraft: he afterwards
contrived to learn Latin, and was for some time an usher in a
school, till he became a tutor in a nobleman's family. The
low-bred, self-taught man, the pedant, and the dependant on the
great contribute to form the Editor of the \emph{Quarterly
  Review}. He is admirably qualified for this situation, which he
has held for some years, by a happy combination of defects,
natural and acquired; and in the event of his death, it will be
difficult to provide him a suitable successor.

Mr. Gifford has no pretensions to be thought a man of genius, of
taste, or even of general knowledge. He merely understands the
mechanical and instrumental part of learning. He is a critic of
the last age, when the different editions of an author, or the
dates of his several performances were all that occupied the
inquiries of a profound scholar, and the spirit of the writer or
the beauties of his style were left to shift for themselves, or
exercise the fancy of the light and superficial reader. In
studying an old author, he has no notion of any thing beyond
adjusting a point, proposing a different reading, or correcting,
by the collation of various copies, an error of the press. In
appreciating a modern one, if it is an enemy, the first thing he
thinks of is to charge him with bad grammar\textemdash he scans
his sentences instead of weighing his sense; or if it is a friend,
the highest compliment he conceives it possible to pay him is,
that his thoughts and expressions are moulded on some hackneyed
model. His standard of \emph{ideal} perfection is what he himself
now is, a person of \emph{mediocre} literary attainments: his
utmost contempt is shewn by reducing any one to what he himself
once was, a person without the ordinary advantages of education
and learning. It is accordingly assumed, with much complacency in
his critical pages, that Tory writers are classical and courtly as
a matter of course; as it is a standing jest and evident truism,
that Whigs and Reformers must be persons of low birth and
breeding\textemdash imputations from one of which he himself has
narrowly escaped, and both of which he holds in suitable
abhorrence. He stands over a contemporary performance with all the
self-conceit and self-importance of a country schoolmaster, tries
it by technical rules, affects not to understand the meaning,
examines the hand-writing, the spelling, shrugs up his shoulders
and chuckles over a slip of the pen, and keeps a sharp look-out
for a false concord and\textemdash a flogging. There is nothing
liberal, nothing humane in his style of judging: it is altogether
petty, captious, and literal. The Editor's political subserviency
adds the last finishing to his ridiculous pedantry and vanity. He
has all his life been a follower in the train of wealth and
power\textemdash strives to back his pretensions on Parnassus by a
place at court, and to gild his reputation as a man of letters by
the smile of greatness. He thinks his works are stamped with
additional value by having his name in the \emph{Red-Book}. He
looks up to the distinctions of rank and station as he does to
those of learning, with the gross and overweening adulation of his
early origin. All his notions are low, upstart, servile. He thinks
it the highest honour to a poet to be patronised by a peer or by
some dowager of quality. He is prouder of a court-livery than of a
laurel-wreath; and is only sure of having established his claims
to respectability by having sacrificed those of independence. He
is a retainer to the Muses; a door-keeper to learning; a lacquey
in the state. He believes that modern literature should wear the
fetters of classical antiquity; that truth is to be weighed in the
scales of opinion and prejudice; that power is equivalent to
right; that genius is dependent on rules; that taste and
refinement of language consist in \emph{word-catching}. Many
persons suppose that Mr. Gifford knows better than he pretends;
and that he is shrewd, artful, and designing.  But perhaps it may
be nearer the mark to suppose that his dulness is guarantee for
his sincerity; or that before he is the tool of the profligacy of
others, he is the dupe of his own jaundiced feelings, and narrow,
hoodwinked perceptions.
\begin{verse}
  \vleftofline{``}Destroy his fib or sophistry: in vain\textemdash
\\
The creature's at his dirty work again!''

\end{verse}
But this is less from choice or perversity, than because he cannot
help it and can do nothing else. He damns a beautiful expression
less out of spite than because he really does not understand it:
any novelty of thought or sentiment gives him a shock from which
he cannot recover for some time, and he naturally takes his
revenge for the alarm and uneasiness occasioned him, without
referring to venal or party motives.  He garbles an author's
meaning, not so much wilfully, as because it is a pain to him to
enlarge his microscopic view to take in the context, when a
particular sentence or passage has struck him as quaint and out of
the way: he fly-blows an author's style, and picks out detached
words and phrases for cynical reprobation, simply because he feels
himself at home, or takes a pride and pleasure in this sort of
petty warfare. He is tetchy and impatient of contradiction; sore
with wounded pride; angry at obvious faults, more angry at
unforeseen beauties. He has the \emph{chalk-stones} in his
understanding, and from being used to long confinement, cannot
bear the slightest jostling or irregularity of motion. He may call
out with the fellow in the \emph{Tempest}\textemdash "I am not
Stephano, but a cramp!'' He would go back to the standard of
opinions, style, the faded ornaments, and insipid formalities that
came into fashion about forty years ago. Flashes of thought,
flights of fancy, idiomatic expressions, he sets down among the
signs of the times\textemdash the extraordinary occurrences of the
age we live in. They are marks of a restless and revolutionary
spirit: they disturb his composure of mind, and threaten (by
implication) the safety of the state. His slow, snail-paced,
bed-rid habits of reasoning cannot keep up with the whirling,
eccentric motion, the rapid, perhaps extravagant combinations of
modern literature. He has long been stationary himself, and is
determined that others shall remain so. The hazarding a paradox is
like letting off a pistol close to his ear: he is alarmed and
offended. The using an elliptical mode of expression (such as he
did not use to find in Guides to the English Tongue) jars him like
coming suddenly to a step in a flight of stairs that you were not
aware of. He \emph{pishes} and \emph{pshaws} at all this,
exercises a sort of interjectional criticism on what excites his
spleen, his envy, or his wonder, and hurls his meagre anathemas
\emph{ex cathedr\^{a}} at all those writers who are indifferent
alike to his precepts and his example!

Mr. Gifford, in short, is possessed of that sort of learning which
is likely to result from an over-anxious desire to supply the want
of the first rudiments of education; that sort of wit, which is
the offspring of ill-humour or bodily pain; that sort of sense,
which arises from a spirit of contradiction and a disposition to
cavil at and dispute the opinions of others; and that sort of
reputation, which is the consequence of bowing to established
authority and ministerial influence. He dedicates to some great
man, and receives his compliments in return. He appeals to some
great name, and the Under-graduates of the two Universities look
up to him as an oracle of wisdom. He throws the weight of his
verbal criticism and puny discoveries in \emph{black-letter}
reading into the gap, that is supposed to be making in the
Constitution by Whigs and Radicals, whom he qualifies without
mercy as dunces and miscreants; and so entitles himself to the
protection of Church and State. The character of his mind is an
utter want of independence and magnanimity in all that he
attempts. He cannot go alone, he must have crutches, a go-cart and
trammels, or he is timid, fretful, and helpless as a child. He
cannot conceive of any thing different from what he finds it, and
hates those who pretend to a greater reach of intellect or
boldness of spirit than himself. He inclines, by a natural and
deliberate bias, to the traditional in laws and government; to the
orthodox in religion; to the safe in opinion; to the trite in
imagination; to the technical in style; to whatever implies a
surrender of individual judgment into the hands of authority, and
a subjection of individual feeling to mechanic rules. If he finds
any one flying in the face of these, or straggling from the beaten
path, he thinks he has them at a notable disadvantage, and falls
foul of them without loss of time, partly to soothe his own sense
of mortified self-consequence, and as an edifying spectacle to his
legitimate friends. He takes none but unfair advantages. He
\emph{twits} his adversaries (that is, those who are not in the
leading-strings of his school or party) with some personal or
accidental defect. If a writer has been punished for a political
libel, he is sure to hear of it in a literary criticism. If a lady
goes on crutches and is out of favour at court, she is reminded of
it in Mr.  Gilford's manly satire. He sneers at people of low
birth or who have not had a college-education, partly to hide his
own want of certain advantages, partly as well-timed flattery to
those who possess them. He has a right to laugh at poor,
unfriended, untitled genius from wearing the livery of rank and
letters, as footmen behind a coronet-coach laugh at the rabble. He
keeps good company, and forgets himself. He stands at the door of
Mr. Murray's shop, and will not let any body pass but the
well-dressed mob, or some followers of the court. To edge into the
\emph{Quarterly} Temple of Fame the candidate must have a diploma
from the Universities, a passport from the Treasury. Otherwise, it
is a breach of etiquette to let him pass, an insult to the better
sort who aspire to the love of letters\textemdash and may chance
to drop in to the \emph{Feast of the Poets}. Or, if he cannot
manage it thus, or get rid of the claim on the bare ground of
poverty or want of school-learning, he \emph{trumps} up an excuse
for the occasion, such as that ``a man was confined in Newgate a
short time before''\textemdash it is not a \emph{lie} on the part
of the critic, it is only an amiable subserviency to the will of
his betters, like that of a menial who is ordered to deny his
master, a sense of propriety, a knowledge of the world, a poetical
and moral license. Such fellows (such is his cue from his
employers) should at any rate be kept out of privileged places:
persons who have been convicted of prose-libels ought not to be
suffered to write poetry\textemdash if the fact was not exactly as
it was stated, it was something of the kind, or it \emph{ought} to
have been so, the assertion was a pious fraud,\textemdash the
public, the court, the prince himself might read the work, but for
this mark of opprobrium set upon it\textemdash it was not to be
endured that an insolent plebeian should aspire to elegance,
taste, fancy\textemdash it was throwing down the barriers which
ought to separate the higher and the lower classes, the loyal and
the disloyal\textemdash the paraphrase of the story of Dante was
therefore to perform quarantine, it was to seem not yet recovered
from the gaol infection, there was to be a taint upon it, as there
was none in it\textemdash and all this was performed by a single
slip of Mr. Gifford's pen! We would willingly believe (if we
could) that in this case there was as much weakness and prejudice
as there was malice and cunning.\textemdash Again, we do not think
it possible that under any circumstances the writer of the
\emph{Verses to Anna} could enter into the spirit or delicacy of
Mr. Keats's poetry. The fate of the latter somewhat resembled that
of
\begin{verse}
  \pcdash{2} ``a bud bit by an envious worm, \\
  Ere it could spread its sweet leaves to the air, \\
  Or dedicate its beauty to the sun.''

\end{verse}
Mr. Keats's ostensible crime was that he had been praised in the
\emph{Examiner Newspaper}: a greater and more unpardonable offence
probably was, that he was a true poet, with all the errors and
beauties of youthful genius to answer for. Mr. Gifford was as
insensible to the one as he was inexorable to the other. Let the
reader judge from the two subjoined specimens how far the one
writer could ever, without a presumption equalled only by a want
of self-knowledge, set himself in judgment on the other.
\begin{verse}
  \vleftofline{``}Out went the taper as she hurried in;\\
  Its little smoke in pallid moonshine died: \\
  She closed the door, she panted, all akin \\
  To spirits of the air and visions wide: \\
  No utter'd syllable, or woe betide!  \\
  But to her heart, her heart was voluble, \\
  Paining with eloquence her balmy side; \\
  As though a tongueless nightingale should swell \\
  Her heart in vain, and die, heart-stifled, in her dell.\\
\vspace{\stanzaskip}
\vleftofline{``}A casement high and triple-arch'd there was, \\
  All garlanded with carven imag'ries \\
  Of fruits, and flowers, and bunches of knot-grass, \\
  And diamonded with panes of quaint device,\\
  Innumerable of stains and splendid dyes, \\
  As are the tiger-moth's deep-damask'd wings; \\
  And in the midst, 'mong thousand heraldries, \\
  And twilight saints and dim emblazonings, \\
  A shielded scutcheon blush'd with blood of queens and kings.\\
\vspace{\stanzaskip}
  \vleftofline{``}Full on this casement shone the wintry moon, \\
  And threw warm gules on Madeline's fair breast, \\
  As down she knelt for Heaven's grace and boon; \\
  Rose-bloom fell on her hands, together prest,\\
  And on her silver cross soft amethyst, \\
  And on her hair a glory, like a Saint: \\
  She seem'd a splendid angel, newly drest, \\
  Save wings, for heaven:\textemdash \\
  Porphyro grew faint: \\
  She knelt, so pure a thing, so free from mortal taint.\\
\vspace{\stanzaskip}
  \vleftofline{``}Anon his heart revives: her vespers done, \\
  Of all its wreathed pearls her hair she frees; \\
  Unclasps her warmed jewels one by one; \\
  Loosens her fragrant boddice; by degrees \\
  Her rich attire creeps rustling to her knees: \\
  Half-hidden, like a mermaid in sea-weed, \\
  Pensive awhile she dreams awake, and sees, \\
  In fancy, fair St. Agnes in her bed, \\
  But dares not look behind, or all the charm is fled.\\
\vspace{\stanzaskip}
\vleftofline{``}Soon trembling in her soft and chilly nest, \\
In sort of wakeful swoon, perplex'd she lay, \\
Until the poppied warmth of sleep oppress'd \\
Her soothed limbs, and soul fatigued away \\
Flown, like a thought, until the morrow-day: \\
Blissfully haven'd both from joy and pain; \\
Clasp'd like a missal where swart Paynims pray;\\
Blinded alike from sunshine and from rain, \\
As though a rose should shut, and be a bud again.''
\sourceatright{\textsc{Eve Of St. Agnes}\hspace{3pc}.}
\end{verse}

With the rich beauties and the dim obscurities of lines like these, let
us contrast the Verses addressed \emph{To a Tuft of early Violets} by the
fastidious author of the Baviad and Mæviad.\textemdash 
\begin{verse}
  \begin{altverse}
    \vleftofline{``}Sweet flowers! that from your humble beds \\
    Thus prematurely dare to rise, \\
    And trust your unprotected heads \\
    To cold Aquarius' watery skies.
  \end{altverse}

  \begin{altverse}
    \vleftofline{``}Retire, retire! \emph{These} tepid airs \\
    Are not the genial brood of May; \\
    \emph{That} sun with light malignant glares, \\
    And flatters only to betray.
  \end{altverse}

  \begin{altverse}
    \vleftofline{``}Stern Winter's reign is not yet past\textemdash \\
    Lo! while your buds prepare to blow, \\
    On icy pinions comes the blast, \\
    And nips your root, and lays you low.
  \end{altverse}

  \begin{altverse}
    \vleftofline{``}Alas, for such ungentle doom!  \\
    But I will shield you; and supply \\
    A kindlier soil on which to bloom, \\
    A nobler bed on which to die.
  \end{altverse}

  \begin{altverse}
    \vleftofline{``}Come then\textemdash 'ere yet the morning ray
  \\
  Has drunk the dew that gems your crest, \\
  And drawn your balmiest sweets away; \\
  O come and grace my Anna's breast.
  \end{altverse}

  \begin{altverse}
    \vleftofline{``}Ye droop, fond flowers!But did ye know \\
    What worth, what goodness there reside, \\
    Your cups with liveliest tints would glow; \\
    And spread their leaves with conscious pride.
  \end{altverse}

  \begin{altverse}
    \vleftofline{``}For there has liberal Nature joined \\
    Her riches to the stores of Art, \\
    And added to the vigorous mind \\
    The soft, the sympathising heart.
  \end{altverse}

  \begin{altverse}
    \vleftofline{``}Come, then\textemdash 'ere yet the morning ray \\
    Has drunk the dew that gems your crest, \\
    And drawn your balmiest sweets away; \\
    O come and grace my Anna's breast.
  \end{altverse}

  \begin{altverse}
    \vleftofline{``}O! I should think\textemdash \emph{that fragrant
      bed} \\
    \emph{Might but hope with you to share}\textemdash\\
\footnote{What an awkward bed-fellow for a tuft of violets!}
    Years of anxiety repaid \\
    By one short hour of transport there.
  \end{altverse}

  \begin{altverse}
    \vleftofline{``}More blest than me, thus shall ye live \\
    Your little day; and when ye die, \\
    Sweet flowers! the grateful Muse shall give \\
    A verse; the sorrowing maid, a sigh.
  \end{altverse}

  \begin{altverse}
    \vleftofline{``}While I alas! no distant date, \\
    Mix with the dust from whence I came, \\
    Without a friend to weep my fate, \\
    Without a stone to tell my name.''
  \end{altverse}
\end{verse}

% separate next LONG! footmark using rule
\renewcommand*{\footnoterule}{%
\centering \rule[0.25\baselineskip]{\textwidth}{0.65pt}
}

We subjoin one more specimen of these ``wild strains"\footnote{
\begin{verse}
  \vleftofline{``}How oft, O Dart! what time the faithful pair \\
  Walk'd forth, the fragrant hour of eve to share, \\
  On thy romantic banks, have my \emph{wild strains} \\
  (Not yet forgot amidst my native plains)\\
  While thou hast sweetly gurgled down the vale.  \\
  Filled up the pause of love's delightful tale!  \\
  While, ever as she read, the conscious maid, \\
  By faultering voice and downcast looks betray'd, \\
  Would blushing on her lover's neck recline, \\
  And with her finger\textemdash point the tenderest line!''
\end{verse}
\attrib{\emph{Mæviad}, pp. 194, 202.}

Yet the author assures us just before, that in these ``wild strains'' ``all
was plain.''
\begin{verse}
  \vleftofline{``}Even then (admire, John Bell! my simple ways) \\
  No heaven and hell danced madly through my lays, \\
  No oaths, no execrations; \emph{all was plain}; \\
  Yet trust me, while thy ever jingling train \\
  Chime their sonorous woes with frigid art, \\
  And shock the reason and revolt the heart; \\
  My hopes and fears, in nature's language drest, \\
  Awakened love in many a gentle breast.''
\end{verse}
\attrib{\emph{Ibid.} v. 185-92.}

If any one else had composed these ``wild strains,'' in which ``all is
plain,'' Mr. Gifford would have accused them of three things, ``1.
Downright nonsense. 2. Downright frigidity. 3. Downright doggrel;'' and
proceeded to anatomise them very cordially in his way. As it is, he is
thrilled with a very pleasing horror at his former scenes of tenderness,
and ``gasps at the recollection'' \emph{of watery Aquarius}!
\emph{he! jam satis est!} ``Why rack a grub\textemdash a butterfly upon a wheel?''} said to be
``\emph{Written two years after the preceding}.'' \textsc{Ecce Iterum Crispinus}.

\begin{verse}
  \begin{altverse}
    \vleftofline{``}I wish I was where Anna lies; \\
    For I am sick of lingering here, \\
    And every hour Affection cries, \\
    Go, and partake her humble bier.
  \end{altverse}

  \begin{altverse}
    \vleftofline{``}I wish I could! for when she died \\
    I lost my all; and life has prov'd \\
    Since that sad hour a dreary void, \\
    A waste unlovely and unlov'd.
  \end{altverse}

  \begin{altverse}
    \vleftofline{``}But who, when I am turn'd to clay, \\
    Shall duly to her grave repair, \\
    And pluck the ragged moss away, \\
    And weeds that have ``no business there?''
  \end{altverse}

  \begin{altverse} 
    \vleftofline{``}And who, with pious hand, shall bring \\
    The flowers she cherish'd, snow-drops cold, \\
    And violets that unheeded spring, \\
    To scatter o'er her hallow'd mould?
  \end{altverse}

  \begin{altverse}
    \vleftofline{``}And who, while Memory loves to dwell \\
    Upon her name for ever dear, \\
    Shall feel his heart with passion swell,\\
    And pour the bitter, bitter tear?
  \end{altverse}

  \begin{altverse}
    \vleftofline{``}I did it; and would fate allow, \\
    Should visit still, should still deplore\textemdash \\
    But health and strength have left me now, \\
    But I, alas! can weep no more.
  \end{altverse}

  \begin{altverse}
    \vleftofline{``}Take then, sweet maid! this simple strain, \\
    The last I offer at thy shrine; \\
    Thy grave must then undeck'd remain, \\
    And all thy memory fade with mine.
  \end{altverse}

  \begin{altverse}
    \vleftofline{``}And can thy soft persuasive look, \\
    That voice that might with music vie, \\
    Thy air that every gazer took, \\
    Thy matchless eloquence of eye,
  \end{altverse}

  \begin{altverse}
    \vleftofline{``}Thy spirits, frolicsome as good, \\
    Thy courage, by no ills dismay'd, \\
    Thy patience, by no wrongs subdued, \\
    Thy gay good-humour\textemdash can they ``fade?''
  \end{altverse}

  \begin{altverse}
    \vleftofline{``}Perhaps\textemdash but sorrow dims my eye: \\
    Cold turf, which I no more must view, \\
    Dear name, which I no more must sigh, \\
    A long, a last, a sad adieu!''
  \end{altverse}
\end{verse}
It may be said in extenuation of the low, mechanic vein of these
impoverished lines, that they were written at an early
age\textemdash they were the inspired production of a youthful
lover! Mr. Gifford was thirty when he wrote them, Mr. Keats died
when he was scarce twenty! Farther it may be said, that
Mr. Gifford hazarded his first poetical attempts under all the
disadvantages of a neglected education: but the same circumstance,
together with a few unpruned redundancies of fancy and
quaintnesses of expression, was made the plea on which Mr. Keats
was hooted out of the world, and his fine talents and wounded
sensibilities consigned to an early grave. In short, the treatment
of this heedless candidate for poetical fame might serve as a
warning, and was intended to serve as a warning to all unfledged
tyros, how they venture upon any such doubtful experiments, except
under the auspices of some lord of the bedchamber or Government
Aristarchus, and how they imprudently associate themselves with
men of mere popular talent or independence of feeling!\textemdash
It is the same in prose works. The Editor scorns to enter the
lists of argument with any proscribed writer of the opposite
party. He does not refute, but denounces him. He makes no
concessions to an adversary, lest they should in some way be
turned against him. He only feels himself safe in the fancied
insignificance of others: he only feels himself superior to those
whom he stigmatizes as the lowest of mankind. All persons are
without common-sense and honesty who do not believe implicitly
(with him) in the immaculateness of Ministers and the divine
origin of Kings.  Thus he informed the world that the author of
\textsc{Table-Talk} was a person who could not write a sentence of
common English and could hardly spell his own name, because he was
not a friend to the restoration of the Bourbons, and had the
assurance to write \emph{Characters of Shakespears Plays} in a
style of criticism somewhat different from Mr. Gifford's. He
charged this writer with imposing on the public by a flowery
style; and when the latter ventured to refer to a work of his,
called \emph{An Essay on the Principles of Human Action}, which
has not a single ornament in it, as a specimen of his original
studies and the proper bias of his mind, the learned critic, with
a shrug of great self-satisfaction, said, ``It was amusing to see
this person, sitting like one of Brouwer's Dutch boors over his
gin and tobacco-pipes, and fancying himself a Leibnitz!''  The
question was, whether the subject of Mr. Gifford's censure had
ever written such a work or not; for if he had, he had amused
himself with something besides gin and tobacco-pipes. But our
Editor, by virtue of the situation he holds, is superior to facts
or arguments: he is accountable neither to the public nor to
authors for what he says of them, but owes it to his employers to
prejudice the work and vilify the writer, if the latter is not
avowedly ready to range himself on the stronger side.\textemdash
The \emph{Quarterly Review}, besides the political \emph{tirades}
and denunciations of suspected writers, intended for the guidance
of the heads of families, is filled up with accounts of books of
Voyages and Travels for the amusement of the younger branches. The
poetical department is almost a sinecure, consisting of mere
summary decisions and a list of quotations. Mr. Croker is
understood to contribute the St. Helena articles and the
liberality, Mr. Canning the practical good sense, Mr. D'Israeli
the good-nature, Mr. Jacob the modesty, Mr. Southey the
consistency, and the Editor himself the chivalrous spirit and the
attacks on Lady Morgan. It is a double crime, and excites a double
portion of spleen in the Editor, when female writers are not
advocates of passive obedience and non-resistance. This Journal,
then, is a depository for every species of political sophistry and
personal calumny. There is no abuse or corruption that does not
there find a jesuitical palliation or a bare-faced
vindication. There we meet the slime of hypocrisy, the varnish of
courts, the cant of pedantry, the cobwebs of the law, the iron
hand of power. Its object is as mischievous as the means by which
it is pursued are odious. The intention is to poison the sources
of public opinion and of individual fame\textemdash to pervert
literature, from being the natural ally of freedom and humanity,
into an engine of priestcraft and despotism, and to undermine the
spirit of the English Constitution and the independence of the
English character.  The Editor and his friends systematically
explode every principle of liberty, laugh patriotism and public
spirit to scorn, resent every pretence to integrity as a piece of
singularity or insolence, and strike at the root of all free
inquiry or discussion, by running down every writer as a vile
scribbler and a bad member of society, who is not a hireling and a
slave. No means are stuck at in accomplishing this laudable
end. Strong in patronage, they trample on truth, justice, and
decency. They claim the privilege of court-favourites. They keep
as little faith with the public, as with their opponents. No
statement in the \emph{Quarterly Review} is to be trusted: there
is no fact that is not misrepresented in it, no quotation that is
not garbled, no character that is not slandered, if it can answer
the purposes of a party to do so. The weight of power, of wealth,
of rank is thrown into the scale, gives its impulse to the
machine; and the whole is under the guidance of Mr. Gifford's
instinctive genius\textemdash of the inborn hatred of servility
for independence, of dulness for talent, of cunning and impudence
for truth and honesty. It costs him no effort to execute his
disreputable task\textemdash in being the tool of a crooked
policy, he but labours in his natural vocation. He patches up a
rotten system as he would supply the chasms in a worm-eaten
manuscript, from a grovelling incapacity to do any thing better;
thinks that if a single iota in the claims of prerogative and
power were lost, the whole fabric of society would fall upon his
head and crush him; and calculates that his best chance for
literary reputation is by \emph{black-balling} one half of the
competitors as Jacobins and levellers, and securing the suffrages
of the other half in his favour as a loyal subject and trusty
partisan!

\renewcommand*{\footnoterule}{}

Mr. Gifford, as a satirist, is violent and abrupt. He takes
obvious or physical defects, and dwells upon them with much labour
and harshness of invective, but with very little wit or spirit. He
expresses a great deal of anger and contempt, but you cannot tell
very well why\textemdash except that he seems to be sore and out
of humour. His satire is mere peevishness and spleen, or something
worse\textemdash personal antipathy and rancour. We are in quite
as much pain for the writer, as for the object of his resentment.
His address to Peter Pindar is laughable from its
outrageousness. He denounces him as a wretch hateful to God and
man, for some of the most harmless and amusing trifles that ever
were written\textemdash and the very good- humour and pleasantry
of which, we suspect, constituted their offence in the eyes of
this Drawcansir.\textemdash His attacks on Mrs. Robinson were
unmanly, and even those on Mr. Merry and the Della-Cruscan School
were much more ferocious than the occasion warranted. A little
affectation and quaintness of style did not merit such severity of
castigation.\footnote{Mr. Merry was even with our author in
  personality of abuse.  See his Lines on the Story of the Ape
  that was given in charge to the ex-tutor.} As a translator,
Mr. Gifford's version of the Roman satirist is the baldest, and,
in parts, the most offensive of all others. We do not know why he
attempted it, unless he had got it in his head that he should thus
follow in the steps of Dryden, as he had already done in those of
Pope in the Baviad and Maeviad. As an editor of old authors,
Mr. Gifford is entitled to considerable praise for the pains he
has taken in revising the text, and for some improvements he has
introduced into it. He had better have spared the notes, in which,
though he has detected the blunders of previous commentators, he
has exposed his own ill-temper and narrowness of feeling more. As
a critic, he has thrown no light on the character and spirit of
his authors. He has shewn no striking power of analysis nor of
original illustration, though he has chosen to exercise his pen on
writers most congenial to his own turn of mind, from their dry and
caustic vein; Massinger, and Ben Jonson. What he will make of
Marlowe, it is difficult to guess. He has none of ``the fiery
quality'' of the poet. Mr. Gifford does not take for his motto on
these occasions\textemdash \emph{Spiritus precipitandus
  est!}\textemdash His most successful efforts in this way are
barely respectable. In general, his observations are petty,
ill-concocted, and discover as little \emph{tact}, as they do a
habit of connected reasoning. Thus, for instance, in attempting to
add the name of Massinger to the list of Catholic poets, our
minute critic insists on the profusion of crucifixes, glories,
angelic visions, garlands of roses, and clouds of incense
scattered through the \emph{Virgin-Martyr,} as evidence of the
theological sentiments meant to be inculcated by the play, when
the least reflection might have taught him, that they proved
nothing but the author's poetical conception of the character and
\emph{costume} of his subject. A writer might, with the same
sinister, short-sighted shrewdness, be accused of Heathenism for
talking of Flora and Ceres in a poem on the Seasons! What are
produced as the exclusive badges and occult proofs of Catholic
bigotry, are nothing but the adventitious ornaments and external
symbols, the gross and sensible language, in a word, the
\emph{poetry} of Christianity in general. What indeed shews the
frivolousness of the whole inference is that Deckar, who is
asserted by our critic to have contributed some of the most
passionate and fantastic of these devotional scenes, is not even
suspected of a leaning to Popery. In like manner, he excuses
Massinger for the grossness of one of his plots (that of the
\emph{Unnatural Combat}) by saying that it was supposed to take
place before the Christian era; by this shallow common-place
persuading himself, or fancying he could persuade others, that the
crime in question (which yet on the very face of the story is made
the ground of a tragic catastrophe) was first made
\emph{statutory} by the Christian religion.

The foregoing is a harsh criticism, and may be thought
illiberal. But as Mr. Gifford assumes a right to say what he
pleases of others\textemdash they may be allowed to speak the
truth of him!
