\chapter[William Godwin]{william godwin} 

The Spirit of the Age was never more fully-shewn than in its
treatment of this writer\textemdash its love of paradox and
change, its dastard submission to prejudice and to the fashion of
the day. Five-and-twenty years ago he was in the very zenith of a
sultry and unwholesome popularity; he blazed as a sun in the
firmament of reputation; no one was more talked of, more looked up
to, more sought after, and wherever liberty, truth, justice was
the theme, his name was not far off:\textemdash now he has sunk
below the horizon, and enjoys the serene twilight of a doubtful
immortality. Mr.  Godwin, during his lifetime, has secured to
himself the triumphs and the mortifications of an extreme
notoriety and of a sort of posthumous fame.

His bark, after being tossed in the revolutionary tempest, now
raised to heaven by all the fury of popular breath, now almost
dashed in pieces, and buried in the quicksands of ignorance, or
scorched with the lightning of momentary indignation, at length
floats on the calm wave that is to bear it down the stream of
time. Mr. Godwin's person is not known, he is not pointed out in
the street, his conversation is not courted, his opinions are not
asked, he is at the head of no cabal, he belongs to no party in
the State, he has no train of admirers, no one thinks it worth his
while even to traduce and vilify him, he has scarcely friend or
foe, the world make a point (as Goldsmith used to say) of taking
no more notice of him than if such an individual had never
existed; he is to all ordinary intents and purposes dead and
buried; but the author of \emph{Political Justice} and of
\emph{Caleb Williams} can never die, his name is an abstraction in
letters, his works are standard in the history of intellect. He is
thought of now like any eminent writer a hundred-and-fifty years
ago, or just as he will be a hundred-and-fifty years hence. He
knows this, and smiles in silent mockery of himself, reposing on
the monument of his fame\textemdash
\begin{quote}
  ``Sedet, in eternumque sedebit infelix Theseus.''
\end{quote}
No work in our time gave such a blow to the
philosophical mind of the country as the celebrated \emph{Enquiry
concerning Political Justice}. Tom Paine was considered for the
time as a Tom Fool to him; Paley an old woman; Edmund Burke a
flashy sophist. Truth, moral truth, it was supposed, had here
taken up its abode; and these were the oracles of thought. ``Throw
aside your books of chemistry,'' said Wordsworth to a young man, a
student in the Temple, ``and read Godwin on Necessity.'' Sad
necessity! Fatal reverse! Is truth then so variable? Is it one
thing at twenty, and another at forty? Is it at a burning heat in
1793, and below \emph{zero} in 1814? Not so, in the name of
manhood and of common sense! Let us pause here a
little.\textemdash Mr. Godwin indulged in extreme opinions, and
carried with him all the most sanguine and fearless understandings
of the time. What then? Because those opinions were overcharged,
were they therefore altogether groundless? Is the very God of our
idolatry all of a sudden to become an abomination and an anathema?
Could so many young men of talent, of education, and of principle
have been hurried away by what had neither truth, nor nature, not
one particle of honest feeling nor the least shew of reason in it?
Is the \emph{Modern Philosophy} (as it has been called) at one
moment a youthful bride, and the next a withered beldame, like the
false Duessa in Spenser? Or is the vaunted edifice of Reason, like
his House of Pride, gorgeous in front, and dazzling to approach,
while ``its hinder parts are ruinous, decayed, and old?'' Has the
main prop, which supported the mighty fabric, been shaken and
given way under the strong grasp of some Samson; or has it not
rather been undermined by rats and vermin? At one time, it almost
seemed, that ``if this failed,
\begin{quote}
  The pillar'd firmament was rottenness,
  \\ And earth's base built of stubble:''
\end{quote}
now scarce a shadow of it remains, it is crumbled to
dust, nor is it even talked of! ``What then, went ye forth for to
see, a reed shaken with the wind?'' Was it for this that our young
gownsmen of the greatest expectation and promise, versed in
classic lore, steeped in dialectics, armed at all points for the
foe, well read, well nurtured, well provided for, left the
University and the prospect of lawn sleeves, tearing asunder the
shackles of the free born spirit, and the cobwebs of
school-divinity, to throw themselves at the feet of the new
Gamaliel, and learn wisdom from him? Was it for this, that
students at the bar, acute, inquisitive, sceptical (here only wild
enthusiasts) neglected for a while the paths of preferment and the
law as too narrow, tortuous, and unseemly to bear the pure and
broad light of reason? Was it for this, that students in medicine
missed their way to Lecturerships and the top of their profession,
deeming lightly of the health of the body, and dreaming only of
the renovation of society and the march of mind? Was it to this
that Mr. Southey's \emph{Inscriptions} pointed? to this that Mr.
Coleridge's \emph{Religious Musings} tended? Was it for this, that
Mr. Godwin himself sat with arms folded, and, ``like Cato, gave
his little senate laws?'' Or rather, like another Prospero,
uttered syllables that with their enchanted breath were to change
the world, and might almost stop the stars in their courses? Oh!
and is all forgot? Is this sun of intellect blotted from the sky?
Or has it suffered total eclipse? Or is it we who make the fancied
gloom, by looking at it through the paltry, broken, stained
fragments of our own interests and prejudices? Were we fools then,
or are we dishonest now? Or was the impulse of the mind less
likely to be true and sound when it arose from high thought and
warm feeling, than afterwards, when it was warped and debased by
the example, the vices, and follies of the world?

The fault, then, of Mr. Godwin's philosophy, in one word, was too
much ambition\textemdash ``by that sin fell the angels!'' He
conceived too nobly of his fellows (the most unpardonable crime
against them, for there is nothing that annoys our self-love so
much as being complimented on imaginary achievements, to which we
are wholly unequal)\textemdash he raised the standard of morality
above the reach of humanity, and by directing virtue to the most
airy and romantic heights, made her path dangerous, solitary, and
impracticable. The author of the \emph{Political Justice} took
abstract reason for the rule of conduct, and abstract good for its
end. He places the human mind on an elevation, from which it
commands a view of the whole line of moral consequences; and
requires it to conform its acts to the larger and more enlightened
conscience which it has thus acquired.  He absolves man from the
gross and narrow ties of sense, custom, authority, private and
local attachment, in order that he may devote himself to the
boundless pursuit of universal benevolence. Mr. Godwin gives no
quarter to the amiable weaknesses of our nature, nor does he stoop
to avail himself of the supplementary aids of an imperfect virtue.
Gratitude, promises, friendship, family affection give way, not
that they may be merged in the opposite vices or in want of
principle; but that the void may be filled up by the disinterested
love of good, and the dictates of inflexible justice, which is
``the law of laws, and sovereign of sovereigns.'' All minor
considerations yield, in his system, to the stern sense of duty,
as they do, in the ordinary and established ones, to the voice of
necessity. Mr. Godwin's theory and that of more approved reasoners
differ only in this, that what are with them the exceptions, the
extreme cases, he makes the every-day rule. No one denies that on
great occasions, in moments of fearful excitement, or when a
mighty object is at stake, the lesser and merely instrumental
points of duty are to be sacrificed without remorse at the shrine
of patriotism, of honour, and of conscience. But the disciple of
the \emph{New School} (no wonder it found so many impugners, even
in its own bosom!)  is to be always the hero of duty; the law to
which he has bound himself never swerves nor relaxes; his feeling
of what is right is to be at all times wrought up to a pitch of
enthusiastic self-devotion; he must become the unshrinking martyr
and confessor of the public good. If it be said that this scheme
is chimerical and impracticable on ordinary occasions, and to the
generality of mankind, well and good; but those who accuse the
author of having trampled on the common feelings and prejudices of
mankind in wantonness or insult, or without wishing to substitute
something better (and only unattainable, because it is better) in
their stead, accuse him wrongfully. We may not be able to launch
the bark of our affections on the ocean-tide of humanity, we may
be forced to paddle along its shores, or shelter in its creeks and
rivulets: but we have no right to reproach the bold and
adventurous pilot, who dared us to tempt the uncertain abyss, with
our own want of courage or of skill, or with the jealousies and
impatience, which deter us from undertaking, or might prevent us
from accomplishing the voyage!

The \emph{Enquiry concerning Political Justice} (it was urged by
its favourers and defenders at the time, and may still be so,
without either profaneness or levity) is a metaphysical and
logical commentary on some of the most beautiful and striking
texts of Scripture. Mr. Godwin is a mixture of the Stoic and of
the Christian philosopher. To break the force of the vulgar
objections and outcry that have been raised against the Modern
Philosophy, as if it were a new and monstrous birth in morals, it
may be worth noticing, that volumes of sermons have been written
to excuse the founder of Christianity for not including friendship
and private affection among its golden rules, but rather excluding
them.\footnote{Shaftesbury made this an objection to Christianity,
which was answered by Foster, Leland, and other eminent divines,
on the ground that Christianity had a higher object in view,
namely, general philanthropy.} Moreover, the answer to the
question, ``Who is thy neighbour?'' added to the divine precept,
``Thou shalt love thy neighbour as thyself,'' is the same as in
the exploded pages of our author,\textemdash ``He to whom we can
do most good.'' In determining this point, we were not to be
influenced by any extrinsic or collateral considerations, by our
own predilections, or the expectations of others, by our
obligations to them or any services they might be able to render
us, by the climate they were born in, by the house they lived in,
by rank or religion, or party, or personal ties, but by the
abstract merits, the pure and unbiassed justice of the case. The
artificial helps and checks to moral conduct were set aside as
spurious and unnecessary, and we came at once to the grand and
simple question\textemdash ``In what manner we could best
contribute to the greatest possible good?'' This was the paramount
obligation in all cases whatever, from which we had no right to
free ourselves upon any idle or formal pretext, and of which each
person was to judge for himself, under the infallible authority of
his own opinion and the inviolable sanction of his
self-approbation. ``There was the rub that made \emph{philosophy}
of so short life!'' Mr. Godwin's definition of morals was the same
as the admired one of law, \emph{reason without passion}; but with
the unlimited scope of private opinion, and in a boundless field
of speculation (for nothing less would satisfy the pretensions of
the New School), there was danger that the unseasoned novice might
substitute some pragmatical conceit of his own for the rule of
right reason, and mistake a heartless indifference for a
superiority to more natural and generous feelings. Our ardent and
dauntless reformer followed out the moral of the parable of the
Good Samaritan into its most rigid and repulsive consequences with
a pen of steel, and let fall his ``trenchant blade'' on every
vulnerable point of human infirmity; but there is a want in his
system of the mild and persuasive tone of the Gospel, where ``all
is conscience and tender heart.'' Man was indeed screwed up, by
mood and figure, into a logical machine, that was to forward the
public good with the utmost punctuality and effect, and it might
go very well on smooth ground and under favourable circumstances;
but would it work up-hill or \emph{against the grain}? It was to
be feared that the proud Temple of Reason, which at a distance and
in stately supposition shone like the palaces of the New
Jerusalem, might (when placed on actual ground) be broken up into
the sordid styes of sensuality, and the petty huckster's shops of
self-interest! Every man (it was proposed\textemdash ``so ran the
tenour of the bond'') was to be a Regulus, a Codrus, a Cato, or a
Brutus\textemdash every woman a Mother of the Gracchi.
\begin{quotation} % 6 dashes
  `` \pcdash{3.5}It was well said,\\
  And 'tis a kind of good deed to say well.''
\end{quotation} But heroes on paper might degenerate into
vagabonds in practice, Corinnas into courtezans. Thus a refined
and permanent individual attachment is intended to supply the
place and avoid the inconveniences of marriage; but vows of
eternal constancy, without church security, are found to be
fragile. A member of the \emph{ideal} and perfect commonwealth of
letters lends another a hundred pounds for immediate and pressing
use; and when he applies for it again, the borrower has still more
need of it than he, and retains it for his own especial, which is
tantamount to the public good. The Exchequer of pure reason, like
that of the State, never refunds. The political as well as the
religious fanatic appeals from the over-weening opinion and claims
of others to the highest and most impartial tribunal, namely, his
own breast. Two persons agree to live together in Chambers on
principles of pure equality and mutual assistance\textemdash but
when it comes to the push, one of them finds that the other always
insists on his fetching water from the pump in Hare-court, and
cleaning his shoes for him. A modest assurance was not the least
indispensable virtue in the new perfectibility code; and it was
hence discovered to be a scheme, like other schemes where there
are all prizes and no blanks, for the accommodation of the
enterprizing and cunning, at the expence of the credulous and
honest. This broke up the system, and left no good odour behind
it! Reason has become a sort of bye-word, and philosophy has
``fallen first into a fasting, then into a sadness, then into a
decline, and last, into the dissolution of which we all
complain!'' This is a worse error than the former: we may be said
to have ``lost the immortal part of ourselves, and what remains is
beastly!''  The point of view from which this matter may be fairly
considered, is two-fold, and may be stated thus:\textemdash In the
first place, it by no means follows, because reason is found not
to be the only infallible or safe rule of conduct, that it is no
rule at all; or that we are to discard it altogether with derision
and ignominy. On the contrary, if not the sole, it is the
principal ground of action; it is ``the guide, the stay and anchor
of our purest thoughts, and soul of all our moral being.'' In
proportion as we strengthen and expand this principle, and bring
our affections and subordinate, but perhaps more powerful motives
of action into harmony with it, it will not admit of a doubt that
we advance to the goal of perfection, and answer the ends of our
creation, those ends which not only morality enjoins, but which
religion sanctions. If with the utmost stretch of reason, man
cannot (as some seemed inclined to suppose) soar up to the God,
and quit the ground of human frailty, yet, stripped wholly of it,
he sinks at once into the brute. If it cannot stand alone, in its
naked simplicity, but requires other props to buttress it up, or
ornaments to set it off; yet without it the moral structure would
fall flat and dishonoured to the ground. Private reason is that
which raises the individual above his mere animal instincts,
appetites and passions: public reason in its gradual progress
separates the savage from the civilized state. Without the one,
men would resemble wild beasts in their dens; without the other,
they would be speedily converted into hordes of barbarians or
banditti. Sir Walter Scott, in his zeal to restore the spirit of
loyalty, of passive obedience and non-resistance as an
acknowledgment for his having been created a Baronet by a Prince
of the House of Brunswick, may think it a fine thing to return in
imagination to the good old times, ``when in Auvergne alone, there
were three hundred nobles whose most ordinary actions were
robbery, rape, and murder,'' when the castle of each Norman baron
was a strong hold from which the lordly proprietor issued to
oppress and plunder the neighbouring districts, and when the Saxon
peasantry were treated by their gay and gallant tyrants as a herd
of loathsome swine\textemdash but for our own parts we beg to be
excused; we had rather live in the same age with the author of
Waverley and Blackwood's Magazine.  Reason is the meter and
alnager in civil intercourse, by which each person's upstart and
contradictory pretensions are weighed and approved or found
wanting, and without which it could not subsist, any more than
traffic or the exchange of commodities could be carried on without
weights and measures. It is the medium of knowledge, and the
polisher of manners, by creating common interests and ideas. Or in
the words of a contemporary writer, ``Reason is the queen of the
moral world, the soul of the universe, the lamp of human life, the
pillar of society, the foundation of law, the beacon of nations,
the golden chain let down from heaven, which links all accountable
and all intelligent natures in one common system\textemdash and in
the vain strife between fanatic innovation and fanatic prejudice,
we are exhorted to dethrone this queen of the world, to blot out
this light of the mind, to deface this fair column, to break in
pieces this golden chain! We are to discard and throw from us with
loud taunts and bitter execrations that reason, which has been the
lofty theme of the philosopher, the poet, the moralist, and the
divine, whose name was not first named to be abused by the
enthusiasts of the French Revolution, or to be blasphemed by the
madder enthusiasts, the advocates of Divine Right, but which is
coeval with, and inseparable from the nature and faculties of
man\textemdash is the image of his Maker stamped upon him at his
birth, the understanding breathed into him with the breath of
life, and in the participation and improvement of which alone he
is raised above the brute creation and his own physical
nature!''\textemdash The overstrained and ridiculous pretensions
of monks and ascetics were never thought to justify a return to
unbridled licence of manners, or the throwing aside of all
decency. The hypocrisy, cruelty, and fanaticism, often attendant
on peculiar professions of sanctity, have not banished the name of
religion from the world. Neither can ``the unreasonableness of the
reason'' of some modern sciolists ``so unreason our reason,'' as
to debar us of the benefit of this principle in future, or to
disfranchise us of the highest privilege of our nature. In the
second place, if it is admitted that Reason alone is not the sole
and self-sufficient ground of morals, it is to Mr. Godwin that we
are indebted for having settled the point. No one denied or
distrusted this principle (before his time) as the absolute judge
and interpreter in all questions of difficulty; and if this is no
longer the case, it is because he has taken this principle, and
followed it into its remotest consequences with more keenness of
eye and steadiness of hand than any other expounder of ethics. His
grand work is (at least) an \emph{experimentum crucis} to shew the
weak sides and imperfections of human reason as the sole law of
human action. By overshooting the mark, or by ``flying an eagle
flight, forth and right on,'' he has pointed out the limit or line
of separation, between what is practicable and what is barely
conceivable\textemdash by imposing impossible tasks on the naked
strength of the will, he has discovered how far it is or is not in
our power to dispense with the illusions of sense, to resist the
calls of affection, to emancipate ourselves from the force of
habit; and thus, though he has not said it himself, has enabled
others to say to the towering aspirations after good, and to the
over-bearing pride of human intellect\textemdash ``Thus far shalt
thou come, and no farther!'' Captain Parry would be thought to
have rendered a service to navigation and his country, no less by
proving that there is no North-West Passage, than if he had
ascertained that there is one: so Mr.  Godwin has rendered an
essential service to moral science, by attempting (in vain) to
pass the Arctic Circle and Frozen Regions, where the understanding
is no longer warmed by the affections, nor fanned by the breeze of
fancy! This is the effect of all bold, original, and powerful
thinking, that it either discovers the truth, or detects where
error lies; and the only crime with which Mr. Godwin can be
charged as a political and moral reasoner is, that he has
displayed a more ardent spirit, and a more independent activity of
thought than others, in establishing the fallacy (if fallacy it
be) of an old popular prejudice that \emph{the Just and True were
one}, by ``championing it to the Outrance,'' and in the final
result placing the Gothic structure of human virtue on an humbler,
but a wider and safer foundation than it had hitherto occupied in
the volumes and systems of the learned. Mr. Godwin is an inventor
in the regions of romance, as well as a skilful and hardy explorer
of those of moral truth. \emph{Caleb Williams} and \emph{St. Leon}
are two of the most splendid and impressive works of the
imagination that have appeared in our times. It is not merely that
these novels are very well for a philosopher to have
produced\textemdash they are admirable and complete in themselves,
and would not lead you to suppose that the author, who is so
entirely at home in human character and dramatic situation, had
ever dabbled in logic or metaphysics. The first of these,
particularly, is a master-piece, both as to invention and
execution. The romantic and chivalrous principle of the love of
personal fame is embodied in the finest possible manner in the
character of Falkland;\footnote{Mr. Fuseli used to object to this
striking delineation a want of historical correctness, inasmuch as
the animating principle of the true chivalrous character was the
sense of honour, not the mere regard to, or saving of,
appearances. This, we think, must be an hypercriticism, from all
we remember of books of chivalry and heroes of romance.} as in
Caleb Williams (who is not the first, but the second character in
the piece) we see the very demon of curiosity personified. Perhaps
the art with which these two characters are contrived to relieve
and set off each other, has never been surpassed in any work of
fiction, with the exception of the immortal satire of
Cervantes. The restless and inquisitive spirit of Caleb Williams,
in search and in possession of his patron's fatal secret, haunts
the latter like a second conscience, plants stings in his tortured
mind, fans the flame of his jealous ambition, struggling with
agonized remorse; and the hapless but noble-minded Falkland at
length falls a martyr to the persecution of that morbid and
overpowering interest, of which his mingled virtues and vices have
rendered him the object. We conceive no one ever began Caleb
Williams that did not read it through: no one that ever read it
could possibly forget it, or speak of it after any length of time,
but with an impression as if the events and feelings had been
personal to himself.  This is the case also with the story of
St. Leon, which, with less dramatic interest and intensity of
purpose, is set off by a more gorgeous and flowing eloquence, and
by a crown of preternatural imagery, that waves over it like a
palm-tree! It is the beauty and the charm of Mr. Godwin's
descriptions that the reader identifies himself with the author;
and the secret of this is, that the author has identified himself
with his personages. Indeed, he has created them. They are the
proper issue of his brain, lawfully begot, not foundlings, nor the
``bastards of his art.'' He is not an indifferent, callous
spectator of the scenes which he himself pourtrays, but without
seeming to feel them.  There is no look of patch-work and
plagiarism, the beggarly copiousness of borrowed wealth; no
tracery-work from worm-eaten manuscripts, from forgotten
chronicles, nor piecing out of vague traditions with fragments and
snatches of old ballads, so that the result resembles a gaudy,
staring transparency, in which you cannot distinguish the daubing
of the painter from the light that shines through the flimsy
colours and gives them brilliancy. Here all is clearly made out
with strokes of the pencil, by fair, not by factitious means. Our
author takes a given subject from nature or from books, and then
fills it up with the ardent workings of his own mind, with the
teeming and audible pulses of his own heart. The effect is entire
and satisfactory in proportion. The work (so to speak) and the
author are one. We are not puzzled to decide upon their respective
pretensions. In reading Mr. Godwin's novels, we know what share of
merit the author has in them. In reading the \emph{Scotch Novels},
we are perpetually embarrassed in asking ourselves this question;
and perhaps it is not altogether a false modesty that prevents the
editor from putting his name in the title-page\textemdash he is
(for any thing we know to the contrary) only a more voluminous
sort of Allen-a-Dale.  At least, we may claim this advantage for
the English author, that the chains with which he rivets our
attention are forged out of his own thoughts, link by link, blow
for blow, with glowing enthusiasm: we see the genuine ore melted
in the furnace of fervid feeling, and moulded into stately and
\emph{ideal} forms; and this is so far better than peeping into an
old iron shop, or pilfering from a dealer in marine stores!  There
is one drawback, however, attending this mode of proceeding, which
attaches generally, indeed, to all originality of composition;
namely, that it has a tendency to a certain degree of monotony. He
who draws upon his own resources, easily comes to an end of his
wealth. Mr.  Godwin, in all his writings, dwells upon one idea or
exclusive view of a subject, aggrandises a sentiment, exaggerates
a character, or pushes an argument to extremes, and makes up by
the force of style and continuity of feeling for what he wants in
variety of incident or ease of manner.  This necessary defect is
observable in his best works, and is still more so in Fleetwood
and Mandeville; the one of which, compared with his more admired
performances, is mawkish, and the other morbid. Mr. Godwin is also
an essayist, an historian\textemdash in short, what is he not,
that belongs to the character of an indefatigable and accomplished
author? His \emph{Life of Chaucer} would have given celebrity to
any man of letters possessed of three thousand a year, with
leisure to write quartos: as the legal acuteness displayed in his
\emph{Remarks on Judge Eyre's Charge to the Jury} would have
raised any briefless barrister to the height of his
profession. This temporary effusion did more\textemdash it gave a
turn to the trials for high treason in the year 1794, and possibly
saved the lives of twelve innocent individuals, marked out as
political victims to the Moloch of Legitimacy, which then skulked
behind a British throne, and had not yet dared to stalk forth (as
it has done since) from its lurking-place, in the face of day, to
brave the opinion of the world. If it had then glutted its maw
with its intended prey (the sharpness of Mr.  Godwin's pen cut the
legal cords with which it was attempted to bind them), it might
have done so sooner, and with more lasting effect. The world do
not know (and we are not sure but the intelligence may startle
Mr. Godwin himself), that he is the author of a volume of Sermons,
and of a Life of Chatham.\footnote{We had forgotten the tragedies
of Antonio and Ferdinand.  Peace be with their \emph{manes}!}

Mr. Fawcett (an old friend and fellow-student of our author, and
who always spoke of his writings with admiration, tinctured with
wonder) used to mention a circumstance with respect to the
last-mentioned work, which may throw some light on the history and
progress of Mr. Godwin's mind. He was anxious to make his
biographical account as complete as he could, and applied for this
purpose to many of his acquaintance to furnish him with anecdotes
or to suggest criticisms. Amongst others Mr.  Fawcett repeated to
him what he thought a striking passage in a speech on
\emph{General Warrants} delivered by Lord Chatham, at which he
(Mr.  Fawcett) had been present. ``Every man's house'' (said this
emphatic thinker and speaker) ``has been called his castle. And
why is it called his castle? Is it because it is defended by a
wall, because it is surrounded with a moat? No, it may be nothing
more than a straw-built shed. It may be open to all the elements:
the wind may enter in, the rain may enter in\textemdash but the
king \emph{cannot} enter in!'' His friend thought that the point
was here palpable enough: but when he came to read the printed
volume, he found it thus \emph{transposed}: ``Every man's house is
his castle. And why is it called so? Is it because it is defended
by a wall, because it is surrounded with a moat? No, it may be
nothing more than a straw-built shed. It may be exposed to all the
elements: the rain may enter into it, \emph{all the winds of
Heaven may whistle round it}, but the king cannot, \&c.'' This was
what Fawcett called a defect of \emph{natural imagination}. He at
the same time admitted that Mr. Godwin had improved his native
sterility in this respect; or atoned for it by incessant activity
of mind and by accumulated stores of thought and powers of
language. In fact, his \emph{forte} is not the spontaneous, but
the voluntary exercise of talent. He fixes his ambition on a high
point of excellence, and spares no pains or time in attaining
it. He has less of the appearance of a man of genius, than any one
who has given such decided and ample proofs of it. He is ready
only on reflection: dangerous only at the rebound. He gathers
himself up, and strains every nerve and faculty with deliberate
aim to some heroic and dazzling atchievement of intellect: but he
must make a career before he flings himself, armed, upon the
enemy, or he is sure to be unhorsed. Or he resembles an eight-day
clock that must be wound up long before it can strike.  Therefore,
his powers of conversation are but limited. He has neither
acuteness of remark, nor a flow of language, both which might be
expected from his writings, as these are no less distinguished by
a sustained and impassioned tone of declamation than by novelty of
opinion or brilliant tracks of invention. In company, Horne Tooke
used to make a mere child of him\textemdash or of any man!
Mr. Godwin liked this treatment\footnote{To be sure, it was
redeemed by a high respect, and by some magnificent
compliments. Once in particular, at his own table, after a good
deal of \emph{badinage} and cross-questioning about his being the
author of the Reply to Judge Eyre's Charge, on Mr. Godwin's
acknowledging that he was, Mr. Tooke said, ``Come here
then,''\textemdash and when his guest went round to his chair, he
took his hand, and pressed it to his lips, saying\textemdash ``I
can do no less for the hand that saved my life!''}, and indeed it
is his foible to fawn on those who use him \emph{cavalierly}, and
to be cavalier to those who express an undue or unqualified
admiration of him. He looks up with unfeigned respect to
acknowledged reputation (but then it must be very well ascertained
before he admits it)\textemdash and has a favourite hypothesis
that Understanding and Virtue are the same thing. Mr. Godwin
possesses a high degree of philosophical candour, and studiously
paid the homage of his pen and person to Mr.  Malthus, Sir James
Macintosh, and Dr. Parr, for their unsparing attacks on him; but
woe to any poor devil who had the hardihood to defend him against
them! In private, the author of \emph{Political Justice} at one
time reminded those who knew him of the metaphysician engrafted on
the Dissenting Minister. There was a dictatorial, captious,
quibbling pettiness of manner. He lost this with the first blush
and awkwardness of popularity, which surprised him in the
retirement of his study; and he has since, with the wear and tear
of society, from being too pragmatical, become somewhat too
careless. He is, at present, as easy as an old glove. Perhaps
there is a little attention to effect in this, and he wishes to
appear a foil to himself. His best moments are with an intimate
acquaintance or two, when he gossips in a fine vein about old
authors, Clarendon's \emph{History of the Rebellion}, or Burnet's
\emph{History of his own Times}; and you perceive by your host's
talk, as by the taste of seasoned wine, that he has a
\emph{cellarage} in his understanding! Mr.  Godwin also has a
correct \emph{acquired} taste in poetry and the drama. He relishes
Donne and Ben Jonson, and recites a passage from either with an
agreeable mixture of pedantry and \emph{bonhommie}. He is not one
of those who do not grow wiser with opportunity and reflection: he
changes his opinions, and changes them for the better. The
alteration of his taste in poetry, from an exclusive admiration of
the age of Queen Anne to an almost equally exclusive one of that
of Elizabeth, is, we suspect, owing to Mr. Coleridge, who some
twenty years ago, threw a great stone into the standing pool of
criticism, which splashed some persons with the mud, but which
gave a motion to the surface and a reverberation to the
neighbouring echoes, which has not since subsided. In common
company, Mr. Godwin either goes to sleep himself, or sets others
to sleep. He is at present engaged in a History of the
Commonwealth of England.\textemdash \emph{Esto perpetua!} In size
Mr. Godwin is below the common stature, nor is his deportment
graceful or animated. His face is, however, fine, with an
expression of placid temper and recondite thought. He is not
unlike the common portraits of Locke. There is a very admirable
likeness of him by Mr. Northcote, which with a more heroic and
dignified air, only does justice to the profound sagacity and
benevolent aspirations of our author's mind. Mr. Godwin has kept
the best company of his time, but he has survived most of the
celebrated persons with whom he lived in habits of intimacy. He
speaks of them with enthusiasm and with discrimination; and
sometimes dwells with peculiar delight on a day passed at John
Kemble's in company with Mr. Sheridan, Mr. Curran,
Mrs. Wolstonecraft and Mrs. Inchbald, when the conversation took a
most animated turn and the subject was of Love. Of all these our
author is the only one remaining. Frail tenure, on which human
life and genius are lent us for a while to improve or to enjoy!
