\chapter[Mr. Coleridge]{mr. coleridge}

The present is an age of talkers, and not of doers; and the reason
is, that the world is growing old. We are so far advanced in the
Arts and Sciences, that we live in retrospect, and doat on past
atchievements.  The accumulation of knowledge has been so great,
that we are lost in wonder at the height it has reached, instead
of attempting to climb or add to it; while the variety of objects
distracts and dazzles the looker-on. What \emph{niche} remains
unoccupied? What path untried? What is the use of doing anything,
unless we could do better than all those who have gone before us?
What hope is there of this? We are like those who have been to see
some noble monument of art, who are content to admire without
thinking of rivalling it; or like guests after a feast, who praise
the hospitality of the donor ``and thank the bounteous
Pan''\textemdash perhaps carrying away some trifling fragments; or
like the spectators of a mighty battle, who still hear its sound
afar off, and the clashing of armour and the neighing of the
war-horse and the shout of victory is in their ears, like the
rushing of innumerable waters!

Mr. Coleridge has ``a mind reflecting ages past:'' his voice is like
the echo of the congregated roar of the ``dark rearward and abyss''
of thought. He who has seen a mouldering tower by the side of a
chrystal lake, hid by the mist, but glittering in the wave below,
may conceive the dim, gleaming, uncertain intelligence of his eye:
he who has marked the evening clouds uprolled (a world of
vapours), has seen the picture of his mind, unearthly,
unsubstantial, with gorgeous tints and ever-varying forms\textemdash 
\begin{quote}
  ``That which was now a horse, even with a thought\\
  The rack dislimns, and makes it indistinct\\
  As water is in water.''
\end{quote}
Our author's mind is (as he himself might express it)
\emph{tangential}.  There is no subject on which he has not touched,
none on which he has rested. With an understanding fertile,
subtle, expansive, ``quick, forgetive, apprehensive,'' beyond all
living precedent, few traces of it will perhaps remain. He lends
himself to all impressions alike; he gives up his mind and liberty
of thought to none. He is a general lover of art and science, and
wedded to no one in particular. He pursues knowledge as a
mistress, with outstretched hands and winged speed; but as he is
about to embrace her, his Daphne turns\textemdash alas! not to a laurel!
Hardly a speculation has been left on record from the earliest
time, but it is loosely folded up in Mr. Coleridge's memory, like
a rich, but somewhat tattered piece of tapestry; we might add
(with more seeming than real extravagance), that scarce a thought
can pass through the mind of man, but its sound has at some time
or other passed over his head with rustling pinions. On whatever
question or author you speak, he is prepared to take up the theme
with advantage\textemdash from Peter Abelard down to Thomas Moore, from the
subtlest metaphysics to the politics of the \emph{Courier}. There is no
man of genius, in whose praise he descants, but the critic seems
to stand above the author, and ``what in him is weak, to
strengthen, what is low, to raise and support:'' nor is there any
work of genius that does not come out of his hands like an
Illuminated Missal, sparkling even in its defects. If
Mr. Coleridge had not been the most impressive talker of his age,
he would probably have been the finest writer; but he lays down
his pen to make sure of an auditor, and mortgages the admiration
of posterity for the stare of an idler. If he had not been a poet,
he would have been a powerful logician; if he had not dipped his
wing in the Unitarian controversy, he might have soared to the
very summit of fancy. But in writing verse, he is trying to
subject the Muse to \emph{transcendental} theories: in his abstract
reasoning, he misses his way by strewing it with flowers. All that
he has done of moment, he had done twenty years ago: since then,
he may be said to have lived on the sound of his own
voice. Mr. Coleridge is too rich in intellectual wealth, to need
to task himself to any drudgery: he has only to draw the sliders
of his imagination, and a thousand subjects expand before him,
startling him with their brilliancy, or losing themselves in
endless obscurity\textemdash 
\begin{quote}
  ``And by the force of blear illusion, \\
  They draw him on to his confusion.''
\end{quote}
What is the little he could add to the stock, compared with the
countless stores that lie about him, that he should stoop to pick
up a name, or to polish an idle fancy? He walks abroad in the
majesty of an universal understanding, eyeing the ``rich strond,''
or golden sky above him, and ``goes sounding on his way,'' in
eloquent accents, uncompelled and free!

Persons of the greatest capacity are often those, who for this
reason do the least; for surveying themselves from the highest
point of view, amidst the infinite variety of the universe, their
own share in it seems trifling, and scarce worth a thought, and
they prefer the contemplation of all that is, or has been, or can
be, to the making a coil about doing what, when done, is no better
than vanity. It is hard to concentrate all our attention and
efforts on one pursuit, except from ignorance of others; and
without this concentration of our faculties, no great progress can
be made in any one thing. It is not merely that the mind is not
capable of the effort; it does not think the effort worth making.
Action is one; but thought is manifold. He whose restless eye
glances through the wide compass of nature and art, will not
consent to have ``his own nothings monstered:'' but he must do this,
before he can give his whole soul to them. The mind, after
``letting contemplation have its fill,'' or
\begin{quote}
  ``Sailing with supreme dominion\\
  Through the azure deep of air,''
\end{quote}
sinks down on the ground, breathless, exhausted, powerless,
inactive; or if it must have some vent to its feelings, seeks the
most easy and obvious; is soothed by friendly flattery, lulled by
the murmur of immediate applause, thinks as it were aloud, and
babbles in its dreams!  A scholar (so to speak) is a more
disinterested and abstracted character than a mere author. The
first looks at the numberless volumes of a library, and says, ``All
these are mine:'' the other points to a single volume (perhaps it
may be an immortal one) and says, ``My name is written on the back
of it.'' This is a puny and groveling ambition, beneath the lofty
amplitude of Mr. Coleridge's mind. No, he revolves in his wayward
soul, or utters to the passing wind, or discourses to his own
shadow, things mightier and more various!\textemdash Let us draw the
curtain, and unlock the shrine. Learning rocked him in his cradle,
and, while yet a child,
\begin{quote}
  ``He lisped in numbers, for the numbers came.''
\end{quote}
At sixteen he wrote his \emph{Ode on Chatterton}, and he still reverts
to that period with delight, not so much as it relates to himself
(for that string of his own early promise of fame rather jars than
otherwise) but as exemplifying the youth of a poet. Mr. Coleridge
talks of himself, without being an egotist, for in him the
individual is always merged in the abstract and general. He
distinguished himself at school and at the University by his
knowledge of the classics, and gained several prizes for Greek
epigrams. How many men are there (great scholars, celebrated names
in literature) who having done the same thing in their youth, have
no other idea all the rest of their lives but of this achievement,
of a fellowship and dinner, and who, installed in academic
honours, would look down on our author as a mere strolling bard!
At Christ's Hospital, where he was brought up, he was the idol of
those among his schoolfellows, who mingled with their bookish
studies the music of thought and of humanity; and he was usually
attended round the cloisters by a group of these (inspiring and
inspired) whose hearts, even then, burnt within them as he talked,
and where the sounds yet linger to mock \textsc{Elia} on his way, still
turning pensive to the past! One of the finest and rarest parts of
Mr. Coleridge's conversation, is when he expatiates on the Greek
tragedians (not that he is not well acquainted, when he pleases,
with the epic poets, or the philosophers, or orators, or
historians of antiquity)\textemdash on the subtle reasonings and melting
pathos of Euripides, on the harmonious gracefulness of Sophocles,
tuning his love-laboured song, like sweetest warblings from a
sacred grove; on the high-wrought trumpet-tongued eloquence of
Æschylus, whose Prometheus, above all, is like an Ode to Fate,
and a pleading with Providence, his thoughts being let loose as
his body is chained on his solitary rock, and his afflicted will
(the emblem of mortality)
\begin{quote}
  ``Struggling in vain with ruthless destiny.''
\end{quote}
As the impassioned critic speaks and rises in his theme, you would
think you heard the voice of the Man hated by the Gods, contending
with the wild winds as they roar, and his eye glitters with the
spirit of Antiquity!

Next, he was engaged with Hartley's tribes of mind, ``etherial
braid, thought-woven,''\textemdash and he busied himself for a year or two
with vibrations and vibratiuncles and the great law of association
that binds all things in its mystic chain, and the doctrine of
Necessity (the mild teacher of Charity) and the Millennium,
anticipative of a life to come\textemdash and he plunged deep into the
controversy on Matter and Spirit, and, as an escape from
Dr. Priestley's Materialism, where he felt himself imprisoned by
the logician's spell, like Ariel in the cloven pine-tree, he
became suddenly enamoured of Bishop Berkeley's
fairy-world,\footnote{
  Mr. Coleridge named his eldest son (the writer of
some beautiful Sonnets) after Hartley, and the second after
Berkeley. The third was called Derwent, after the river of that
name. Nothing can be more characteristic of his mind than this
circumstance. All his ideas indeed are like a river, flowing on
for ever, and still murmuring as it flows, discharging its waters
and still replenished\textemdash 
\begin{quote}
  ``And so by many winding nooks it strays, With willing sport to
  the wild ocean!''
\end{quote}} and
used in all companies to build the universe, like a brave poetical
fiction, of fine words\textemdash and he was deep-read in Malebranche, and
in Cudworth's Intellectual System (a huge pile of learning,
unwieldy, enormous) and in Lord Brook's hieroglyphic theories, and
in Bishop Butler's Sermons, and in the Duchess of Newcastle's
fantastic folios, and in Clarke and South and Tillotson, and all
the fine thinkers and masculine reasoners of that age\textemdash and
Leibnitz's \emph{Pre-established Harmony} reared its arch above his
head, like the rainbow in the cloud, covenanting with the hopes of
man\textemdash and then he fell plump, ten thousand fathoms down (but his
wings saved him harmless) into the \emph{hortus siccus} of Dissent,
where he pared religion down to the standard of reason and
stripped faith of mystery, and preached Christ crucified and the
Unity of the Godhead, and so dwelt for a while in the spirit with
John Huss and Jerome of Prague and Socinus and old John Zisca, and
ran through Neal's History of the Puritans, and Calamy's
Non-Conformists' Memorial, having like thoughts and passions with
them\textemdash but then Spinoza became his God, and he took up the vast
chain of being in his hand, and the round world became the centre
and the soul of all things in some shadowy sense, forlorn of
meaning, and around him he beheld the living traces and the
sky-pointing proportions of the mighty Pan\textemdash but poetry redeemed
him from this spectral philosophy, and he bathed his heart in
beauty, and gazed at the golden light of heaven, and drank of the
spirit of the universe, and wandered at eve by fairy-stream or
fountain,
\begin{quote} % 3 emdashes
  `` \pcdash{2} When he saw nought but beauty, \\
  When he heard the voice of that Almighty One \\
  In every breeze that blew, or wave that murmured''\textemdash 
\end{quote}
and wedded with truth in Plato's shade, and in the writings of
Proclus and Plotinus saw the ideas of things in the eternal mind,
and unfolded all mysteries with the Schoolmen and fathomed the
depths of Duns Scotus and Thomas Aquinas, and entered the third
heaven with Jacob Behmen, and walked hand in hand with Swedenborg
through the pavilions of the New Jerusalem, and sung his faith in
the promise and in the word in his \emph{Religious Musings}\textemdash and
lowering himself from that dizzy height, poised himself on
Milton's wings, and spread out his thoughts in charity with the
glad prose of Jeremy Taylor, and wept over Bowles's Sonnets, and
studied Cowper's blankverse, and betook himself to Thomson's
Castle of Indolence, and sported with the wits of Charles the
Second's days and of Queen Anne, and relished Swift's style and
that of the John Bull (Arbuthnot's we mean, not Mr. Croker's) and
dallied with the British Essayists and Novelists, and knew all
qualities of more modern writers with a learned spirit, Johnson,
and Goldsmith, and Junius, and Burke, and Godwin, and the Sorrows
of Werter, and Jean Jacques Rousseau, and Voltaire, and Marivaux,
and Crebillon, and thousands more\textemdash now ``laughed with Rabelais in
his easy chair'' or pointed to Hogarth, or afterwards dwelt on
Claude's classic scenes or spoke with rapture of Raphael, and
compared the women at Rome to figures that had walked out of his
pictures, or visited the Oratory of Pisa, and described the works
of Giotto and Ghirlandaio and Massaccio, and gave the moral of the
picture of the Triumph of Death, where the beggars and the
wretched invoke his dreadful dart, but the rich and mighty of the
earth quail and shrink before it; and in that land of siren sights
and sounds, saw a dance of peasant girls, and was charmed with
lutes and gondolas,\textemdash or wandered into Germany and lost himself in
the labyrinths of the Hartz Forest and of the Kantean philosophy,
and amongst the cabalistic names of Fichtè and Schelling and
Lessing, and God knows who\textemdash this was long after, but all the
former while, he had nerved his heart and filled his eyes with
tears, as he hailed the rising orb of liberty, since quenched in
darkness and in blood, and had kindled his affections at the blaze
of the French Revolution, and sang for joy when the towers of the
Bastile and the proud places of the insolent and the oppressor
fell, and would have floated his bark, freighted with fondest
fancies, across the Atlantic wave with Southey and others to seek
for peace and freedom\textemdash 
\begin{quote}
  ``In Philarmonia's undivided dale!''
\end{quote}
Alas! ``Frailty, thy name is \emph{Genius}!''\textemdash What is become of all this
mighty heap of hope, of thought, of learning, and humanity? It has
ended in swallowing doses of oblivion and in writing paragraphs in
the \emph{Courier}.\textemdash Such, and so little is the mind of man!

It was not to be supposed that Mr. Coleridge could keep on at the
rate he set off; he could not realize all he knew or thought, and
less could not fix his desultory ambition; other stimulants
supplied the place, and kept up the intoxicating dream, the fever
and the madness of his early impressions. Liberty (the
philosopher's and the poet's bride) had fallen a victim,
meanwhile, to the murderous practices of the hag, Legitimacy.
Proscribed by court-hirelings, too romantic for the herd of vulgar
politicians, our enthusiast stood at bay, and at last turned on
the pivot of a subtle casuistry to the \emph{unclean side:} but his
discursive reason would not let him trammel himself into a
poet-laureate or stamp-distributor, and he stopped, ere he had
quite passed that well-known ``bourne from whence no traveller
returns''\textemdash and so has sunk into torpid, uneasy repose, tantalized
by useless resources, haunted by vain imaginings, his lips idly
moving, but his heart forever still, or, as the shattered chords
vibrate of themselves, making melancholy music to the ear of
memory! Such is the fate of genius in an age, when in the unequal
contest with sovereign wrong, every man is ground to powder who is
not either a born slave, or who does not willingly and at once
offer up the yearnings of humanity and the dictates of reason as a
welcome sacrifice to besotted prejudice and loathsome power.

Of all Mr. Coleridge's productions, the \emph{Ancient Mariner} is the
only one that we could with confidence put into any person's
hands, on whom we wished to impress a favourable idea of his
extraordinary powers. Let whatever other objections be made to it,
it is unquestionably a work of genius\textemdash of wild, irregular,
overwhelming imagination, and has that rich, varied movement in
the verse, which gives a distant idea of the lofty or changeful
tones of Mr. Coleridge's voice. In the \emph{Christobel}, there is one
splendid passage on divided friendship. The \emph{Translation of
Schiller's Wallenstein} is also a masterly production in its kind,
faithful and spirited. Among his smaller pieces there are
occasional bursts of pathos and fancy, equal to what we might
expect from him; but these form the exception, and not the
rule. Such, for instance, is his affecting Sonnet to the author of
the Robbers.
\begin{verse}
  Schiller! that hour I would have wish'd to die, \\
  \vin If through the shudd'ring midnight I had sent \\
  \vin From the dark dungeon of the tower time-rent,

  That fearful voice, a famish'd father's cry\textemdash 

  That in no after-moment aught less vast \\
  \vin Might stamp me mortal! A triumphant shout \\
  \vin Black horror scream'd, and all her goblin rout
  
  From the more with'ring scene diminish'd pass'd.

  Ah! Bard tremendous in sublimity! \\
  \vin Could I behold thee in thy loftier mood,
    
  Wand'ring at eve, with finely frenzied eye, \\
  \vin Beneath some vast old tempest-swinging wood! \\
  \vin Awhile, with mute awe gazing, I would brood,
  
  Then weep aloud in a wild ecstasy.
\end{verse}
His Tragedy, entitled \emph{Remorse}, is full of beautiful and striking
passages, but it does not place the author in the first rank of
dramatic writers. But if Mr. Coleridge's works do not place him in
that rank, they injure instead of conveying a just idea of the
man, for he himself is certainly in the first class of general
intellect.

If our author's poetry is inferior to his conversation, his prose
is utterly abortive. Hardly a gleam is to be found in it of the
brilliancy and richness of those stores of thought and language
that he pours out incessantly, when they are lost like drops of
water in the ground. The principal work, in which he has attempted
to embody his general views of things, is the \textsc{Friend}, of which,
though it contains some noble passages and fine trains of thought,
prolixity and obscurity are the most frequent characteristics.

No two persons can be conceived more opposite in character or
genius than the subject of the present and of the preceding
sketch. Mr. Godwin, with less natural capacity, and with fewer
acquired advantages, by concentrating his mind on some given
object, and doing what he had to do with all his might, has
accomplished much, and will leave more than one monument of a
powerful intellect behind him; Mr. Coleridge, by dissipating his,
and dallying with every subject by turns, has done little or
nothing to justify to the world or to posterity, the high opinion
which all who have ever heard him converse, or known him
intimately, with one accord entertain of him. Mr. Godwin's
faculties have kept house, and plied their task in the work-shop
of the brain, diligently and effectually: Mr. Coleridge's have
gossipped away their time, and gadded about from house to house,
as if life's business were to melt the hours in listless
talk. Mr. Godwin is intent on a subject, only as it concerns
himself and his reputation; he works it out as a matter of duty,
and discards from his mind whatever does not forward his main
object as impertinent and vain. Mr. Coleridge, on the other hand,
delights in nothing but episodes and digressions, neglects
whatever he undertakes to perform, and can act only on spontaneous
impulses, without object or method. ``He cannot be constrained by
mastery.'' While he should be occupied with a given pursuit, he is
thinking of a thousand other things; a thousand tastes, a thousand
objects tempt him, and distract his mind, which keeps open house,
and entertains all comers; and after being fatigued and amused
with morning calls from idle visitors, finds the day consumed and
its business unconcluded. Mr. Godwin, on the contrary, is somewhat
exclusive and unsocial in his habits of mind, entertains no
company but what he gives his whole time and attention to, and
wisely writes over the doors of his understanding, his fancy, and
his senses\textemdash ``No admittance except on business.'' He has
none of that fastidious refinement and false delicacy, which might
lead him to balance between the endless variety of modern
attainments. He does not throw away his life (nor a single
half-hour of it) in adjusting the claims of different
accomplishments, and in choosing between them or making himself
master of them all. He sets about his task, (whatever it may be)
and goes through it with spirit and fortitude. He has the
happiness to think an author the greatest character in the world,
and himself the greatest author in it. Mr. Coleridge, in writing
an harmonious stanza, would stop to consider whether there was not
more grace and beauty in a \emph{Pas de trois}, and would not
proceed till he had resolved this question by a chain of
metaphysical reasoning without end.  Not so Mr. Godwin. That is
best to him, which he can do best. He does not waste himself in
vain aspirations and effeminate sympathies. He is blind, deaf,
insensible to all but the trump of Fame. Plays, operas, painting,
music, ball-rooms, wealth, fashion, titles, lords, ladies, touch
him not\textemdash all these are no more to him than to the
magician in his cell, and he writes on to the end of the chapter,
through good report and evil report. \emph{Pingo in
eternitatem}\textemdash is his motto. He neither envies nor
admires what others are, but is contented to be what he is, and
strives to do the utmost he can. Mr. Coleridge has flirted with
the Muses as with a set of mistresses: Mr. Godwin has been married
twice, to Reason and to Fancy, and has to boast no short-lived
progeny by each.  So to speak, he has \emph{valves} belonging to
his mind, to regulate the quantity of gas admitted into it, so
that like the bare, unsightly, but well-compacted steam-vessel, it
cuts its liquid way, and arrives at its promised end: while
Mr. Coleridge's bark, ``taught with the little nautilus to sail,''
the sport of every breath, dancing to every wave,
\begin{quote} ``Youth at its prow, and Pleasure at its helm,''
\end{quote} flutters its gaudy pennons in the air, glitters in the
sun, but we wait in vain to hear of its arrival in the destined
harbour. Mr. Godwin, with less variety and vividness, with less
subtlety and susceptibility both of thought and feeling, has had
firmer nerves, a more determined purpose, a more comprehensive
grasp of his subject, and the results are as we find them. Each
has met with his reward: for justice has, after all, been done to
the pretensions of each; and we must, in all cases, use means to
ends!

% previously missing
It was a misfortune to any man of talent to be borin in the latter
end of the last century. Genius stopped the way of Legitimacy, and
therefore it was to be abated, crushed, or set aside as a
nuisance. The spirit of monarchy was at variance with the spirit
of the age. The flame of liberty, the light of intellect, was to
be extinguished with the sword \textemdash or with slander, whose
edge is sharper than the sword. The war between power and reason
was carried on by the first of these abroad \textemdash by the
last at home. No quarter was given (then or now) by the
Government-critics, the authorised censors of the press, to those
who followed the dictates of independence, who listened to the
voice of tempter, Fancy. Instead of gathering fruits and flowers,
immortal fruits and amaranthine flowers, they soon found
themselves beset not only by a host of prejudices, but assailed
with all the engines of power, by nicknames, by lies, by all the
arts of malice, interest and hypocrisy, without the possibility of
their defending themselves ``from the pelting of the pitiless
storm,'' that poured down upon them from the strong-holds of
corruption and authority. The philosophers, the dry abstract
reasoners, submitted to this reverse pretty well, and armed
themselves with patience ``as with triple steel,'' to bear
discomiture, persecution, and disgrace. But the poets, the
creatures of sympathy, could not stand the frowns both of king and
people. They did not like to be shut out when places and pensions,
when the critic's praises, and the laurel-wreath were about to be
distributed. They did not stomach being \emph{sent to Coventry},
and Mr. Coleridge sounded a retreat for them by the help of
casuistry, and a musical voice.\textemdash ``His words were hollow,
but they pleased the ear'' of his friends of the Lake School, who
turned back disgusted and panic-struck from the dry desert of
unpopularity, like Hassan the camel-driver,
\begin{quote}
``And curs'd the hour, the curs'd the luckless day,\\
When first from Shiraz' walls they bent their way.
\end{quote}

They are safely inclosed there, but Mr. Coleridge did not enter
with them; pitching his tent upon the barren waste without, and
having no abiding place nor city of refuge!