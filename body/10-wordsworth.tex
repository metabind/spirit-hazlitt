\chapter[Mr. Wordsworth]{mr. wordsworth}

Mr. Wordsworth's genius is a pure emanation of the Spirit of the
Age.  Had he lived in any other period of the world, he would
never have been heard of. As it is, he has some difficulty to
contend with the hebetude of his intellect, and the meanness of
his subject. With him ``lowliness is young ambition's ladder:''
but he finds it a toil to climb in this way the steep of Fame. His
homely Muse can hardly raise her wing from the ground, nor spread
her hidden glories to the sun. He has ``no figures nor no
fantasies, which busy \emph{passion} draws in the brains of men:''
neither the gorgeous machinery of mythologic lore, nor the
splendid colours of poetic diction. His style is vernacular: he
delivers household truths.  He sees nothing loftier than human
hopes; nothing deeper than the human heart. This he probes, this
he tampers with, this he poises, with all its incalculable weight
of thought and feeling, in his hands; and at the same time calms
the throbbing pulses of his own heart, by keeping his eye ever
fixed on the face of nature. If he can make the life-blood flow
from the wounded breast, this is the living colouring with which
he paints his verse: if he can assuage the pain or close up the
wound with the balm of solitary musing, or the healing power of
plants and herbs and ``skyey influences,'' this is the sole
triumph of his art. He takes the simplest elements of nature and
of the human mind, the mere abstract conditions inseparable from
our being, and tries to compound a new system of poetry from them;
and has perhaps succeeded as well as any one could. ``\emph{Nihil
  humani a me alienum puto}''\textemdash is the motto of his
works. He thinks nothing low or indifferent of which this can be
affirmed: every thing that professes to be more than this, that is
not an absolute essence of truth and feeling, he holds to be
vitiated, false, and spurious. In a word, his poetry is founded on
setting up an opposition (and pushing it to the utmost length)
between the natural and the artificial: between the spirit of
humanity, and the spirit of fashion and of the world!

It is one of the innovations of the time. It partakes of, and is
carried along with, the revolutionary movement of our age: the
political changes of the day were the model on which he formed and
conducted his poetical experiments. His Muse (it cannot be denied,
and without this we cannot explain its character at all) is a
levelling one. It proceeds on a principle of equality, and strives
to reduce all things to the same standard. It is distinguished by
a proud humility. It relies upon its own resources, and disdains
external shew and relief. It takes the commonest events and
objects, as a test to prove that nature is always interesting from
its inherent truth and beauty, without any of the ornaments of
dress or pomp of circumstances to set it off. Hence the
unaccountable mixture of seeming simplicity and real abstruseness
in the \emph{Lyrical Ballad}. Fools have laughed at, wise men
scarcely understand them. He takes a subject or a story merely as
pegs or loops to hang thought and feeling on; the incidents are
trifling, in proportion to his contempt for imposing appearances;
the reflections are profound, according to the gravity and the
aspiring pretensions of his mind. His popular, inartificial style
gets rid (at a blow) of all the trappings of verse, of all the
high places of poetry: ``the cloud-capt towers, the solemn
temples, the gorgeous palaces'' are swept to the ground, and
``like the baseless fabric of a vision, leave not a wreck
behind.''  All the traditions of learning, all the superstitions
of age, are obliterated and effaced. We begin \emph{de novo}, on a
\emph{tabula rasa} of poetry. The purple pall, the nodding plume
of tragedy are exploded as mere pantomime and trick, to return to
the simplicity of truth and nature. Kings, queens, priests,
nobles, the altar and the throne, the distinctions of rank, birth,
wealth, power, ``the judge's robe, the marshall's truncheon, the
ceremony that to great ones 'longs'' are not to be found here. The
author tramples on the pride of art with greater pride. The Ode
and Epode, the Strophe and the Antistrophe, he laughs to
scorn. The harp of Homer, the trump of Pindar and of Alcaeus are
still.  The decencies of costume, the decorations of vanity are
stripped off without mercy as barbarous, idle, and Gothic. The
jewels in the crisped hair, the diadem on the polished brow are
thought meretricious, theatrical, vulgar; and nothing contents his
fastidious taste beyond a simple garland of flowers. Neither does
he avail himself of the advantages which nature or accident holds
out to him. He chooses to have his subject a foil to his
invention, to owe nothing but to himself. He gathers manna in the
wilderness, he strikes the barren rock for the gushing
moisture. He elevates the mean by the strength of his own
aspirations; he clothes the naked with beauty and grandeur from
the store of his own recollections. No cypress-grove loads his
verse with perfumes: but his imagination lends a sense of joy
\begin{verse} \vleftofline{``}To the bare trees and mountains
bare, \\ And grass in the green field.''
\end{verse} No storm, no shipwreck startles us by its horrors: but
the rainbow lifts its head in the cloud, and the breeze sighs
through the withered fern.  No sad vicissitude of fate, no
overwhelming catastrophe in nature deforms his page: but the
dew-drop glitters on the bending flower, the tear collects in the
glistening eye.
\begin{verse} \vleftofline{``}Beneath the hills, along the flowery
vales,\\ The generations are prepared; the pangs, \\ The internal
pangs are ready; the dread strife \\ Of poor humanity's afflicted
will, \\ Struggling in vain with ruthless destiny.''
\end{verse} As the lark ascends from its low bed on fluttering
wing, and salutes the morning skies; so Mr. Wordsworth's
unpretending Muse, in russet guise, scales the summits of
reflection, while it makes the round earth its footstool, and its
home!

Possibly a good deal of this may be regarded as the effect of
disappointed views and an inverted ambition. Prevented by native
pride and indolence from climbing the ascent of learning or
greatness, taught by political opinions to say to the vain pomp
and glory of the world, ``I hate ye,'' seeing the path of
classical and artificial poetry blocked up by the cumbrous
ornaments of style and turgid \emph{common-places}, so that
nothing more could be achieved in that direction but by the most
ridiculous bombast or the tamest servility; he has turned back
partly from the bias of his mind, partly perhaps from a judicious
policy\textemdash has struck into the sequestered vale of humble
life, sought out the Muse among sheep-cotes and hamlets and the
peasant's mountain-haunts, has discarded all the tinsel pageantry
of verse, and endeavoured (not in vain) to aggrandise the trivial
and add the charm of novelty to the familiar. No one has shewn the
same imagination in raising trifles into importance: no one has
displayed the same pathos in treating of the simplest feelings of
the heart. Reserved, yet haughty, having no unruly or violent
passions, (or those passions having been early suppressed,)
Mr. Wordsworth has passed his life in solitary musing, or in daily
converse with the face of nature. He exemplifies in an eminent
degree the power of \emph{association}; for his poetry has no
other source or character. He has dwelt among pastoral scenes,
till each object has become connected with a thousand feelings, a
link in the chain of thought, a fibre of his own heart. Every one
is by habit and familiarity strongly attached to the place of his
birth, or to objects that recal the most pleasing and eventful
circumstances of his life. But to the author of the \emph{Lyrical
Ballads}, nature is a kind of home; and he may be said to take a
personal interest in the universe. There is no image so
insignificant that it has not in some mood or other found the way
into his heart: no sound that does not awaken the memory of other
years.\textemdash
\begin{verse} \vleftofline{``}To him the meanest flower that blows
can give \\ Thoughts that do often lie too deep for tears.''
\end{verse} 
\fixspacing{1.1}{The daisy looks up to him with sparkling eye as an old
acquaintance: the cuckoo haunts him with sounds of early youth not
to be expressed: a linnet's nest startles him with boyish delight:
an old withered thorn is weighed down with a heap of
recollections: a grey cloak, seen on some wild moor, torn by the
wind, or drenched in the rain, afterwards becomes an object of
imagination to him: even the lichens on the rock have a life and
being in his thoughts. He has described all these objects in a way
and with an intensity of feeling that no one else had done before
him, and has given a new view or aspect of nature. He is in this
sense the most original poet now living, and the one whose
writings could the least be spared: for they have no substitute
elsewhere. The vulgar do not read them, the learned, who see all
things through books, do not understand them, the great despise,
the fashionable may ridicule them: but the author has created
himself an interest in the heart of the retired and lonely student
of nature, which can never die. Persons of this class will still
continue to feel what he has felt: he has expressed what they
might in vain wish to express, except with glistening eye and
faultering tongue! There is a lofty philosophic tone, a thoughtful
humanity, infused into his pastoral vein. Remote from the passions
and events of the great world, he has communicated interest and
dignity to the primal movements of the heart of man, and ingrafted
his own conscious reflections on the casual thoughts of hinds and
shepherds.  Nursed amidst the grandeur of mountain scenery, he has
stooped to have a nearer view of the daisy under his feet, or
plucked a branch of white-thorn from the spray: but in describing
it, his mind seems imbued with the majesty and solemnity of the
objects around him\textemdash the tall rock lifts its head in the
erectness of his spirit; the cataract roars in the sound of his
verse; and in its dim and mysterious meaning, the mists seem to
gather in the hollows of Helvellyn, and the forked Skiddaw hovers
in the distance. There is little mention of mountainous scenery in
Mr. Wordsworth's poetry; but by internal evidence one might be
almost sure that it was written in a mountainous country, from its
bareness, its simplicity, its loftiness and its depth!}

His later philosophic productions have a somewhat different
character.  They are a departure from, a dereliction of his first
principles. They are classical and courtly. They are polished in
style, without being gaudy; dignified in subject, without
affectation. They seem to have been composed not in a cottage at
Grasmere, but among the half-inspired groves and stately
recollections of Cole-Orton. We might allude in particular, for
examples of what we mean, to the lines on a Picture by Claude
Lorraine, and to the exquisite poem, entitled \emph{Laodamia}. The
last of these breathes the pure spirit of the finest fragments of
antiquity\textemdash the sweetness, the gravity, the strength, the
beauty and the langour of death\textemdash
\begin{verse} \vleftofline{``}Calm contemplation and majestic
pains.''
\end{verse} Its glossy brilliancy arises from the perfection of
the finishing, like that of careful sculpture, not from gaudy
colouring\textemdash the texture of the thoughts has the
smoothness and solidity of marble. It is a poem that might be read
aloud in Elysium, and the spirits of departed heroes and sages
would gather round to listen to it! Mr. Wordsworth's philosophic
poetry, with a less glowing aspect and less tumult in the veins
than Lord Byron's on similar occasions, bends a calmer and keener
eye on mortality; the impression, if less vivid, is more pleasing
and permanent; and we confess it (perhaps it is a want of taste
and proper feeling) that there are lines and poems of our
author's, that we think of ten times for once that we recur to any
of Lord Byron's. Or if there are any of the latter's writings,
that we can dwell upon in the same way, that is, as lasting and
heart-felt sentiments, it is when laying aside his usual pomp and
pretension, he descends with Mr. Wordsworth to the common ground
of a disinterested humanity. It may be considered as
characteristic of our poet's writings, that they either make no
impression on the mind at all, seem mere \emph{nonsense-verses},
or that they leave a mark behind them that never wears out. They
either
\begin{quote} ``Fall blunted from the indurated
breast''\textemdash
\end{quote} 
\fixspacing{1.2}{without any perceptible result, or they absorb it like a
passion. To one class of readers he appears sublime, to another
(and we fear the largest) ridiculous. He has probably realised
Milton's wish,\textemdash ``and fit audience found, though few:''
but we suspect he is not reconciled to the alternative. There are
delightful passages in the \textsc{Excursion}, both of natural
description and of inspired reflection (passages of the latter
kind that in the sound of the thoughts and of the swelling
language resemble heavenly symphonies, mournful \emph{requiems}
over the grave of human hopes); but we must add, in justice and in
sincerity, that we think it impossible that this work should ever
become popular, even in the same degree as the \emph{Lyrical
  Ballads}. It affects a system without having any intelligible
clue to one; and instead of unfolding a principle in various and
striking lights, repeats the same conclusions till they become
flat and insipid. Mr. Wordsworth's mind is obtuse, except as it is
the organ and the receptacle of accumulated feelings: it is not
analytic, but synthetic; it is reflecting, rather than
theoretical. The \textsc{Excursion}, we believe, fell stillborn from the
press.  There was something abortive, and clumsy, and ill-judged
in the attempt.  It was long and laboured. The personages, for the
most part, were low, the fare rustic: the plan raised expectations
which were not fulfilled, and the effect was like being ushered
into a stately hall and invited to sit down to a splendid banquet
in the company of clowns, and with nothing but successive courses
of apple-dumplings served up. It was not even \emph{toujours
  perdrix}!}

Mr. Wordsworth, in his person, is above the middle size, with
marked features, and an air somewhat stately and Quixotic. He
reminds one of some of Holbein's heads, grave, saturnine, with a
slight indication of sly humour, kept under by the manners of the
age or by the pretensions of the person. He has a peculiar
sweetness in his smile, and great depth and manliness and a rugged
harmony, in the tones of his voice. His manner of reading his own
poetry is particularly imposing; and in his favourite passages his
eye beams with preternatural lustre, and the meaning labours
slowly up from his swelling breast. No one who has seen him at
these moments could go away with an impression that he was a ``man
of no mark or likelihood.'' Perhaps the comment of his face and
voice is necessary to convey a full idea of his poetry. His
language may not be intelligible, but his manner is not to be
mistaken. It is clear that he is either mad or inspired. In
company, even in a \emph{tête-à-tête}, Mr.  Wordsworth is often
silent, indolent, and reserved. If he is become verbose and
oracular of late years, he was not so in his better days.  He
threw out a bold or an indifferent remark without either effort or
pretension, and relapsed into musing again. He shone most (because
he seemed most roused and animated) in reciting his own poetry, or
in talking about it. He sometimes gave striking views of his
feelings and trains of association in composing certain passages;
or if one did not always understand his distinctions, still there
was no want of interest\textemdash there was a latent meaning
worth inquiring into, like a vein of ore that one Cannot exactly
hit upon at the moment, but of which there are sure
indications. His standard of poetry is high and severe, almost to
exclusiveness. He admits of nothing below, scarcely of any thing
above himself. It is fine to hear him talk of the way in which
certain subjects should have been treated by eminent poets,
according to his notions of the art. Thus he finds fault with
Dryden's description of Bacchus in the \emph{Alexander's Feast},
as if he were a mere good-looking youth, or boon
companion\textemdash
\begin{verse} \vleftofline{``}Flushed with a purple grace,\\ He
shews his honest face''\textemdash
\end{verse} 
\fixspacing{1.1}{instead of representing the God returning from
  the conquest of India, crowned with vine-leaves, and drawn by
  panthers, and followed by troops of satyrs, of wild men and
  animals that he had tamed. You would thank, in hearing him speak
  on this subject, that you saw Titian's picture of the meeting of
  \emph{Bacchus and Ariadne}\textemdash so classic were his
  conceptions, so glowing his style. Milton is his great idol, and
  he sometimes dares to compare himself with him. His Sonnets,
  indeed, have something of the same high-raised tone and
  prophetic spirit. Chaucer is another prime favourite of his, and
  he has been at the pains to modernise some of the Canterbury
  Tales. Those persons who look upon Mr. Wordsworth as a merely
  puerile writer, must be rather at a loss to account for his
  strong predilection for such geniuses as Dante and Michael
  Angelo. We do not think our author has any very cordial sympathy
  with Shakespear. How should he? Shakespear was the least of an
  egotist of any body in the world. He does not much relish the
  variety and scope of dramatic composition. ``He hates those
  interlocutions between Lucius and Caius.'' Yet Mr. Wordsworth
  himself wrote a tragedy when he was young; and we have heard the
  following energetic lines quoted from it, as put into the mouth
  of a person smit with remorse for some rash crime:}
\begin{verse} \pcdash{2}``Action is momentary, \\ The motion of a
muscle this way or that; \\ Suffering is long, obscure, and
infinite!''
\end{verse}
\fixspacing{1.1}{
Perhaps for want of light and shade, and the unshackled spirit of
the drama, this performance was never brought forward. Our critic
has a great dislike to Gray, and a fondness for Thomson and
Collins. It is mortifying to hear him speak of Pope and Dryden,
whom, because they have been supposed to have all the possible
excellences of poetry, he will allow to have none. Nothing,
however, can be fairer, or more amusing, than the way in which he
sometimes exposes the unmeaning verbiage of modern poetry. Thus,
in the beginning of Dr. Johnson's \emph{Vanity of Human
  Wishes}\textemdash}
\begin{quote}
  ``Let observation with extensive view \\
  Survey mankind from China to Peru''\textemdash
\end{quote}
he says there is a total want of imagination accompanying the
words, the same idea is repeated three times under the disguise of
a different phraseology: it comes to this\textemdash ``let
\emph{observation}, with extensive \emph{observation, observe}
mankind;'' or take away the first line, and the second,
\begin{quote}
  ``Survey mankind from China to Peru,''
\end{quote}
literally conveys the whole. Mr. Wordsworth is, we
must say, a perfect Drawcansir as to prose writers. He complains
of the dry reasoners and matter-of-fact people for their want of
\emph{passion}; and he is jealous of the rhetorical declaimers and
rhapsodists as trenching on the province of poetry. He condemns
all French writers (as well of poetry as prose) in the lump. His
list in this way is indeed small. He approves of Walton's Angler,
Paley, and some other writers of an inoffensive modesty of
pretension. He also likes books of voyages and travels, and
Robinson Crusoe. In art, he greatly esteems Bewick's wood-cuts,
and Waterloo's sylvan etchings. But he sometimes takes a higher
tone, and gives his mind fair play. We have known him enlarge with
a noble intelligence and enthusiasm on Nicolas Poussin's fine
landscape-compositions, pointing out the unity of design that
pervades them, the superintending mind, the imaginative principle
that brings all to bear on the same end; and declaring he would
not give a rush for any landscape that did not express the time of
day, the climate, the period of the world it was meant to
illustrate, or had not this character of \emph{wholeness} in
it. His eye also does justice to Rembrandt's fine and masterly
effects. In the way in which that artist works something out of
nothing, and transforms the stump of a tree, a common figure into
an \emph{ideal} object, by the gorgeous light and shade thrown
upon it, he perceives an analogy to his own mode of investing the
minute details of nature with an atmosphere of sentiment; and in
pronouncing Rembrandt to be a man of genius, feels that he
strengthens his own claim to the title. It has been said of
Mr. Wordsworth, that ``he hates conchology, that he hates the
Venus of Medicis.'' But these, we hope, are mere epigrams and
\emph{jeux-d'esprit}, as far from truth as they are free from
malice; a sort of running satire or critical clenches\textemdash
\begin{verse} \vleftofline{``}Where one for sense and one for
rhyme \\ Is quite sufficient at one time.''
\end{verse} We think, however, that if Mr. Wordsworth had been a
more liberal and candid critic, he would have been a more sterling
writer. If a greater number of sources of pleasure had been open
to him, he would have communicated pleasure to the world more
frequently. Had he been less fastidious in pronouncing sentence on
the works of others, his own would have been received more
favourably, and treated more leniently.  The current of his
feelings is deep, but narrow; the range of his understanding is
lofty and aspiring rather than discursive. The force, the
originality, the absolute truth and identity with which he feels
some things, makes him indifferent to so many others. The
simplicity and enthusiasm of his feelings, with respect to nature,
renders him bigotted and intolerant in his judgments of men and
things. But it happens to him, as to others, that his strength
lies in his weakness; and perhaps we have no right to complain. We
might get rid of the cynic and the egotist, and find in his stead
a common-place man. We should ``take the good the Gods provide
us:'' a fine and original vein of poetry is not one of their most
contemptible gifts, and the rest is scarcely worth thinking of,
except as it may be a mortification to those who expect perfection
from human nature; or who have been idle enough at some period of
their lives, to deify men of genius as possessing claims above
it. But this is a chord that jars, and we shall not dwell upon it.

Lord Byron we have called, according to the old proverb, ``the
spoiled child of fortune:'' Mr. Wordsworth might plead, in
mitigation of some peculiarities, that he is ``the spoiled child
of disappointment.'' We are convinced, if he had been early a
popular poet, he would have borne his honours meekly, and would
have been a person of great \emph{bonhommie} and frankness of
disposition. But the sense of injustice and of undeserved ridicule
sours the temper and narrows the views. To have produced works of
genius, and to find them neglected or treated with scorn, is one
of the heaviest trials of human patience. We exaggerate our own
merits when they are denied by others, and are apt to grudge and
cavil at every particle of praise bestowed on those to whom we
feel a conscious superiority. In mere self-defence we turn against
the world, when it turns against us; brood over the undeserved
slights we receive; and thus the genial current of the soul is
stopped, or vents itself in effusions of petulance and
self-conceit. Mr. Wordsworth has thought too much of contemporary
critics and criticism; and less than he ought of the award of
posterity, and of the opinion, we do not say of private friends,
but of those who were made so by their admiration of his
genius. He did not court popularity by a conformity to established
models, and he ought not to have been surprised that his
originality was not understood as a matter of course. He has
\emph{gnawed too much on the bridle}; and has often thrown out
crusts to the critics, in mere defiance or as a point of honour
when he was challenged, which otherwise his own good sense would
have withheld. We suspect that Mr. Wordsworth's feelings are a
little morbid in this respect, or that he resents censure more
than he is gratified by praise. Otherwise, the tide has turned
much in his favour of late years\textemdash he has a large body of
determined partisans\textemdash and is at present sufficiently in
request with the public to save or relieve him from the last
necessity to which a man of genius can be reduced\textemdash that
of becoming the God of his own idolatry!

