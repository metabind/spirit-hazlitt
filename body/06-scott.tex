\chapter[Sir Walter Scott]{sir walter scott}

Sir Walter Scott is undoubtedly the most popular writer of the
age\textemdash the ``lord of the ascendant'' for the time being. He is just
half what the human intellect is capable of being: if you take the
universe, and divide it into two parts, he knows all that it \emph{has
been}; all that it \emph{is to be} is nothing to him. His is a mind
brooding over antiquity\textemdash scorning ``the present ignorant time.'' He
is ``laudator temporis acti''\textemdash a \emph{prophesier} of things past. The
old world is to him a crowded map; the new one a dull, hateful
blank. He dotes on all well- authenticated superstitions; he
shudders at the shadow of innovation.  His retentiveness of
memory, his accumulated weight of interested prejudice or romantic
association have overlaid his other faculties. The cells of his
memory are vast, various, full even to bursting with life and
motion; his speculative understanding is empty, flaccid, poor, and
dead. His mind receives and treasures up every thing brought to it
by tradition or custom\textemdash it does not project itself beyond this
into the world unknown, but mechanically shrinks back as from the
edge of a prejudice. The land of pure reason is to his
apprehension like \emph{Van Dieman's Land};\textemdash barren, miserable,
distant, a place of exile, the dreary abode of savages, convicts,
and adventurers. Sir Walter would make a bad hand of a description
of the \emph{Millennium}, unless he could lay the scene in Scotland
five hundred years ago, and then he would want facts and
worm-eaten parchments to support his drooping style.  Our
historical novelist firmly thinks that nothing \emph{is} but what \emph{has
been}\textemdash that the moral world stands still, as the material one was
supposed to do of old\textemdash and that we can never get beyond the point
where we actually are without utter destruction, though every
thing changes and will change from what it was three hundred years
ago to what it is now,\textemdash from what it is now to all that the
bigoted admirer of the good old times most dreads and hates!

It is long since we read, and long since we thought of our
author's poetry. It would probably have gone out of date with the
immediate occasion, even if he himself had not contrived to banish
it from our recollection. It is not to be denied that it had great
merit, both of an obvious and intrinsic kind. It abounded in vivid
descriptions, in spirited action, in smooth and flowing
versification. But it wanted \emph{character}. It was poetry ``of no
mark or likelihood.'' It slid out of the mind as soon as read, like
a river; and would have been forgotten, but that the public
curiosity was fed with ever-new supplies from the same teeming
liquid source. It is not every man that can write six quarto
volumes in verse, that are caught up with avidity, even by
fastidious judges. But what a difference between \emph{their}
popularity and that of the Scotch Novels! It is true, the public
read and admired the Lay of the Last Minstrel, Marmion, and so
on, and each individual was contented to read and admire because
the public did so: but with regard to the prose-works of the same
(supposed) author, it is quite another-guess sort of thing. Here
every one stands forward to applaud on his own ground, would be
thought to go before the public opinion, is eager to extol his
favourite characters louder, to understand them better than every
body else, and has his own scale of comparative excellence for
each work, supported by nothing but his own enthusiastic and
fearless convictions. It must be amusing to the \emph{Author of
Waverley} to hear his readers and admirers (and are not these the
same thing?[A]) quarrelling which of his novels is the best,
opposing character to character, quoting passage against passage,
striving to surpass each other in the extravagance of their
encomiums, and yet unable to settle the precedence, or to do the
author's writings justice\textemdash so various, so equal, so transcendant
are their merits! His volumes of poetry were received as
fashionable and well-dressed acquaintances: we are ready to tear
the others in pieces as old friends. There was something
meretricious in Sir Walter's ballad-rhymes; and like those who
keep opera \emph{figurantes}, we were willing to have our admiration
shared, and our taste confirmed by the town: but the Novels are
like the betrothed of our hearts, bone of our bone, and flesh of
our flesh, and we are jealous that any one should be as much
delighted or as thoroughly acquainted with their beauties as
ourselves. For which of his poetical heroines would the reader
break a lance so soon as for Jeanie Deans?  What \emph{Lady of the
Lake} can compare with the beautiful Rebecca? We believe the late
Mr. John Scott went to his death-bed (though a painful and
premature one) with some degree of satisfaction, inasmuch as he
had penned the most elaborate panegyric on the \emph{Scotch Novels}
that had as yet appeared!\textemdash The \emph{Epics} are not poems, so much as
metrical romances.  There is a glittering veil of verse thrown
over the features of nature and of old romance. The deep incisions
into character are ``skinned and filmed over''\textemdash the details are lost
or shaped into flimsy and insipid decorum; and the truth of
feeling and of circumstance is translated into a tinkling sound, a
tinsel common-place. It must be owned, there is a power in true
poetry that lifts the mind from the ground of reality to a higher
sphere, that penetrates the inert, scattered, incoherent materials
presented to it, and by a force and inspiration of its own, melts
and moulds them into sublimity and beauty. But Sir Walter (we
contend, under correction) has not this creative impulse, this
plastic power, this capacity of reacting on his first
impressions. He is a learned, a literal, a matter-of-fact
expounder of truth or fable:\footnote{Just as Cobbett is a
  \emph{matter-of-fact reasoner}.} he does not soar above and look
down upon his subject, imparting his own lofty views and feelings
to his descriptions of nature\textemdash he relies upon it, is raised by it,
is one with it, or he is nothing. A poet is essentially a \emph{maker};
that is, he must atone for what he loses in individuality and
local resemblance by the energies and resources of his own
mind. The writer of whom we speak is deficient in these last. He
has either not the faculty or not the will to impregnate his
subject by an effort of pure invention. The execution also is much
upon a par with the more ephemeral effusions of the press. It is
light, agreeable, effeminate, diffuse. Sir Walter's Muse is a
\emph{Modern Antique}. The smooth, glossy texture of his verse
contrasts happily with the quaint, uncouth, rugged materials of
which it is composed; and takes away any appearance of heaviness
or harshness from the body of local traditions and obsolete
costume. We see grim knights and iron armour; but then they are
woven in silk with a careless, delicate hand, and have the
softness of flowers. The poet's figures might be compared to old
\footnote{St. Ronan's Well.} tapestries copied on the finest velvet:\textemdash they are not like
Raphael's \emph{Cartoons}, but they are very like Mr. Westall's
drawings, which accompany, and are intended to illustrate
them. This facility and grace of execution is the more remarkable,
as a story goes that not long before the appearance of the \emph{Lay of
the Last Minstrel} Sir Walter (then Mr.) Scott, having, in the
company of a friend, to cross the Frith of Forth in a ferry-boat,
they proposed to beguile the time by writing a number of verses on
a given subject, and that at the end of an hour's hard study, they
found they had produced only six lines between them. ``It is
plain,'' said the unconscious author to his fellow-labourer, ``that
you and I need never think of getting our living by writing
poetry!'' In a year or so after this, he set to work, and poured
out quarto upon quarto, as if they had been drops of water. As to
the rest, and compared with true and great poets, our Scottish
Minstrel is but a metre ballad-monger. We would rather have
written one song of Burns, or a single passage in Lord Byron's
\emph{Heaven and Earth}, or one of Wordsworth's fancies and
good-nights, than all his epics. What is he to Spenser, over
whose immortal, ever-amiable verse beauty hovers and trembles, and
who has shed the purple light of Fancy, from his ambrosial wings,
over all nature? What is there of the might of Milton, whose head
is canopied in the blue serene, and who takes us to sit with him
there? What is there (in his ambling rhymes) of the deep pathos of
Chaucer? Or of the o'er-informing power of Shakespear, whose eye,
watching alike the minutest traces of characters and the strongest
movements of passion, ``glances from heaven to earth, from earth to
heaven,'' and with the lambent flame of genius, playing round each
object, lights up the universe in a robe of its own radiance? Sir
Walter has no voluntary power of combination: all his associations
(as we said before) are those of habit or of tradition. He is a
mere narrative and descriptive poet, garrulous of the old
time. The definition of his poetry is a pleasing superficiality.

Not so of his \textsc{Novels and Romances}. There we turn over a new
leaf\textemdash another and the same\textemdash the same in matter, but in form, in
power how different! The author of Waverley has got rid of the
tagging of rhymes, the eking out of syllables, the supplying of
epithets, the colours of style, the grouping of his characters,
and the regular march of events, and comes to the point at once,
and strikes at the heart of his subject, without dismay and
without disguise. His poetry was a lady's waiting-maid, dressed
out in cast-off finery: his prose is a beautiful, rustic nymph,
that, like Dorothea in Don Quixote, when she is surprised with
dishevelled tresses bathing her naked feet in the brook, looks
round her, abashed at the admiration her charms have excited! The
grand secret of the author's success in these latter productions
is that he has completely got rid of the trammels of authorship;
and torn off at one rent (as Lord Peter got rid of so many yards
of lace in the \emph{Tale of a Tub}) all the ornaments of fine writing
and worn-out sentimentality.  All is fresh, as from the hand of
nature: by going a century or two back and laying the scene in a
remote and uncultivated district, all becomes new and startling in
the present advanced period.\textemdash Highland manners, characters,
scenery, superstitions, Northern dialect and costume, the wars,
the religion, and politics of the sixteenth and seventeenth
centuries, give a charming and wholesome relief to the fastidious
refinement and over-laboured lassitude of modern readers, like
the effect of plunging a nervous valetudinarian into a
cold-bath. The \emph{Scotch Novels}, for this reason, are not so much
admired in Scotland as in England. The contrast, the transition is
less striking. From the top of the Calton-Hill, the inhabitants of
``Auld Reekie'' can descry, or fancy they descry the peaks of Ben
Lomond and the waving outline of Rob Roy's country: we who live at
the southern extremity of the island can only catch a glimpse of
the billowy scene in the descriptions of the Author of
Waverley. The mountain air is most bracing to our languid nerves,
and it is brought us in ship-loads from the neighbourhood of
Abbot's-Ford. There is another circumstance to be taken into the
account. In Edinburgh there is a little opposition and something
of the spirit of cabal between the partisans of works proceeding
from Mr.  Constable's and Mr. Blackwood's shops. Mr. Constable
gives the highest prices; but being the Whig bookseller, it is
grudged that he should do so. An attempt is therefore made to
transfer a certain share of popularity to the second-rate Scotch
novels, the embryo fry, the little airy of \emph{ricketty} children,
issuing through Mr. Blackwood's shop-door.  This operates a
diversion, which does not affect us here. The Author of Waverley
wears the palm of legendary lore alone. Sir Walter may, indeed,
surfeit us: his imitators make us sick! It may be asked, it has
been asked, ``Have we no materials for romance in England? Must we
look to Scotland for a supply of whatever is original and striking
in this kind?'' And we answer\textemdash ``Yes!'' Every foot of soil is with us
worked up: nearly every movement of the social machine is
calculable. We have no room left for violent catastrophes; for
grotesque quaintnesses; for wizard spells. The last skirts of
ignorance and barbarism are seen hovering (in Sir Walter's pages)
over the Border. We have, it is true, gipsies in this country as
well as at the Cairn of Derncleugh: but they live under clipped
hedges, and repose in camp-beds, and do not perch on crags, like
eagles, or take shelter, like sea-mews, in basaltic subterranean
caverns. We have heaths with rude heaps of stones upon them: but
no existing superstition converts them into the Geese of
Micklestane-Moor, or sees a Black Dwarf groping among them. We
have sects in religion: but the only thing sublime or ridiculous
in that way is Mr. Irving, the Caledonian preacher, who ``comes
like a satyr staring from the woods, and yet speaks like an
orator!'' We had a Parson Adams not quite a hundred years ago\textemdash a
Sir Roger de Coverley rather more than a hundred! Even Sir Walter
is ordinarily obliged to pitch his angle (strong as the hook is) a
hundred miles to the North of the ``Modern Athens'' or a century
back. His last work,\footnote{No! For we met with a young lady who kept a
circulating library and a milliner's-shop, in a watering-place in
the country, who, when we inquired for the \emph{Scotch Novels}, spoke
indifferently about them, said they were ``so dry she could hardly
get through them,'' and recommended us to read \emph{Agnes}. We never
thought of it before; but we would venture to lay a wager that
there are many other young ladies in the same situation, and who
think ``Old Mortality'' ``dry.''}indeed, is mystical, is romantic in
nothing but the title-page. Instead of a holy-water sprinkle
dipped in dew, he has given us a fashionable watering-place\textemdash and
we see what he has made of it. He must not come down from his
fastnesses in traditional barbarism and native rusticity: the
level, the littleness, the frippery of modern civilization will
undo him as it has undone us!

Sir Walter has found out (oh, rare discovery) that facts are
better than fiction; that there is no romance like the romance of
real life; and that if we can but arrive at what men feel, do, and
say in striking and singular situations, the result will be ``more
lively, audible, and full of vent,'' than the fine-spun cobwebs of
the brain. With reverence be it spoken, he is like the man who
having to imitate the squeaking of a pig upon the stage, brought
the animal under his coat with him. Our author has conjured up the
actual people he has to deal with, or as much as he could get of
them, in ``their habits as they lived.'' He has ransacked old
chronicles, and poured the contents upon his page; he has squeezed
out musty records; he has consulted wayfaring pilgrims, bed-rid
sibyls; he has invoked the spirits of the air; he has conversed
with the living and the dead, and let them tell their story their
own way; and by borrowing of others, has enriched his own genius
with everlasting variety, truth, and freedom. He has taken his
materials from the original, authentic sources, in large concrete
masses, and not tampered with or too much frittered them away. He
is only the amanuensis of truth and history. It is impossible to
say how fine his writings in consequence are, unless we could
describe how fine nature is. All that portion of the history of
his country that he has touched upon (wide as the scope is) the
manners, the personages, the events, the scenery, lives over again
in his volumes. Nothing is wanting\textemdash the illusion is
complete. There is a hurtling in the air, a trampling of feet upon
the ground, as these perfect representations of human character or
fanciful belief come thronging back upon our imaginations. We will
merely recall a few of the subjects of his pencil to the reader's
recollection; for nothing we could add, by way of note or
commendation, could make the impression more vivid.

\fixspacing{1.1}{There is (first and foremost, because the earliest of our
acquaintance) the Baron of Bradwardine, stately, kind-hearted,
whimsical, pedantic; and Flora MacIvor (whom even \emph{we} forgive for
her Jacobitism), the fierce Vich Ian Vohr, and Evan Dhu, constant
in death, and Davie Gellatly roasting his eggs or turning his
rhymes with restless volubility, and the two stag-hounds that met
Waverley, as fine as ever Titian painted, or Paul Veronese:\textemdash then
there is old Balfour of Burley, brandishing his sword and his
Bible with fire-eyed fury, trying a fall with the insolent,
gigantic Bothwell at the 'Change-house, and vanquishing him at the
noble battle of Loudonhill; there is Bothwell himself, drawn to
the life, proud, cruel, selfish, profligate, but with the
love-letters of the gentle Alice (written thirty years before),
and his verses to her memory, found in his pocket after his death:
in the same volume of \emph{Old Mortality} is that lone figure, like a
figure in Scripture, of the woman sitting on the stone at the
turning to the mountain, to warn Burley that there is a lion in
his path; and the fawning Claverhouse, beautiful as a panther,
smooth-looking, blood-spotted; and the fanatics, Macbriar and
Mucklewrath, crazed with zeal and sufferings; and the inflexible
Morton, and the faithful Edith, who refused to ``give her hand to
another while her heart was with her lover in the deep and dead
sea.'' And in The Heart of Mid-Lothian we have Effie Deans (that
sweet, faded flower) and Jeanie, her more than sister, and old
David Deans, the patriarch of St. Leonard's Crags, and Butler, and
Dumbiedikes, eloquent in his silence, and Mr. Bartoline
Saddle-tree and his prudent helpmate, and Porteous swinging in the
wind, and Madge Wildfire, full of finery and madness, and her
ghastly mother.\textemdash Again, there is Meg Merrilies, standing on her
rock, stretched on her bier with ``her head to the east,'' and Dirk
Hatterick (equal to Shakespear's Master Barnardine), and Glossin,
the soul of an attorney, and Dandy Dinmont, with his terrier-pack
and his pony Dumple, and the fiery Colonel Mannering, and the
modish old counsellor Pleydell, and Dominie Sampson,\footnote{Perhaps the finest scene in all these novels, is that
where the Dominie meets his pupil, Miss Lucy, the morning after
her brother's arrival.} and Rob
Roy (like the eagle in his eyry), and Baillie Nicol Jarvie, and
the inimitable Major Galbraith, and Rashleigh Osbaldistone, and
Die Vernon, the best of secret-keepers; and in the \emph{Antiquary},
the ingenious and abstruse Mr. Jonathan Oldbuck, and the old
beadsman Edie Ochiltree, and that preternatural figure of old
Edith Elspeith, a living shadow, in whom the lamp of life had been
long extinguished, had it not been fed by remorse and
thick-coming recollections; and that striking picture of the
effects of feudal tyranny and fiendish pride, the unhappy Earl of
Glenallan; and the Black Dwarf, and his friend Habbie of the
Heughfoot (the cheerful hunter), and his cousin Grace Armstrong,
fresh and laughing like the morning; and the \emph{Children of the
Mint}, and the baying of the blood-hound that tracks their steps
at a distance (the hollow echoes are in our ears now), and Amy and
her hapless love, and the villain Varney, and the deep voice of
George of Douglas\textemdash and the immoveable Balafre, and Master Oliver
the Barber in Quentin Durward\textemdash and the quaint humour of the
Fortunes of Nigel, and the comic spirit of Peveril of the
Peak\textemdash and the fine old English romance of Ivanhoe. What a list of
names! What a host of associations! What a thing is human life!
What a power is that of genius! What a world of thought and
feeling is thus rescued from oblivion! How many hours of heartfelt
satisfaction has our author given to the gay and thoughtless! How
many sad hearts has he soothed in pain and solitude! It is no
wonder that the public repay with lengthened applause and
gratitude the pleasure they receive. He writes as fast as they can
read, and he does not write himself down. He is always in the
public eye, and we do not tire of him. His worst is better than
any other person's best. His \emph{backgrounds} (and his later works
are little else but back-grounds capitally made out) are more
attractive than the principal figures and most complicated actions
of other writers. His works (taken together) are almost like a new
edition of human nature.  This is indeed to be an author!}

The political bearing of the \emph{Scotch Novels} has been a
considerable recommendation to them. They are a relief to the
mind, rarefied as it has been with modern philosophy, and heated
with ultra-radicalism. At a time also, when we bid fair to revive
the principles of the Stuarts, it is interesting to bring us
acquainted with their persons and misfortunes. The candour of Sir
Walter's historic pen levels our bristling prejudices on this
score, and sees fair play between Roundheads and Cavaliers,
between Protestant and Papist. He is a writer reconciling all the
diversities of human nature to the reader. He does not enter into
the distinctions of hostile sects or parties, but treats of the
strength or the infirmity of the human mind, of the virtues or
vices of the human breast, as they are to be found blended in the
whole race of mankind. Nothing can shew more handsomely or be more
gallantly executed. There was a talk at one time that our author
was about to take Guy Faux for the subject of one of his novels,
in order to put a more liberal and humane construction on the
Gunpowder Plot than our ``No Popery'' prejudices have hitherto
permitted. Sir Walter is a professed \emph{clarifier} of the age from
the vulgar and still lurking old-English antipathy to Popery and
Slavery. Through some odd process of \emph{servile} logic, it should
seem, that in restoring the claims of the Stuarts by the courtesy
of romance, the House of Brunswick are more firmly seated in point
of fact, and the Bourbons, by collateral reasoning, become
legitimate! In any other point of view, we cannot possibly
conceive how Sir Walter imagines ``he has done something to revive
the declining spirit of loyalty'' by these novels. His loyalty is
founded on would-be treason: he props the actual throne by the
shadow of rebellion. Does he really think of making us enamoured
of the ``good old times'' by the faithful and harrowing portraits he
has drawn of them? Would he carry us back to the early stages of
barbarism, of clanship, of the feudal system as ``a consummation
devoutly to be wished?'' Is he infatuated enough, or does he so
dote and drivel over his own slothful and self-willed prejudices,
as to believe that he will make a single convert to the beauty of
Legitimacy, that is, of lawless power and savage bigotry, when he
himself is obliged to apologise for the horrors he describes, and
even render his descriptions credible to the modern reader by
referring to the authentic history of these delectable times?\footnote{``And here we cannot but think it necessary to offer
some better proof than the incidents of an idle tale, to vindicate
the melancholy representation of manners which has been just laid
before the reader. It is grievous to think that those valiant
Barons, to whose stand against the crown the liberties of England
were indebted for their existence, should themselves have been
such dreadful oppressors, and capable of excesses, contrary not
only to the laws of England, but to those of nature and
humanity. But alas! we have only to extract from the industrious
Henry one of those numerous passages which he has collected from
contemporary historians, to prove that fiction itself can hardly
reach the dark reality of the horrors of the period.''
The description given by the author of the Saxon Chronicle of the
cruelties exercised in the reign of King Stephen by the great
barons and lords of castles, who were all Normans, affords a
strong proof of the excesses of which they were capable when their
passions were inflamed.  'They grievously oppressed the poor
people by building castles; and when they were built, they filled
them with wicked men or rather devils, who seized both men and
women who they imagined had any money, threw them into prison, and
put them to more cruel tortures than the martyrs ever
endured. They suffocated some in mud, and suspended others by the
feet, or the head, or the thumbs, kindling fires below them. They
squeezed the heads of some with knotted cords till they pierced
their brains, while they threw others into dungeons swarming with
serpents, snakes, and toads.' But it would be cruel to put the
reader to the pain of perusing the remainder of the
description.\textemdash \emph{Henry's Hist}. edit. 1805, vol.  vii. p. 346.}
He is indeed so besotted as to the moral of his own story, that he
has even the blindness to go out of his way to have a fling at
\emph{flints} and \emph{dungs} (the contemptible ingredients, as he would
have us believe, of a modern rabble) at the very time when he is
describing a mob of the twelfth century\textemdash a mob (one should think)
after the writer's own heart, without one particle of modern
philosophy or revolutionary politics in their composition, who
were to a man, to a hair, just what priests, and kings, and nobles
\emph{let} them be, and who were collected to witness (a spectacle
proper to the times) the burning of the lovely Rebecca at a stake
for a sorceress, because she was a Jewess, beautiful and innocent,
and the consequent victim of insane bigotry and unbridled
profligacy. And it is at this moment (when the heart is kindled
and bursting with indignation at the revolting abuses of
self-constituted power) that Sir Walter \emph{stops the press} to have
a sneer at the people, and to put a spoke (as he thinks) in the
wheel of upstart innovation! This is what he ``calls backing his
friends''\textemdash it is thus he administers charms and philtres to our
love of Legitimacy, makes us conceive a horror of all reform,
civil, political, or religious, and would fain put down the
\emph{Spirit of the Age}. The author of Waverley might just as well get
up and make a speech at a dinner at Edinburgh, abusing
Mr. Mac-Adam for his improvements in the roads, on the ground that
they were nearly \emph{impassable} in many places ``sixty years since;''
or object to Mr. Peel's Police-Bill, by insisting that
Hounslow-Heath was formerly a scene of greater interest and terror
to highwaymen and travellers, and cut a greater figure in the
Newgate-Calendar than it does at present.\textemdash Oh! Wickliff, Luther,
Hampden, Sidney, Somers, mistaken Whigs, and thoughtless Reformers
in religion and politics, and all ye, whether poets or
philosophers, heroes or sages, inventors of arts or sciences,
patriots, benefactors of the human race, enlighteners and
civilisers of the world, who have (so far) reduced opinion to
reason, and power to law, who are the cause that we no longer burn
witches and heretics at slow fires, that the thumb-screws are no
longer applied by ghastly, smiling judges, to extort confession of
imputed crimes from sufferers for conscience sake; that men are no
longer strung up like acorns on trees without judge or jury, or
hunted like wild beasts through thickets and glens, who have
abated the cruelty of priests, the pride of nobles, the divinity
of kings in former times; to whom we owe it, that we no longer
wear round our necks the collar of Gurth the swineherd, and of
Wamba the jester; that the castles of great lords are no longer
the dens of banditti, from whence they issue with fire and sword,
to lay waste the land; that we no longer expire in loathsome
dungeons without knowing the cause, or have our right hands struck
off for raising them in self-defence against wanton insult; that
we can sleep without fear of being burnt in our beds, or travel
without making our wills; that no Amy Robsarts are thrown down
trap-doors by Richard Varneys with impunity; that no Red Reiver of
Westburn-Flat sets fire to peaceful cottages; that no Claverhouse
signs cold-blooded death-warrants in sport; that we have no
Tristan the Hermit, or Petit- Andrè, crawling near us, like
spiders, and making our flesh creep, and our hearts sicken within
us at every moment of our lives\textemdash ye who have produced this change
in the face of nature and society, return to earth once more, and
beg pardon of Sir Walter and his patrons, who sigh at not being
able to undo all that you have done! Leaving this question, there
are two other remarks which we wished to make on the Novels. The
one was, to express our admiration at the good-nature of the
mottos, in which the author has taken occasion to remember and
quote almost every living author (whether illustrious or obscure)
but himself\textemdash an indirect argument in favour of the general opinion
as to the source from which they spring\textemdash and the other was, to
hint our astonishment at the innumerable and incessant in-stances
of bad and slovenly English in them, more, we believe, than in any
other works now printed. We should think the writer could not
possibly read the manuscript after he has once written it, or
overlook the press.

If there were a writer, who ``born for the universe''\pcdash{1}
\begin{quotation}
  ``\pcdash{3}Narrow'd his mind, \\
  And to party gave up what was meant for mankind\textemdash''
\end{quotation}
who, from the height of his genius looking abroad into nature, and
scanning the recesses of the human heart, ``winked and shut his
apprehension up'' to every thought or purpose that tended to the
future good of mankind\textemdash who, raised by affluence, the reward of
successful industry, and by the voice of fame above the want of
any but the most honourable patronage, stooped to the unworthy
arts of adulation, and abetted the views of the great with the
pettifogging feelings of the meanest dependant on office\textemdash who,
having secured the admiration of the public (with the probable
reversion of immortality), shewed no respect for himself, for that
genius that had raised him to distinction, for that nature which
he trampled under foot\textemdash who, amiable, frank, friendly, manly in
private life, was seized with the dotage of age and the fury of a
woman, the instant politics were concerned\textemdash who reserved all his
candour and comprehensiveness of view for history, and vented his
littleness, pique, resentment, bigotry, and intolerance on his
contemporaries\textemdash who took the wrong side, and defended it by unfair
means\textemdash who, the moment his own interest or the prejudices of
others interfered, seemed to forget all that was due to the pride
of intellect, to the sense of manhood\textemdash who, praised, admired by
men of all parties alike, repaid the public liberality by striking
a secret and envenomed blow at the reputation of every one who was
not the ready tool of power\textemdash who strewed the slime of rankling
malice and mercenary scorn over the bud and promise of genius,
because it was not fostered in the hot-bed of corruption, or
warped by the trammels of servility\textemdash who supported the worst
abuses of authority in the worst spirit\textemdash who joined a gang of
desperadoes to spread calumny, contempt, infamy, wherever they
were merited by honesty or talent on a different side\textemdash who
officiously undertook to decide public questions by private
insinuations, to prop the throne by nicknames, and the altar by
lies\textemdash who being (by common consent) the finest, the most humane
and accomplished writer of his age, associated himself with and
encouraged the lowest panders of a venal press; deluging,
nauseating the public mind with the offal and garbage of
Billingsgate abuse and vulgar \emph{slang}; shewing no remorse, no
relenting or compassion towards the victims of this nefarious and
organized system of party-proscription, carried on under the mask
of literary criticism and fair discussion, insulting the
misfortunes of some, and trampling on the early grave of others\textemdash 
\begin{quote}
  ``Who would not grieve if such a man there be? \\
  Who would not weep if Atticus were he?''
\end{quote}
But we believe there is no other age or country of the world (but
ours), in which such genius could have been so degraded!
