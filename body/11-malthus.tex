\chapter[Mr. Malthus]{mr. malthus}

Mr. Malthus may be considered as one of those rare and fortunate
writers who have attained a \emph{scientific} reputation in
questions of moral and political philosophy. His name undoubtedly
stands very high in the present age, and will in all probability
go down to posterity with more or less of renown or obloquy. It
was said by a person well qualified to judge both from strength
and candour of mind, that ``it would take a thousand years at
least to answer his work on Population.'' He has certainly thrown
a new light on that question, and changed the aspect of political
economy in a decided and material point of view\textemdash whether
he has not also endeavoured to spread a gloom over the hopes and
more sanguine speculations of man, and to cast a slur upon the
face of nature, is another question. There is this to be said for
Mr. Malthus, that in speaking of him, one knows what one is
talking about. He is something beyond a mere name\textemdash one
has not to \emph{beat the bush} about his talents, his
attainments, his vast reputation, and leave off without knowing
what it all amounts to\textemdash he is not one of those great
men, who set themselves off and strut and fret an hour upon the
stage, during a day-dream of popularity, with the ornaments and
jewels borrowed from the common stock, to which nothing but their
vanity and presumption gives them the least individual
claim\textemdash he has dug into the mine of truth, and brought up
ore mixed with dross! In weighing his merits we come at once to
the question of what he has done or failed to do. It is a specific
claim that he sets up. When we speak of Mr. Malthus, we mean the
\emph{Essay on Population}; and when we mention the Essay on
Population, we mean a distinct leading proposition, that stands
out intelligibly from all trashy pretence, and is a ground on
which to fix the levers that may move the world, backwards or
forwards. He has not left opinion where he found it; he has
advanced or given it a wrong bias, or thrown a stumbling-block in
its way. In a word, his name is not stuck, like so many others, in
the firmament of reputation, nobody knows why, inscribed in great
letters, and with a transparency of \textsc{Talents, Genius,
Learning} blazing round it\textemdash it is tantamount to an idea,
it is identified with a principle, it means that \emph{the
population cannot go on perpetually increasing without pressing on
the limits of the means of subsistence, and that a check of some
kind or other must, sooner or later, be opposed to it}. This is
the essence of the doctrine which Mr. Malthus has been the first
to bring into general notice, and as we think, to establish beyond
the fear of contradiction. Admitting then as we do the prominence
and the value of his claims to public attention, it yet remains a
question, how far those claims are (as to the talent displayed in
them) strictly original; how far (as to the logical accuracy with
which he has treated the subject) he has introduced foreign and
doubtful matter into it; and how far (as to the spirit in which he
has conducted his inquiries, and applied a general principle to
particular objects) he has only drawn fair and inevitable
conclusions from it, or endeavoured to tamper with and wrest it to
sinister and servile purposes. A writer who shrinks from following
up a well-founded principle into its untoward consequences from
timidity or false delicacy, is not worthy of the name of a
philosopher: a writer who assumes the garb of candour and an
inflexible love of truth to garble and pervert it, to crouch to
power and pander to prejudice, deserves a worse title than that of
a sophist!

Mr. Malthus's first octavo volume on this subject (published in
the year 1798) was intended as an answer to Mr. Godwin's
\emph{Enquiry concerning Political Justice}. It was well got up
for the purpose, and had an immediate effect. It was what in the
language of the ring is called \emph{a facer}. It made Mr. Godwin
and the other advocates of Modern Philosophy look about them. It
may be almost doubted whether Mr. Malthus was in the first
instance serious in many things that he threw out, or whether he
did not hazard the whole as an amusing and extreme paradox, which
might puzzle the reader as it had done himself in an idle moment,
but to which no practical consequence whatever could attach. This
state of mind would probably continue till the irritation of
enemies and the encouragement of friends convinced him that what
he had at first exhibited as an idle fancy was in fact a very
valuable discovery, or ``like the toad ugly and venomous, had yet
a precious jewel in its head.'' Such a supposition would at least
account for some things in the original Essay, which scarcely any
writer would venture upon, except as professed exercises of
ingenuity, and which have been since in part retracted. But a
wrong bias was thus given, and the author's theory was thus
rendered warped, disjointed, and sophistical from the very outset.

Nothing could in fact be more illogical (not to say absurd) than
the whole of Mr. Malthus's reasoning applied as an answer
(\emph{par excellence}) to Mr. Godwin's book, or to the theories
of other Utopian philosophers.  Mr. Godwin was not singular, but
was kept in countenance by many authorities, both ancient and
modern, in supposing a state of society possible in which the
passions and wills of individuals would be conformed to the
general good, in which the knowledge of the best means of
promoting human welfare and the desire of contributing to it would
banish vice and misery from the world, and in which, the
stumbling-blocks of ignorance, of selfishness, and the indulgence
of gross appetite being removed, all things would move on by the
mere impulse of wisdom and virtue, to still higher and higher
degrees of perfection and happiness. Compared with the lamentable
and gross deficiencies of existing institutions, such a view of
futurity as barely possible could not fail to allure the gaze and
tempt the aspiring thoughts of the philanthropist and the
philosopher: the hopes and the imaginations of speculative men
could not but rush forward into this ideal world as into a
\emph{vacuum} of good; and from ``the mighty stream of tendency''
(as Mr. Wordsworth in the cant of the day calls it,) there was
danger that the proud monuments of time-hallowed institutions,
that the strong-holds of power and corruption, that ``the
Corinthian capitals of polished society,'' with the base and
pediments, might be overthrown and swept away as by a
hurricane. There were not wanting persons whose ignorance, whose
fears, whose pride, or whose prejudices contemplated such an
alternative with horror; and who would naturally feel no small
obligation to the man who should relieve their apprehensions from
the stunning roar of this mighty change of opinion that thundered
at a distance, and should be able, by some logical apparatus or
unexpected turn of the argument, to prevent the vessel of the
state from being hurried forward with the progress of improvement,
and dashed in pieces down the tremendous precipice of human
perfectibility. Then comes Mr.  Malthus forward with the
geometrical and arithmetical ratios in his hands, and holds them
out to his affrighted contemporaries as the only means of
salvation. ``For'' (so argued the author of the Essay) ``let the
principles of Mr. Godwin's Enquiry and of other similar works be
carried literally and completely into effect; let every corruption
and abuse of power be entirely got rid of; let virtue, knowledge,
and civilization be advanced to the greatest height that these
visionary reformers would suppose; let the passions and appetites
be subjected to the utmost control of reason and influence of
public opinion: grant them, in a word, all that they ask, and the
more completely their views are realized, the sooner will they be
overthrown again, and the more inevitable and fatal will be the
catastrophe. For the principle of population will still prevail,
and from the comfort, ease, and plenty that will abound, will
receive an increasing force and \emph{impetus}; the number of
mouths to be fed will have no limit, but the food that is to
supply them cannot keep pace with the demand for it; we must come
to a stop somewhere, even though each square yard, by extreme
improvements in cultivation, could maintain its man: in this state
of things there will be no remedy, the wholesome checks of vice
and misery (which have hitherto kept this principle within bounds)
will have been done away; the voice of reason will be unheard; the
passions only will bear sway; famine, distress, havoc, and dismay
will spread around; hatred, violence, war, and bloodshed will be
the infallible consequence, and from the pinnacle of happiness,
peace, refinement, and social advantage, we shall be hurled once
more into a profounder abyss of misery, want, and barbarism than
ever, by the sole operation of the principle of
population!''\textemdash Such is a brief abstract of the argument
of the Essay.  Can any thing be less conclusive, a more complete
fallacy and \emph{petitio principii}? Mr. Malthus concedes, he
assumes a state of perfectibility, such as his opponents imagined,
in which the general good is to obtain the entire mastery of
individual interests, and reason of gross appetites and passions;
and then he argues that such a perfect structure of society will
fall by its own weight, or rather be undermined by the principle
of population, because in the highest possible state of the
subjugation of the passions to reason, they will be absolutely
lawless and unchecked, and because as men become enlightened,
quick sighted and public-spirited, they will shew themselves
utterly blind to the consequences of their actions, utterly
indifferent to their own well-being and that of all succeeding
generations, whose fate is placed in their hands. This we conceive
to be the boldest paralogism that ever was offered to the world,
or palmed upon willing credulity. Against whatever other scheme of
reform this objection might be valid, the one it was brought
expressly to overturn was impregnable against it, invulnerable to
its slightest graze. Say that the Utopian reasoners are
visionaries, unfounded; that the state of virtue and knowledge
they suppose, in which reason shall have become all-in-all, can
never take place, that it is inconsistent with the nature of man
and with all experience, well and good\textemdash but to say that
society will have attained this high and ``palmy state,'' that
reason will have become the master- key to all our motives, and
that when arrived at its greatest power it will cease to act at
all, but will fall down dead, inert, and senseless before the
principle of population, is an opinion which one would think few
people would choose to advance or assent to, without strong
inducements for maintaining or believing it.

The fact, however, is, that Mr. Malthus found this argument entire
(the principle and the application of it) in an obscure and almost
forgotten work published about the middle of the last century,
entitled \emph{Various Prospects of Mankind, Nature, and
Providence}, by a Scotch gentleman of the name of Wallace. The
chapter in this work on the Principle of Population, considered as
a bar to all ultimate views of human improvement, was probably
written to amuse an idle hour, or read as a paper to exercise the
wits of some literary society in the Northern capital, and no
farther responsibility or importance annexed to it. Mr.  Malthus,
by adopting and setting his name to it, has given it sufficient
currency and effect. It sometimes happens that one writer is the
first to discover a certain principle or lay down a given
observation, and that another makes an application of, or draws a
remote or an immediate inference from it, totally unforeseen by
the first, and from which, in all probability, he might have
widely dissented. But this is not so in the present
instance. Mr. Malthus has borrowed (perhaps without consciousness,
at any rate without acknowledgment) both the preliminary
statement, that the increase in the supply of food ``from a
limited earth and a limited fertility'' must have an end, while
the tendency to increase in the principle of population has none,
without some external and forcible restraint on it, and the
subsequent use made of this statement as an insuperable bar to all
schemes of Utopian or progressive improvement\textemdash both
these he has borrowed (whole) from Wallace, with all their
imperfections on their heads, and has added more and greater ones
to them out of his own store. In order to produce something of a
startling and dramatic effect, he has strained a point or two. In
order to quell and frighten away the bugbear of Modern Philosophy,
he was obliged to make a sort of monster of the principle of
population, which was brought into the field against it, and which
was to swallow it up quick. No half-measures, no middle course of
reasoning would do. With a view to meet the highest possible power
of reason in the new order of things, Mr. Malthus saw the
necessity of giving the greatest possible physical weight to the
antagonist principle, and he accordingly lays it down that its
operation is mechanical and irresistible. He premises these two
propositions as the basis of all his reasoning, 1. \emph{That food
is necessary to man}; 2. \emph{That the desire to propagate the
species is an equally indispensable law of our
existence}:\textemdash thus making it appear that these two wants
or impulses are equal and coordinate principles of action. If this
double statement had been true, the whole scope and structure of
his reasoning (as hostile to human hopes and sanguine
speculations) would have been irrefragable; but as it is not true,
the whole (in that view) falls to the ground. According to
Mr. Malthus's octavo edition, the sexual passion is as necessary
to be gratified as the appetite of hunger, and a man can no more
exist without propagating his species than he can live without
eating. Were it so, neither of these passions would admit of any
excuses, any delay, any restraint from reason or foresight; and
the only checks to the principle of population must be vice and
misery. The argument would be triumphant and complete.  But there
is no analogy, no parity in the two cases, such as our author here
assumes. No man can live for any length of time without food; many
persons live all their lives without gratifying the other sense.
The longer the craving after food is unsatisfied, the more
violent, imperious, and uncontroulable the desire becomes; whereas
the longer the gratification of the sexual passion is resisted,
the greater force does habit and resolution acquire over it; and,
generally speaking, it is a well-known fact, attested by all
observation and history, that this latter passion is subject more
or less to controul from personal feelings and character, from
public opinions and the institutions of society, so as to lead
either to a lawful and regulated indulgence, or to partial or
total abstinence, according to the dictates of \emph{moral
restraint}, which latter check to the inordinate excesses and
unheard-of consequences of the principle of population, our
author, having no longer an extreme case to make out, admits and
is willing to patronize in addition to the two former and
exclusive ones of \emph{vice and misery}, in the second and
remaining editions of his work. Mr. Malthus has shewn some
awkwardness or even reluctance in softening down the harshness of
his first peremptory decision. He sometimes grants his grand
exception cordially, proceeds to argue stoutly, and to try
conclusions upon it; at other times he seems disposed to cavil
about or retract it:\textemdash ``the influence of moral restraint
is very inconsiderable, or none at all.'' It is indeed difficult
(more particularly for so formal and nice a reasoner as
Mr. Malthus) to piece such contradictions plausibly or gracefully
together. We wonder how \emph{he} manages it\textemdash how
\emph{any one} should attempt it! The whole question, the
\emph{gist} of the argument of his early volume turned upon this,
``Whether vice and misery were the \emph{only} actual or possible
checks to the principle of population?'' He then said they were,
and farewell to building castles in the air: he now says that
\emph{moral restraint} is to be coupled with these, and that its
influence depends greatly on the state of laws and
manners\textemdash and Utopia stands where it did, a great way off
indeed, but not turned \emph{topsy-turvy} by our magician's wand!
Should we ever arrive there, that is, attain to a state of
\emph{perfect moral restraint}, we shall not be driven headlong
back into Epicurus's stye for want of the only possible checks to
population, \emph{vice and misery}; and in proportion as we
advance that way, that is, as the influence of moral restraint is
extended, the necessity for vice and misery will be diminished,
instead of being increased according to the first alarm given by
the Essay. Again, the advance of civilization and of population in
consequence with the same degree of moral restraint (as there
exists in England at this present time, for instance) is a good,
and not an evil\textemdash but this does not appear from the
Essay. The Essay shews that population is not (as had been
sometimes taken for granted) an abstract and unqualified good; but
it led many persons to suppose that it was an abstract and
unqualified evil, to be checked only by vice and misery, and
producing, according to its encouragement a greater quantity of
vice and misery; and this error the author has not been at
sufficient pains to do away. Another thing, in which Mr. Malthus
attempted to \emph{clench} Wallace's argument, was in giving to
the disproportionate power of increase in the principle of
population and the supply of food a mathematical form, or reducing
it to the arithmetical and geometrical ratios, in which we believe
Mr. Malthus is now generally admitted, even by his friends and
admirers, to have been wrong. There is evidently no inherent
difference in the principle of increase in food or population;
since a grain of corn, for example, will propagate and multiply
itself much faster even than the human species.  A bushel of wheat
will sow a field; that field will furnish seed for twenty
others. So that the limit to the means of subsistence is only the
want of room to raise it in, or, as Wallace expresses it, ``a
limited fertility and a limited earth.'' Up to the point where the
earth or any given country is fully occupied or cultivated, the
means of subsistence naturally increase in a geometrical ratio,
and will more than keep pace with the natural and unrestrained
progress of population; and beyond that point, they do not go on
increasing even in Mr. Malthus's arithmetical ratio, but are
stationary or nearly so. So far, then, is this proportion from
being universally and mathematically true, that in no part of the
world or state of society does it hold good. But our theorist, by
laying down this double ratio as a law of nature, gains this
advantage, that at all times it seems as if, whether in new or
old-peopled countries, in fertile or barren soils, the population
was pressing hard on the means of subsistence; and again, it seems
as if the evil increased with the progress of improvement and
civilization; for if you cast your eye at the scale which is
supposed to be calculated upon true and infallible \emph{data},
you find that when the population is at 8, the means of
subsistence are at 4; so that here there is only a \emph{deficit}
of one half; but when it is at 32, they have only got to 6, so
that here there is a difference of 26 in 32, and so on in
proportion; the farther we proceed, the more enormous is the mass
of vice and misery we must undergo, as a consequence of the
natural excess of the population over the means of subsistence and
as a salutary check to its farther desolating progress. The
mathematical Table, placed at the front of the Essay, therefore
leads to a secret suspicion or a bare-faced assumption, that we
ought in mere kindness and compassion to give every sort of
indirect and under-hand encouragement (to say the least) to the
providential checks of vice and misery; as the sooner we arrest
this formidable and paramount evil in its course, the less
opportunity we leave it of doing incalculable
mischief. Accordingly, whenever there is the least talk of
colonizing new countries, of extending the population, or adding
to social comforts and improvements, Mr. Malthus conjures up his
double ratios, and insists on the alarming results of advancing
them a single step forward in the series. By the same rule, it
would be better to return at once to a state of barbarism; and to
take the benefit of acorns and scuttle-fish, as a security against
the luxuries and wants of civilized life. But it is not our
ingenious author's wish to hint at or recommend any alterations in
existing institutions; and he is therefore silent on that
unpalatable part of the subject and natural inference from his
principles.

Mr. Malthus's ``gospel is preached to the poor.'' He lectures them
on economy, on morality, the regulation of their passions (which,
he says, at other times, are amenable to no restraint) and on the
ungracious topic, that ``the laws of nature, which are the laws of
God, have doomed them and their families to starve for want of a
right to the smallest portion of food beyond what their labour
will supply, or some charitable hand may hold out in compassion.''
This is illiberal, and it is not philosophical. The laws of nature
or of God, to which the author appeals, are no other than a
limited fertility and a limited earth.  Within those bounds, the
rest is regulated by the laws of man. The division of the produce
of the soil, the price of labour, the relief afforded to the poor,
are matters of human arrangement: while any charitable hand can
extend relief, it is a proof that the means of subsistence are not
exhausted in themselves, that ``the tables are not full!''
Mr. Malthus says that the laws of nature, which are the laws of
God, have rendered that relief physically impossible; and yet he
would abrogate the poor-laws by an act of the legislature, in
order to take away that \emph{impossible} relief, which the laws
of God deny, and which the laws of man \emph{actually} afford. We
cannot think that this view of his subject, which is prominent and
dwelt on at great length and with much pertinacity, is dictated
either by rigid logic or melting charity! A labouring man is not
allowed to knock down a hare or a partridge that spoils his
garden: a country-squire keeps a pack of hounds: a lady of quality
rides out with a footman behind her, on two sleek, well-fed
horses. We have not a word to say against all this as exemplifying
the spirit of the English Constitution, as a part of the law of
the land, or as an artful distribution of light and shade in the
social picture; but if any one insists at the same time that ``the
laws of nature, which are the laws of God, have doomed the poor
and their families to starve,'' because the principle of
population has encroached upon and swallowed up the means of
subsistence, so that not a mouthful of food is left \emph{by the
grinding law of necessity} for the poor, we beg leave to deny both
fact and inference\textemdash and we put it to Mr. Malthus whether
we are not, in strictness, justified in doing so?

We have, perhaps, said enough to explain our feeling on the
subject of Mr. Malthus's merits and defects. We think he had the
opportunity and the means in his hands of producing a great work
on the principle of population; but we believe he has let it slip
from his having an eye to other things besides that broad and
unexplored question. He wished not merely to advance to the
discovery of certain great and valuable truths, but at the same
time to overthrow certain unfashionable paradoxes by exaggerated
statements\textemdash to curry favour with existing prejudices and
interests by garbled representations. He has, in a word, as it
appears to us on a candid retrospect and without any feelings of
controversial asperity rankling in our minds, sunk the philosopher
and the friend of his species (a character to which he might have
aspired) in the sophist and party-writer. The period at which
Mr. Malthus came forward teemed with answers to Modern Philosophy,
with antidotes to liberty and humanity, with abusive Histories of
the Greek and Roman republics, with fulsome panegyrics on the
Roman Emperors (at the very time when we were reviling Buonaparte
for his strides to universal empire) with the slime and offal of
desperate servility\textemdash and we cannot but consider the
Essay as one of the poisonous ingredients thrown into the cauldron
of Legitimacy ``to make it thick and slab.'' Our author has,
indeed, so far done service to the cause of truth, that he has
counteracted many capital errors formerly prevailing as to the
universal and indiscriminate encouragement of population under all
circumstances; but he has countenanced opposite errors, which if
adopted in theory and practice would be even more mischievous, and
has left it to future philosophers to follow up the principle,
that some check must be provided for the unrestrained progress of
population, into a set of wiser and more humane
consequences. Mr. Godwin has lately attempted an answer to the
Essay (thus giving Mr. Malthus a \emph{Roland for his Oliver}) but
we think he has judged ill in endeavouring to invalidate the
principle, instead of confining himself to point out the
misapplication of it. There is one argument introduced in this
Reply, which will, perhaps, amuse the reader as a sort of
metaphysical puzzle.

``It has sometimes occurred to me whether Mr. Malthus did not
catch the first hint of his geometrical ratio from a curious
passage of Judge Blackstone, on consanguinity, which is as
follows:\textemdash

``The doctrine of lineal consanguinity is sufficiently plain and
obvious; but it is at the first view astonishing to consider the
number of lineal ancestors which every man has within no very
great number of degrees: and so many different bloods is a man
said to contain in his veins, as he hath lineal ancestors. Of
these he hath two in the first ascending degree, his own parents;
he hath four in the second, the parents of his father and the
parents of his mother; he hath eight in the third, the parents of
his two grandfathers and two grandmothers; and by the same rule of
progression, he hath an hundred and twenty-eight in the seventh; a
thousand and twenty-four in the tenth; and at the twentieth
degree, or the distance of twenty generations, every man hath
above a million of ancestors, as common arithmetic will
demonstrate.

``This will seem surprising to those who are unacquainted with the
increasing power of progressive numbers; but is palpably evident
from the following table of a geometrical progression, in which
the first term is 2, and the denominator also 2; or, to speak more
intelligibly, it is evident, for that each of us has two ancestors
in the first degree; the number of which is doubled at every
remove, because each of our ancestors had also two ancestors of
his own.

\begin{center}
\begin{tabular}{*{5}{c}}
\small\emph{Lineal Degrees.} &&&& \small\emph{Number of Ancestors}.\\
1  & \textemdash & \textemdash & \textemdash & 2         \\
2  & \textemdash & \textemdash & \textemdash & 4         \\
3  & \textemdash & \textemdash & \textemdash & 8         \\
4  & \textemdash & \textemdash & \textemdash & 16        \\
5  & \textemdash & \textemdash & \textemdash & 32        \\
6  & \textemdash & \textemdash & \textemdash & 64        \\
7  & \textemdash & \textemdash & \textemdash & 128       \\
8  & \textemdash & \textemdash & \textemdash & 256       \\
9  & \textemdash & \textemdash & \textemdash & 512       \\
10 & \textemdash & \textemdash & \textemdash & 1024      \\
11 & \textemdash & \textemdash & \textemdash & 2048      \\
12 & \textemdash & \textemdash & \textemdash & 4096      \\
13 & \textemdash & \textemdash & \textemdash & 8192      \\
14 & \textemdash & \textemdash & \textemdash & 16,384    \\
15 & \textemdash & \textemdash & \textemdash & 32,768    \\
16 & \textemdash & \textemdash & \textemdash & 65,536    \\
17 & \textemdash & \textemdash & \textemdash & 131,072   \\
18 & \textemdash & \textemdash & \textemdash & 262,144   \\
19 & \textemdash & \textemdash & \textemdash & 524,288   \\
20 & \textemdash & \textemdash & \textemdash & 1,048,576 \\
\end{tabular}
\end{center}

``This argument, however,'' (proceeds Mr. Godwin) ``from Judge
Blackstone of a geometrical progression would much more naturally
apply to Montesquieu's hypothesis of the depopulation of the
world, and prove that the human species is hastening fast to
extinction, than to the purpose for which Mr. Malthus has employed
it. An ingenious sophism might be raised upon it, to shew that the
race of mankind will ultimately terminate in unity. Mr. Malthus,
indeed, should have reflected, that it is much more certain that
every man has had ancestors than that he will have posterity, and
that it is still more doubtful, whether he will have posterity to
twenty or to an indefinite number of generations.''\textemdash
\textsc{Enquiry Concerning Population}, p. 100.

Mr. Malthus's style is correct and elegant; his tone of
controversy mild and gentlemanly; and the care with which he has
brought his facts and documents together, deserves the highest
praise. He has lately quitted his favourite subject of population,
and broke a lance with Mr. Ricardo on the question of rent and
value. The partisans of Mr. Ricardo, who are also the admirers of
Mr. Malthus, say that the usual sagacity of the latter has here
failed him, and that he has shewn himself to be a very illogical
writer. To have said this of him formerly on another ground, was
accounted a heresy and a piece of presumption not easily to be
forgiven. Indeed Mr. Malthus has always been a sort of ``darling
in the public eye,'' whom it was unsafe to meddle with. He has
contrived to make himself as many friends by his attacks on the
schemes of \emph{Human Perfectibility} and on the
\emph{Poor-Laws}, as Mandeville formerly procured enemies by his
attacks on \emph{Human Perfections} and on \emph{Charity-Schools};
and among other instances that we might mention, \emph{Plug}
Pulteney, the celebrated miser, of whom Mr. Burke said on his
having a large estate left him, ``that now it was to be hoped he
would \emph{set up a pocket-handkerchief},'' was so enamoured with
the saving schemes and humane economy of the Essay, that he
desired a friend to find out the author and offer him a church
living! This liberal intention was (by design or accident)
unhappily frustrated.