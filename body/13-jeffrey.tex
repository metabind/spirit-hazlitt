\chapter[Mr. Jeffrey]{mr. jeffrey}

The \emph{Quarterly Review} arose out of the \emph{Edinburgh}, not
as a corollary, but in contradiction to it. An article had
appeared in the latter on Don Pedro Cevallos, which stung the
Tories to the quick by the free way in which it spoke of men and
things, and something must be done to check these \emph{escapades}
of the \emph{Edinburgh}. It was not to be endured that the truth
should \emph{out} in this manner, even occasionally and half in
jest. A startling shock was thus given to established prejudices,
the mask was taken off from grave hypocrisy, and the most serious
consequences were to be apprehended. The persons who wrote in this
Review seemed ``to have their hands full of truths'', and now and
then, in a fit of spleen or gaiety, let some of them fly; and
while this practice continued, it was impossible to say that the
Monarchy or the Hierarchy was safe. Some of the arrows glanced,
others might stick, and in the end prove fatal. It was not the
principles of the \emph{Edinburgh Review}, but the spirit that was
looked at with jealousy and alarm. The principles were by no means
decidedly hostile to existing institutions: but the spirit was
that of fair and free discussion; a field was open to argument and
wit; every question was tried upon its own ostensible merits, and
there was no foul play. The tone was that of a studied
impartiality (which many called \emph{trimming}) or of a sceptical
indifference. This tone of impartiality and indifference, however,
did not at all suit those who profited or existed by abuses, who
breathed the very air of corruption. They know well enough that
``those who are not \emph{for} them are \emph{against} them.''
They wanted a publication impervious alike to truth and candour;
that, hood-winked itself, should lead public opinion blindfold;
that should stick at nothing to serve the turn of a party; that
should be the exclusive organ of prejudice, the sordid tool of
power; that should go the whole length of want of principle in
palliating every dishonest measure, of want of decency in defaming
every honest man; that should prejudge every question, traduce
every opponent; that should give no quarter to fair inquiry or
liberal sentiment; that should be ``ugly all over with
hypocrisy'', and present one foul blotch of servility,
intolerance, falsehood, spite, and ill-manners. The
\emph{Quarterly Review} was accordingly set up.
\begin{verse}
    \vleftofline{``}Sithence no fairy lights, no quickning ray, Nor stir of\\
    pulse, nor object to entice Abroad the spirits; but the\\
    cloister'd heart Sits squat at home, like Pagod in a niche\\
    Obscure!''

\end{verse}
This event was accordingly hailed (and the omen has been
fulfilled!) as a great relief to all those of his Majesty's
subjects who are firmly convinced that the only way to have things
remain exactly as they are is to put a stop to all inquiries
whether they are right or wrong, and that if you cannot answer a
man's arguments, you may at least try to take away his character.

We do not implicitly bow to the political opinions, nor to the
critical decisions of the \emph{Edinburgh Review}; but we must do
justice to the talent with which they are supported, and to the
tone of manly explicitness in which they are
delivered.\footnote{The style of philosophical criticism, which
  has been the boast of the Edinburgh Review, was first introduced
  into the Monthly Review about the year 1796, in a series of
  articles by Mr. William Taylor, of Norwich.} They are eminently
characteristic of the Spirit of the Age; as it is the express
object of the \emph{Quarterly Review} to discountenance and
extinguish that spirit, both in theory and practice. The
\emph{Edinburgh Review} stands upon the ground of opinion; it
asserts the supremacy of intellect: the pre-eminence it claims is
from an acknowledged superiority of talent and information and
literary attainment, and it does not build one tittle of its
influence on ignorance, or prejudice, or authority, or personal
malevolence. It takes up a question, and argues it \emph{pro} and
\emph{con} with great knowledge and boldness and skill; it points
out an absurdity, and runs it down, fairly, and according to the
evidence adduced. In the former case, its conclusions may be
wrong, there may be a bias in the mind of the writer, but he
states the arguments and circumstances on both sides, from which a
judgment is to be formed\textemdash it is not his cue, he has
neither the effrontery nor the meanness to falsify facts or to
suppress objections. In the latter case, or where a vein of
sarcasm or irony is resorted to, the ridicule is not barbed by
some allusion (false or true) to private history; the object of it
has brought the infliction on himself by some literary folly or
political delinquency which is referred to as the understood and
justifiable provocation, instead of being held up to scorn as a
knave for not being a tool, or as a blockhead for thinking for
himself. In the \emph{Edinburgh Review} the talents of those on
the opposite side are always extolled \emph{pleno ore}\textemdash
in the \emph{Quarterly Review} they are denied altogether, and the
justice that is in this way withheld from them is compensated by a
proportionable supply of personal abuse. A man of genius who is a
lord, and who publishes with Mr. Murray, may now and then stand as
good a chance as a lord who is not a man of genius and who
publishes with Messrs. Longman: but that is the utmost extent of
the impartiality of the \emph{Quarterly}. From its account you
would take Lord Byron and Mr.  Stuart Rose for two very pretty
poets; but Mr. Moore's Magdalen Muse is sent to Bridewell without
mercy, to beat hemp in silk-stockings. In the \emph{Quarterly}
nothing is regarded but the political creed or external
circumstances of a writer: in the \emph{Edinburgh} nothing is ever
adverted to but his literary merits. Or if there is a bias of any
kind, it arises from an affectation of magnanimity and candour in
giving heaped measure to those on the aristocratic side in
politics, and in being critically severe on others. Thus Sir
Walter Scott is lauded to the skies for his romantic powers,
without any allusion to his political demerits (as if this would
be compromising the dignity of genius and of criticism by the
introduction of party-spirit)\textemdash while Lord Byron is
called to a grave moral reckoning. There is, however, little of
the cant of morality in the \emph{Edinburgh Review}\textemdash and
it is quite free from that of religion. It keeps to its province,
which is that of criticism\textemdash or to the discussion of
debateable topics, and acquits itself in both with force and
spirit.  This is the natural consequence of the composition of the
two Reviews.  The one appeals with confidence to its own
intellectual resources, to the variety of its topics, to its very
character and existence as a literary journal, which depend on its
setting up no pretensions but those which it can make good by the
talent and ingenuity it can bring to bear upon them\textemdash it
therefore meets every question, whether of a lighter or a graver
cast, on its own grounds; the other \emph{blinks} every question,
for it has no confidence but in \emph{the powers that
  be}\textemdash shuts itself up in the impregnable fastnesses of
authority, or makes some paltry, cowardly attack (under cover of
anonymous criticism) on individuals, or dispenses its award of
merit entirely according to the rank or party of the writer. The
faults of the \emph{Edinburgh Review} arise out of the very
consciousness of critical and logical power. In political
questions it relies too little on the broad basis of liberty and
humanity, enters too much into mere dry formalities, deals too
often in \emph{moot-points}, and descends too readily to a sort of
special-pleading in defence of \emph{home} truths and natural
feelings: in matters of taste and criticism, its tone is sometimes
apt to be supercilious and \emph{cavalier} from its habitual
faculty of analysing defects and beauties according to given
principles, from its quickness in deciding, from its facility in
illustrating its views. In this latter department it has been
guilty of some capital oversights. The chief was in its treatment
of the \emph{Lyrical Ballads} at their first appearance\textemdash
not in its ridicule of their puerilities, but in its denial of
their beauties, because they were included in no school, because
they were reducible to no previous standard or theory of poetical
excellence. For this, however, considerable reparation has been
made by the prompt and liberal spirit that has been shewn in
bringing forward other examples of poetical genius. Its capital
sin, in a doctrinal point of view, has been (we shrewdly suspect)
in the uniform and unqualified encouragement it has bestowed on
Mr. Malthus's system.  We do not mean that the \emph{Edinburgh
  Review} was to join in the general \emph{hue and cry} that was
raised against this writer; but while it asserted the soundness of
many of his arguments, and yielded its assent to the truths he has
divulged, it need not have screened his errors. On this subject
alone we think the \emph{Quarterly} has the advantage of it. But
as the \emph{Quarterly Review} is a mere mass and tissue of
prejudices on all subjects, it is the foible of the
\emph{Edinburgh Review} to affect a somewhat fastidious air of
superiority over prejudices of all kinds, and a determination not
to indulge in any of the amiable weaknesses of our nature, except
as it can give a reason for the faith that is in it.  Luckily, it
is seldom reduced to this alternative: ``reasons'' are with it
``as plenty as blackberries!''

Mr. Jeffrey is the Editor of the \emph{Edinburgh Review,} and is
understood to have contributed nearly a fourth part of the
articles from its commencement. No man is better qualified for
this situation; nor indeed so much so. He is certainly a person in
advance of the age, and yet perfectly fitted both from knowledge
and habits of mind to put a curb upon its rash and headlong
spirit. He is thoroughly acquainted with the progress and
pretensions of modern literature and philosophy; and to this he
adds the natural acuteness and discrimination of the logician with
the habitual caution and coolness of his profession. If the
\emph{Edinburgh Review} may be considered as the organ of or at
all pledged to a party, that party is at least a respectable one,
and is placed in the middle between two extremes. The Editor is
bound to lend a patient hearing to the most paradoxical opinions
and extravagant theories which have resulted in our times from the
``infinite agitation of wit'', but he is disposed to qualify them
by a number of practical objections, of speculative doubts, of
checks and drawbacks, arising out of actual circumstances and
prevailing opinions, or the frailties of human nature.  He has a
great range of knowledge, an incessant activity of mind; but the
suspension of his judgment, the well-balanced moderation of his
sentiments, is the consequence of the very discursiveness of his
reason.  What may be considered as \emph{a commonplace} conclusion
is often the result of a comprehensive view of all the
circumstances of a case. Paradox, violence, nay even originality
of conception is not seldom owing to our dwelling long and
pertinaciously on some one part of a subject, instead of attending
to the whole. Mr. Jeffrey is neither a bigot nor an enthusiast. He
is not the dupe of the prejudices of others, nor of his own. He is
not wedded to any dogma, he is not long the sport of any whim;
before he can settle in any fond or fantastic opinion, another
starts up to match it, like beads on sparkling wine. A too
restless display of talent, a too undisguised statement of all
that can be said for and against a question, is perhaps the great
fault that is to be attributed to him. Where there is so much
power and prejudice to contend with in the opposite scale, it may
be thought that the balance of truth can hardly be held with a
slack or an even hand; and that the infusion of a little more
visionary speculation, of a little more popular indignation into
the great Whig Review would be an advantage both to itself and to
the cause of freedom. Much of this effect is chargeable less on an
Epicurean levity of feeling or on party-trammels, than on real
sanguineness of disposition, and a certain fineness of
professional tact. Our sprightly Scotchman is not of a desponding
and gloomy turn of mind. He argues well for the future hopes of
mankind from the smallest beginnings, watches the slow, gradual,
reluctant growth of liberal views, and smiling sees the aloe of
Reform blossom at the end of a hundred years; while the habitual
subtlety of his mind makes him perceive decided advantages where
vulgar ignorance or passion sees only doubts and difficulty; and a
flaw in an adversary's argument stands him instead of the shout of
a mob, the votes of a majority, or the fate of a pitched
battle. The Editor is satisfied with his own conclusions, and does
not make himself uneasy about the fate of mankind. The issue, he
thinks, will verify his moderate and well-founded
expectations.\textemdash We believe also that late events have
given a more decided turn to Mr.  Jeffrey's mind, and that he
feels that as in the struggle between liberty and slavery, the
views of the one party have been laid bare with their success, so
the exertions on the other side should become more strenuous, and
a more positive stand be made against the avowed and appalling
encroachments of priestcraft and arbitrary power.

The characteristics of Mr. Jeffrey's general style as a writer
correspond, we think, with what we have here stated as the
characteristics of his mind. He is a master of the foils; he makes
an exulting display of the dazzling fence of wit and argument. His
strength consists in great range of knowledge, an equal
familiarity with the principles and the details of a subject, and
in a glancing brilliancy and rapidity of style. Indeed, we doubt
whether the brilliancy of his manner does not resolve itself into
the rapidity, the variety and aptness of his illustrations. His
pen is never at a loss, never stands still; and would dazzle for
this reason alone, like an eye that is ever in motion. Mr. Jeffrey
is far from a flowery or affected writer; he has few tropes or
figures, still less any odd startling thoughts or quaint
innovations in expression:\textemdash but he has a constant supply
of ingenious solutions and pertinent examples; he never proses,
never grows dull, never wears an argument to tatters; and by the
number, the liveliness and facility of his transitions, keeps up
that appearance of vivacity, of novel and sparkling effect, for
which others are too often indebted to singularity of combination
or tinsel ornaments.

It may be discovered, by a nice observer, that Mr. Jeffrey's style
of composition is that of a person accustomed to public
speaking. There is no pause, no meagreness, no inanimateness, but
a flow, a redundance and volubility like that of a stream or of a
rolling-stone. The language is more copious than select, and
sometimes two or three words perform the office of one. This
copiousness and facility is perhaps an advantage in
\emph{extempore} speaking, where no stop or break is allowed in
the discourse, and where any word or any number of words almost is
better than coming to a dead stand; but in written compositions it
gives an air of either too much carelessness or too much
labour. Mr. Jeffrey's excellence, as a public speaker, has
betrayed him into this peculiarity.  He makes fewer \emph{blots}
in addressing an audience than any one we remember to have
heard. There is not a hair's-breadth space between any two of his
words, nor is there a single expression either ill-chosen or out
of its place. He speaks without stopping to take breath, with
ease, with point, with elegance, and without ``spinning the thread
of his verbosity finer than the staple of his argument.'' He may
be said to weave words into any shapes he pleases for use or
ornament, as the glass-blower moulds the vitreous fluid with his
breath; and his sentences shine like glass from their polished
smoothness, and are equally transparent. His style of eloquence,
indeed, is remarkable for neatness, for correctness, and
epigrammatic point; and he has applied this as a standard to his
written compositions, where the very same degree of correctness
and precision produces, from the contrast between writing and
speaking, an agreeable diffuseness, freedom, and animation.
Whenever the Scotch advocate has appeared at the bar of the
English House of Lords, he has been admired by those who were in
the habit of attending to speeches there, as having the greatest
fluency of language and the greatest subtlety of distinction of
any one of the profession.  The law-reporters were as little able
to follow him from the extreme rapidity of his utterance as from
the tenuity and evanescent nature of his reasoning.

Mr. Jeffrey's conversation is equally lively, various, and
instructive.  There is no subject on which he is not \emph{au
  fait}: no company in which he is not ready to scatter his pearls
for sport. Whether it be politics, or poetry, or science, or
anecdote, or wit, or raillery, he takes up his cue without effort,
without preparation, and appears equally incapable of tiring
himself or his hearers. His only difficulty seems to be not to
speak, but to be silent. There is a constitutional buoyancy and
elasticity of mind about him that cannot subside into repose, much
less sink into dulness. There may be more original talkers,
persons who occasionally surprise or interest you more; few, if
any, with a more uninterrupted flow of cheerfulness and animal
spirits, with a greater fund of information, and with fewer
specimens of the \emph{bathos} in their conversation. He is never
absurd, nor has he any favourite points which he is always
bringing forward. It cannot be denied that there is something
bordering on petulance of manner, but it is of that least
offensive kind which may be accounted for from merit and from
success, and implies no exclusive pretensions nor the least
particle of ill-will to others. On the contrary, Mr. Jeffrey is
profuse of his encomiums and admiration of others, but still with
a certain reservation of a right to differ or to blame. He cannot
rest on one side of a question: he is obliged by a mercurial habit
and disposition to vary his point of view.  If he is ever tedious,
it is from an excess of liveliness: he oppresses from a sense of
airy lightness. He is always setting out on a fresh scent: there
are always \emph{relays} of topics; the harness is put to, and he
rattles away as delightfully and as briskly as ever. New causes
are called; he holds a brief in his hand for every possible
question.  This is a fault. Mr. Jeffrey is not obtrusive, is not
impatient of opposition, is not unwilling to be interrupted; but
what is said by another, seems to make no impression on him; he is
bound to dispute, to answer it, as if he was in Court, or as if it
were in a paltry Debating Society, where young beginners were
trying their hands. This is not to maintain a character, or for
want of good-nature\textemdash it is a thoughtless habit. He
cannot help cross-examining a witness, or stating the adverse view
of the question. He listens not to judge, but to reply.  In
consequence of this, you can as little tell the impression your
observations make on him as what weight to assign to
his. Mr. Jeffrey shines in mixed company; he is not good in a
\emph{tete-a-tete}. You can only shew your wisdom or your wit in
general society: but in private your follies or your weaknesses
are not the least interesting topics; and our critic has neither
any of his own to confess, nor does he take delight in hearing
those of others. Indeed in Scotland generally, the display of
personal character, the indulging your whims and humours in the
presence of a friend, is not much encouraged\textemdash every one
there is looked upon in the light of a machine or a collection of
topics. They turn you round like a cylinder to see what use they
can make of you, and drag you into a dispute with as little
ceremony as they would drag out an article from an
Encyclopedia. They criticise every thing, analyse every thing,
argue upon every thing, dogmatise upon every thing; and the bundle
of your habits, feelings, humours, follies and pursuits is
regarded by them no more than a bundle of old clothes. They stop
you in a sentiment by a question or a stare, and cut you short in
a narrative by the time of night. The accomplished and ingenious
person of whom we speak, has been a little infected by the tone of
his countrymen\textemdash he is too didactic, too pugnacious, too
full of electrical shocks, too much like a voltaic battery, and
reposes too little on his own excellent good sense, his own love
of ease, his cordial frankness of disposition and unaffected
candour. He ought to have belonged to us!

The severest of critics (as he has been sometimes termed) is the
best-natured of men. Whatever there may be of wavering or
indecision in Mr. Jeffrey's reasoning, or of harshness in his
critical decisions, in his disposition there is nothing but
simplicity and kindness. He is a person that no one knows without
esteeming, and who both in his public connections and private
friendships, shews the same manly uprightness and unbiassed
independence of spirit. At a distance, in his writings, or even in
his manner, there may be something to excite a little uneasiness
and apprehension: in his conduct there is nothing to except
against.  He is a person of strict integrity himself, without
pretence or affectation; and knows how to respect this quality in
others, without prudery or intolerance. He can censure a friend or
a stranger, and serve him effectually at the same time. He
expresses his disapprobation, but not as an excuse for closing up
the avenues of his liberality. He is a Scotchman without one
particle of hypocrisy, of cant, of servility, or selfishness in
his composition. He has not been spoiled by fortune\textemdash has
not been tempted by power\textemdash is firm without violence,
friendly without weakness\textemdash a critic and even-tempered, a
casuist and an honest man\textemdash and amidst the toils of his
profession and the distractions of the world, retains the gaiety,
the unpretending carelessness and simplicity of youth. Mr. Jeffrey
in his person is slight, with a countenance of much expression,
and a voice of great flexibility and acuteness of tone.
