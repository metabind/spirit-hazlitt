\chapter[Mr. Campbell and Mr. Crabbe]
{mr. campbell {\Large and} mr. crabbe}

Mr. Campbell may be said to hold a place (among modern poets) between
Lord Byron and Mr. Rogers. With much of the glossy splendour, the
pointed vigour, and romantic interest of the one, he possesses the
fastidious refinement, the classic elegance of the other. Mr. Rogers, as
a writer, is too effeminate, Lord Byron too extravagant: Mr. Campbell is
neither. The author of the \emph{Pleasures of Memory} polishes his lines till
they sparkle with the most exquisite finish; he attenuates them into the
utmost degree of trembling softness: but we may complain, in spite of
the delicacy and brilliancy of the execution, of a want of strength
and solidity. The author of the \emph{Pleasures of Hope}, with a richer and
deeper vein of thought and imagination, works it out into figures of
equal grace and dazzling beauty, avoiding on the one hand the tinsel of
flimsy affectation, and on the other the vices of a rude and barbarous
negligence. His Pegasus is not a rough, skittish colt, running wild
among the mountains, covered with bur-docks and thistles, nor a tame,
sleek pad, unable to get out of the same ambling pace, but a beautiful
\emph{manege}-horse, full of life and spirit in itself, and subject to the
complete controul of the rider. Mr. Campbell gives scope to his feelings
and his fancy, and embodies them in a noble and naturally interesting
subject; and he at the same time conceives himself called upon (in these
days of critical nicety) to pay the exactest attention to the expression
of each thought, and to modulate each line into the most faultless
harmony. The character of his mind is a lofty and self-scrutinising
ambition, that strives to reconcile the integrity of general design with
the perfect elaboration of each component part, that aims at striking
effect, but is jealous of the means by which this is to be produced.
Our poet is not averse to popularity (nay, he is tremblingly alive to
it)\textemdash but self-respect is the primary law, the indispensable condition
on which it must be obtained. We should dread to point out (even if we
could) a false concord, a mixed metaphor, an imperfect rhyme in any of
Mr. Campbell's productions; for we think that all his fame would hardly
compensate to him for the discovery. He seeks for perfection, and
nothing evidently short of it can satisfy his mind. He is a \emph{high
finisher} in poetry, whose every work must bear inspection, whose
slightest touch is precious\textemdash not a coarse dauber who is contented to
impose on public wonder and credulity by some huge, ill-executed design,
or who endeavours to wear out patience and opposition together by a load
of lumbering, feeble, awkward, improgressive lines\textemdash on the contrary, Mr.
Campbell labours to lend every grace of execution to his subject, while
he borrows his ardour and inspiration from it, and to deserve the
laurels he has earned, by true genius and by true pains. There is an
apparent consciousness of this in most of his writings. He has attained
to great excellence by aiming at the greatest, by a cautious and yet
daring selection of topics, and by studiously (and with a religious
horror) avoiding all those faults which arise from grossness, vulgarity,
haste, and disregard of public opinion. He seizes on the highest point
of eminence, and strives to keep it to himself\textemdash he ``snatches a grace
beyond the reach of art,'' and will not let it go\textemdash he steeps a single
thought or image so deep in the Tyrian dyes of a gorgeous imagination,
that it throws its lustre over a whole page\textemdash every where vivid \emph{ideal}
forms hover (in intense conception) over the poet's verse, which
ascends, like the aloe, to the clouds, with pure flowers at its top. Or
to take an humbler comparison (the pride of genius must sometimes stoop
to the lowliness of criticism) Mr. Campbell's poetry often reminds us of
the purple gilliflower, both for its colour and its scent, its glowing
warmth, its rich, languid, sullen hue,
\begin{verse}
  \vleftofline{``}Yet sweeter than the lids of Juno's eyes, \\
  Or Cytherea's breath!''
\end{verse}
There are those who complain of the little that Mr. Campbell has done
in poetry, and who seem to insinuate that he is deterred by his own
reputation from making any further or higher attempts. But after having
produced two poems that have gone to the heart of a nation, and are
gifts to a world, he may surely linger out the rest of his life in a
dream of immortality. There are moments in our lives so exquisite that
all that remains of them afterwards seems useless and barren; and there
are lines and stanzas in our author's early writings in which he may
be thought to have exhausted all the sweetness and all the essence of
poetry, so that nothing farther was left to his efforts or his ambition.
Happy is it for those few and fortunate worshippers of the Muse (not
a subject of grudging or envy to others) who already enjoy in their
life-time a foretaste of their future fame, who see their names
accompanying them, like a cloud of glory, from youth to age,
\begin{verse}
  \vleftofline{``}And by the vision splendid,\\
  Are on their way attended''\textemdash 
\end{verse}
and who know that they have built a shrine for the thoughts and
feelings, that were most dear to them, in the minds and memories
of other men, till the language which they lisped in childhood is
forgotten, or the human heart shall beat no more!

The \emph{Pleasures of Hope} alone would not have called forth these remarks
from us; but there are passages in the \emph{Gertrude of Wyoming} of so rare
and ripe a beauty, that they challenge, as they exceed all praise.
Such, for instance, is the following peerless description of Gertrude's
childhood:\textemdash 
\begin{verse} % verse 90-120 stanza xi-xiii 1809
  \vleftofline{``}A lov'd bequest\textemdash and I may half impart\textemdash\\
  To those that feel the strong paternal tie,\\
  How like a new existence in his heart\\
  That living flow'r uprose beneath his eye,\\
  Dear as she was, from cherub infancy,\\
  From hours when she would round his garden play,\\
  To time when as the ripening years went by,\\
  Her lovely mind could culture well repay,\\
  And more engaging grew from pleasing day to day.\\[\stanzaskip]

  \vleftofline{``}I may not paint those thousand infant charms\\
  (Unconscious fascination, undesign'd!)\\
  The orison repeated in his arms,\\
  For God to bless her sire and all mankind;\\
  The book, the bosom on his knee reclined,\\
  Or how sweet fairy-lore he heard her con\\
  (The play-mate ere the teacher of her mind):\\
  All uncompanion'd else her years had gone,\\
  Till now in Gertrude's eyes their ninth blue summer shone.\\[\stanzaskip]

  \vleftofline{``}And summer was the tide, and sweet the hour,\\
  When sire and daughter saw, with fleet descent,\\
  An Indian from his bark approach their bow'r,\\
  Of buskin'd limb and swarthy lineament;\\
  The red wild feathers on his brow were blent,\\
  And bracelets bound the arm that help'd to light\\
  A boy, who seem'd, as he beside him went,\\
  Of Christian vesture and complexion bright,\\
  Led by his dusky guide like morning brought by night.''\\
\end{verse}
In the foregoing stanzas we particularly admire the line\textemdash 
\begin{verse}
  ``Till now in Gertrude's eyes their ninth blue summer shone.''
\end{verse}
It appears to us like the ecstatic union of natural beauty and poetic
fancy, and in its playful sublimity resembles the azure canopy mirrored
in the smiling waters, bright, liquid, serene, heavenly! A great outcry,
we know, has prevailed for some time past against poetic diction and
affected conceits, and, to a certain degree, we go along with it; but
this must not prevent us from feeling the thrill of pleasure when we see
beauty linked to beauty, like kindred flame to flame, or from applauding
the voluptuous fancy that raises and adorns the fairy fabric of thought,
that nature has begun! Pleasure is ``scattered in stray-gifts o'er the
earth''\textemdash beauty streaks the ``famous poet's page'' in occasional lines of
inconceivable brightness; and wherever this is the case, no splenetic
censures or ``jealous leer malign,'' no idle theories or cold indifference
should hinder us from greeting it with rapture.\textemdash There are other parts
of this poem equally delightful, in which there is a light startling as
the red-bird's wing; a perfume like that of the magnolia; a music
like the murmuring of pathless woods or of the everlasting ocean. We
conceive, however, that Mr. Campbell excels chiefly in sentiment and
imagery. The story moves slow, and is mechanically conducted, and rather
resembles a Scotch canal carried over lengthened aqueducts and with a
number of \emph{locks} in it, than one of those rivers that sweep in their
majestic course, broad and full, over Transatlantic plains and lose
themselves in rolling gulfs, or thunder down lofty precipices. But in
the centre, the inmost recesses of our poet's heart, the pearly dew of
sensibility is distilled and collects, like the diamond in the mine, and
the structure of his fame rests on the crystal columns of a polished
imagination. We prefer the \emph{Gertrude} to the \emph{Pleasures of Hope},
because with perhaps less brilliancy, there is more of tenderness and
natural imagery in the former. In the \emph{Pleasures of Hope} Mr. Campbell
had not completely emancipated himself from the trammels of the more
artificial style of poetry\textemdash from epigram, and antithesis, and hyperbole.
The best line in it, in which earthly joys are said to be\textemdash 
\begin{verse}
  ``Like angels' visits, few and far between''\textemdash 
\end{verse}
is a borrowed one.\footnote{\begin{quote}
    ``Like angels' visits, short and far between.''\textemdash
    \emph{Blair's Grave}.
\end{quote}
} But in the \emph{Gertrude of Wyoming} ``we perceive a
softness coming over the heart of the author, and the scales and crust
of formality that fence in his couplets and give them a somewhat
glittering and rigid appearance, fall off,'' and he has succeeded in
engrafting the wild and more expansive interest of the romantic school
of poetry on classic elegance and precision. After the poem we have
just named, Mr. Campbell's \textsc{Songs} are the happiest efforts of his
Muse:\textemdash breathing freshness, blushing like the morn, they seem, like
clustering roses, to weave a chaplet for love and liberty; or their
bleeding words gush out in mournful and hurried succession, like ``ruddy
drops that visit the sad heart'' of thoughtful Humanity. The \emph{Battle of
Hohenlinden} is of all modern compositions the most lyrical in spirit
and in sound. To justify this encomium, we need only recall the lines to
the reader's memory.
\begin{verse}
  \vleftofline{``}On Linden, when the sun was low,\\
  All bloodless lay th' untrodden snow,\\
  And dark as winter was the flow\\
  Of Iser, rolling rapidly.\\[\stanzaskip]

  But Linden saw another sight,\\
  When the drum beat at dead of night,\\
  Commanding fires of death to light\\
  The darkness of her scenery.\\[\stanzaskip]

  By torch and trumpet fast array'd,\\
  Each horseman drew his battle blade,\\
  And furious every charger neigh'd,\\
  To join the dreadful revelry.\\[\stanzaskip]

  Then shook the hills with thunder riv'n,\\
  Then rush'd the steed to battle driv'n,\\
  And louder than the bolts of heav'n\\
  Far flash'd the red artillery.\\[\stanzaskip]

  But redder yet that light shall glow\\
  On Linden's hills of stained snow,\\
  And bloodier yet the torrent flow\\
  Of Iser, rolling rapidly.\\[\stanzaskip]

  'Tis morn, but scarce yon level sun\\
  Can pierce the war-clouds, rolling\footnote{Is not this word, which occurs in the last line but one,
(as well as before) an instance of that repetition, which we so often
meet with in the most correct and elegant writers?} dun,\\
  Where furious Frank and fiery Hun\\
  Shout in their sulph'rous canopy.\\[\stanzaskip]

  The combat deepens.  On, ye brave,\\
  Who rush to glory, or the grave!\\
  Wave, Munich! all thy banners wave!\\
  And charge with all thy chivalry!\\[\stanzaskip]

  Few, few shall part, where many meet!\\
  The snow shall be their winding-sheet,\\
  And every turf beneath their feet\\
  Shall be a soldier's sepulchre."\\
\end{verse}
Mr. Campbell's prose-criticisms on contemporary and other poets (which
have appeared in the New Monthly Magazine) are in a style at once
chaste, temperate, guarded, and just.

Mr. Crabbe presents an entire contrast to
Mr. Campbell:\textemdash the one is the most ambitious and aspiring of living
poets, the other the most humble and prosaic. If the poetry of the one
is like the arch of the rainbow, spanning and adorning the earth, that
of the other is like a dull, leaden cloud hanging over it. Mr. Crabbe's
style might be cited as an answer to Audrey's question\textemdash ``Is poetry
a true thing?'' There are here no ornaments, no flights of fancy, no
illusions of sentiment, no tinsel of words. His song is one sad reality,
one unraised, unvaried note of unavailing woe. Literal fidelity serves
him in the place of invention; he assumes importance by a number of
petty details; he rivets attention by being tedious. He not only deals
in incessant matters of fact, but in matters of fact of the most
familiar, the least animating, and the most unpleasant kind; but he
relies for the effect of novelty on the microscopic minuteness with
which he dissects the most trivial objects\textemdash and for the interest he
excites, on the unshrinking determination with which he handles the most
painful. His poetry has an official and professional air. He is called
in to cases of difficult births, of fractured limbs, or breaches of the
peace; and makes out a parochial list of accidents and offences. He
takes the most trite, the most gross and obvious and revolting part of
nature, for the subject of his elaborate descriptions; but it is Nature
still, and Nature is a great and mighty Goddess! It is well for the
Reverend Author that it is so. Individuality is, in his theory, the only
definition of poetry. Whatever \emph{is}, he hitches into rhyme. Whoever
makes an exact image of any thing on the earth, however deformed or
insignificant, according to him, must succeed\textemdash and he himself has
succeeded. Mr. Crabbe is one of the most popular and admired of our
living authors. That he is so, can be accounted for on no other
principle than the strong ties that bind us to the world about us, and
our involuntary yearnings after whatever in any manner powerfully and
directly reminds us of it. His Muse is not one of \emph{the Daughters of
Memory}, but the old toothless, mumbling dame herself, doling out the
gossip and scandal of the neighbourhood, recounting \emph{totidem verbis et
literis}, what happens in every place of the kingdom every hour in the
year, and fastening always on the worst as the most palatable morsels.
But she is a circumstantial old lady, communicative, scrupulous, leaving
nothing to the imagination, harping on the smallest grievances, a
village-oracle and critic, most veritable, most identical, bringing us
acquainted with persons and things just as they chanced to exist, and
giving us a local interest in all she knows and tells. Mr. Crabbe's
Helicon is choked up with weeds and corruption; it reflects no light
from heaven, it emits no cheerful sound: no flowers of love, of hope,
or joy spring up near it, or they bloom only to wither in a moment. Our
poet's verse does not put a spirit of youth in every thing, but a spirit
of fear, despondency, and decay: it is not an electric spark to kindle
or expand, but acts like the torpedo's touch to deaden or contract. It
lends no dazzling tints to fancy, it aids no soothing feelings in the
heart, it gladdens no prospect, it stirs no wish; in its view the
current of life runs slow, dull, cold, dispirited, half under ground,
muddy, and clogged with all creeping things. The world is one vast
infirmary; the hill of Parnassus is a penitentiary, of which our author
is the overseer: to read him is a penance, yet we read on! Mr. Crabbe,
it must be confessed, is a repulsive writer. He contrives to ``turn
diseases to commodities,'' and makes a virtue of necessity. He puts us
out of conceit with this world, which perhaps a severe divine should do;
yet does not, as a charitable divine ought, point to another. His morbid
feelings droop and cling to the earth, grovel where they should soar;
and throw a dead weight on every aspiration of the soul after the good
or beautiful. By degrees we submit, and are reconciled to our fate, like
patients to the physician, or prisoners in the condemned cell. We can
only explain this by saying, as we said before, that Mr. Crabbe gives
us one part of nature, the mean, the little, the disgusting, the
distressing; that he does this thoroughly and like a master, and we
forgive all the rest.

Mr. Crabbe's first poems were published so long ago as the year 1782,
and received the approbation of Dr. Johnson only a little before he
died. This was a testimony from an enemy; for Dr. Johnson was not an
admirer of the simple in style or minute in description. Still he was an
acute, strong-minded man, and could see truth when it was presented to
him, even through the mist of his prejudices and his foibles. There was
something in Mr. Crabbe's intricate points that did not, after all, so
ill accord with the Doctor's purblind vision; and he knew quite
enough of the petty ills of life to judge of the merit of our poet's
descriptions, though he himself chose to slur them over in high-sounding
dogmas or general invectives. Mr. Crabbe's earliest poem of the
\emph{Village} was recommended to the notice of Dr. Johnson by Sir Joshua
Reynolds; and we cannot help thinking that a taste for that sort of
poetry, which leans for support on the truth and fidelity of its
imitations of nature, began to display itself much about that time, and,
in a good measure, in consequence of the direction of the public taste
to the subject of painting. Book-learning, the accumulation of wordy
common-places, the gaudy pretensions of poetical fiction, had enfeebled
and perverted our eye for nature. The study of the fine arts, which came
into fashion about forty years ago, and was then first considered as a
polite accomplishment, would tend imperceptibly to restore it. Painting
is essentially an imitative art; it cannot subsist for a moment on empty
generalities: the critic, therefore, who had been used to this sort of
substantial entertainment, would be disposed to read poetry with the
eye of a connoisseur, would be little captivated with smooth, polished,
unmeaning periods, and would turn with double eagerness and relish to
the force and precision of individual details, transferred, as it were,
to the page from the canvas. Thus an admirer of Teniers or Hobbima
might think little of the pastoral sketches of Pope or Goldsmith; even
Thompson describes not so much the naked object as what he sees in his
mind's eye, surrounded and glowing with the mild, bland, genial vapours
of his brain:\textemdash but the adept in Dutch interiors, hovels, and pig-styes
must find in Mr. Crabbe a man after his own heart. He is the very thing
itself; he paints in words, instead of colours: there is no other
difference. As Mr. Crabbe is not a painter, only because he does not use
a brush and colours, so he is for the most part a poet, only because
he writes in lines of ten syllables. All the rest might be found in a
newspaper, an old magazine, or a county-register. Our author is himself
a little jealous of the prudish fidelity of his homely Muse, and tries
to justify himself by precedents. He brings as a parallel instance of
merely literal description, Pope's lines on the gay Duke of Buckingham,
beginning ``In the worst inn's worst room see Villiers lies!'' But surely
nothing can be more dissimilar. Pope describes what is striking, Crabbe
would have described merely what was there. The objects in Pope stand
out to the fancy from the mixture of the mean with the gaudy, from the
contrast of the scene and the character. There is an appeal to the
imagination; you see what is passing in a poetical point of view. In
Crabbe there is no foil, no contrast, no impulse given to the mind. It
is all on a level and of a piece. In fact, there is so little connection
between the subject-matter of Mr. Crabbe's lines and the ornament of
rhyme which is tacked to them, that many of his verses read like serious
burlesque, and the parodies which have been made upon them are hardly so
quaint as the originals.

Mr. Crabbe's great fault is certainly that he is a sickly, a querulous,
a uniformly dissatisfied poet. He sings the country; and he sings it in
a pitiful tone. He chooses this subject only to take the charm out of
it, and to dispel the illusion, the glory, and the dream, which had
hovered over it in golden verse from Theocritus to Cowper. He sets out
with professing to overturn the theory which had hallowed a shepherd's
life, and made the names of grove and valley music to our ears, in order
to give us truth in its stead; but why not lay aside the fool's cap and
bells at once? Why not insist on the unwelcome reality in plain prose?
If our author is a poet, why trouble himself with statistics? If he is a
statistic writer, why set his ill news to harsh and grating verse? The
philosopher in painting the dark side of human nature may have reason
on his side, and a moral lesson or remedy in view. The tragic poet, who
shews the sad vicissitudes of things and the disappointments of the
passions, at least strengthens our yearnings after imaginary good, and
lends wings to our desires, by which we, ``at one bound, high overleap
all bound'' of actual suffering. But Mr. Crabbe does neither. He gives
us discoloured paintings of life; helpless, repining, unprofitable,
unedifying distress. He is not a philosopher, but a sophist, a
misanthrope in verse; a namby-pamby Mandeville, a Malthus turned
metrical romancer. He professes historical fidelity; but his vein is not
dramatic; nor does he give us the \emph{pros} and \emph{cons} of that versatile
gipsey, Nature. He does not indulge his fancy, or sympathise with us, or
tell us how the poor feel; but how he should feel in their situation,
which we do not want to know. He does not weave the web of their lives
of a mingled yarn, good and ill together, but clothes them all in the
same dingy linsey-woolsey, or tinges them with a green and yellow
melancholy. He blocks out all possibility of good, cancels the hope, or
even the wish for it as a weakness; check-mates Tityrus and Virgil at
the game of pastoral cross-purposes, disables all his adversary's white
pieces, and leaves none but black ones on the board. The situation of a
country clergyman is not necessarily favourable to the cultivation of
the Muse. He is set down, perhaps, as he thinks, in a small curacy for
life, and he takes his revenge by imprisoning the reader's imagination
in luckless verse. Shut out from social converse, from learned colleges
and halls, where he passed his youth, he has no cordial fellow-feeling
with the unlettered manners of the \emph{Village} or the \emph{Borough}; and he
describes his neighbours as more uncomfortable and discontented than
himself. All this while he dedicates successive volumes to rising
generations of noble patrons; and while he desolates a line of coast
with sterile, blighting lines, the only leaf of his books where honour,
beauty, worth, or pleasure bloom, is that inscribed to the Rutland
family! We might adduce instances of what we have said from every page
of his works: let one suffice\textemdash 
\begin{verse}
  \vleftofline{``}Thus by himself compelled to live each day,\\
  To wait for certain hours the tide's delay;\\
  At the same times the same dull views to see,\\
  The bounding marsh-bank and the blighted tree;\\
  The water only when the tides were high,\\
  When low, the mud half-covered and half-dry;\\
  The sun-burnt tar that blisters on the planks,\\
  And bank-side stakes in their uneven ranks;\\
  Heaps of entangled weeds that slowly float,\\
  As the tide rolls by the impeded boat.\\
  When tides were neap, and in the sultry day,\\
  Through the tall bounding mud-banks made their way,\\
  Which on each side rose swelling, and below\\
  The dark warm flood ran silently and slow;\\
  There anchoring, Peter chose from man to hide,\\
  There hang his head, and view the lazy tide\\
  In its hot slimy channel slowly glide;\\
  Where the small eels, that left the deeper way\\
  For the warm shore, within the shallows play;\\
  Where gaping muscles, left upon the mud,\\
  Slope their slow passage to the fall'n flood:\\
  Here dull and hopeless he'd lie down and trace\\
  How side-long crabs had crawled their crooked race;\\
  Or sadly listen to the tuneless cry\\
  Of fishing gull or clanging golden-eye;\\
  What time the sea-birds to the marsh would come,\\
  And the loud bittern, from the bull-rush home,\\
  Gave from the salt ditch-side the bellowing boom:\\
  He nursed the feelings these dull scenes produce\\
  And loved to stop beside the opening sluice;\\
  Where the small stream, confined in narrow bound,\\
  Ran with a dull, unvaried, saddening sound;\\
  Where all, presented to the eye or ear,\\
  Oppressed the soul with misery, grief, and fear.''
\end{verse}
This is an exact \emph{fac-simile} of some of the most unlovely parts of the
creation. Indeed the whole of Mr. Crabbe's \emph{Borough}, from which the
above passage is taken, is done so to the life, that it seems almost
like some sea-monster, crawled out of the neighbouring slime, and
harbouring a breed of strange vermin, with a strong local scent of
tar and bulge-water. Mr. Crabbe's \emph{Tales} are more readable than his
\emph{Poems}; but in proportion as the interest increases, they become more
oppressive. They turn, one and all, upon the same sort of teazing,
helpless, mechanical, unimaginative distress;\textemdash and though it is not
easy to lay them down, you never wish to take them up again. Still in
this way, they are highly finished, striking, and original portraits,
worked out with an eye to nature, and an intimate knowledge of the
small and intricate folds of the human heart. Some of the best are
the \emph{Confidant}, the story of \emph{Silly Shore}, the \emph{Young Poet}, the
\emph{Painter}. The episode of \emph{Phoebe Dawson} in the \emph{Village}, is one of
the most tender and pensive; and the character of the methodist parson
who persecutes the sailor's widow with his godly, selfish love, is one
of the most profound. In a word, if Mr. Crabbe's writings do not add
greatly to the store of entertaining and delightful fiction, yet they
will remain ``as a thorn in the side of poetry,'' perhaps for a century to
come!
