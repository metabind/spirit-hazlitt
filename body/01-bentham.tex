% Level 0
% \chapter[<toc-title>][<head-title>]{<title>}

\chapter[Jeremy Bentham]{jeremy bentham} 
% start a new page 
% place page number & chapter title in header
% add entry in ToC using optional title
% no header on first chapter page
% page number is placed in footer on first chapter page

Mr. Bentham is one of those persons who verify the old adage, that
``A prophet has no honour, except out of his own country.'' His
reputation lies at the circumference; and the lights of his
understanding are reflected, with increasing lustre, on the other
side of the globe. His name is little known in England, better in
Europe, best of all in the plains of Chili and the mines of
Mexico. He has offered constitutions for the New World, and
legislated for future times. The people of Westminster, where he
lives, hardly know of such a person; but the Siberian savage has
received cold comfort from his lunar aspect, and may say to him
with Caliban\textemdash ``I know thee, and thy dog and thy bush!''
The tawny Indian may hold out the hand of fellowship to him across
the \textsc{Great Pacific}. We believe that the Empress Catherine
corresponded with him; and we know that the Emperor Alexander
called upon him, and presented him with his miniature in a gold
snuff-box, which the philosopher, to his eternal honour,
returned. Mr. Hobhouse is a greater man at the hustings, Lord
Rolle at Plymouth Dock; but Mr. Bentham would carry it hollow, on
the score of popularity, at Paris or Pegu. The reason is, that our
author's influence is purely intellectual. He has devoted his life
to the pursuit of abstract and general truths, and to those
studies\textemdash
\begin{quote} ``That waft a \emph{thought} from Indus to the
Pole''\textemdash
\end{quote} and has never mixed himself up with personal intrigues
or party politics. He once, indeed, stuck up a hand-bill to say
that he (Jeremy Bentham) being of sound mind, was of opinion that
Sir Samuel Romilly was the most proper person to represent
Westminster; but this was the whim of the moment. Otherwise, his
reasonings, if true at all, are true everywhere alike: his
speculations concern humanity at large, and are not confined to
the hundred or the bills of mortality. It is in moral as in
physical magnitude. The little is seen best near: the great
appears in its proper dimensions, only from a more commanding
point of view, and gains strength with time, and elevation from
distance!

Mr. Bentham is very much among philosophers what La Fontaine was
among poets:\textemdash in general habits and in all but his
professional pursuits, he is a mere child. He has lived for the
last forty years in a house in Westminster, overlooking the Park,
like an anchoret in his cell, reducing law to a system, and the
mind of man to a machine. He scarcely ever goes out, and sees very
little company. The favoured few, who have the privilege of the
\emph{entr\'{e}e}, are always admitted one by one. He does not
like to have witnesses to his conversation. He talks a great deal,
and listens to nothing but facts. When any one calls upon him, he
invites them to take a turn round his garden with him (Mr. Bentham
is an economist of his time, and sets apart this portion of it to
air and exercise)\textemdash and there you may see the lively old
man, his mind still buoyant with thought and with the prospect of
futurity, in eager conversation with some Opposition Member, some
expatriated Patriot, or Transatlantic Adventurer, urging the
extinction of Close Boroughs, or planning a code of laws for some
``lone island in the watery waste,'' his walk almost amounting to
a run, his tongue keeping pace with it in shrill, cluttering
accents, negligent of his person, his dress, and his manner,
intent only on his grand theme of \textsc{Utility}\textemdash or pausing,
perhaps, for want of breath and with lack-lustre eye to point out
to the stranger a stone in the wall at the end of his garden
(overarched by two beautiful cotton-trees) \emph{Inscribed to the
Prince of Poets}, which marks the house where Milton formerly
lived. To shew how little the refinements of taste or fancy enter
into our author's system, he proposed at one time to cut down
these beautiful trees, to convert the garden where he had breathed
the air of Truth and Heaven for near half a century into a paltry
\emph{Chreistomathic School}, and to make Milton's house (the
cradle of Paradise Lost) a thoroughfare, like a three-stalled
stable, for the idle rabble of Westminster to pass backwards and
forwards to it with their cloven hoofs. Let us not, however, be
getting on too fast\textemdash Milton himself taught school! There
is something not altogether dissimilar between Mr. Bentham's
appearance, and the portraits of Milton, the same silvery tone, a
few dishevelled hairs, a peevish, yet puritanical expression, an
irritable temperament corrected by habit and discipline. Or in
modern times, he is something between Franklin and Charles Fox,
with the comfortable double-chin and sleek thriving look of the
one, and the quivering lip, the restless eye, and animated
acuteness of the other. His eye is quick and lively; but it
glances not from object to object, but from thought to thought. He
is evidently a man occupied with some train of fine and inward
association.  He regards the people about him no more than the
flies of a summer. He meditates the coming age. He hears and sees
only what suits his purpose, or some ``foregone conclusion;'' and
looks out for facts and passing occurrences in order to put them
into his logical machinery and grind them into the dust and powder
of some subtle theory, as the miller looks out for grist to his
mill! Add to this physiognomical sketch the minor points of
costume, the open shirt-collar, the single-breasted coat, the
old-fashioned half-boots and ribbed stockings; and you will find
in Mr.  Bentham's general appearance a singular mixture of boyish
simplicity and of the venerableness of age. In a word, our
celebrated jurist presents a striking illustration of the
difference between the \emph{philosophical} and the \emph{regal}
look; that is, between the merely abstracted and the merely
personal. There is a lackadaisical \emph{bonhommie} about his
whole aspect, none of the fierceness of pride or power; an
unconscious neglect of his own person, instead of a stately
assumption of superiority; a good-humoured, placid intelligence,
instead of a lynx-eyed watchfulness, as if it wished to make
others its prey, or was afraid they might turn and rend him; he is
a beneficent spirit, prying into the universe, not lording it over
it; a thoughtful spectator of the scenes of life, or ruminator on
the fate of mankind, not a painted pageant, a stupid idol set up
on its pedestal of pride for men to fall down and worship with
idiot fear and wonder at the thing themselves have made, and
which, without that fear and wonder, would in itself be nothing!

Mr. Bentham, perhaps, over-rates the importance of his own
theories. He has been heard to say (without any appearance of
pride or affectation) that ``he should like to live the remaining
years of his life, a year at a time at the end of the next six or
eight centuries, to see the effect which his writings would by
that time have had upon the world.'' Alas!  his name will hardly
live so long! Nor do we think, in point of fact, that Mr. Bentham
has given any new or decided impulse to the human mind.  He cannot
be looked upon in the light of a discoverer in legislation or
morals. He has not struck out any great leading principle or
parent-truth, from which a number of others might be deduced; nor
has he enriched the common and established stock of intelligence
with original observations, like pearls thrown into wine. One
truth discovered is immortal, and entitles its author to be so:
for, like a new substance in nature, it cannot be destroyed. But
Mr. Bentham's \emph{forte} is arrangement; and the form of truth,
though not its essence, varies with time and circumstance. He has
methodised, collated, and condensed all the materials prepared to
his hand on the subjects of which he treats, in a masterly and
scientific manner; but we should find a difficulty in adducing
from his different works (however elaborate or closely reasoned)
any new element of thought, or even a new fact or
illustration. His writings are, therefore, chiefly valuable as
\emph{books of reference}, as bringing down the account of
intellectual inquiry to the present period, and disposing the
results in a compendious, connected, and tangible shape; but books
of reference are chiefly serviceable for facilitating the
acquisition of knowledge, and are constantly liable to be
superseded and to grow out of fashion with its progress, as the
scaffolding is thrown down as soon as the building is
completed. Mr.  Bentham is not the first writer (by a great many)
who has assumed the principle of \textsc{Utility} as the
foundation of just laws, and of all moral and political
reasoning:\textemdash his merit is, that he has applied this
principle more closely and literally; that he has brought all the
objections and arguments, more distinctly labelled and ticketted,
under this one head, and made a more constant and explicit
reference to it at every step of his progress, than any other
writer. Perhaps the weak side of his conclusions also is, that he
has carried this single view of his subject too far, and not made
sufficient allowance for the varieties of human nature, and the
caprices and irregularities of the human will. ``He has not
allowed for the \emph{wind}.'' It is not that you can be said to
see his favourite doctrine of Utility glittering everywhere
through his system, like a vein of rich, shining ore (that is not
the nature of the material)\textemdash but it might be plausibly
objected that he had struck the whole mass of fancy, prejudice,
passion, sense, whim, with his petrific, leaden mace, that he had
``bound volatile Hermes,'' and reduced the theory and practice of
human life to a \emph{caput mortuum} of reason, and dull,
plodding, technical calculation. The gentleman is himself a
capital logician; and he has been led by this circumstance to
consider man as a logical animal. We fear this view of the matter
will hardly hold water.  If we attend to the \emph{moral} man, the
constitution of his mind will scarcely be found to be built up of
pure reason and a regard to consequences: if we consider the
\emph{criminal} man (with whom the legislator has chiefly to do)
it will be found to be still less so.

Every pleasure, says Mr. Bentham, is equally a good, and is to be
taken into the account as such in a moral estimate, whether it be
the pleasure of sense or of conscience, whether it arise from the
exercise of virtue or the perpetration of crime. We are afraid the
human mind does not readily come into this doctrine, this
\emph{ultima ratio philosophorum}, interpreted according to the
letter. Our moral sentiments are made up of sympathies and
antipathies, of sense and imagination, of understanding and
prejudice. The soul, by reason of its weakness, is an aggregating
and an exclusive principle; it clings obstinately to some things,
and violently rejects others. And it must do so, in a great
measure, or it would act contrary to its own nature. It needs
helps and stages in its progress, and ``all appliances and means
to boot,'' which can raise it to a partial conformity to truth and
good (the utmost it is capable of) and bring it into a tolerable
harmony with the universe. By aiming at too much, by dismissing
collateral aids, by extending itself to the farthest verge of the
conceivable and possible, it loses its elasticity and vigour, its
impulse and its direction. The moralist can no more do without the
intermediate use of rules and principles, without the 'vantage
ground of habit, without the levers of the understanding, than the
mechanist can discard the use of wheels and pulleys, and perform
every thing by simple motion. If the mind of man were competent to
comprehend the whole of truth and good, and act upon it at once,
and independently of all other considerations, Mr. Bentham's plan
would be a feasible one, and \emph{the truth, the whole truth, and
nothing but the truth} would be the best possible ground to place
morality upon. But it is not so. In ascertaining the rules of
moral conduct, we must have regard not merely to the nature of the
object, but to the capacity of the agent, and to his fitness for
apprehending or attaining it. Pleasure is that which is so in
itself: good is that which approves itself as such on reflection,
or the idea of which is a source of satisfaction.  All pleasure is
not, therefore (morally speaking) equally a good; for all pleasure
does not equally bear reflecting on. There are some tastes that
are sweet in the mouth and bitter in the belly; and there is a
similar contradiction and anomaly in the mind and heart of
man. Again, what would become of the \emph{Posthaec meminisse
juvabit} of the poet, if a principle of fluctuation and reaction
is not inherent in the very constitution of our nature, or if all
moral truth is a mere literal truism? We are not, then, so much to
inquire what certain things are abstractedly or in themselves, as
how they affect the mind, and to approve or condemn them
accordingly. The same object seen near strikes us more powerfully
than at a distance: things thrown into masses give a greater blow
to the imagination than when scattered and divided into their
component parts. A number of mole-hills do not make a mountain,
though a mountain is actually made up of atoms: so moral truth
must present itself under a certain aspect and from a certain
point of view, in order to produce its full and proper effect upon
the mind. The laws of the affections are as necessary as those of
optics. A calculation of consequences is no more equivalent to a
sentiment, than a \emph{seriatim} enumeration of square yards or
feet touches the fancy like the sight of the Alps or Andes!

To give an instance or two of what we mean. Those who on pure
cosmopolite principles, or on the ground of abstract humanity
affect an extraordinary regard for the Turks and Tartars, have
been accused of neglecting their duties to their friends and
next-door neighbours. Well, then, what is the state of the
question here? One human being is, no doubt, as much worth in
himself, independently of the circumstances of time or place, as
another; but he is not of so much value to us and our
affections. Could our imagination take wing (with our speculative
faculties) to the other side of the globe or to the ends of the
universe, could our eyes behold whatever our reason teaches us to
be possible, could our hands reach as far as our thoughts or
wishes, we might then busy ourselves to advantage with the
Hottentots, or hold intimate converse with the inhabitants of the
Moon; but being as we are, our feelings evaporate in so large a
space\textemdash we must draw the circle of our affections and
duties somewhat closer\textemdash the heart hovers and fixes
nearer home. It is true, the bands of private, or of local and
natural affection are often, nay in general, too tightly strained,
so as frequently to do harm instead of good: but the present
question is whether we can, with safety and effect, be wholly
emancipated from them?  Whether we should shake them off at
pleasure and without mercy, as the only bar to the triumph of
truth and justice? Or whether benevolence, constructed upon a
logical scale, would not be merely \emph{nominal}, whether duty,
raised to too lofty a pitch of refinement, might not sink into
callous indifference or hollow selfishness? Again, is it not to
exact too high a strain from humanity, to ask us to qualify the
degree of abhorrence we feel against a murderer by taking into our
cool consideration the pleasure he may have in committing the
deed, and in the prospect of gratifying his avarice or his
revenge? We are hardly so formed as to sympathise at the same
moment with the assassin and his victim. The degree of pleasure
the former may feel, instead of extenuating, aggravates his guilt,
and shews the depth of his malignity.  Now the mind revolts
against this by mere natural antipathy, if it is itself
well-disposed; or the slow process of reason would afford but a
feeble resistance to violence and wrong. The will, which is
necessary to give consistency and promptness to our good
intentions, cannot extend so much candour and courtesy to the
antagonist principle of evil: virtue, to be sincere and practical,
cannot be divested entirely of the blindness and impetuosity of
passion! It has been made a plea (half jest, half earnest) for the
horrors of war, that they promote trade and manufactures. It has
been said, as a set-off for the atrocities practised upon the
negro slaves in the West Indies, that without their blood and
sweat, so many millions of people could not have sugar to sweeten
their tea. Fires and murders have been argued to be beneficial, as
they serve to fill the newspapers, and for a subject to talk
of\textemdash this is a sort of sophistry that it might be
difficult to disprove on the bare scheme of contingent utility;
but on the ground that we have stated, it must pass for a mere
irony. What the proportion between the good and the evil will
really be found in any of the supposed cases, may be a question to
the understanding; but to the imagination and the heart, that is,
to the natural feelings of mankind, it admits of none!

Mr. Bentham, in adjusting the provisions of a penal code, lays too
little stress on the cooperation of the natural prejudices of
mankind, and the habitual feelings of that class of persons for
whom they are more particularly designed. Legislators (we mean
writers on legislation) are philosophers, and governed by their
reason: criminals, for whose controul laws are made, are a set of
desperadoes, governed only by their passions. What wonder that so
little progress has been made towards a mutual understanding
between the two parties! They are quite a different species, and
speak a different language, and are sadly at a loss for a common
interpreter between them. Perhaps the Ordinary of Newgate bids as
fair for this office as any one. What should Mr. Bentham, sitting
at ease in his arm-chair, composing his mind before he begins to
write by a prelude on the organ, and looking out at a beautiful
prospect when he is at a loss for an idea, know of the principles
of action of rogues, outlaws, and vagabonds? No more than
Montaigne of the motions of his cat! If sanguine and
tender-hearted philanthropists have set on foot an inquiry into
the barbarity and the defects of penal laws, the practical
improvements have been mostly suggested by reformed cut-throats,
turnkeys, and thief-takers. What even can the Honourable House,
who when the Speaker has pronounced the well-known, wished-for
sounds ``That this house do now adjourn,'' retire, after voting a
royal crusade or a loan of millions, to lie on down, and feed on
plate in spacious palaces, know of what passes in the hearts of
wretches in garrets and night-cellars, petty pilferers and
marauders, who cut throats and pick pockets with their own hands?
The thing is impossible. The laws of the country are, therefore,
ineffectual and abortive, because they are made by the rich for
the poor, by the wise for the ignorant, by the respectable and
exalted in station for the very scum and refuse of the
community. If Newgate would resolve itself into a committee of the
whole Press-yard, with Jack Ketch at its head, aided by
confidential persons from the county prisons or the Hulks, and
would make a clear breast, some \emph{data} might be found out to
proceed upon; but as it is, the \emph{criminal mind} of the
country is a book sealed, no one has been able to penetrate to the
inside! Mr. Bentham, in his attempts to revise and amend our
criminal jurisprudence, proceeds entirely on his favourite
principle of Utility.  Convince highwaymen and house-breakers that
it will be for their interest to reform, and they will reform and
lead honest lives; according to Mr. Bentham. He says, ``All men
act from calculation, even madmen reason.'' And, in our opinion,
he might as well carry this maxim to Bedlam or St. Luke's, and
apply it to the inhabitants, as think to coerce or overawe the
inmates of a gaol, or those whose practices make them candidates
for that distinction, by the mere dry, detailed convictions of the
understanding. Criminals are not to be influenced by reason; for
it is of the very essence of crime to disregard consequences both
to ourselves and others. You may as well preach philosophy to a
drunken man, or to the dead, as to those who are under the
instigation of any mischievous passion. A man is a drunkard, and
you tell him he ought to be sober; he is debauched, and you ask
him to reform; he is idle, and you recommend industry to him as
his wisest course; he gambles, and you remind him that he may be
ruined by this foible; he has lost his character, and you advise
him to get into some reputable service or lucrative situation;
vice becomes a habit with him, and you request him to rouse
himself and shake it off; he is starving, and you warn him that if
he breaks the law, he will be hanged. None of this reasoning
reaches the mark it aims at. The culprit, who violates and suffers
the vengeance of the laws, is not the dupe of ignorance, but the
slave of passion, the victim of habit or necessity. To argue with
strong passion, with inveterate habit, with desperate
circumstances, is to talk to the winds. Clownish ignorance may
indeed be dispelled, and taught better; but it is seldom that a
criminal is not aware of the consequences of his act, or has not
made up his mind to the alternative.  They are, in general,
\emph{too knowing by half}. You tell a person of this stamp what
is his interest; he says he does not care about his interest, or
the world and he differ on that particular. But there is one point
on which he must agree with them, namely, what \emph{they} think
of his conduct, and that is the only hold you have of him. A man
may be callous and indifferent to what happens to himself; but he
is never indifferent to public opinion, or proof against open
scorn and infamy. Shame, then, not fear, is the sheet-anchor of
the law. He who is not afraid of being pointed at as a
\emph{thief}, will not mind a month's hard labour. He who is
prepared to take the life of another, is already reckless of his
own.  But every one makes a sorry figure in the pillory; and the
being launched from the New Drop lowers a man in his own
opinion. The lawless and violent spirit, who is hurried by
headstrong self-will to break the laws, does not like to have the
ground of pride and obstinacy struck from under his feet. This is
what gives the \emph{swells} of the metropolis such a dread of the
\emph{tread-mill}\textemdash it makes them ridiculous. It must be
confessed, that this very circumstance renders the reform of
criminals nearly hopeless. It is the apprehension of being
stigmatized by public opinion, the fear of what will be thought
and said of them, that deters men from the violation of the laws,
while their character remains unimpeached; but honour once lost,
all is lost. The man can never be himself again! A citizen is like
a soldier, a part of a machine, who submits to certain hardships,
privations, and dangers, not for his own ease, pleasure, profit,
or even conscience, but\textemdash\emph{for shame}. What is it
that keeps the machine together in either case? Not punishment or
discipline, but sympathy. The soldier mounts the breach or stands
in the trenches, the peasant hedges and ditches, or the mechanic
plies his ceaseless task, because the one will not be called a
\emph{coward}, the other a \emph{rogue}: but let the one turn
deserter and the other vagabond, and there is an end of him. The
grinding law of necessity, which is no other than a name, a
breath, loses its force; he is no longer sustained by the good
opinion of others, and he drops out of his place in society, a
useless clog! Mr. Bentham takes a culprit, and puts him into what
he calls a \emph{Panopticon}, that is, a sort of circular prison,
with open cells, like a glass bee-hive. He sits in the middle, and
sees all the other does. He gives him work to do, and lectures him
if he does not do it. He takes liquor from him, and society, and
liberty; but he feeds and clothes him, and keeps him out of
mischief; and when he has convinced him, by force and reason
together, that this life is for his good, he turns him out upon
the world a reformed man, and as confident of the success of his
handy-work, as the shoemaker of that which he has just taken off
the last, or the Parisian barber in Sterne, of the buckle of his
wig. ``Dip it in the ocean,'' said the perruquier, ``and it will
stand!'' But we doubt the durability of our projector's
patchwork. Will our convert to the great principle of Utility work
when he is from under Mr. Bentham's eye, because he was forced to
work when under it? Will he keep sober, because he has been kept
from liquor so long? Will he not return to loose company, because
he has had the pleasure of sitting vis-a-vis with a philosopher of
late? Will he not steal, now that his hands are untied? Will he
not take the road, now that it is free to him? Will he not call
his benefactor all the names he can set his tongue to, the moment
his back is turned? All this is more than to be feared. The charm
of criminal life, like that of savage life, consists in liberty,
in hardship, in danger, and in the contempt of death, in one word,
in extraordinary excitement; and he who has tasted of it, will no
more return to regular habits of life, than a man will take to
water after drinking brandy, or than a wild beast will give over
hunting its prey.  Miracles never cease, to be sure; but they are
not to be had wholesale, or \emph{to order}. Mr. Owen, who is
another of these proprietors and patentees of reform, has lately
got an American savage with him, whom he carries about in great
triumph and complacency, as an antithesis to his \emph{New View of
Society}, and as winding up his reasoning to what it mainly
wanted, an epigrammatic point. Does the benevolent visionary of
the Lanark cotton-mills really think this \emph{natural man} will
act as a foil to his \emph{artificial man}? Does he for a moment
imagine that his \emph{Address to the higher and middle classes},
with all its advantages of fiction, makes any thing like so
interesting a romance as \emph{Hunter's Captivity among the North
American Indians?} Has he any thing to shew, in all the apparatus
of New Lanark and its desolate monotony, to excite the thrill of
imagination like the blankets made of wreaths of snow under which
the wild wood-rovers bury themselves for weeks in winter? Or the
skin of a leopard, which our hardy adventurer slew, and which
served him for great coat and bedding? Or the rattle-snake that he
found by his side as a bedfellow? Or his rolling himself into a
ball to escape from him? Or his suddenly placing himself against a
tree to avoid being trampled to death by the herd of wild
buffaloes, that came rushing on like the sound of thunder? Or his
account of the huge spiders that prey on bluebottles and gilded
flies in green pathless forests; or of the great Pacific Ocean,
that the natives look upon as the gulf that parts time from
eternity, and that is to waft them to the spirits of their
fathers? After all this, Mr. Hunter must find Mr. Owen and his
parallellograms trite and flat, and will, we suspect, take an
opportunity to escape from them!

Mr. Bentham's method of reasoning, though comprehensive and exact,
labours under the defect of most systems\textemdash it is too
\emph{topical}. It includes every thing; but it includes every
thing alike. It is rather like an inventory, than a valuation of
different arguments. Every possible suggestion finds a place, so
that the mind is distracted as much as enlightened by this
perplexing accuracy. The exceptions seem as important as the
rule. By attending to the minute, we overlook the great; and in
summing up an account, it will not do merely to insist on the
number of items without considering their amount. Our author's
page presents a very nicely dove-tailed mosaic pavement of legal
common-places. We slip and slide over its even surface without
being arrested any where. Or his view of the human mind resembles
a map, rather than a picture: the outline, the disposition is
correct, but it wants colouring and relief. There is a
technicality of manner, which renders his writings of more value
to the professional inquirer than to the general reader. Again,
his style is unpopular, not to say unintelligible. He writes a
language of his own, that \emph{darkens knowledge}. His works have
been translated into French\textemdash they ought to be translated
into English. People wonder that Mr. Bentham has not been
prosecuted for the boldness and severity of some of his
invectives. He might wrap up high treason in one of his
inextricable periods, and it would never find its way into
Westminster-Hall. He is a kind of Manuscript author\textemdash he
writes a cypher-hand, which the vulgar have no key to. The
construction of his sentences is a curious framework with pegs and
hooks to hang his thoughts upon, for his own use and guidance, but
almost out of the reach of every body else. It is a barbarous
philosophical jargon, with all the repetitions, parentheses,
formalities, uncouth nomenclature and verbiage of law-Latin; and
what makes it worse, it is not mere verbiage, but has a great deal
of acuteness and meaning in it, which you would be glad to pick
out if you could. In short, Mr. Bentham writes as if he was
allowed but a single sentence to express his whole view of a
subject in, and as if, should he omit a single circumstance or
step of the argument, it would be lost to the world for ever, like
an estate by a flaw in the title-deeds. This is over-rating the
importance of our own discoveries, and mistaking the nature and
object of language altogether. Mr. Bentham has \emph{acquired}
this disability\textemdash it is not natural to him. His admirable
little work \emph{On Usury}, published forty years ago, is clear,
easy, and vigorous. But Mr.  Bentham has shut himself up since
then ``in nook monastic,'' conversing only with followers of his
own, or with ``men of Ind,'' and has endeavoured to overlay his
natural humour, sense, spirit, and style with the dust and cobwebs
of an obscure solitude. The best of it is, he thinks his present
mode of expressing himself perfect, and that whatever may be
objected to his law or logic, no one can find the least fault with
the purity, simplicity, and perspicuity of his style.

Mr. Bentham, in private life, is an amiable and exemplary
character.  He is a little romantic, or so; and has dissipated
part of a handsome fortune in practical speculations. He lends an
ear to plausible projectors, and, if he cannot prove them to be
wrong in their premises or their conclusions, thinks himself bound
\emph{in reason} to stake his money on the venture. Strict
logicians are licensed visionaries. Mr.  Bentham is half-brother
to the late Mr. Speaker Abbott\footnote{Now Lord
  Colchester.}\textemdash\emph{Proh pudor}!  He was educated at
Eton, and still takes our novices to task about a passage in
Homer, or a metre in Virgil. He was afterwards at the University,
and he has described the scruples of an ingenuous youthful mind
about subscribing the articles, in a passage in his
\emph{Church-of-Englandism}, which smacks of truth and honour
both, and does one good to read it in an age, when ``to be
honest'' (or not to laugh at the very idea of honesty) ``is to be
one man picked out of ten thousand!''  Mr. Bentham relieves his
mind sometimes, after the fatigue of study, by playing on a fine
old organ, and has a relish for Hogarth's prints. He turns wooden
utensils in a lathe for exercise, and fancies he can turn men in
the same manner. He has no great fondness for poetry, and can
hardly extract a moral out of Shakespeare. His house is warmed and
lighted by steam. He is one of those who prefer the artificial to
the natural in most things, and think the mind of man
omnipotent. He has a great contempt for out-of-door prospects, for
green fields and trees, and is for referring every thing to
Utility. There is a little narrowness in this; for if all the
sources of satisfaction are taken away, what is to become of
utility itself? It is, indeed, the great fault of this able and
extraordinary man, that he has concentrated his faculties and
feelings too entirely on one subject and pursuit, and has not
``looked enough abroad into universality.''\footnote{Lord Bacon's
  Advancement of Learning.}


