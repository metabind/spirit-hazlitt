\chapter[Lord Byron]{lord byron} %07

Lord Byron and Sir Walter Scott are among writers now living\footnote{This Essay was written just before Lord Byron's death.} the two,
who would carry away a majority of suffrages as the greatest geniuses of
the age. The former would, perhaps, obtain the preference with the fine
gentlemen and ladies (squeamishness apart)\textemdash the latter with the critics
and the vulgar. We shall treat of them in the same connection, partly
on account of their distinguished pre-eminence, and partly because they
afford a complete contrast to each other. In their poetry, in their
prose, in their politics, and in their tempers no two men can be more
unlike. If Sir Walter Scott may be thought by some to have been
\begin{quote}
  ``Born universal heir to all humanity,''
\end{quote}
it is plain Lord Byron can set up no such pretension. He is, in a
striking degree, the creature of his own will. He holds no communion
with his kind; but stands alone, without mate or fellow\textemdash 
\begin{quote}
  ``As if a man were author of himself, \\
  And owned no other kin.''
\end{quote}
He is like a solitary peak, all access to which is cut off not more by
elevation than distance. He is seated on a lofty eminence, cloud-capt,
or reflecting the last rays of setting suns; and in his poetical moods,
reminds us of the fabled Titans, retired to a ridgy steep, playing on
their Pan's-pipes, and taking up ordinary men and things in their hands
with haughty indifference. He raises his subject to himself, or tramples
on it: he neither stoops to, nor loses himself in it. He exists not by
sympathy, but by antipathy. He scorns all things, even himself. Nature
must come to him to sit for her picture\textemdash he does not go to her. She must
consult his time, his convenience, and his humour; and wear a \emph{sombre}
or a fantastic garb, or his Lordship turns his back upon her. There is
no ease, no unaffected simplicity of manner, no ``golden mean.'' All is
strained, or petulant in the extreme. His thoughts are sphered and
crystalline; his style ``prouder than when blue Iris bends;'' his spirit
fiery, impatient, wayward, indefatigable. Instead of taking his
impressions from without, in entire and almost unimpaired masses, he
moulds them according to his own temperament, and heats the materials
of his imagination in the furnace of his passions.\textemdash Lord Byron's verse
glows like a flame, consuming every thing in its way; Sir Walter Scott's
glides like a river, clear, gentle, harmless. The poetry of the first
scorches, that of the last scarcely warms. The light of the one proceeds
from an internal source, ensanguined, sullen, fixed; the other reflects
the hues of Heaven, or the face of nature, glancing vivid and various.
The productions of the Northern Bard have the rust and the freshness
of antiquity about them; those of the Noble Poet cease to startle
from their extreme ambition of novelty, both in style and matter. Sir
Walter's rhymes are ``silly sooth''\textemdash 
\begin{quote}
  ``And dally with the innocence of thought, \\
  Like the old age''\textemdash

\end{quote}
his Lordship's Muse spurns \emph{the olden time}, and affects all the
supercilious airs of a modern fine lady and an upstart. The object of
the one writer is to restore us to truth and nature: the other chiefly
thinks how he shall display his own power, or vent his spleen, or
astonish the reader either by starting new subjects and trains of
speculation, or by expressing old ones in a more striking and emphatic
manner than they have been expressed before. He cares little what it is
he says, so that he can say it differently from others. This may account
for the charges of plagiarism which have been repeatedly brought against
the Noble Poet\textemdash if he can borrow an image or sentiment from another, and
heighten it by an epithet or an allusion of greater force and beauty
than is to be found in the original passage, he thinks he shews his
superiority of execution in this in a more marked manner than if
the first suggestion had been his own. It is not the value of the
observation itself he is solicitous about; but he wishes to shine by
contrast\textemdash even nature only serves as a foil to set off his style. He
therefore takes the thoughts of others (whether contemporaries or not)
out of their mouths, and is content to make them his own, to set his
stamp upon them, by imparting to them a more meretricious gloss, a
higher relief, a greater loftiness of tone, and a characteristic
inveteracy of purpose. Even in those collateral ornaments of modern
style, slovenliness, abruptness, and eccentricity (as well as in
terseness and significance), Lord Byron, when he pleases, defies
competition and surpasses all his contemporaries. Whatever he does, he
must do in a more decided and daring manner than any one else\textemdash he lounges
with extravagance, and yawns so as to alarm the reader! Self-will,
passion, the love of singularity, a disdain of himself and of others
(with a conscious sense that this is among the ways and means of
procuring admiration) are the proper categories of his mind: he is a
lordly writer, is above his own reputation, and condescends to the Muses
with a scornful grace!

Lord Byron, who in his politics is a \emph{liberal}, in his genius is haughty
and aristocratic: Walter Scott, who is an aristocrat in principle, is
popular in his writings, and is (as it were) equally \emph{servile} to nature
and to opinion. The genius of Sir Walter is essentially imitative, or
``denotes a foregone conclusion:'' that of Lord Byron is self-dependent;
or at least requires no aid, is governed by no law, but the impulses of
its own will. We confess, however much we may admire independence of
feeling and erectness of spirit in general or practical questions, yet
in works of genius we prefer him who bows to the authority of nature,
who appeals to actual objects, to mouldering superstitions, to history,
observation, and tradition, before him who only consults the pragmatical
and restless workings of his own breast, and gives them out as oracles
to the world. We like a writer (whether poet or prose-writer) who takes
in (or is willing to take in) the range of half the universe in feeling,
character, description, much better than we do one who obstinately and
invariably shuts himself up in the Bastile of his own ruling passions.
In short, we had rather be Sir Walter Scott (meaning thereby the Author
of Waverley) than Lord Byron, a hundred times over. And for the reason
just given, namely, that he casts his descriptions in the mould of
nature, ever-varying, never tiresome, always interesting and always
instructive, instead of casting them constantly in the mould of his
own individual impressions. He gives us man as he is, or as he was, in
almost every variety of situation, action, and feeling. Lord Byron
makes man after his own image, woman after his own heart; the one is
a capricious tyrant, the other a yielding slave; he gives us the
misanthrope and the voluptuary by turns; and with these two characters,
burning or melting in their own fires, he makes out everlasting centos
of himself. He hangs the cloud, the film of his existence over all
outward things\textemdash sits in the centre of his thoughts, and enjoys dark
night, bright day, the glitter and the gloom ``in cell monastic''\textemdash we see
the mournful pall, the crucifix, the death's heads, the faded chaplet of
flowers, the gleaming tapers, the agonized brow of genius, the wasted
form of beauty\textemdash but we are still imprisoned in a dungeon, a curtain
intercepts our view, we do not breathe freely the air of nature or of
our own thoughts\textemdash the other admired author draws aside the curtain, and
the veil of egotism is rent, and he shews us the crowd of living men and
women, the endless groups, the landscape back-ground, the cloud and
the rainbow, and enriches our imaginations and relieves one passion
by another, and expands and lightens reflection, and takes away that
tightness at the breast which arises from thinking or wishing to think
that there is nothing in the world out of a man's self!\textemdash In this point
of view, the Author of Waverley is one of the greatest teachers of
morality that ever lived, by emancipating the mind from petty, narrow,
and bigotted prejudices: Lord Byron is the greatest pamperer of those
prejudices, by seeming to think there is nothing else worth encouraging
but the seeds or the full luxuriant growth of dogmatism and
self-conceit. In reading the \emph{Scotch Novels}, we never think about
the author, except from a feeling of curiosity respecting our unknown
benefactor: in reading Lord Byron's works, he himself is never absent
from our minds. The colouring of Lord Byron's style, however rich and
dipped in Tyrian dyes, is nevertheless opaque, is in itself an object
of delight and wonder: Sir Walter Scott's is perfectly transparent. In
studying the one, you seem to gaze at the figures cut in stained glass,
which exclude the view beyond, and where the pure light of Heaven is
only a means of setting off the gorgeousness of art: in reading the
other, you look through a noble window at the clear and varied landscape
without. Or to sum up the distinction in one word, Sir Walter Scott is
the most \emph{dramatic} writer now living; and Lord Byron is the least so.
It would be difficult to imagine that the Author of Waverley is in the
smallest degree a pedant; as it would be hard to persuade ourselves that
the author of Childe Harold and Don Juan is not a coxcomb, though a
provoking and sublime one. In this decided preference given to Sir
Walter Scott over Lord Byron, we distinctly include the prose-works of
the former; for we do not think his poetry alone by any means entitles
him to that precedence. Sir Walter in his poetry, though pleasing and
natural, is a comparative trifler: it is in his anonymous productions
that he has shewn himself for what he is!\textemdash 

\emph{Intensity} is the great and prominent distinction of Lord Byron's
writings. He seldom gets beyond force of style, nor has he produced any
regular work or masterly whole. He does not prepare any plan beforehand,
nor revise and retouch what he has written with polished accuracy. His
only object seems to be to stimulate himself and his readers for the
moment\textemdash to keep both alive, to drive away \emph{ennui}, to substitute a
feverish and irritable state of excitement for listless indolence or
even calm enjoyment. For this purpose he pitches on any subject at
random without much thought or delicacy\textemdash he is only impatient to
begin\textemdash and takes care to adorn and enrich it as he proceeds with
``thoughts that breathe and words that burn.'' He composes (as he himself
has said) whether he is in the bath, in his study, or on horseback\textemdash he
writes as habitually as others talk or think\textemdash and whether we have the
inspiration of the Muse or not, we always find the spirit of the man
of genius breathing from his verse. He grapples with his subject, and
moves, penetrates, and animates it by the electric force of his own
feelings. He is often monotonous, extravagant, offensive; but he is
never dull, or tedious, but when he writes prose. Lord Byron does not
exhibit a new view of nature, or raise insignificant objects into
importance by the romantic associations with which he surrounds them;
but generally (at least) takes common-place thoughts and events, and
endeavours to express them in stronger and statelier language than
others. His poetry stands like a Martello tower by the side of his
subject. He does not, like Mr. Wordsworth, lift poetry from the ground,
or create a sentiment out of nothing. He does not describe a daisy or a
periwinkle, but the cedar or the cypress: not ``poor men's cottages, but
princes' palaces.'' His Childe Harold contains a lofty and impassioned
review of the great events of history, of the mighty objects left as
wrecks of time, but he dwells chiefly on what is familiar to the mind of
every school-boy; has brought out few new traits of feeling or thought;
and has done no more than justice to the reader's preconceptions by the
sustained force and brilliancy of his style and imagery. Lord Byron's
earlier productions, \emph{Lara}, the \emph{Corsair}, \&c. were wild and gloomy
romances, put into rapid and shining verse. They discover the madness
of poetry, together with the inspiration: sullen, moody, capricious,
fierce, inexorable, gloating on beauty, thirsting for revenge, hurrying
from the extremes of pleasure to pain, but with nothing permanent,
nothing healthy or natural. The gaudy decorations and the morbid
sentiments remind one of flowers strewed over the face of death! In
his \emph{Childe Harold} (as has been just observed) he assumes a lofty and
philosophic tone, and reasons high of providence, fore-knowledge, will,
and fate. He takes the highest points in the history of the world,
and comments on them from a more commanding eminence: he shews us the
crumbling monuments of time, he invokes the great names, the
mighty spirit of antiquity. The universe is changed into a stately
mausoleum:\textemdash in solemn measures he chaunts a hymn to fame. Lord Byron has
strength and elevation enough to fill up the moulds of our classical and
time-hallowed recollections, and to rekindle the earliest aspirations of
the mind after greatness and true glory with a pen of fire. The names of
Tasso, of Ariosto, of Dante, of Cincinnatus, of Caesar, of Scipio, lose
nothing of their pomp or their lustre in his hands, and when he begins
and continues a strain of panegyric on such subjects, we indeed sit
down with him to a banquet of rich praise, brooding over imperishable
glories,
\begin{quote}
  ``Till Contemplation has her fill.''

\end{quote}
Lord Byron seems to cast himself indignantly from ``this bank and shoal
of time,'' or the frail tottering bark that bears up modern reputation,
into the huge sea of ancient renown, and to revel there with untired,
outspread plume. Even this in him is spleen\textemdash his contempt of his
contemporaries makes him turn back to the lustrous past, or project
himself forward to the dim future!\textemdash Lord Byron's
tragedies, Faliero,\footnote{
  \begin{quote}
    ``Don Juan was my Moscow, and Faliero \\
    My Leipsic, and my Mont St. Jean seems Cain,'' \\
\sourceatright{\emph{Don Juan}, Canto. XI.\hspace{5pc}}
\end{quote} 
}
Sardanapalus, \&c. are not equal to his other works. They want the
essence of the drama. They abound in speeches and descriptions, such as
he himself might make either to himself or others, lolling on his couch
of a morning, but do not carry the reader out of the poet's mind to the
scenes and events recorded. They have neither action, character,
nor interest, but are a sort of \emph{gossamer} tragedies, spun out, and
glittering, and spreading a flimsy veil over the face of nature. Yet
he spins them on. Of all that he has done in this way the \emph{Heaven and
Earth} (the same subject as Mr. Moore's \emph{Loves of the Angels}) is the
best. We prefer it even to \emph{Manfred}. \emph{Manfred} is merely himself,
with a fancy-drapery on: but in the dramatic fragment published in the
\emph{Liberal}, the space between Heaven and Earth, the stage on which
his characters have to pass to and fro, seems to fill his Lordship's
imagination; and the Deluge, which he has so finely described, may be
said to have drowned all his own idle humours.

We must say we think little of our author's turn for satire. His
``English Bards and Scotch Reviewers'' is dogmatical and insolent, but
without refinement or point. He calls people names, and tries to
transfix a character with an epithet, which does not stick, because
it has no other foundation than his own petulance and spite; or he
endeavours to degrade by alluding to some circumstance of external
situation. He says of Mr. Wordsworth's poetry, that ``it is his
aversion.'' That may be: but whose fault is it? This is the satire of
a lord, who is accustomed to have all his whims or dislikes taken for
gospel, and who cannot be at the pains to do more than signify his
contempt or displeasure. If a great man meets with a rebuff which he
does not like, he turns on his heel, and this passes for a repartee.
The Noble Author says of a celebrated barrister and critic, that he was
``born in a garret sixteen stories high.'' The insinuation is not true; or
if it were, it is low. The allusion degrades the person who makes, not
him to whom it is applied. This is also the satire of a person of birth
and quality, who measures all merit by external rank, that is, by
his own standard. So his Lordship, in a ``Letter to the Editor of My
Grandmother's Review,'' addresses him fifty times as \emph{my dear Robarts};
nor is there any other wit in the article. This is surely a mere
assumption of superiority from his Lordship's rank, and is the sort of
\emph{quizzing} he might use to a person who came to hire himself as a valet
to him at \emph{Long's}\textemdash the waiters might laugh, the public will not. In
like manner, in the controversy about Pope, he claps Mr. Bowles on the
back with a coarse facetious familiarity, as if he were his chaplain
whom he had invited to dine with him, or was about to present to a
benefice. The reverend divine might submit to the obligation, but he has
no occasion to subscribe to the jest. If it is a jest that Mr. Bowles
should be a parson, and Lord Byron a peer, the world knew this before;
there was no need to write a pamphlet to prove it.

The \emph{Don Juan} indeed has great power; but its power is owing to the
force of the serious writing, and to the oddity of the contrast between
that and the flashy passages with which it is interlarded. From the
sublime to the ridiculous there is but one step. You laugh and are
surprised that any one should turn round and \emph{travestie} himself: the
drollery is in the utter discontinuity of ideas and feelings. He makes
virtue serve as a foil to vice; \emph{dandyism} is (for want of any other) a
variety of genius. A classical intoxication is followed by the splashing
of soda-water, by frothy effusions of ordinary bile. After the lightning
and the hurricane, we are introduced to the interior of the cabin and
the contents of wash-hand basins. The solemn hero of tragedy plays
\emph{Scrub} in the farce. This is ``very tolerable and not to be endured.''
The Noble Lord is almost the only writer who has prostituted his talents
in this way. He hallows in order to desecrate; takes a pleasure in
defacing the images of beauty his hands have wrought; and raises our
hopes and our belief in goodness to Heaven only to dash them to the
earth again, and break them in pieces the more effectually from the very
height they have fallen. Our enthusiasm for genius or virtue is thus
turned into a jest by the very person who has kindled it, and who thus
fatally quenches the sparks of both. It is not that Lord Byron is
sometimes serious and sometimes trifling, sometimes profligate, and
sometimes moral\textemdash but when he is most serious and most moral, he is only
preparing to mortify the unsuspecting reader by putting a pitiful \emph{hoax}
upon him. This is a most unaccountable anomaly. It is as if the eagle
were to build its eyry in a common sewer, or the owl were seen soaring
to the mid-day sun. Such a sight might make one laugh, but one would not
wish or expect it to occur more than once!\footnote{This censure applies to the first Cantos of \textsc{Don Juan} much
more than to the last. It has been called a \textsc{Tristram Shandy} in rhyme: it
is rather a poem written about itself.}

In fact, Lord Byron is the spoiled child of fame as well as fortune.
He has taken a surfeit of popularity, and is not contented to delight,
unless he can shock the public. He would force them to admire in spite
of decency and common sense\textemdash he would have them read what they would
read in no one but himself, or he would not give a rush for their
applause. He is to be ``a chartered libertine,'' from whom insults are
favours, whose contempt is to be a new incentive to admiration. His
Lordship is hard to please: he is equally averse to notice or neglect,
enraged at censure and scorning praise. He tries the patience of the
town to the very utmost, and when they shew signs of weariness or
disgust, threatens to \emph{discard} them. He says he will write on, whether
he is read or not. He would never write another page, if it were not
to court popular applause, or to affect a superiority over it. In this
respect also, Lord Byron presents a striking contrast to Sir Walter
Scott. The latter takes what part of the public favour falls to his
share, without grumbling (to be sure he has no reason to complain) the
former is always quarrelling with the world about his \emph{modicum} of
applause, the \emph{spolia opima} of vanity, and ungraciously throwing the
offerings of incense heaped on his shrine back in the faces of his
admirers. Again, there is no taint in the writings of the Author of
Waverley, all is fair and natural and above-board: he never outrages
the public mind. He introduces no anomalous character: broaches no
staggering opinion. If he goes back to old prejudices and superstitions
as a relief to the modern reader, while Lord Byron floats on swelling
paradoxes\textemdash 
\begin{quote}
  ``Like proud seas under him;''

\end{quote}
if the one defers too much to the spirit of antiquity, the other
panders to the spirit of the age, goes to the very edge of extreme and
licentious speculation, and breaks his neck over it. Grossness and
levity are the playthings of his pen. It is a ludicrous circumstance
that he should have dedicated his \emph{Cain} to the worthy Baronet! Did the
latter ever acknowledge the obligation? We are not nice, not very nice;
but we do not particularly approve those subjects that shine chiefly
from their rottenness: nor do we wish to see the Muses drest out in
the flounces of a false or questionable philosophy, like \emph{Portia} and
\emph{Nerissa} in the garb of Doctors of Law. We like metaphysics as well as
Lord Byron; but not to see them making flowery speeches, nor dancing a
measure in the fetters of verse. We have as good as hinted, that his
Lordship's poetry consists mostly of a tissue of superb common-places;
even his paradoxes are common-place. They are familiar in the schools:
they are only new and striking in his dramas and stanzas, by being out
of place. In a word, we think that poetry moves best within the circle
of nature and received opinion: speculative theory and subtle casuistry
are forbidden ground to it. But Lord Byron often wanders into this
ground wantonly, wilfully, and unwarrantably. The only apology we can
conceive for the spirit of some of Lord Byron's writings, is the spirit
of some of those opposed to him. They would provoke a man to write any
thing. ``Farthest from them is best.'' The extravagance and license of the
one seems a proper antidote to the bigotry and narrowness of the other.
The first \emph{Vision of Judgment} was a set-off to the second, though
\begin{quote}
  ``None but itself could be its parallel.''

\end{quote}
Perhaps the chief cause of most of Lord Byron's errors is, that he is
that anomaly in letters and in society, a Noble Poet. It is a double
privilege, almost too much for humanity. He has all the pride of birth
and genius. The strength of his imagination leads him to indulge in
fantastic opinions; the elevation of his rank sets censure at defiance.
He becomes a pampered egotist. He has a seat in the House of Lords, a
niche in the Temple of Fame. Every-day mortals, opinions, things are not
good enough for him to touch or think of. A mere nobleman is, in his
estimation, but ``the tenth transmitter of a foolish face:'' a mere man of
genius is no better than a worm. His Muse is also a lady of quality.
The people are not polite enough for him: the Court not sufficiently
intellectual. He hates the one and despises the other. By hating and
despising others, he does not learn to be satisfied with himself. A
fastidious man soon grows querulous and splenetic. If there is nobody
but ourselves to come up to our idea of fancied perfection, we easily
get tired of our idol. When a man is tired of what he is, by a natural
perversity he sets up for what he is not. If he is a poet, he pretends
to be a metaphysician: if he is a patrician in rank and feeling, he
would fain be one of the people. His ruling motive is not the love of
the people, but of distinction not of truth, but of singularity. He
patronizes men of letters out of vanity, and deserts them from caprice,
or from the advice of friends. He embarks in an obnoxious publication to
provoke censure, and leaves it to shift for itself for fear of scandal.
We do not like Sir Walter's gratuitous servility: we like Lord Byron's
preposterous \emph{liberalism} little better. He may affect the principles of
equality, but he resumes his privilege of peerage, upon occasion. His
Lordship has made great offers of service to the Greeks\textemdash money and
horses. He is at present in Cephalonia, waiting the event!

\bulletrule

We had written thus far when news came of the death of Lord Byron, and
put an end at once to a strain of somewhat peevish invective, which was
intended to meet his eye, not to insult his memory. Had we known that we
were writing his epitaph, we must have done it with a different feeling.
As it is, we think it better and more like himself, to let what we had
written stand, than to take up our leaden shafts, and try to melt them
into ``tears of sensibility,'' or mould them into dull praise, and an
affected shew of candour. We were not silent during the author's
life-time, either for his reproof or encouragement (such us we
could give, and \emph{he} did not disdain to accept) nor can we now turn
undertakers' men to fix the glittering plate upon his coffin, or fall
into the procession of popular woe.\textemdash Death cancels every thing but
truth; and strips a man of every thing but genius and virtue. It is a
sort of natural canonization. It makes the meanest of us sacred\textemdash it
installs the poet in his immortality, and lifts him to the skies. Death
is the great assayer of the sterling ore of talent. At his touch the
drossy particles fall off, the irritable, the personal, the gross, and
mingle with the dust\textemdash the finer and more ethereal part mounts with the
winged spirit to watch over our latest memory and protect our bones from
insult. We consign the least worthy qualities to oblivion, and cherish
the nobler and imperishable nature with double pride and fondness.
Nothing could shew the real superiority of genius in a more striking
point of view than the idle contests and the public indifference about
the place of Lord Byron's interment, whether in Westminster-Abbey or
his own family-vault. A king must have a coronation\textemdash a nobleman a
funeral-procession.\textemdash The man is nothing without the pageant. The poet's
cemetery is the human mind, in which he sows the seeds of never ending
thought\textemdash his monument is to be found in his works:
\begin{quote}
  ``Nothing can cover his high fame but Heaven; \\
  No pyramids set off his memory, \\
  But the eternal substance of his greatness.''
\end{quote}
Lord Byron is dead: he also died a martyr to his zeal in the cause of
freedom, for the last, best hopes of man. Let that be his excuse and his
epitaph!