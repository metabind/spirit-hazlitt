\chapter[Mr. T. Moore and Mr. Leigh Hunt]
{mr. t. moore {\normalsize and} mr. leigh hunt}

\begin{verse}
  \vleftofline{``}Or winglet of the fairy humming-bird,\\
  Like atoms of the rainbow fluttering round.''\\
\sourceatright{\textsc{Campbell}.\hspace{5pc}}
\end{verse}

The lines placed at the head of this sketch, from a contemporary writer,
appear to us very descriptive of Mr. Moore's poetry. His verse is like
a shower of beauty; a dance of images; a stream of music; or like the
spray of the water-fall, tinged by the morning-beam with rosy light.
The characteristic distinction of our author's style is this continuous
and incessant flow of voluptuous thoughts and shining allusions. He
ought to write with a crystal pen on silver paper. His subject is set
off by a dazzling veil of poetic diction, like a wreath of flowers
gemmed with innumerous dewdrops, that weep, tremble, and glitter in
liquid softness and pearly light, while the song of birds ravishes
the ear, and languid odours breathe around, and Aurora opens Heaven's
smiling portals, Peris and nymphs peep through the golden glades, and an
Angel's wing glances over the glossy scene.
\begin{verse}
  \vleftofline{``}No dainty flower or herb that grows on ground,\\
  No arboret with painted blossoms drest,\\
  And smelling sweet, but there it might be found\\
  To bud out fair, and its sweet smells throw all around.\\ 
\vspace{\stanzaskip}
  No tree, whose branches did not bravely spring;\\
  No branch, whereon a fine bird did not sit;\\
  No bird, but did her shrill notes sweetly sing;\\
  No song, but did contain a lovely dit:\\
  Trees, branches, birds, and songs were framed fit\\
  For to allure frail minds to careless ease.'' \dots
\end{verse}
Mr. Campbell's imagination is fastidious and select; and hence,
though we meet with more exquisite beauties in his writings, we
meet with them more rarely: there is comparatively a dearth of
ornament. But Mr.  Moore's strictest economy is ``wasteful and
superfluous excess:'' he is always liberal, and never at a loss;
for sooner than not stimulate and delight the reader, he is
willing to be tawdry, or superficial, or common-place. His Muse
must be fine at any rate, though she should paint, and wear
cast-off decorations. Rather than have any lack of excitement, he
repeats himself; and ``Eden, and Eblis, and cherub-smiles'' fill
up the pauses of the sentiment with a sickly monotony.\textemdash
It has been too much our author's object to pander to the
artificial taste of the age; and his productions, however
brilliant and agreeable, are in consequence somewhat meretricious
and effeminate. It was thought formerly enough to have an
occasionally fine passage in the progress of a story or a poem,
and an occasionally striking image or expression in a fine passage
or description. But this style, it seems, was to be exploded as
rude, Gothic, meagre, and dry. Now all must be raised to the same
tantalising and preposterous level. There must be no pause, no
interval, no repose, no gradation. Simplicity and truth yield up
the palm to affectation and grimace. The craving of the public
mind after novelty and effect is a false and uneasy appetite that
must be pampered with fine words at every step\textemdash we must
be tickled with sound, startled with shew, and relieved by the
importunate, uninterrupted display of fancy and verbal tinsel as
much as possible from the fatigue of thought or shock of
feeling. A poem is to resemble an exhibition of fireworks, with a
continual explosion of quaint figures and devices, flash after
flash, that surprise for the moment, and leave no trace of light
or warmth behind them. Or modern poetry in its retrograde progress
comes at last to be constructed on the principles of the modern
\textsc{Opera}, where an attempt is made to gratify every sense at every
instant, and where the understanding alone is insulted and the
heart mocked. It is in this view only that we can discover that
Mr. Moore's poetry is vitiated or immoral,\textemdash it seduces
the taste and enervates the imagination. It creates a false
standard of reference, and inverts or decompounds the natural
order of association, in which objects strike the thoughts and
feelings.  His is the poetry of the bath, of the toilette, of the
saloon, of the fashionable world; not the poetry of nature, of the
heart, or of human life. He stunts and enfeebles equally the
growth of the imagination and the affections, by not taking the
seed of poetry and sowing it in the ground of truth, and letting
it expand in the dew and rain, and shoot up to heaven,
\begin{verse}
  \vleftofline{``}And spread its sweet leaves to the air, \\
  Or dedicate its beauty to the sun,''\textemdash 
\end{verse}
instead of which he anticipates and defeats his own object, by
plucking flowers and blossoms from the stem, and setting them in
the ground of idleness and folly\textemdash or in the cap of his
own vanity, where they soon wither and disappear, ``dying or ere
they sicken!'' This is but a sort of child's play, a short-sighted
ambition. In Milton we meet with many prosaic lines, either
because the subject does not require raising or because they are
necessary to connect the story, or serve as a relief to other
passages\textemdash there is not such a thing to be found in all
Mr. Moore's writings. His volumes present us with ``a perpetual
feast of nectar'd sweets''\textemdash but we cannot
add,\textemdash ``where no crude surfeit reigns.'' He indeed cloys
with sweetness; he obscures with splendour; he fatigues with
gaiety. We are stifled on beds of roses\textemdash we literally
lie ``on the rack of restless ecstacy.'' His flowery fancy ``looks
so fair and smells so sweet, that the sense aches at it.'' His
verse droops and languishes under a load of beauty, like a bough
laden with fruit. His gorgeous style is like ``another morn risen
on mid-noon.'' There is no passage that is not made up of blushing
lines, no line that is not enriched with a sparkling metaphor, no
image that is left unadorned with a double epithet\textemdash all
his verbs, nouns, adjectives, are equally glossy, smooth, and
beautiful. Every stanza is transparent with light, perfumed with
odours, floating in liquid harmony, melting in luxurious,
evanescent delights. His Muse is never contented with an offering
from one sense alone, but brings another rifled charm to match it,
and revels in a fairy round of pleasure. The interest is not
dramatic, but melo-dramatic\textemdash it is a mixture of
painting, poetry, and music, of the natural and preternatural, of
obvious sentiment and romantic costume. A rose is a \emph{Gul}, a
nightingale a \emph{Bulbul}. We might fancy ourselves in an
eastern harem, amidst Ottomans, and otto of roses, and veils and
spangles, and marble pillars, and cool fountains, and Arab maids
and Genii, and magicians, and Peris, and cherubs, and what not?
Mr. Moore has a little mistaken the art of poetry for the
\emph{cosmetic art}. He does not compose an historic group, or
work out a single figure; but throws a variety of elementary
sensations, of vivid impressions together, and calls it a
description. He makes out an inventory of beauty\textemdash the
smile on the lips, the dimple on the cheeks, \emph{item}, golden
locks, \emph{item}, a pair of blue wings, \emph{item}, a silver
sound, with breathing fragrance and radiant light, and thinks it a
character or a story. He gets together a number of fine things and
fine names, and thinks that, flung on heaps, they make up a fine
poem. This dissipated, fulsome, painted, patch-work style may
succeed in the levity and languor of the \emph{boudoir}, or might
have been adapted to the Pavilions of royalty, but it is not the
style of Parnassus, nor a passport to Immortality. It is not the
taste of the ancients, ``'tis not classical lore''\textemdash nor
the fashion of Tibullus, or Theocritus, or Anacreon, or Virgil, or
Ariosto, or Pope, or Byron, or any great writer among the living
or the dead, but it is the style of our English Anacreon, and it
is (or was) the fashion of the day! Let one example (and that an
admired one) taken from \emph{Lalla Rookh}, suffice to explain the
mystery and soften the harshness of the foregoing criticism.
\begin{verse}
  \vleftofline{``}Now upon Syria's land of roses\\
  Softly the light of eve reposes,\\
  And like a glory, the broad sun\\
  Hangs over sainted Lebanon:\\
  Whose head in wintry grandeur towers,\\
  And whitens with eternal sleet,\\
  While summer, in a vale of flowers,\\
  Is sleeping rosy at his feet.\\
  To one who look'd from upper air,\\
  O'er all th' enchanted regions there,\\
  How beauteous must have been the glow,\\
  The life, the sparkling from below!\\
  Fair gardens, shining streams, with ranks\\
  Of golden melons on their banks,\\
  More golden where the sun-light falls,\textemdash \\
  Gay lizards, glittering on the walls\\
  Of ruin'd shrines, busy and bright\\
  As they were all alive with light;\textemdash \\
  And yet more splendid, numerous flocks\\
  Of pigeons, settling on the rocks,\\
  With their rich, restless wings, that gleam\\
  Variously in the crimson beam\\
  Of the warm west, as if inlaid\\
  With brilliants from the mine, or made\\
  Of tearless rainbows, such as span\\
  The unclouded skies of Peristan!\\
  And then, the mingling sounds that come\\
  Of shepherd's ancient reed, with hum\\
  Of the wild bees of Palestine,\\
  Banquetting through the flowery vales\textemdash \\
  And, Jordan, those sweet banks of thine,\\
  And woods, so full of nightingales.''\textemdash 
\end{verse}
The following lines are the very perfection of Della Cruscan sentiment,
and affected orientalism of style. The Peri exclaims on finding that old
talisman and hackneyed poetical machine, ``a penitent tear''\textemdash 
\begin{verse}
  \vleftofline{``}Joy, joy forever! my task is done\textemdash \\
  The gates are pass'd, and Heaven is won!\\
  Oh! am I not happy? I am, I am\textemdash \\
  To thee, sweet Eden! how dark and sad\\
  Are the diamond turrets of Shadukiam,\\
  And the fragrant bowers of Amberabad.''
\end{verse}
There is in all this a play of fancy, a glitter of words, a shallowness
of thought, and a want of truth and solidity that is wonderful, and
that nothing but the heedless, rapid glide of the verse could render
tolerable:\pcdash{1} it seems that the poet, as well as the lover,
\begin{verse}
  \vleftofline{``}May bestride the Gossamer,\\
  That wantons in the idle, summer air,\\
  And yet not fall, so light is vanity!''
\end{verse}
Mr. Moore ought not to contend with serious difficulties or with entire
subjects. He can write verses, not a poem. There is no principle of
massing or of continuity in his productions\textemdash neither height nor breadth
nor depth of capacity. There is no truth of representation, no strong
internal feeling\textemdash but a continual flutter and display of affected airs
and graces, like a finished coquette, who hides the want of symmetry by
extravagance of dress, and the want of passion by flippant forwardness
and unmeaning sentimentality. All is flimsy, all is florid to excess.
His imagination may dally with insect beauties, with Rosicrucian spells;
may describe a butterfly's wing, a flower-pot, a fan: but it should not
attempt to span the great outlines of nature, or keep pace with the
sounding march of events, or grapple with the strong fibres of the human
heart. The great becomes turgid in his hands, the pathetic insipid. If
Mr. Moore were to describe the heights of Chimboraco, instead of the
loneliness, the vastness and the shadowy might, he would only think
of adorning it with roseate tints, like a strawberry-ice, and would
transform a magician's fortress in the Himmalaya (stripped of its
mysterious gloom and frowning horrors) into a jeweller's toy, to be set
upon a lady's toilette. In proof of this, see above ``the diamond turrets
of Shadukiam,'' \&c. The description of Mokanna in the fight, though
it has spirit and grandeur of effect, has still a great alloy of the
mock-heroic in it. The route of blood and death, which is otherwise well
marked, is infested with a swarm of ``fire-fly'' fancies.
\begin{verse}
  \vleftofline{``}In vain Mokanna, 'midst the general flight,\\
  Stands, like the red moon, in some stormy night.\\
  Among the fugitive clouds, that hurrying by,\\
  Leave only her unshaken in the sky.''
\end{verse}
This simile is fine, and would have been perfect, but that the moon is
not red, and that she seems to hurry by the clouds, not they by her. The
description of the warrior's youthful adversary,
\begin{verse}
 \pcdash{2}``Whose coming seems\\
  A light, a glory, such as breaks in dreams.''\textemdash 
\end{verse}
is fantastic and enervated\textendash a field of battle has
nothing to do with dreams:\textendash and again, the two lines
immediately after,
\begin{verse}
  \vleftofline{``}And every sword, true as o'er billows dim\\
  The needle tracks the load-star, following him''\textemdash 
\end{verse}
\fixspacing{3}{are a mere piece of enigmatical ingenuity and scientific
\emph{mimminee-pimminee.}}

We cannot except the \emph{Irish Melodies} from the same censure. If these
national airs do indeed express the soul of impassioned feeling in his
countrymen, the case of Ireland is hopeless. If these prettinesses pass
for patriotism, if a country can heave from its heart's core only these
vapid, varnished sentiments, lip-deep, and let its tears of blood
evaporate in an empty conceit, let it be governed as it has been. There
are here no tones to waken Liberty, to console Humanity. Mr. Moore
converts the wild harp of Erin into a musical snuff-box\footnote{Compare his songs with Burns's.}!\textemdash We \emph{do}
except from this censure the author's political squibs, and the ``Two-
penny Post-bag.'' These are essences, are ``nests of spicery'', bitter and
sweet, honey and gall together. No one can so well describe the set
speech of a dull formalist\footnote{\begin{verse}
  \vleftofline{``}There was a little man, and he had a little soul,\\
  And he said, Little soul, let us try,'' \&c.\textemdash 
\end{verse}
Parody on
\begin{verse}
   ``There was a little man, and he had a little gun.''\textemdash 
\end{verse}
One should think this exquisite ridicule of a pedantic effusion might
have silenced for ever the automaton that delivered it: but the
official personage in question at the close of the Session addressed an
extra-official congratulation to the Prince Regent on a bill that had
\emph{not} passed\textemdash as if to repeat and insist upon our errors were to justify
them.}, or the flowing locks of a Dowager,
\begin{verse}
   ``In the manner of Ackermann's dresses for May.''
\end{verse}
His light, agreeable, polished style pierces through the body of the
court\textemdash hits off the faded graces of ``an Adonis of fifty'', weighs the
vanity of fashion in tremulous scales, mimics the grimace of affectation
and folly, shews up the littleness of the great, and spears a phalanx of
statesmen with its glittering point as with a diamond broach.
\begin{verse}
  \vleftofline{``}In choosing songs the Regent named\\
  \vleftofline{`}Had I a heart for falsehood fram'd:'\\
  While gentle Hertford begg'd and pray'd\\
  For `Young I am, and sore afraid.' ''
\end{verse}
Nothing in Pope or Prior ever surpassed the delicate insinuation
and adroit satire of these lines, and hundreds more of our author's
composition. We wish he would not take pains to make us think of them
with less pleasure than formerly.\textemdash The ``Fudge Family'' is in the same
spirit, but with a little falling-off. There is too great a mixture of
undisguised Jacobinism and fashionable \emph{slang}. The ``divine Fanny Bias''
and ``the mountains \emph{à la Russe}'' figure in somewhat quaintly with
Buonaparte and the Bourbons. The poet also launches the lightning of
political indignation; but it rather plays round and illumines his own
pen than reaches the devoted heads at which it is aimed!

Mr. Moore is in private life an amiable and estimable man. The
embellished and voluptuous style of his poetry, his unpretending origin,
and his \emph{mignon} figure soon introduced him to the notice of the
great, and his gaiety, his wit, his good-humour, and many agreeable
accomplishments fixed him there, the darling of his friends and the idol
of fashion. If he is no longer familiar with Royalty as with his garter,
the fault is not his\textemdash his adherence to his principles caused the
separation\textemdash his love of his country was the cloud that intercepted the
sunshine of court-favour. This is so far well. Mr. Moore vindicates his
own dignity; but the sense of intrinsic worth, of wide-spread fame, and
of the intimacy of the great makes him perhaps a little too fastidious
and \emph{exigeant} as to the pretensions of others. He has been so long
accustomed to the society of Whig Lords, and so enchanted by the smile
of beauty and fashion, that he really fancies himself one of the \emph{set},
to which he is admitted on sufferance, and tries very unnecessarily to
keep others out of it. He talks familiarly of works that are or are
not read ``in \emph{our} circle;'' and seated smiling and at his ease in a
coronet-coach, enlivening the owner by his brisk sallies and Attic
conceits, is shocked, as he passes, to see a Peer of the realm shake
hands with a poet. There is a little indulgence of spleen and envy, a
little servility and pandering to aristocratic pride in this proceeding.
Is Mr. Moore bound to advise a Noble Poet to get as fast as possible out
of a certain publication, lest he should not be able to give an
account at Holland or at Lansdown House, how his friend Lord B\textemdash \textemdash had
associated himself with his friend L. H\textemdash \textemdash ? Is he afraid that the
``Spirit of Monarchy'' will eclipse the ``Fables for the Holy Alliance'' in
virulence and plain speaking? Or are the members of the ``Fudge Family''
to secure a monopoly for the abuse of the Bourbons and the doctrine of
Divine Right? Because he is genteel and sarcastic, may not others be
paradoxical and argumentative? Or must no one bark at a Minister or
General, unless they have been first dandled, like a little French
pug-dog, in the lap of a lady of quality? Does Mr. Moore insist on the
double claim of birth and genius as a title to respectability in all
advocates of the popular side\textemdash but himself? Or is he anxious to keep the
pretensions of his patrician and plebeian friends quite separate, so
as to be himself the only point of union, a sort of \emph{double meaning},
between the two? It is idle to think of setting bounds to the weakness
and illusions of self-love as long as it is confined to a man's own
breast; but it ought not to be made a plea for holding back the powerful
hand that is stretched out to save another struggling with the tide
of popular prejudice, who has suffered shipwreck of health, fame and
fortune in a common cause, and who has deserved the aid and the good
wishes of all who are (on principle) embarked in the same cause by equal
zeal and honesty, if not by equal talents to support and to adorn it!

We shall conclude the present article with a short notice of an
individual who, in the cast of his mind and in political
principle, bears no very remote resemblance to the patriot and wit
just spoken of, and on whose merits we should descant at greater
length, but that personal intimacy might be supposed to render us
partial. It is well when personal intimacy produces this effect;
and when the light, that dazzled us at a distance, does not on a
closer inspection turn out an opaque substance. This is a charge
that none of his friends will bring against Mr. Leigh Hunt. He
improves upon acquaintance. The author translates admirably into
the man. Indeed the very faults of his style are virtues in the
individual. His natural gaiety and sprightliness of manner, his
high animal spirits, and the \emph{vinous} quality of his mind,
produce an immediate fascination and intoxication in those who
come in contact with him, and carry off in society whatever in his
writings may to some seem flat and impertinent. From great
sanguineness of temper, from great quickness and unsuspecting
simplicity, he runs on to the public as he does at his own
fire-side, and talks about himself, forgetting that he is not
always among friends. His look, his tone are required to point
many things that he says: his frank, cordial manner reconciles you
instantly to a little over-bearing, over-weening self-
complacency. ``To be admired, he needs but to be seen:'' but
perhaps he ought to be seen to be fully appreciated. No one ever
sought his society who did not come away with a more favourable
opinion of him: no one was ever disappointed, except those who had
entertained idle prejudices against him. He sometimes trifles with
his readers, or tires of a subject (from not being urged on by the
stimulus of immediate sympathy)\textemdash but in conversation he
is all life and animation, combining the vivacity of the
school-boy with the resources of the wit and the taste of the
scholar. The personal character, the spontaneous impulses, do not
appear to excuse the author, unless you are acquainted with his
situation and habits\textemdash like some proud beauty who gives
herself what we think strange airs and graces under a mask, but
who is instantly forgiven when she shews her face. We have said
that Lord Byron is a sublime coxcomb: why should we not say that
Mr. Hunt is a delightful one? There is certainly an exuberance of
satisfaction in his manner which is more than the strict logical
premises warrant, and which dull and phlegmatic constitutions know
nothing of, and cannot understand till they see it. He is the only
poet or literary man we ever knew who puts us in mind of Sir John
Suckling or Killigrew or Carew; or who united rare intellectual
acquirements with outward grace and natural gentility.  Mr. Hunt
ought to have been a gentleman born, and to have patronised men of
letters. He might then have played, and sung, and laughed, and
talked his life away; have written manly prose, elegant verse; and
his \emph{Story of Rimini} would have been praised by
Mr. Blackwood. As it is, there is no man now living who at the
same time writes prose and verse so well, with the exception of
Mr. Southey (an exception, we fear, that will be little palatable
to either of these gentlemen). His prose writings, however,
display more consistency of principle than the laureate's: his
verses more taste. We will venture to oppose his Third Canto of
the \emph{Story of Rimini} for classic elegance and natural
feeling to any equal number of lines from Mr. Southey's Epics or
from Mr. Moore's Lalla Rookh. In a more gay and conversational
style of writing, we think his \emph{Epistle to Lord Byron} on his
going abroad, is a masterpiece;\textemdash and the \emph{Feast of
  the Poets} has run through several editions. A light, familiar
grace, and mild unpretending pathos are the characteristics of his
more sportive or serious writings, whether in poetry or prose. A
smile plays round the features of the one; a tear is ready to
start from the thoughtful gaze of the other. He perhaps takes too
little pains, and indulges in too much wayward caprice in both. A
wit and a poet, Mr. Hunt is also distinguished by fineness of tact
and sterling sense: he has only been a visionary in humanity, the
fool of virtue. What then is the drawback to so many shining
qualities, that has made them useless, or even hurtful to their
owner? His crime is, to have been Editor of the \emph{Examiner}
ten years ago, when some allusion was made in it to the age of the
present king, and that, though his Majesty has grown older, our
luckless politician is no wiser than he was then!
