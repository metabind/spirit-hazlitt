\chapter[Mr. Brougham and Sir F. Burdett]
{mr. brougham {\normalsize and} sir f. burdett}

There is a class of eloquence which has been described and
particularly insisted on, under the style and title of \emph{Irish
  Eloquence}: there is another class which it is not absolutely
unfair to oppose to this, and that is the Scotch. The first of
these is entirely the offspring of \emph{impulse}: the last of
\emph{mechanism}. The one is as full of fancy as it is bare of
facts: the other excludes all fancy, and is weighed down with
facts. The one is all fire, the other all ice: the one nothing but
enthusiasm, extravagance, eccentricity; the other nothing but
logical deductions, and the most approved postulates. The one
without scruple, nay, with reckless zeal, throws the reins loose
on the neck of the imagination: the other pulls up with a
curbbridle, and starts at every casual object it meets in the way
as a bug-bear. The genius of Irish oratory stands forth in the
naked majesty of untutored nature, its eye glancing wildly round
on all objects, its tongue darting forked fire: the genius of
Scottish eloquence is armed in all the panoply of the schools; its
drawling, ambiguous dialect seconds its circumspect dialectics;
from behind the vizor that guards its mouth and shadows its
pent-up brows, it sees no visions but its own set purpose, its own
\emph{data}, and its own dogmas. It ``has no figures, nor no
fantasies,'' but ``those which busy care draws in the brains of
men,'' or which set off its own superior acquirements and
wisdom. It scorns to ``tread the primrose path of
dalliance''\textemdash it shrinks back from it as from a precipice,
and keeps in the iron rail-way of the understanding. Irish
oratory, on the contrary, is a sort of aeronaut: it is always
going up in a balloon, and breaking its neck, or coming down in
the parachute. It is filled full with gaseous matter, with whim
and fancy, with alliteration and antithesis, with heated passion
and bloated metaphors, that burst the slender, silken covering of
sense; and the airy pageant, that glittered in empty space and
rose in all the bliss of ignorance, flutters and sinks down to its
native bogs! If the Irish orator riots in a studied neglect of his
subject and a natural confusion of ideas, playing with words,
ranging them into all sorts of fantastic combinations, because in
the unlettered void or chaos of his mind there is no obstacle to
their coalescing into any shapes they please, it must be confessed
that the eloquence of the Scotch is encumbered with an excess of
knowledge, that it cannot get on for a crowd of difficulties, that
it staggers under a load of topics, that it is so environed in the
forms of logic and rhetoric as to be equally precluded from
originality or absurdity, from beauty or deformity:\textemdash the
plea of humanity is lost by going through the process of law, the
firm and manly tone of principle is exchanged for the wavering and
pitiful cant of policy, the living bursts of passion are reduced
to a defunct \emph{common-place}, and all true imagination is
buried under the dust and rubbish of learned models and imposing
authorities. If the one is a bodiless phantom, the other is a
lifeless skeleton: if the one in its feverish and hectic
extravagance resembles a sick man's dream, the other is akin to
the sleep of death\textemdash cold, stiff, unfeeling, monumental!
Upon the whole, we despair less of the first than of the last, for
the principle of life and motion is, after all, the primary
condition of all genius. The luxuriant wildness of the one may be
disciplined, and its excesses sobered down into reason; but the
dry and rigid formality of the other can never burst the shell or
husk of oratory. It is true that the one is disfigured by the
puerilities and affectation of a Phillips; but then it is redeemed
by the manly sense and fervour of a Plunket, the impassioned
appeals and flashes of wit of a Curran, and by the golden tide of
wisdom, eloquence, and fancy, that flowed from the lips of a
Burke. In the other, we do not sink so low in the negative series;
but we get no higher in the ascending scale than a Mackintosh or a
Brougham.\footnote{ Mr. Brougham is not a Scotchman literally, but
  by adoption.} It may be suggested that the late Lord Erskine
enjoyed a higher reputation as an orator than either of these: but
he owed it to a dashing and graceful manner, to presence of mind,
and to great animation in delivering his sentiments. Stripped of
these outward and personal advantages, the matter of his speeches,
like that of his writings, is nothing, or perfectly inert and
dead. Mr. Brougham is from the North of England, but he was
educated in Edinburgh, and represents that school of politics and
political economy in the House.  He differs from Sir James
Mackintosh in this, that he deals less in abstract principles, and
more in individual details. He makes less use of general topics,
and more of immediate facts. Sir James is better acquainted with
the balance of an argument in old authors; Mr. Brougham with the
balance of power in Europe. If the first is better versed in the
progress of history, no man excels the last in a knowledge of the
course of exchange. He is apprised of the exact state of our
exports and imports, and scarce a ship clears out its cargo at
Liverpool or Hull, but he has notice of the bill of lading. Our
colonial policy, prison-discipline, the state of the Hulks,
agricultural distress, commerce and manufactures, the Bullion
question, the Catholic question, the Bourbons or the Inquisition,
``domestic treason, foreign levy,'' nothing can come amiss to
him\textemdash he is at home in the crooked mazes of rotten
boroughs, is not baffled by Scotch law, and can follow the meaning
of one of Mr. Canning's speeches. With so many resources, with
such variety and solidity of information, Mr. Brougham is rather a
powerful and alarming, than an effectual debater. In so many
details (which he himself goes through with unwearied and
unshrinking resolution) the spirit of the question is lost to
others who have not the same voluntary power of attention or the
same interest in hearing that he has in speaking; the original
impulse that urged him forward is forgotten in so wide a field, in
so interminable a career. If he can, others \emph{cannot} carry
all he knows in their heads at the same time; a rope of
circumstantial evidence does not hold well together, nor drag the
unwilling mind along with it (the willing mind hurries on before
it, and grows impatient and absent)\textemdash he moves in an
unmanageable procession of facts and proofs, instead of coming to
the point at once\textemdash and his premises (so anxious is he to
proceed on sure and ample grounds) overlay and block up his
conclusion, so that you cannot arrive at it, or not till the first
fury and shock of the onset is over. The ball, from the too great
width of the \emph{calibre} from which it is sent, and from
striking against such a number of hard, projecting points, is
almost spent before it reaches its destination. He keeps a ledger
or a debtor-and-creditor account between the Government and the
Country, posts so much actual crime, corruption, and injustice
against so much contingent advantage or sluggish prejudice, and at
the bottom of the page brings in the balance of indignation and
contempt, where it is due.  But people are not to be
\emph{calculated into} contempt or indignation on abstract
grounds; for however they may submit to this process where their
own interests are concerned, in what regards the public good we
believe they must see and feel instinctively, or not at all. There
is (it is to be lamented) a good deal of froth as well as strength
in the popular spirit, which will not admit of being
\emph{decanted} or served out in formal driblets; nor will spleen
(the soul of Opposition) bear to be corked up in square patent
bottles, and kept for future use! In a word, Mr. Brougham's is
ticketed and labelled eloquence, registered and in numeros (like
the successive parts of a Scotch Encyclopedia)\textemdash it is
clever, knowing, imposing, masterly, an extraordinary display of
clearness of head, of quickness and energy of thought, of
application and industry; but it is not the eloquence of the
imagination or the heart, and will never save a nation or an
individual from perdition.

Mr. Brougham has one considerable advantage in debate: he is overcome
by no false modesty, no deference to others. But then, by a natural
consequence or parity of reasoning, he has little sympathy with other
people, and is liable to be mistaken in the effect his arguments will
have upon them. He relies too much, among other things, on the patience
of his hearers, and on his ability to turn every thing to his own
advantage. He accordingly goes to the full length of \emph{his tether} (in
vulgar phrase) and often overshoots the mark. \emph{C'est dommage}. He has no
reserve of discretion, no retentiveness of mind or check upon himself.
He needs, with so much wit,
\begin{quote}
  ``As much again to govern it.''

\end{quote}
He cannot keep a good thing or a shrewd piece of information in his
possession, though the letting it out should mar a cause. It is not
that he thinks too much of himself, too little of his cause: but he is
absorbed in the pursuit of truth as an abstract inquiry, he is led away
by the headstrong and over-mastering activity of his own mind. He is
borne along, almost involuntarily, and not impossibly against his better
judgment, by the throng and restlessness of his ideas as by a crowd
of people in motion. His perceptions are literal, tenacious,
\emph{epileptic}\textemdash his understanding voracious of facts, and equally
communicative of them\textemdash and he proceeds to
\begin{verse}
  ``\pcdash{3} Pour out all as plain \\
  As downright Shippen or as old Montaigne''\textemdash

\end{verse}
without either the virulence of the one or the \emph{bonhommie} of
the other.  The repeated, smart, unforeseen discharges of the
truth jar those that are next him. He does not dislike this state
of irritation and collision, indulges his curiosity or his
triumph, till by calling for more facts or hazarding some extreme
inference, he urges a question to the verge of a precipice, his
adversaries urge it \emph{over}, and he himself shrinks back from
the consequence\textemdash
\begin{quote}
  ``Scared at the sound himself has made!''

\end{quote}
Mr. Brougham has great fearlessness, but not equal firmness; and
after going too far on the \emph{forlorn hope}, turns short round
without due warning to others or respect for himself. He is
adventurous, but easily panic-struck; and sacrifices the vanity of
self-opinion to the necessity of self-preservation. He is too
improvident for a leader, too petulant for a partisan; and does
not sufficiently consult those with whom he is supposed to act in
concert. He sometimes leaves them in the lurch, and is sometimes
left in the lurch by them. He wants the principle of
co-operation. He frequently, in a fit of thoughtless levity, gives
an unexpected turn to the political machine, which alarms older
and more experienced heads: if he was not himself the first to get
out of harm's way and escape from the danger, it would be
well!\textemdash We hold, indeed, as a general rule, that no man
born or bred in Scotland can be a great orator, unless he is a
mere quack; or a great statesman unless he turns plain knave. The
national gravity is against the first: the national caution is
against the last. To a Scotchman if a thing \emph{is, it is};
there is an end of the question with his opinion about it. He is
positive and abrupt, and is not in the habit of conciliating the
feelings or soothing the follies of others. His only way therefore
to produce a popular effect is to sail with the stream of
prejudice, and to vent common dogmas, ``the total grist, unsifted,
husks and all,'' from some evangelical pulpit. This may answer,
and it has answered. On the other hand, if a Scotchman, born or
bred, comes to think at all of the feelings of others, it is not
as they regard them, but as their opinion reacts on his own
interest and safety. He is therefore either pragmatical and
offensive, or if he tries to please, he becomes cowardly and
fawning. His public spirit wants pliancy; his selfish compliances
go all lengths. He is as impracticable as a popular partisan, as
he is mischievous as a tool of Government. We do not wish to press
this argument farther, and must leave it involved in some degree
of obscurity, rather than bring the armed intellect of a whole
nation on our heads.

Mr. Brougham speaks in a loud and unmitigated tone of voice,
sometimes almost approaching to a scream. He is fluent, rapid,
vehement, full of his subject, with evidently a great deal to say,
and very regardless of the manner of saying it. As a lawyer, he
has not hitherto been remarkably successful. He is not profound in
cases and reports, nor does he take much interest in the peculiar
features of a particular cause, or shew much adroitness in the
management of it. He carries too much weight of metal for ordinary
and petty occasions: he must have a pretty large question to
discuss, and must make \emph{thorough-stitch} work of it. He,
however, had an encounter with Mr. Phillips the other day, and
shook all his tender blossoms, so that they fell to the ground,
and withered in an hour; but they soon bloomed again! Mr. Brougham
writes almost, if not quite, as well as he speaks. In the midst of
an Election contest he comes out to address the populace, and goes
back to his study to finish an article for the Edinburgh Review;
sometimes indeed wedging three or four articles (in the shape of
\emph{refaccimentos} of his own pamphlets or speeches in
parliament) into a single number. Such indeed is the activity of
his mind that it appears to require neither repose, nor any other
stimulus than a delight in its own exercise. He can turn his hand
to any thing, but he cannot be idle. There are few intellectual
accomplishments which he does not possess, and possess in a very
high degree. He speaks French (and, we believe, several other
modern languages) fluently: is a capital mathematician, and
obtained an introduction to the celebrated Carnot in this latter
character, when the conversation turned on squaring the circle,
and not on the propriety of confining France within the natural
boundary of the Rhine. Mr. Brougham is, in fact, a striking
instance of the versatility and strength of the human mind, and
also in one sense of the length of human life, if we make a good
use of our time. There is room enough to crowd almost every art
and science into it. If we pass ``no day without a line,'' visit
no place without the company of a book, we may with ease fill
libraries or empty them of their contents. Those who complain of
the shortness of life, let it slide by them without wishing to
seize and make the most of its golden minutes. The more we do, the
more we can do; the more busy we are, the more leisure we have. If
any one possesses any advantage in a considerable degree, he may
make himself master of nearly as many more as he pleases, by
employing his spare time and cultivating the waste faculties of
his mind. While one person is determining on the choice of a
profession or study, another shall have made a fortune or gained a
merited reputation. While one person is dreaming over the meaning
of a word, another will have learnt several languages. It is not
incapacity, but indolence, indecision, want of imagination, and a
proneness to a sort of mental tautology, to repeat the same images
and tread the same circle, that leaves us so poor, so dull, and
inert as we are, so naked of acquirement, so barren of resources!
While we are walking backwards and forwards between Charing-Cross
and Temple-Bar, and sitting in the same coffee-house every day, we
might make the grand tour of Europe, and visit the Vatican and the
Louvre. Mr. Brougham, among other means of strengthening and
enlarging his views, has visited, we believe, most of the courts,
and turned his attention to most of the Constitutions of the
continent. He is, no doubt, a very accomplished, active-minded,
and admirable person.

Sir Francis Burdett, in many respects, affords a contrast to the
foregoing character. He is a plain, unaffected, unsophisticated
English gentleman. He is a person of great reading too and
considerable information, but he makes very little display of
these, unless it be to quote Shakespear, which he does often with
extreme aptness and felicity.  Sir Francis is one of the most
pleasing speakers in the House, and is a prodigious favourite of
the English people. So he ought to be: for he is one of the few
remaining examples of the old English understanding and old
English character. All that he pretends to is common sense and
common honesty; and a greater compliment cannot be paid to these
than the attention with which he is listened to in the House of
Commons. We cannot conceive a higher proof of courage than the
saying things which he has been known to say there; and we have
seen him blush and appear ashamed of the truths he has been
obliged to utter, like a bashful novice. He could not have uttered
what he often did there, if, besides his general respectability,
he had not been a very honest, a very good-tempered, and a very
good-looking man. But there was evidently no wish to shine, nor
any desire to offend: it was painful to him to hurt the feelings
of those who heard him, but it was a higher duty in him not to
suppress his sincere and earnest convictions. It is wonderful how
much virtue and plain-dealing a man may be guilty of with
impunity, if he has no vanity, or ill-nature, or duplicity to
provoke the contempt or resentment of others, and to make them
impatient of the superiority he sets up over them. We do not
recollect that Sir Francis ever endeavoured to atone for any
occasional indiscretions or intemperance by giving the Duke of
York credit for the battle of Waterloo, or congratulating
Ministers on the confinement of Buonaparte at St. Helena. There is
no honest cause which he dares not avow: no oppressed individual
that he is not forward to succour. He has the firmness of manhood
with the unimpaired enthusiasm of youthful feeling about him. His
principles are mellowed and improved, without having become less
sound with time: for at one period he sometimes appeared to come
charged to the House with the petulance and caustic
sententiousness he had imbibed at Wimbledon Common. He is never
violent or in extremes, except when the people or the parliament
happen to be out of their senses; and then he seems to regret the
necessity of plainly telling them he thinks so, instead of pluming
himself upon it or exulting over impending calamities. There is
only one error he seems to labour under (which, we believe, he
also borrowed from Mr. Horne Tooke or Major Cartwright), the
wanting to go back to the early times of our Constitution and
history in search of the principles of law and liberty. He might
as well
\begin{quote}
  ``Hunt half a day for a forgotten dream.''

\end{quote}
Liberty, in our opinion, is but a modern invention (the growth of
books and printing)\textemdash and whether new or old, is not the
less desirable. A man may be a patriot, without being an
antiquary. This is the only point on which Sir Francis is at all
inclined to a tincture of pedantry. In general, his love of
liberty is pure, as it is warm and steady: his humanity is
unconstrained and free. His heart does not ask leave of his head
to feel; nor does prudence always keep a guard upon his tongue or
his pen. No man writes a better letter to his Constituents than
the member for Westminster; and his compositions of that kind
ought to be good, for they have occasionally cost him dear. He is
the idol of the people of Westminster: few persons have a greater
number of friends and well-wishers; and he has still greater
reason to be proud of his enemies, for his integrity and
independence have made them so. Sir Francis Burdett has often been
left in a Minority in the House of Commons, with only one or two
on his side. We suspect, unfortunately for his country, that
History will be found to enter its protest on the same side of the
question!
