%%% Chapter Heading Style %%%
%% See Section 6.5 of The Memoir Manual for more info

% The heading represents the content and style used to introduce a
% new sectional division, e.g. a chapter.

%% predefined chapter heading styles:
% default
% section
% hangnum
% companion
% article
% reparticle

% Chapters *always* start on a new page.
% Which page, recto or verso, do chapters start on?
% For onesided paper, openany (the next page).
% For twosided paper, there are 3 options:
% 1. openright: the next recto.
% 2. openleft: the next verso.
% 3. openany: the next page.
% Because we used the openright option, our chapters will begin
% on the next recto by default.
% However, we can modify this behavior by calling the respective
% declarations: \openright,\openleft,\openany.

% Okay, so what happens when a chapter heading is created?
% A number of chapter heading primitives are run in order:

%% COMMAND SUMMARY %%
%-----------------------------------------------------------------
% \clearforchapter
% \chapterheadstart
% \printchaptername, \chapternamenum, \printchapternum
% \printchapternoname
% \afterchapternum
% \printchaptertitle
% \memendofchapterhook

%% VARIABLE SUMMARY %%
%\beforechapskip
%\midchapskip
%\afterchapskip
%\chapnamefont
%\chapnumfont
%\chaptitlefont


%-----------------------------------------------------------------
% MAKE CHAPTER HEADING BEGIN
%*****************************************************************

% Henry Colburn 1825 Chapter Heading Style 
% for The Spirit of the Age by Hazlitt

% no chapter name or number
% solitary chapter title centered vertically on recto in caps
% no page header or footer
% blank verso, no page header or footer
% chapter title centered 3 parts down on recto
% followed by a ~0.5in rule
% followed by indented paragraph
% page header consists of book title centered and page number
% no footer

% NOTE: \@chapapp stores the chapter name
\makechapterstyle{colburn}{
% 1. Used to pick new page.
% \def\clearforchapter{\cleartorecto}
% 2. Called before printing chapter title and number. Inserts
%    \beforechapskip (50pts) by default.
\def\chapterheadstart{\vspace*{0.2\textheight}}
% 3. Printing "CHAPTER" name and "#N" number
% 3a. Prints the name using \chapnamefont.
\def\printchaptername{\chapnamefont \@chapapp}
% 3b. Prints the text between name and number.
\def\chapternamenum{\space}
% 3c. Prints the number using \chapnumfont.
\def\printchapternum{\chapnumfont \thechapter}
% 3*. Called if no name or num is used.
\def\printchapternonum{}
% 4. Called after printing number. Inserts \midchapskip (20pt) by
%    default.
\def\afterchapternum{\par\nobreak\vskip \midchapskip}
% 5. Prints the <title> using \chaptitlefont.
\renewcommand*{\chaptitlefont}
    {\addfontfeature{WordSpace=2}\normalfont\Huge\bfseries\scshape}
\def\printchaptertitle##1{
    \fontspec{\millerdsc}
    \centering
    \chaptitlefont
    ##1.
}
% 6. Called after printing title.  Inserts \afterchapskip (40pt)
%    by default.
%\def\afterchaptertitle{\par\nobreak\vskip \afterchapskip}
\def\afterchaptertitle{%
    \par 
    \drop{1}
    \rule{0.25\textwidth}{0.4pt}
    \par
    \drop{2}
}
% 7. Not set by default, but runs last.
\def\memendofchapterhook{\thispagestyle{chapterspirit}}
}

%*****************************************************************
% MAKE CHAPTER HEADING END

% The first paragraph of a chapter can be styled using:
\indentafterchapter
% \noindentafterchapter (default)

\chapterstyle{colburn}

